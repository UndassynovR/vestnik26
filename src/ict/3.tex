%% DONE
\id{ГРНТИ 28.23.02}{}

\begin{articleheader}
\sectionwithauthors{А.Ж.Танирбергенов, С.К.Серикбаева, Б.Тасуов, Г.Ш.Мусагулова, Л.Ақзуллақызы, Б.К.Жарменова}{МЕТОДЫ КОНТРОЛЯ УТОМЛЯЕМОСТИ ВОДИТЕЛЕЙ С ИСПОЛЬЗОВАНИЕМ ТЕХНОЛОГИЙ МАШИННОГО ОБУЧЕНИЯ}

{\bfseries \textsuperscript{1}А.Ж. Танирбергенов\authorid,
\textsuperscript{1}С.К. Серикбаева\textsuperscript{\envelope } \authorid,
\textsuperscript{2}Б. Тасуов\authorid,
\textsuperscript{3}Г.Ш. Мусагулова\authorid,
\textsuperscript{3}Л. Ақзуллақызы\authorid,
\textsuperscript{3}Б.К. Жарменова\authorid}
\end{articleheader}

\begin{affiliation}
\emph{\textsuperscript{1}Евразийский национальный университет имени Л.Н.Гумилева, Астана, Казахстан,}

\emph{\textsuperscript{2}Таразский региональный университет им.М. Х. Дулати, Тараз, Казахстан,}

\emph{\textsuperscript{3}Кызылординский университет имени Коркыт Ата, Кызылорда, Казахстан}

\raggedright \textsuperscript{\envelope }\emph{Корреспондент-автор: \href{mailto:inf_8585@mail.ru}{\nolinkurl{inf\_8585@mail.ru}}}
\end{affiliation}

В статье рассматриваются методы разработки интеллектуальной системы
мониторинга состояния усталости водителей с применением технологий
машинного обучения. Усталость водителя является одной из ведущих причин
дорожно-транспортных происшествий, особенно на длительных маршрутах и
при ночных сменах. Предложенная модель на основе коэффициента пропорции
глаз (EAR) и классификатора с использованием метода опорных векторов
(SVM) обеспечивает эффективное детектирование морганий и других
признаков усталости в режиме реального времени. Особое внимание уделено
устойчивости модели к изменениям условий освещения и ориентации головы,
что повышает надежность системы в сложных эксплуатационных условиях.
Особенностью разработанной системы является устойчивость к изменяющимся
условиям, включая изменения освещения и углов наклона головы водителя,
что улучшает надежность модели в сложных эксплуатационных условиях. В
статье отмечается, что применение данной технологии возможно в различных
интеллектуальных транспортных системах, поскольку тестирование показало
высокие показатели точности и минимальное количество ложных
срабатываний.

В результате тестирования предложенной системы были получены высокие
показатели точности, что делает ее подходящей для использования в
интеллектуальных транспортных системах. Ключевым преимуществом
предлагаемой системы является её устойчивость к изменениям условий
освещения и ориентации головы водителя, что значительно повышает
точность и надежность работы системы в сложных эксплуатационных
условиях.

{\bfseries Ключевые слова:} усталость водителя, мониторинг состояния,
машинное обучение, детектирование морганий, интеллектуальная система,
коэффициент пропорции глаз, SVM, компьютерное зрение.

\begin{articleheader}
{\bfseries МАШИНАЛЫҚ ОҚЫТУ ТЕХНОЛОГИЯЛАРЫН ҚОЛДАНА ОТЫРЫП ЖҮРГІЗУШІЛЕРДІҢ
ШАРШАҒЫШТЫҒЫН БАҚЫЛАУ ӘДІСТЕРІ}

{\bfseries
\textsuperscript{1}А.Ж. Танирбергенов,
\textsuperscript{1}С.К. Серикбаева\textsuperscript{\envelope },
\textsuperscript{2}Б. Тасуов,
\textsuperscript{3}Г.Ш. Мусагулова,
\textsuperscript{3}Л. Ақзуллақызы,
\textsuperscript{3}Б.К. Жарменова}
\end{articleheader}

\begin{affiliation}
\emph{\textsuperscript{1}Л.Н.Гумилев атындағы Еуразия ұлттық университеті, Астана қ., Қазақстан,}

\emph{\textsuperscript{2}М.Х.Дулати атындағы Тараз өңірлік университеті, Тараз қ., Қазақстан,}

\emph{\textsuperscript{3}Қорқыт Ата атындағы Қызылорда университеті, Қызылорда қ., Қазақстан,}

\emph{e-mail: inf\_8585@mail.ru}
\end{affiliation}

Мақалада машиналық оқыту технологияларын қолдана отырып, жүргізушілердің
шаршау жай-күйін мониторингілеудің зияткерлік жүйесін әзірлеу әдістері
қарастырылады. Жүргізушінің шаршауы жол-көлік оқиғаларының, әсіресе ұзақ
бағыттар мен түнгі ауысымдар кезіндегі басты себептерінің бірі болып
табылады. Ұсынылған модель көз пропорциясының коэффициенті (EAR) және
тірек векторлары (SVM) әдісін пайдалана отырып жіктеуіш негізінде нақты
уақыт режимінде морганиялар мен шаршаудың басқа да белгілерін тиімді
анықтауды қамтамасыз етеді. Үлгінің жарықтандыру жағдайларының өзгеруіне
және бастың бағдарына орнықтылығына ерекше назар аударылады, бұл күрделі
пайдалану жағдайларында жүйенің сенімділігін арттырады. Әзірленген
жүйенің ерекшелігі күрделі пайдалану жағдайларында модельдің
сенімділігін жақсартатын жарықтандыру мен жүргізуші басының көлбеу
бұрыштарының өзгеруін қоса алғанда, өзгермелі жағдайларға төзімділік
болып табылады. Мақалада аталған технологияны әртүрлі зияткерлік көлік
жүйелерінде қолдануға болатындығы атап өтілген, себебі тестілеу жоғары
дәлдік көрсеткіштерін және жалған іске қосылулардың ең аз санын
көрсетті.

Ұсынылған жүйені тестілеу нәтижесінде жоғары дәлдік көрсеткіштері
алынды, бұл оны зияткерлік көлік жүйелерінде пайдалану үшін қолайлы
етеді. Ұсынылып отырған жүйенің негізгі артықшылығы оның жарықтандыру
жағдайларының өзгеруіне және жүргізуші басының бағдарына төзімділігі
болып табылады, бұл күрделі пайдалану жағдайларында жүйенің жұмысының
дәлдігі мен сенімділігін едәуір арттырады.

{\bfseries Түйін сөздер:} жүргізушінің шаршауы, жағдайды бақылау, Машиналық
оқыту, жыпылықтауды анықтау, интеллектуалды жүйе, көздің пропорция
коэффициенті, SVM, компьютерлік көру.

\begin{articleheader}
{\bfseries METHODS OF DRIVER FATIGUE CONTROL USING MACHINE LEARNING
TECHNOLOGIES}

{\bfseries
\textsuperscript{1}А. Tanirbergenov,
\textsuperscript{1}S. Serikbayeva\textsuperscript{\envelope },
\textsuperscript{2}B. Tassuov,
\textsuperscript{3}G. Mussagulova,
\textsuperscript{3}L. Akzullakyzy,
\textsuperscript{3}B. Zharmenova}
\end{articleheader}

\begin{affiliation}
\emph{\textsuperscript{1}L.N. Gumilyov Eurasian National University, Astana, Kazakhstan,}

\emph{\textsuperscript{2}Taraz Regional University named after M.Kh. Dulaty, Taraz, Kazakhstan,}

\emph{\textsuperscript{3}Korkyt Ata Kyzylorda University, Kyzylorda,
Kazakhstan,}

\emph{e-mail: \href{mailto:inf_8585@mail.ru}{\nolinkurl{inf\_8585@mail.ru}}}
\end{affiliation}

The article discusses methods for developing an intelligent driver
fatigue monitoring system using machine learning technologies. Driver
fatigue is one of the leading causes of road accidents, especially on
long routes and night shifts. The proposed eye proportion ratio (EAR)
and support vector classifier (SVM) model provides effective real-time
detection of blinks and other signs of fatigue. Particular attention is
paid to the model' s resistance to changes in lighting
conditions and head orientation, which increases the reliability of the
system in difficult operating conditions. A feature of the developed
system is resistance to changing conditions, including changes in
lighting and tilt angles of the driver' s head, which
improves the reliability of the model in difficult operating conditions.
The article notes that the application of this technology is possible in
various intelligent transport systems, since testing has shown high
accuracy rates and a minimum number of false positives.

As a result of testing the proposed system, high accuracy indicators
were obtained, which makes it suitable for use in intelligent transport
systems. The key advantage of the proposed system is its resistance to
changes in lighting conditions and orientation of the
driver' s head, which significantly increases the
accuracy and reliability of the system in difficult operating
conditions.

{\bfseries Keywords:} driver fatigue, condition monitoring, machine
learning, blink detection, intelligent system, eye proportion
coefficient, SVM, computer vision.

\begin{multicols}{2}
{\bfseries Введение} Методы интеллектуальной системы контроля усталостного
состояния водителей с использованием технологий машинного обучения"
представляет собой обзор современных подходов и технологий, направленных
на обеспечение безопасности дорожного движения за счет раннего выявления
усталости водителей.

Усталость за рулем является широко распространенным явлением,
возникающим в результате длительного вождения или недостатка сна. Это
серьезная потенциальная угроза для безопасности дорожного движения, о
чем свидетельствует статистика: в США ежегодно происходит около 100 000
дорожно-транспортных происшествий, связанных с усталостью водителей, в
результате которых 400 000 человек получают ранения, а 1550 теряют
жизнь. В связи с этим исследования по обнаружению усталости во время
вождения становятся актуальной научно-практической задачей во всем мире.

Для повышения безопасности на дорогах необходимо регулярно проверять
состояние водителей и оценивать их манеру вождения. Прогнозирование
поведения водителя представляет собой важную часть интеллектуальных
транспортных систем и играет ключевую роль в их разработке. В ходе ряда
исследований ведущими производителями автомобилей были разработаны и
успешно внедрены несколько методик контроля состояния водителей, включая
определение сонливости и рассеянности.

В этом контексте используются различные аппаратные компоненты, такие как
мобильные камеры и датчики. Информация, полученная с гироскопа,
акселерометра и глобальной системы позиционирования (GPS), собирается
для выявления критических закономерностей, связанных с поведением
водителя. С развитием искусственного интеллекта, Интернета вещей и
технологий компьютерного зрения появляются более усовершенствованные
системы мониторинга состояния водителя и определения степени усталости,
что делает автомобили более «умными» и способными предотвращать аварии
на дорогах. Основными компонентами таких систем являются
микроконтроллеры и различные датчики, включая датчики моргания глаз,
ударов, датчики определения алкоголя и уровня топлива. API GPS и Google
Maps используются для отслеживания местоположения автомобиля. Такие
интегрированные системы могут значительно повысить безопасность
дорожного движения и снизить количество ДТП, связанных с усталостью
водителей.

{\bfseries Обзор литературы.} Проблема утомляемости является одним из
ключевых факторов дорожно-транспортных происшествий, особенно на
длительных маршрутах и при ночных сменах. Водитель, находящийся в
состоянии усталости, теряет концентрацию, замедляется реакция, и
возрастает риск аварийных ситуаций, что ставит под угрозу не только его
собственную жизнь, но и жизнь других участников дорожного движения.

С развитием современных технологий и доступностью датчиков различных
типов появилась возможность автоматизировать процесс мониторинга
состояния водителей. Системы контроля усталости, использующие методы
машинного обучения, способны анализировать множество факторов в реальном
времени, включая поведенческие, физиологические и внешние параметры,
такие как движения глаз, положение головы, частоту морганий и другие.
Использование алгоритмов искусственного интеллекта позволяет не только
повысить точность диагностики усталости, но и адаптировать систему под
особенности каждого водителя, что способствует снижению ложных
срабатываний и повышению общей эффективности системы. На данный момент
на рынке уже существуют различные системы мониторинга усталости, но
большинство из них сталкиваются с проблемами низкой точности при сложных
внешних условиях, а также высокой стоимостью оборудования, что
ограничивает их повсеместное использование. Более того, многие решения
основаны на ограниченном наборе данных, что снижает их универсальность и
адаптивность к различным сценариям эксплуатации. Именно в этом контексте
возникает необходимость разработки новых подходов, сочетающих в себе
высокую точность, устойчивость к внешним факторам и доступность. Одним
из наиболее перспективных решений является применение глубоких нейронных
сетей и методов машинного обучения для анализа данных, поступающих с
различных сенсоров в транспортных средствах. Такие системы способны в
режиме реального времени предсказывать наступление усталости водителя,
что дает возможность вовремя предупреждать его о необходимости отдыха
или смены водителя. В этой связи стоит выделить роль моделей
компьютерного зрения, анализа биометрических данных, а также интеграции
с системами предупреждения и автоматического контроля транспортных
средств.

В статье {[}1{]} рассматриваются перспективы и возможности использования
интеллектуальных систем мониторинга усталости водителя для повышения
безопасности на дорогах. Отмечается, что благодаря развитию современных
технологий, данные системы способствуют значительному снижению
количества дорожно-транспортных происшествий. Проведён анализ
показывает, что интеллектуальные системы способны на ранних этапах
выявлять отклонения в поведении водителя, что позволяет оперативно
генерировать предупреждающие сигналы и оповещения. Сделан вывод, что
наибольшая эффективность в контроле состояния водителя достигается за
счёт комбинирования различных методов и приёмов интеллектуального
анализа данных.

В статье {[}2{]} рассматриваются актуальные вопросы, касающиеся контроля
состояния водителя автомобиля. Подчёркивается, что в настоящее время
разработан широкий спектр методов, которые можно разделить на
физиологические, поведенческие и автомобильные. Системы, основанные на
физиологических и поведенческих данных, демонстрируют высокий уровень
точности и надежности в режиме реального времени. Выявлено, что
эффективность описанных методов может быть значительно повышена с
помощью технологий интеллектуального анализа, таких как нейронные сети и
компьютерное зрение. Сделан вывод о том, что интеллектуальные технологии
позволяют эффективно управлять рисками, связанными с усталостью и
отвлечением внимания водителя.

В статье {[}3{]} проведён анализ методов детектирования утомления
водителей, используемых в современной литературе. Обсуждается широкий
спектр методов, применяемых для оценки функционального состояния
человека, которое представляет собой интегральный комплекс характеристик
функций и качеств, определяющих успешность выполнения различных видов
деятельности. Функциональное состояние организма напрямую влияет на
физическое и психическое состояние человека, а также на результаты его
труда, обучения и творчества. Оценка динамического поведения водителя в
последние годы становится одним из самых актуальных направлений
исследований. Динамическая оценка включает продолжительный мониторинг,
который позволяет определять функциональные состояния.

В работе {[}4{]} рассматривается подход к распознаванию стиля вождения
водителя транспортного средства с использованием сенсоров смартфона.
Основное внимание уделено методам анализа данных, полученных с
акселерометра, гироскопа и GPS-датчика мобильного устройства, для
выявления характерных особенностей поведения водителя. Предложенный
подход позволяет классифицировать стили вождения, такие как агрессивный,
умеренный и спокойный, что может быть полезным для повышения
безопасности дорожного движения, оценки навыков водителя и разработки
персонализированных рекомендаций.

В работе {[}5{]} рассматриваются возможности прогнозирования аварийности
водителей на основе анализа их поведенческих характеристик и факторов,
влияющих на вероятность совершения дорожно-транспортных происшествий. В
исследовании акцентируется внимание на таких аспектах, как стаж
вождения, индивидуальный стиль управления транспортным средством и
склонность к рисковому поведению. Применение статистических методов
анализа и современных технологий, включая телематические устройства и
сенсоры, позволяет выявлять потенциально опасные модели поведения
водителей, что может быть полезно для повышения уровня безопасности
дорожного движения и разработки превентивных мер.

В работе {[}6{]} рассматриваются методы и средства контроля состояния
водителя автомобиля, направленные на повышение безопасности дорожного
движения. Основное внимание уделено современным технологиям, позволяющим
мониторить физиологические и поведенческие параметры водителя, такие как
частота сердечных сокращений, электрическая активность кожи, движения
глаз и головы. Обсуждаются возможности использования систем
видеонаблюдения, датчиков и интеллектуальных алгоритмов для
своевременного выявления признаков утомления, снижения концентрации и
других факторов, влияющих на управление транспортным средством. Также
анализируются перспективы интеграции таких систем в современные
автомобили.

Авторы {[}7{]} рассматривают различные подходы к обнаружению сонливости
водителей, включая анализ физиологических сигналов, характеристик лица и
стиля вождения. Обсуждаются преимущества и ограничения каждого метода, а
также предлагается использование гибридных систем для повышения точности
и надежности детекции.

В работе {[}8{]} представлены методы одновременного обнаружения
усталости и отвлечения водителя с использованием подходов на основе
компьютерного зрения и машинного обучения. Применяются сети глубокого
обучения для анализа изображений лица водителя и выявления признаков
усталости и отвлечения.

Исследование {[}9{]} предлагает систему обнаружения сонливости водителя
в реальном времени, объединяющую методы глубокого обучения и библиотеку
OpenCV. Система использует ключевые точки лица для определения признаков
усталости и может быть интегрирована в современные автомобили для
повышения безопасности на дорогах. Авторы представляют легковесную
нейронную сеть в сочетании с детектором лицевых ориентиров для выявления
усталости водителя в реальном времени. Модель обучена на специально
созданном наборе данных и достигает высокой точности, что позволяет
использовать ее в мобильных приложениях для предотвращения аварий.

Обнаружение морганий глаз является важной задачей в различных областях,
связанных с компьютерным зрением и взаимодействием человека с
компьютером. Например, контроль за морганием может использоваться для
мониторинга усталости водителей, чтобы предотвратить аварии, вызванные
сонливостью. В системах, направленных на улучшение здоровья
пользователей, длительное отсутствие морганий может свидетельствовать о
зрительной усталости и синдроме "сухого глаза", что является актуальной
проблемой для людей, работающих за компьютером. Моргание также играет
важную роль в интерфейсах "человек-компьютер", которые позволяют людям с
ограниченными физическими возможностями взаимодействовать с устройствами
через мимику и жесты, а также используется для защиты от подделки при
распознавании лиц.

Существующие методы обнаружения морганий можно разделить на две
категории: активные и пассивные. Активные методы более надежны, но
требуют специального оборудования, которое может быть дорогим и
неудобным в использовании. Например, для активных систем используются
инфракрасные камеры, которые фиксируют отражение света от глаз, или
носимые устройства, такие как очки с встроенными камерами для наблюдения
за глазами пользователя. Пассивные системы, напротив, используют
стандартные камеры, что делает их более доступными, но такие системы
могут быть менее точными из-за влияния внешних факторов, таких как
освещение или положение головы.

Многие пассивные методы для автоматического обнаружения морганий
основываются на оценке движения в области глаз. Чаще всего лицо и глаза
сначала детектируются с помощью алгоритмов вроде детектора Виолы-Джонса.
Затем движения в области глаз отслеживаются либо с помощью оптического
потока, либо путем вычисления разности между последовательными кадрами и
применения адаптивного порогового значения {[}10{]} . Другие подходы
используют шаблоны для корреляции открытых и закрытых глаз, проекции
интенсивности изображения в области глаз или активные модели формы для
определения контуров век.

Основной недостаток этих подходов заключается в том, что они предъявляют
строгие требования к условиям съемки, таким как ориентация головы
относительно камеры, разрешение изображения, освещение и динамика
движения. В особенности методы, использующие необработанную
интенсивность изображения, могут быть чувствительны к изменениям внешней
среды, несмотря на высокую производительность в реальном времени.

В последнее время в компьютерном зрении появились более надежные
детекторы лицевых ориентиров (landmark detectors), которые способны с
высокой точностью определять ключевые точки на изображениях
человеческого лица, такие как уголки глаз и контуры век. Эти детекторы
обучаются на так называемых наборах данных "в дикой природе", что делает
их устойчивыми к изменению освещенности, выражению лица и даже умеренным
поворотам головы. Средняя ошибка определения лицевых ориентиров в
современных системах составляет менее пяти процентов от расстояния между
зрачками. Современные методы позволяют детектировать ориентиры с
частотой значительно выше реального времени, что открывает новые
возможности для использования этих технологий в повседневных
устройствах.

Таким образом, в данной работе предлагается простой, но эффективный
алгоритм для детектирования морганий на основе современных детекторов
лицевых ориентиров. Из положения ориентиров выводится одна скалярная
величина - коэффициент пропорции глаз (EAR), который характеризует
степень открытия глаз на каждом кадре видео. Получив последовательность
значений EAR для каждого кадра, мы используем классификатор на основе
машины опорных векторов (SVM), чтобы детектировать моргание как
определенную последовательность изменений этого показателя во временном
окне.

{\bfseries Методы и материалы.} Ключевым элементом модели является
коэффициент пропорции глаз (Eye Aspect Ratio, EAR), который позволяет
оценить состояние глаз водителя. Этот коэффициент вычисляется на основе
ключевых точек, представляющих контуры глаза. EAR остаётся практически
постоянным при открытых глазах, стремится к нулю при их закрытии и
устойчив к изменениям положения головы или масштаба изображения.
Алгоритм работы включает детектирование лица, идентификацию глаз на
каждом кадре видеопотока и вычисление усреднённого значения EAR для
повышения точности. Такая система может эффективно обнаруживать моргания
и длительное закрытие глаз, сигнализирующее о сонливости {[}11{]}.

Для повышения точности детектирования применяется классификатор на
основе метода опорных векторов (SVM), который анализирует
последовательность значений EAR в течение временного окна, охватывающего
12 кадров. Это позволяет учитывать контекст и минимизировать ложные
срабатывания, вызванные зевотой или другими изменениями выражения лица.
Обучение модели проводится на размеченных видеопоследовательностях с
положительными (закрытые глаза) и отрицательными (открытые глаза)
примерами, что обеспечивает её адаптацию к реальным условиям {[}12{]}.

Предложенный метод демонстрирует высокую производительность в условиях
реального времени, благодаря минимальным вычислительным затратам и
устойчивости к внешним факторам, таким как освещение и положение головы.
Это делает его особенно полезным для использования в автомобильных
системах мониторинга усталости водителя. Однако ограничения, связанные с
фиксированной длиной моргания и использованием 2D-изображений, требуют
дальнейшего развития, например, через внедрение адаптивных алгоритмов и
трёхмерного анализа данных.

\emph{Математика модели.} Модель для детектирования морганий опирается
на вычисление коэффициента пропорции глаз (Eye Aspect Ratio, EAR) и
анализ его временных изменений для определения состояния глаз водителя.
EAR представляет собой величину, которая характеризует степень открытия
глаз на основе геометрии контуров глаза {[}13{]}. Это значение
используется для того, чтобы определить, открыты ли глаза в данный
момент времени или закрыты. После вычисления EAR для каждого кадра,
временные изменения этого коэффициента анализируются с помощью
классификационных методов, таких как метод опорных векторов (Support
Vector Machine, SVM). В этом разделе мы рассмотрим математическую основу
EAR, принципы его инвариантности, анализ временных изменений и
особенности применения SVM для классификации морганий.

Коэффициент EAR рассчитывается на основе местоположения ключевых точек
на глазах, которые получены с помощью детектора лицевых ориентиров. Для
каждого глаза выбираются шесть ключевых точек, расположенных вокруг
верхнего и нижнего века. Эти точки обозначаются как p1, p2, p3, p4, p5 и
p6. Коэффициент EAR определяется как отношение вертикальных расстояний
между парами точек (p2-p6 и p3-p5) к горизонтальному расстоянию между
уголками глаза (p1-p4). Формула выглядит следующим образом:

\begin{equation}
  EAR=\frac{||p_2-p_6||+||p_3-p_5||}{2\cdot ||p_1-p_4||}
\end{equation}

Здесь (p2-p6 и p3-p5) - это вертикальные расстояния между
соответствующими точками век, а (p1-p4) - это горизонтальное расстояние
между уголками глаза. Таким образом, коэффициент EAR отражает
соотношение вертикальных и горизонтальных размеров глаза. Когда глаз
полностью открыт, значения вертикальных расстояний больше, что приводит
к относительно высокому значению EAR. Когда глаз закрыт, вертикальные
расстояния стремятся к нулю, и коэффициент EAR уменьшается.

Коэффициент пропорции глаз является удобным показателем для отслеживания
состояния глаз, так как он остаётся стабильным при открытых глазах и
резко снижается при их закрытии. Важно отметить, что значение EAR можно
рассматривать как индикатор моргания или длительного закрытия глаз, что
особенно полезно для задач мониторинга усталости водителей.

Одним из главных преимуществ коэффициента EAR является его относительная
инвариантность к изменениям масштаба изображения и ориентации головы.
Это означает, что при изменении дистанции до камеры или при небольших
поворотах головы коэффициент остаётся стабильным, что делает его
надёжным признаком для мониторинга глаз. Это достигается за счёт того,
что формула EAR включает отношения между вертикальными и горизонтальными
расстояниями, которые пропорционально изменяются при масштабировании
изображения {[}14{]}. Важно отметить, что при сильных поворотах или
наклонах головы точность детекции может ухудшаться, но для большинства
реальных условий EAR достаточно устойчив к этим изменениям.

\emph{Коэффициент EAR показывает степень открытия глаза:}

- когда глаз открыт, значение EAR находится в определенном диапазоне
(обычно около 0.2 -0.3 в зависимости от конкретного человека и условий);

- когда глаз закрывается (например, при моргании), коэффициент EAR
снижается, приближаясь к нулю.

Этот простой, но эффективный показатель позволяет количественно оценить
изменение состояния глаз в реальном времени.

Использование временных изменений EAR позволяет более точно отслеживать
паттерны поведения, такие как:

- длительное закрытие глаз;

- повышенная частота морганий;

- длительные моргания, которые могут свидетельствовать о растущей
усталости.

Точность детектирования ключевых точек на лице напрямую влияет на
правильность вычисления коэффициента пропорции глаз (EAR) и общее
качество системы мониторинга состояния усталости водителей. Для оценки
этой точности используется нормализованная ошибка, основанная на средних
Евклидовых расстояниях между истинными и предсказанными координатами
ключевых точек.

Формула для расчета ошибки выглядит следующим образом:

\begin{equation}
    \epsilon = \ \frac{100}{kN}\sum_{i = 1}^{N}{\parallel x_{i} - \widehat{x_{i}} \parallel}^{2}
\end{equation}

где:

ϵ - общая ошибка детектирования в процентах;

Κ - количество ключевых точек (например, глаза, уголки век и т.д.);

N - количество изображений в наборе данных;

Xi - истинные координаты ключевых точек на iii-м изображении;

(x\_i ) ̂ - предсказанные координаты тех же ключевых точек;

∥⋅∥2 - Евклидова норма (расстояние) между истинными и предсказанными
координатами.

Эта формула используется для получения среднего значения ошибки по всем
изображениям и ключевым точкам. Умножение на 100100100 позволяет
выразить результат в процентах, что облегчает интерпретацию точности
модели. Чем меньше значение ошибки ϵ\textbackslash epsilonϵ, тем точнее
модель предсказывает положение ключевых точек, что, в свою очередь,
улучшает точность расчета коэффициента EAR.

Данный метод используется для постоянного мониторинга точности
детектирования ключевых точек. Он помогает минимизировать ошибки в
позиционировании глаз и век, что особенно важно в реальном времени,
когда необходимо быстро и точно детектировать моргания для
предупреждения усталости водителей.

{\bfseries Результаты и обсуждение.} Модель, разработанная для
идентификации личности по изображению ладони, продемонстрировала высокую
точность и стабильность в процессе тестирования. Основной задачей
системы является выделение областей интереса (ROI) на изображении
ладони, предсказание ключевых точек и классификация изображения в
соответствии с ранее обученными классами {[}15{]}.

Предложенная модель была протестирована на различных наборах данных для
обнаружения морганий в видеопоследовательностях. Основные результаты
были получены с использованием детекторов лицевых ориентиров Chehra и
Intraface, которые позволили точно локализовать ключевые точки вокруг
глаз и вычислить коэффициент пропорции глаз (EAR). Здесь подробно
описаны основные этапы работы модели с графиками и визуальными
примерами.

Для оценки работы модели использовались графики изменения коэффициента
EAR во времени. На рисунке 1 ниже показан пример того, как коэффициент
EAR изменяется для последовательности кадров из видеоролика. Каждый пик
на графике представляет момент, когда глаза открыты, а падения ---
моменты, когда глаза закрываются, что соответствует морганию.

\emph{График изменения EAR:}

- синяя линия представляет коэффициент EAR, который остаётся на уровне
около 0.3--0.4 при открытых глазах;

- когда глаз закрывается, значение EAR резко падает почти до нуля, что
показывает момент моргания;

- рядом с этим графиком также отображены результаты классификации SVM и
ручные метки, указывающие на реальное наличие морганий.
\end{multicols}

\begin{figure}[H]
	\centering
	\includegraphics[width=0.6\textwidth]{media/ict/image15}
	\caption*{Рис.1 Сравнение обнаружения ориентиров на лице с использованием Chehra и Intraface}
\end{figure}

\begin{multicols}{2}
\emph{Пример детектирования морганий:}

- на рисунке 2 показано, как пороговый метод (EAR Threshold) может
ошибочно

фиксировать моргание во время движения головы или изменения выражения
лица;

- классификатор SVM успешно отличает такие случайные движения от
реальных

морганий, анализируя изменения коэффициента EAR в более длинной
временной последовательности.

\emph{Детектирование морганий при разных условиях}

Модель была протестирована в различных условиях, включая изменения
освещения, ношение очков и повороты головы. На изображении ниже показаны
скриншоты из видеоролика с участником, который носит очки. Несмотря на
присутствие очков, модель точно детектировала моргания. Это ещё раз
подчеркивает устойчивость метода к визуальным помехам и различным
условиям съёмки.

\emph{Пример работы модели на участниках с очками:}

- красные линии на изображении показывают автоматически детектированные

ориентиры глаз, которые помогают вычислить коэффициент EAR;

- даже при наличии очков, детектор ориентиров эффективно справляется с

локализацией глаз.
\end{multicols}

\begin{figure}[H]
	\centering
	\includegraphics[width=0.6\textwidth]{media/ict/image16}
	\caption*{Рис. 2 - Детектирования морганий}
\end{figure}

\begin{multicols}{2}
Важным аспектом для работы модели в реальных условиях является её
устойчивость к поворотам головы и изменению угла зрения. На изображении
ниже показан пример работы модели, когда участник слегка поворачивает
голову в сторону. Как видно из графика EAR, даже при таких изменениях
положения головы, модель продолжает точно детектировать моменты
моргания.

1. На изображении виден поворот головы участника относительно камеры.

2. Модель всё ещё успешно детектирует моргания благодаря инвариантности
коэффициента EAR к изменениям масштаба и ориентации.

В ходе тестирования предложенной модели были получены следующие выводы:

- точность модели на различных наборах данных составила 90--99\%, в
зависимости от

условий съёмки и набора данных;

- классификатор SVM значительно улучшил точность по сравнению с простыми

пороговыми методами, особенно в сложных условиях, таких как улыбки,
ношение очков и изменения в положении головы;

-гибкость и устойчивость модели позволяют её использовать в широком
диапазоне приложений, начиная от мониторинга водителей для
предотвращения усталости, до использования в системах биометрической
идентификации и интерфейсах для людей с ограниченными возможностями.

На графиках представлены результаты сравнения точности определения
ключевых точек лица с помощью алгоритмов Chehra, Intraface, а также их
версий с уменьшенной моделью (Chehra-small и Intraface-small).

На первом графике показана ошибка локализации всех ключевых точек лица,
измеренная в процентах от межзрачкового расстояния (IOD). Чем выше
кривая, тем точнее модель. Видно, что Intraface и Chehra показывают
примерно схожую точность при малых ошибках, но Intraface имеет небольшое
преимущество. Уменьшенные версии моделей (Chehra-small и
Intraface-small) демонстрируют заметно худшие результаты, особенно при
значительных ошибках локализации.

На втором графике приведены данные только для ключевых точек глаз. Здесь
Intraface также имеет преимущество перед Chehra, особенно при низких
значениях ошибки локализации. Уменьшенные версии моделей также
показывают худшие результаты по сравнению с полными моделями, что
указывает на снижение точности при уменьшении размера модели.

Intraface демонстрирует лучшие результаты по сравнению с Chehra,
особенно на изображениях с низким разрешением. Это делает его более
подходящим для приложений, где важна высокая точность детекции
ориентиров даже при плохом качестве изображения.
\end{multicols}

\begin{figure}[H]
	\centering
	\includegraphics[width=0.45\textwidth]{media/ict/image17}
	\caption*{Рис.3. - сравнивающих производительность систем обнаружения ориентиров лица - Chehra, Intraface, Chehra-small и Intraface-small}
\end{figure}

\begin{multicols}{2}
{\bfseries Выводы.} Предложенная в статье модель для детектирования
морганий на основе коэффициента пропорции глаз (EAR) и классификатора
SVM демонстрирует высокую эффективность и применимость в задачах
реального времени. Модель основывается на детекторах лицевых ориентиров,
которые с высокой точностью распознают ключевые точки на лице, даже при
изменениях освещенности, выражений лица и поворотах головы. Это делает
модель устойчивой и адаптируемой к широкому спектру условий, что имеет
важное значение для различных приложений компьютерного зрения.

В этом исследовании представлен алгоритм для обнаружения морганий глаз в
режиме реального времени с использованием точек лицевых ориентиров. Он
находит применение в таких задачах, как мониторинг внимательности
оператора и предотвращение синдрома компьютерного зрения. Используя
современные детекторы ориентиров, исследование решает задачи, связанные
с изменениями положения головы, освещением и мимикой, что обеспечивает
надёжность и точность обнаружения морганий.

Основной вклад этой работы заключается в интеграции детекции лицевых
ориентиров с классификатором на основе машины опорных векторов (SVM) для
точного обнаружения морганий. Алгоритм использует новый признак --
коэффициент соотношения сторон глаза (EAR), который вычисляется на
основе положения ориентиров, чтобы оценить степень открытия глаза. Такой
подход превосходит передовые методы по производительности и скорости
работы в режиме реального времени на стандартных наборах данных.

Использованные в исследовании детекторы лицевых ориентиров достаточно
надёжны для отслеживания движений глаз в различных условиях. Их
способность точно улавливать изменения в открытии глаз обеспечивает
прочную основу для процесса детекции морганий. В работе подчёркивается
высокая точность и скорость этих детекторов, что имеет решающее значение
для приложений в реальном времени.

Предложенный алгоритм вычисляет коэффициент EAR на основе ориентиров и
обучает SVM для обнаружения морганий на последовательности кадров. Такая
комбинация повышает точность детекции морганий, учитывая временные
изменения, а не полагаясь только на отдельные изображения. Классификатор
SVM также различает моргания и другие движения глаз, такие как зевота
или намеренное закрытие глаз.

Одной из главных проблем предыдущих методов была их чувствительность к
условиям окружающей среды, таким как разрешение изображения или
ориентация головы. Это исследование демонстрирует, что использование
наборов данных, собранных в реальных условиях, позволяет методу детекции
ориентиров хорошо обобщать данные для различных сценариев, что повышает
надёжность системы обнаружения морганий.

В отличие от традиционных методов, основанных на оптическом потоке или
разнице интенсивности, предложенный метод использует более точный подход
на основе лицевых ориентиров. Это не только улучшает точность
обнаружения морганий, но и снижает вычислительную нагрузку, что делает
его подходящим для приложений в реальном времени.

Несмотря на высокую производительность алгоритма в большинстве условий,
допущение фиксированной продолжительности моргания является
ограничением. Так как у каждого человека моргание происходит по-разному,
адаптивный подход мог бы улучшить точность. Также использование
2D-оценки коэффициента открытия глаз может ограничить работу при
экстремальных поворотах головы, что требует рассмотрения 3D-подхода в
будущем.

В будущем исследовании можно сосредоточиться на улучшении адаптации
алгоритма к индивидуальным паттернам морганий и повышении точности
оценки состояния глаз при сильных поворотах головы. Также исследование
3D-методов детекции ориентиров может помочь решить проблемы, связанные с
вращениями головы, что ещё больше повысит надёжность системы.
\end{multicols}

\begin{center}
{\bfseries Литература}
\end{center}

\begin{references}
1. Кириллова Е.С., Сериков С.А. Интеллектуальная система безопасности
водителя, использующая обнаружение усталости // Международный журнал
гуманитарных и естественных наук.- 2024.- № 4-3 (91).- C.7-10. DOI
10.2441/2800-1000-2024-4-3-7-10

2. Кириллова Е.С., Сериков С.А. Методы и средства контроля состояния
водителя автомобиля // Международный журнал гуманитарных и естественных
наук. -2024. -№ 3-2 (90).- C.169-172. DOI
10.24412/2500-1000-2024-3-2-169-172

3. Булыгин А.О., Кашевник А.М.Анализ современных исследований в области
детектирования утомления водителя в кабине транспортного средства //
Системы анализа и обработки данных. - 2021. - № 3 (83).- С.19-36. DOI
10.17212/2782-2001-3-19-36

4. Лашков И.Б. Анализ поведения водителя при управлении транспортным
средством с использованием фронтальной камеры смартфона //
Информационно-управляющие системы.- 2017.- № 4. - C.7-18. DOI
10.15217/issn1684-8853.2017.4.7

5. Лашков, И.Б., Подход к распознаванию стиля вождения водителя
транспортного средства на основе использования сенсоров смартфона //
Информационно-управляющие системы.- 2018.- № 5.- С.2-12. DOI
10.31799/1684-8853-2018-5-2-12

6. Лобанова Ю. И. О возможностях прогноза аварийности водителей //
Психология. \\Психофизиология.- 2017.- № 10 (1).- C.74-87. DOI
10.14529/psy170108

7. Nasri I. et al. A Review of Driver Drowsiness Detection Systems:
Techniques, Advantages and \\Limitations. - 2022. DOI
10.48550/arXiv.2206.07489

8. Wu D. Improving automatic detection of driver fatigue and distraction
using machine learning.-2024.\\//arXiv preprint arXiv:2401.10213.

9. Singh Sengar S., Kumar A., Singh O. VigilEye-\/-Artificial
Intelligence-based Real-time Driver \\Drowsiness Detection.-2024. DOI
10.48550/arXiv.2406.15646

Jose J. et al. SleepyWheels: An Ensemble Model for Drowsiness Detection
leading to Accident Prevention.-2022. DOI 10.48550/arXiv.2211.00718

10.L. M. Bergasa, J. Nuevo, M. A. Sotelo, and M. Vazquez. Real-time
system for monitoring driver vigilance. In IEEE Intelligent Vehicles
Symposium, 2004
DOI~\href{https://doi.org/10.1109/IVS.2004.1336359}{10.1109/IVS.2004.1336359}.

11.T. Danisman, I. Bilasco, C. Djeraba, and N. Ihaddadene. Drowsy driver
detection system using eye blink patterns. In Machine and Web
Intelligence (ICMWI).2010.

DOI
\href{http://dx.doi.org/10.1109/ICMWI.2010.5648121}{10.1109/ICMWI.2010.5648121}

12.A. Sahayadhas, K. Sundaraj, and M. Murugappan. Detecting driver
drowsiness based on sensors: A review.// Sensors.-2012-Vol.12(12).-
P.16937-16953.\href{https://doi.org/10.3390/s121216937}{DOI
/10.3390/s121216937}

13.W. H. Lee, E. C. Lee, and K. E. Park. Blink detection robust to
various facial poses//Journal of \\Neuroscience Methods. - 2010.-
Vol.193(2):356-72.
DOI~\href{https://doi.org/10.1016/j.jneumeth.2010.08.034}{10.1016/j.jneumeth.2010.08.034}

14.Medicton group. The system I4Control. http:// www.i4tracking.cz/.
Date of address- 14.12.2024

15.D. Torricelli, M. Goffredo, S. Conforto, and M. Schmid. An adaptive
blink detector to initialize and update a view-basedremote eye gaze
tracking system in a natural scenario// Pattern Recognition Letters.-
2009.-Vol. 30(12).-P.1144
-1150.\href{https://doi.org/10.1016/j.patrec.2009.05.014}{DOI
/10.1016/j.patrec.2009.05.014}
\end{references}

\begin{center}
{\bfseries References}
\end{center}

\begin{references}
1.Kirillova E.S., Serikov S.A. Intellektual' naja sistema
bezopasnosti voditelja, ispol' zujushhaja \\obnaruzhenie
ustalosti // Mezhdunarodnyj zhurnal gumanitarnyh i estestvennyh nauk.-
2024.- № 4-3 (91).- C.7-10. DOI 10.2441/2800-1000-2024-4-3-7-10. {[}in
Russian{]}

2.Kirillova E.S., Serikov S.A. Metody i sredstva kontrolja sostojanija
voditelja avtomobilja // \\Mezhdunarodnyj zhurnal gumanitarnyh i
estestvennyh nauk. -2024. -№ 3-2 (90).- C.169-172. DOI\\
10.24412/2500-1000-2024-3-2-169-172 .{[}in Russian{]}

3.Bulygin A.O., Kashevnik A.M.Analiz sovremennyh issledovanij v oblasti
detektirovanija utomlenija voditelja v kabine transportnogo sredstva //
Sistemy analiza i obrabotki dannyh. - 2021. - № 3 (83).- S.19-36. DOI
10.17212/2782-2001-3-19-36. {[}in Russian{]}

4. Lashkov I.B. Analiz povedenija voditelja pri upravlenii transportnym
sredstvom s ispol' zovaniem frontal' noj
kamery smartfona // Informacionno-upravljajushhie sistemy.- 2017.- № 4.
- C.7-18. DOI \\10.15217/issn1684-8853.2017.4.7. {[}in Russian{]}

5. Lashkov, I.B., Podhod k raspoznavaniju stilja vozhdenija voditelja
transportnogo sredstva na osnove ispol' zovanija sensorov
smartfona // Informacionno-upravljajushhie sistemy.- 2018.- № 5.-
S.2-12. DOI 10.31799/1684-8853-2018-5-2-12. {[}in Russian{]}

6. Lobanova Ju. I. O vozmozhnostjah prognoza avarijnosti voditelej //
Psihologija. Psihofiziologija.- 2017.- № 10 (1).- C.74-87. DOI
10.14529/psy170108. {[}in Russian{]}

7. Nasri I. et al. A Review of Driver Drowsiness Detection Systems:
Techniques, Advantages and \\Limitations. - 2022. DOI
10.48550/arXiv.2206.07489

8. Wu D. Improving automatic detection of driver fatigue and distraction
using machine learning.-2024.\\//arXiv preprint arXiv:2401.10213.

9. Singh Sengar S., Kumar A., Singh O. VigilEye-\/-Artificial
Intelligence-based Real-time Driver \\Drowsiness Detection.-2024. DOI
10.48550/arXiv.2406.15646

Jose J. et al. SleepyWheels: An Ensemble Model for Drowsiness Detection
leading to Accident Prevention.-2022. DOI 10.48550/arXiv.2211.00718

10.L. M. Bergasa, J. Nuevo, M. A. Sotelo, and M. Vazquez. Real-time
system for monitoring driver vigilance. In IEEE Intelligent Vehicles
Symposium, 2004
DOI~\href{https://doi.org/10.1109/IVS.2004.1336359}{10.1109/IVS.2004.1336359}.

11.T. Danisman, I. Bilasco, C. Djeraba, and N. Ihaddadene. Drowsy driver
detection system using eye blink patterns. In Machine and Web
Intelligence (ICMWI).2010.

DOI
\href{http://dx.doi.org/10.1109/ICMWI.2010.5648121}{10.1109/ICMWI.2010.5648121}

12.A. Sahayadhas, K. Sundaraj, and M. Murugappan. Detecting driver
drowsiness based on sensors: A review.// Sensors.-2012-Vol.12(12).-
P.16937-16953.\href{https://doi.org/10.3390/s121216937}{DOI
/10.3390/s121216937}

13.W. H. Lee, E. C. Lee, and K. E. Park. Blink detection robust to
various facial poses//Journal of \\Neuroscience Methods. - 2010.-
Vol.193(2):356-72.
DOI~\href{https://doi.org/10.1016/j.jneumeth.2010.08.034}{10.1016/j.jneumeth.2010.08.034}

14.Medicton group. The system I4Control. http:// www.i4tracking.cz/.
Date of address- 14.12.2024

15.D. Torricelli, M. Goffredo, S. Conforto, and M. Schmid. An adaptive
blink detector to initialize and update a view-basedremote eye gaze
tracking system in a natural scenario// Pattern Recognition Letters.-
2009.-Vol. 30(12).-P.1144
-1150.\href{https://doi.org/10.1016/j.patrec.2009.05.014}{DOI
/10.1016/j.patrec.2009.05.014}
\end{references}

\begin{authorinfo}
\emph{{\bfseries Сведения об авторах}}

Танирбергенов А.Ж{\bfseries .}- и.о.доцент, заведующий кафедрой
криптологии, Евразийского национального университета им.Л. Н. Гумилева,
Астана, Казахстан, е-mail:
\href{mailto:t.adilbek@mail.ru}{\nolinkurl{t.adilbek@mail.ru}};

Серикбаева С.К. - PhD, старший преподаватель кафедры информационных
систем Евразийского национального университета им. Л. Н. Гумилева,
Астана, Казахстан, е-mail:
\href{mailto:inf_8585@mail.ru}{\nolinkurl{inf\_8585@mail.ru}};

Тасуов Б. - ассоцированный профессор кафедры Физика и информатика
Таразского регионального университета имени М.Х. Дулати, Тараз,
Казахстан, е-mail:
\href{mailto:b.tasuov@dulaty.kz}{\nolinkurl{b.tasuov@dulaty.kz}};

Мусагулова Г.Ш. - Кызылординский университет имени Коркыт Ата, старший
преподаватель образовательной программы «Информатика и
информационно-коммуникационные технологии», г. Кызылорда, Казахстан,
e-mail: \\\href{mailto:erkegulia@mail.ru}{\nolinkurl{erkegulia@mail.ru}};

Акзуллакызы Л. - Кызылординский университет имени Коркыт Ата, старший
преподаватель образовательной программы «Информатика и
информационно-коммуникационные технологии», г. Кызылорда, Казахстан,
e-mail: \\\href{mailto:la.z_1986@mail.ru}{\nolinkurl{la.z\_1986@mail.ru}};

Жарменова Б. К. - Кызылординский университет имени Коркыт Ата, старший
преподаватель образовательной программы «Информатика и
информационно-коммуникационные технологии», г. Кызылорда, Казахстан,
e-mail: \\\href{mailto:81_bota@mail.ru}{\nolinkurl{81\_bota@mail.ru}}

\emph{{\bfseries Information about authors}}

Tanirbergenov А.Adilbek - Acting Associate Professor, Head of the
Department of Cryptology, L.N. Gumilyov Eurasian National University,
Astana, Kazakhstan, е-mail:
\href{mailto:t.adilbek@mail.ru}{\nolinkurl{t.adilbek@mail.ru}};

Serikbayeva S. - PhD, Senior Lecturer of the Department of Information
Systems, L.N. Gumilyov Eurasian National University, Astana, Kazakhstan,
е-mail: \href{mailto:inf_8585@mail.ru}{\nolinkurl{inf\_8585@mail.ru}};

Tassuov B.- Associate Professor, Department of Physics and Informatics,
Taraz Regional University named after M.Kh. Dulaty, Taraz, Kazakhstan,
е-mail:
\href{mailto:b.tasuov@dulaty.kz}{\nolinkurl{b.tasuov@dulaty.kz}};

Mussagulova G. -, Korkyt Ata Kyzylorda University, senior lecturer of
the educational program "Informatics and Information Communication
Technologies", Kyzylorda, Kazakhstan,

e-mail: \href{mailto:erkegulia@mail.ru}{\nolinkurl{erkegulia@mail.ru}};

Akzullakyzy L{\bfseries .} - Korkyt Ata Kyzylorda University, senior
lecturer of the educational program "Informatics and Information
Communication Technologies", Kyzylorda, Kazakhstan,

e-mail: \href{mailto:laz_1986@mail.ru}{\nolinkurl{laz\_1986@mail.ru}};

Zharmenova B. - Korkyt Ata Kyzylorda University, senior lecturer of the
educational program "Informatics and Information Communication
Technologies", Kyzylorda, Kazakhstan,

e-mail: \href{mailto:81_bota@mail.ru}{\nolinkurl{81\_bota@mail.ru}}
\end{authorinfo}
