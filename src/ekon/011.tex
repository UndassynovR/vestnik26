%% DONE
\id{IRSTI 06.61.53}{https://doi.org/10.58805/kazutb.v.1.26-907}

\begin{articleheader}
\sectionwithauthors{Sh. Smagulova A. Omarov, A. Alibekova, B. Omarova}{FEATURES OF ECONOMIC DEVELOPMENT OF THE TRANSPORT SYSTEM THE CITY OF ALMATY}

{\bfseries  
\textsuperscript{1}Sh. Smagulova\textsuperscript{\envelope } \alink{https://orcid.org/0000-0002-8455-4531},
\textsuperscript{2}A. Omarov\alink{https://orcid.org/0009-0008-7261-9028},
\textsuperscript{3}A. Alibekova\alink{https://orcid.org/0009-0001-4048-9520},
\textsuperscript{4}B. Omarova\alink{https://orcid.org/0009-0002-4756-9352}}
\end{articleheader}

\begin{affiliation}
\textsuperscript{\emph{1,3}}\emph{Kenzhegali Sagadiyev University of International Business, Almaty, Kazakhstan,}

\emph{\textsuperscript{2,4} International university of transport and humanities, Almaty, Kazakhstan,}

\raggedright \textsuperscript{\envelope }{\em Correspondent-author: \href{mailto:shsmagulova@mail.ru}{\nolinkurl{shsmagulova@mail.ru}}}
\end{affiliation}

Passenger transport is an important part of urban infrastructure, and
economic efficiency is crucial for the functioning of cities. The study
uses the methods of multiple linear regression, correlation analysis and
dispersion analysis. The aspects of increasing the investment
attractiveness of the transport sector are investigated.

The results of the analysis show that the number of subway cars,
passenger transportation revenues, CPI and fixed capital investments
have a significant impact on passenger transportation revenues in
Almaty. The multiple regression model showed a high degree of
explanation of data fluctuations. In particular, the following factors
were assessed here: the number of passenger buses, trolleybuses, subway
cars, transportation tariffs and the amount of investment, etc.
Correlation analysis showed that there is an important relationship
between the number of passenger vehicles and transportation revenues,
the rate of consumer price index, transportation tariffs and the volume
of investment in fixed capital.

The purpose of the study is to analyze the economic factors affecting
passenger transportation revenues and the level of GRP in Almaty. The
regional development program "Almaty - 2025" is aimed at modernizing the
transport infrastructure, improving the quality of services and
increasing investment in the transport sector. The data obtained
indicate the importance of organizing the transport infrastructure and
the level of development of economic factors in generating income. The
paper provides recommendations for improving the development of the
transport system of the city of Almaty.

{\bfseries Keywords}: transport infrastructure, econometric model,
passenger transportation, the city of Almaty

\begin{articleheader}
{\bfseries ОСОБЕННОСТИ ЭКОНОМИЧЕСКОГО РАЗВИТИЯ ТРАНСПОРТНОЙ СИСТЕМЫ ГОРОДА АЛМАТЫ}

{\bfseries  
\textsuperscript{1}Ш. Смагулова\textsuperscript{\envelope },
\textsuperscript{2}A. Омаров,
\textsuperscript{3}А. Алибекова,
\textsuperscript{4}Б. Омарова}
\end{articleheader}

\begin{affiliation}
\emph{\textsuperscript{1,3} Университет Международного Бизнеса имени Кенжегали Сагадиева, Алматы, Казахстан,}

\emph{\textsuperscript{2,4} Международный Транспортно-Гуманитарный Университет, Алматы, Казахстан,}

\emph{e-mail: \href{mailto:shsmagulova@mail.ru}{\nolinkurl{shsmagulova@mail.ru}}}
\end{affiliation}

Пассажирский транспорт является важной частью городской инфраструктуры,
и экономическая эффективность имеет решающее значение для
функционирования городов. В исследовании используются методы
множественной линейной регрессии, корреляционного анализа и
дисперсионного анализа. Исследуются аспекты повышения инвестиционной
привлекательности транспортного сектора.

Результаты анализа показывают, что количество вагонов метрополитена,
доходы от пассажирских перевозок, ИПЦ и основные капитальные вложения
оказывают значительное влияние на доходы от пассажирских перевозок в
Алматы. Модель множественной регрессии показала высокую степень
объяснения колебаний данных. В частности, здесь были оценены такие
факторы: количество пассажирских автобусов, троллейбусов, вагонов метро,
тарифов на перевозки и количество инвестиций и др. Корреляционный анализ
показал, что существует важная связь между количеством пассажирских
транспортных средств и доходами от перевозок, темпами индекса
потребительских цен, тарифами на перевозки и объемом инвестиций в
основной капитал.

Целью исследования является анализ экономических факторов, влияющих на
доходы от пассажирских перевозок и уровень ВРП в городе Алматы.
Региональная программа развития "Алматы - 2025" направлена на
модернизацию транспортной инфраструктуры, повышение качества услуг и
увеличение инвестиций в транспортный сектор. Полученные данные
свидетельствуют о важности организации транспортной инфраструктуры и
уровня развития экономических факторов в получении дохода. В работе
приведены рекомендации по совершенствованию развития транспортной
системы города Алматы.

{\bfseries Ключевые слова:} транспортная инфраструктура, эконометрическая
модель, пассажирские перевозки, город Алматы

\begin{articleheader}
{\bfseries АЛМАТЫ ҚАЛАСЫНЫҢ КӨЛІК ЖҮЙЕСІНІҢ ЭКОНОМИКАЛЫҚ ДАМУ ЕРЕКШЕЛІКТЕРІ}

{\bfseries  
\textsuperscript{1}Ш. Смагулова\textsuperscript{\envelope },  
\textsuperscript{2}A. Омаров,  
\textsuperscript{3}А. Алибекова,  
\textsuperscript{4}Б. Омарова}
\end{articleheader}

\begin{affiliation}
\emph{\textsuperscript{1,3} Кенжеғали Сағадиев атындағы Халықаралық Бизнес Университеті, Алматы, Қазақстан,}

\emph{\textsuperscript{2,4} Халықаралық Көлік-Гуманитарлық Университеті, Алматы, Қазақстан,}

\emph{e-mail: \href{mailto:shsmagulova@mail.ru}{\nolinkurl{shsmagulova@mail.ru}}}
\end{affiliation}

Жолаушылар көлігі қалалық инфрақұрылымның маңызды бөлігі болып табылады,
ал экономикалық тиімділік қалалардың жұмыс істеуі үшін өте маңызды.
Зерттеуде көп сызықтық регрессия, корреляциялық талдау және дисперсияны
талдау әдістері қолданылады. Көлік секторының инвестициялық
тартымдылығын арттыру аспектілері зерттеледі.

Талдау нәтижелері Алматы қаласындағы жолаушылар кірісіне метро
вагондарының саны, жолаушылар кірісі, ТБИ және негізгі капитал салымдары
айтарлықтай әсер ететінін көрсетті. Көптік регрессия моделі деректердегі
ауытқуларды түсіндірудің жоғары дәрежесін көрсетті. Атап айтқанда, мұнда
келесі факторлар бағаланды: жолаушылар автобустарының, троллейбустардың,
метро вагондарының саны, тасымалдау тарифтері мен инвестиция көлемі және
т.б. Корреляциялық талдау жолаушылар көлігінің саны мен тасымалдаудан
түскен кіріс, тұтыну бағаларының индексінің нормасы, тасымалдау
тарифтері және негізгі капиталға инвестиция көлемі арасында маңызды
байланыс бар екенін көрсетті.

Зерттеудің мақсаты -- Алматы қаласындағы жолаушылар тасымалының кірісіне
және ЖӨӨ деңгейіне әсер ететін экономикалық факторларды талдау. «Алматы
-- 2025» өңірлік даму бағдарламасы көлік инфрақұрылымын жаңғыртуға,
қызмет көрсету сапасын арттыруға және көлік саласына инвестиция көлемін
арттыруға бағытталған. Алынған мәліметтер көлік инфрақұрылымын
ұйымдастырудың маңыздылығын және табыс алудағы экономикалық факторлардың
даму деңгейін көрсетеді. Жұмыста Алматы қаласының көлік жүйесін дамытуды
жақсарту бойынша ұсыныстар берілген.

{\bfseries Түйін сөздер:} көлік инфрақұрылымы, эконометрикалық модель,
жолаушылар тасымалы, Алматы қаласы

\begin{multicols}{2}
{\bfseries Introduction.} The transport system is an integral part of the
urban infrastructure, which plays a significant role in ensuring
economic and social mobility. This is especially true for megacities
such as Almaty, where the efficiency of transport is directly related to
the quality of life of the population and the dynamics of socio-economic
development. Analysis of the state and development of the transport
system is becoming an urgent task due to the rapid growth of the urban
population, urbanization and growing demand for transport services.

In addition, according to the "Concept of Investment Policy of
Kazakhstan until 2029" (Resolution of the Government of the Republic of
Kazakhstan dated October 18, 2024, No. 868), the transport sector
includes important measures to increase investment in: construction of
new roads, renewal of freight transport and improvement of the
infrastructure of the transport hub.

Investments in fixed assets and modernization of vehicles are key
factors in increasing the economic efficiency of passenger
transportation systems. An integral part of this process are public and
private investments aimed at developing infrastructure and improving the
quality of transport services. The main objective of the study is to
determine the factors that have the greatest impact on the revenues
received from passenger transportation in Almaty. The study of factors
affecting passenger transportation revenues not only identifies the main
factors of economic activity in the region, but also provides
recommendations for optimizing the city' s transport
system. The study is conducted within the framework of the «Strategic
Plan of the Almaty Development Program -- 2025», which is aimed at
modernizing and developing transport infrastructure, improving the
quality of services and the efficiency of urban transport systems. The
inclusion of initiatives such as updating vehicles, expanding and
improving transport routes is part of a long-term plan aimed at
improving the quality of life in the city.

In recent years, the transport system of large cities has been actively
studied by researchers, since efficient transport infrastructure plays
an important role in economic development and quality of life of the
population. Many studies are aimed at studying the impact of transport
factors on the economic development of cities.

Thus, energy policy in European countries stimulates economical and
rational energy consumption, as well as increased energy productivity
through the use of innovative and digital technologies. In particular,
the work of Gruetzmacher, Vaz, Ferreira (2025) is devoted to the
sustainability of transport activities in the EU. Here, scientists have
implemented an economic analysis of the transport industry within the
framework of applying alternative ``Benefit-of-the-Doubt'' models
{[}1{]}. The authors concluded that in order to reduce the negative
impacts on harmful emissions from transport, it is necessary to develop
a strategy for improving green technologies and implement energy
efficiency methods.

At the same time, today the organization and development of underground
transport infrastructure "CCS" (Peng et al. 2024) is of significant
importance for reducing harmful CO2 emissions. We believe that the use
of the "CCS" system will provide an opportunity and incentives for the
promotion of environmental monitoring projects and transport
infrastructure sensing technology {[}2{]}. In addition, such systems
will lead to increased transport safety in cities.

An analysis of existing studies on the impact of transport systems on
the economic development of cities has shown that this issue is still
relevant and multifaceted.

In general, transport infrastructure plays an important role in
increasing population mobility, accessibility to markets and quality of
life. However, the degree of its impact on the economic development of
the city remains uneven. Almaty, as the largest economic center of
Kazakhstan, faces a number of problems related to the modernization of
the transport system, increasing tariffs and the impact of inflation.
These factors not only affect production, but also create general
conditions for the socio-economic well-being of the city.

Thus, this article describes the main relationship between the economic
indicators of Almaty and revenues from passenger transportation. It is
expected that the development of transport infrastructure and increased
investment in the transport sector will have a significant impact on the
growth of revenues from passenger transportation. To better understand
these relationships, multiple correlation analysis was conducted to
identify the most important factors affecting the efficiency of the
transport sector. Addressing these issues is essential to developing an
effective passenger transport management strategy and increasing the
economic sustainability of urban transport systems.

{\bfseries Materials and methods.} There are many approaches and models
that explore the relationship between transport infrastructure and
economic performance. However, the question of how the transport system
affects the long-term economic development of the city remains an active
topic of discussion in the scientific community.

In a European study, the authors Mamcarz et al. (2023) conducted
economic and mathematical modeling on determining the role of public
transport in Europe for consumers of these services {[}3{]}. Using a
survey, the researchers proved that there are both positive and negative
results of satisfaction with transport services among 642 respondents.
For example, it was found that waiting time and free travel on public
transport seem to be a positive factor in the consumption of services.

There are works that reveal the level of influence of different types of
transport on travel in cities (Mouratidis et al. 2023). For example,
here, using the methodology of quantitative and qualitative assessment
of statistical data, it was checked how passengers are satisfied with
public transport services in Greek cities. Based on the results
obtained, it was found that it is necessary to develop urban transport
infrastructure, modernize the pedestrian and road system {[}4{]}.

Castagna, Lobo, Coppola, Couto (2024) carried out a comparative
assessment of the performance of 23 European metro systems using
econometric regression based on the Cobb-Douglas production function.
The findings demonstrated high efficiency of metro service companies in
the short term {[}5{]}. This characterizes the achievement of maximum
profitability and profit from the operation of subways in the first
years. In subsequent years, operating costs for the maintenance of
subways increase significantly.

Therefore, given the social significance of the metro, local and state
support for this type of transport should be provided. At the beginning
of the 21-st century, bus rapid transit (BRT) appears to be the most
innovative type of transportation for urban passengers.

However, as Alnsour (2023) notes, the introduction of a new type of
"BRT" presents certain organizational, technical and economic risks
{[}6{]}. In this regard, the analysis of the questionnaire by A. Alnsour
using the SPSS software showed that it is necessary to assess the above
risks. For this, already at the initial stage it is necessary to
carefully consider the planning process, then at the stage of technical
operation and practical implementation of BRT. The results of the
modeling revealed the following risks of the transport system:
volatility of fuel prices, underdeveloped infrastructure, traffic
violations, corruption, etc. Accordingly, for the normal course of BRT
activities, it was recommended to consider the noted risk factors.

First of all, we are talking about a constant influx of investments in
the development of transport infrastructure and optimization of local
management of city routes and bus schedules. The purpose of this study
is to analyze the influence of various factors, including types of
transport on the level of GRP using the example of Almaty. To study
these relationships, correlation analysis and the multiple linear
regression method are used, allowing us to assess how various
independent variables affect the dependent variable. Statistical tests
(t-statistics) were conducted, including regression coefficient analysis
to assess the importance of the impact of each independent variable on
the poverty level. It is important to note that the significance level
of 0.05 is adopted for statistical significance, meaning that if the
p-value of a variable is below this threshold, it is considered
statistically significant for the model. After analysis using the
multiple linear regression method, a model will be created to measure
the impact of each factor on the level of GRP.

The regression coefficient reflects the intensity and direction of the
relationship between the independent and dependent variables. High
coefficients indicate a significant impact on the level of GRP, while
\emph{p}-values \hspace{0pt}\hspace{0pt}demonstrate the statistical
significance of these relationships. Large standard errors or high
p-values \hspace{0pt}\hspace{0pt}may indicate instability or lack of
significance of the estimate, which requires careful interpretation of
the results. The overall significance of the model and individual
predictors is assessed using these statistical tests. The following
hypotheses were tested in the study:

Null Hypothesis (\emph{H₀}): Economic indicators such as fares, consumer
price index, and fixed capital investment, as well as the number of
passenger buses, trolleybuses, and subway, do not have a statistically
significant effect on passenger transportation revenues in Almaty.

Alternative Hypothesis 1 (\emph{H1}): The number of passenger buses,
trolleybuses, and subway has a statistically significant effect on
passenger transportation revenues in Almaty.

Alternative Hypothesis 2 (\emph{H2}): Changes in the fare index and CPI
significantly affect passenger transportation revenues in Almaty.

Alternative Hypothesis 3 (\emph{H3}): Fixed capital investment plays an
important role in increasing passenger transportation revenues in
Almaty.

The study examines the following main questions.

1) The role of transport infrastructure in generating revenues. How do
the number and type of transport (buses, trolleybuses, metro cars)
affect the efficiency of transportation? This clearly shows to what
extent the number of transports is a decisive factor in the economic
efficiency of transportation.

2) Economic factors affecting revenues. How do the fare index and the
consumer price index (CPI) affect revenues from passenger
transportation? This issue analyzes how changes in tariff policy and
inflation processes affect revenues from transportation.

3) Investments in infrastructure and their impact on the transport
economy. How do capital investments, including the modernization of
vehicles and the development of infrastructure, affect revenues from
passenger transportation? The problem is to assess the effectiveness of
public and private investments in the development of the
city' s transport sector.

This analysis is based on the longitudinal construction of time series
of the study, since it covers changes in indicators over a long period
of time (14 years).

Sample. Sample type: secondary data. Number of observations: 140
observations (from 2010 to 2023) for each variable (one for each year).

The econometric model is presented as follows:

\emph{Y = β0 + Х1 * β1+ Х2 * β2 + Х3 * β3 + Х4е * β4 + Х5 * β5 + Х6 * β6
+} Х7 \emph{* β7 +ϵ}

where:

1. \emph{Y}--- Gross regional product, million tenge, dependent
variable.

2. Independent variables:

\emph{X1} --- Number of passenger buses, units

\emph{X2} --- Number of passenger trolleybuses, units

\emph{X3} --- Number of passenger subway cars

\emph{X4} --- Revenue from passenger transportation, million tenge

\emph{X5} --- Indices of transportation tariffs (in \%)

\emph{X6} --- Rates of CPI \%

\emph{X7} --- Investments in fixed capital, in \%

3. \emph{β0} is a constant (intercept).

\emph{β1, β2, β3, β4, β5, β6, β7} --- coefficients for each independent
variable, showing the magnitude and direction of their influence on GRP.

4. \emph{ϵ} --- error coefficient.

Expected results: Assessment of the influence of factors: determination
of the degree and nature of the influence of independent variables on
GRP.

Statistical significance: Identification of statistically significant
influencing factors to confirm or refute hypotheses.

Thus, the research methodology offers a comprehensive approach to the
analysis of GRP, aimed at identifying the main factors and assessing
their impact on the development of effective socio-economic strategies
in the Almaty transport system.

{\bfseries Results and discussions.} Currently, the world is undergoing an
active process of urbanization (Larriva et al. 2023). The growth of the
territory and population of cities requires the same pace of
construction and new introduction of public transport routes {[}7{]}.
Larriva et al. (2023) used econometric methodology to determine the
level of influence of urban public transport projects on the well-being
of different socio-economic groups. Thus, the researchers analyzed the
influence of the first metro line on the adoption of rational decisions
on planning investments in the development and mobility of transport
infrastructure.

The transport system plays a very important role in the economic
development of the city. It directly or indirectly affects various
aspects of life, including production, employment, consumption and
distribution of goods and services.

Today, there is significant urban sprawl in the world (Giduturi, 2015).
For the efficient operation of urban space, the availability of
passenger transportation and services is necessary {[}8{]}. It is noted
that the efficiency of urban planning should be observed in order to
ensure the sustainability of urban transport infrastructure.

In the modern world, transport is an important element of infrastructure
that contributes not only to the efficient movement of people and goods,
but also to economic growth. Almaty, as the largest economic and
cultural center of Kazakhstan, has a developed transport system, which
includes buses, trolleybuses, metro and other types of transport. In
recent years, the load on the transport infrastructure has increased due
to the growth of the urban population and increased economic activity.
Thus, the efficient operation of the city' s transport
system directly affects the economy.

This analysis is devoted to identifying and assessing the factors
affecting the gross regional product. The study was conducted using
multiple regression, using various transport and economic variables to
assess the impact of various factors on passenger revenues. Among the
included variables are the number of passenger buses, trolleybuses,
metro cars, the tariff index for transportation, the consumer price
index (CPI) and investment in fixed assets. The main objective of the
study is to determine the statistical significance of each factor and
their impact on passenger transportation revenues (Table 1).
\end{multicols}

\begin{table}[H]
\caption*{Table 1 - Initial data}
\centering
\resizebox{\linewidth}{!}{%
\begin{tblr}{
  row{even} = {c},
  row{1} = {c},
  row{3} = {c},
  row{5} = {c},
  row{7} = {c},
  row{9} = {c},
  row{11} = {c},
  row{13} = {c},
  row{15} = {c},
  cell{1}{1} = {r=2}{},
  cell{17}{1} = {c=9}{},
  vlines,
  hlines,
}
Years                                                                                                                                         & Y                                          & Х1                                   & Х2                                          & Х3                                  & Х4                                                        & Х5                                            & Х6                  & Х7                                      \\
                                                                                                                                              & {Gross\\regional\\product,\\million tenge} & {Number of\\passenger\\buses,\\units} & {Number of\\passenger\\trolleybuses,\\units} & {Number of\\passenger\\subway cars} & {Revenue from\\passenger\\transportation,\\million tenge} & {Indices of\\transportation\\tariffs (in \%)} & {Rates\\of\\CPI \%} & {Investments\\in fixed\\capital, in \%} \\
2~010,00                                                                                                                                      & 3 923 412,6                                & 1 564,00                             & 191,00                                      & 0,00                                & 80~000,00                                                 & 100,20                                        & 107,80              & 100,40                                  \\
2~011,00                                                                                                                                      & 4 860 213,9                                & 1 780,00                             & 113,00                                      & 7,00                                & 90~000,00                                                 & 100,10                                        & 107,40              & 100,60                                  \\
2~012,00                                                                                                                                      & 5 715 879,2                                & 1 580,00                             & 136,00                                      & 7,00                                & 110~000,00                                                & 128,30                                        & 106,00              & 106,30                                  \\
2~013,00                                                                                                                                      & 7 127 916,4                                & 1 709,00                             & 239,00                                      & 7,00                                & 130~000,00                                                & 102,50                                        & 104,80              & 102,50                                  \\
2~014,00                                                                                                                                      & 8 143 570,2                                & 1 855,00                             & 212,00                                      & 7,00                                & 140~000,00                                                & 113,50                                        & 107,40              & 94,80                                   \\
2~015,00                                                                                                                                      & 9 100 006,0                                & 9 327,00                             & 135,00                                      & 7,00                                & 160~000,00                                                & 153,80                                        & 113,60              & 102,40                                  \\
2~016,00                                                                                                                                      & 10 601 347,8                               & 9 058,00                             & 139,00                                      & 7,00                                & 180~000,00                                                & 107,80                                        & 108,50              & 104,00                                  \\
2~017,00                                                                                                                                      & 11 893 225,9                               & 8 433,00                             & 160,00                                      & 7,00                                & 219~661,50                                                & 100,60                                        & 107,10              & 104,80                                  \\
2~018,00                                                                                                                                      & 12 132 649,7                               & 8 315,00                             & 172,00                                      & 8,00                                & 257~365,20                                                & 110,10                                        & 105,30              & 110,90                                  \\
2~019,00                                                                                                                                      & 13 546 958,4                               & 8 758,00                             & 180,00                                      & 9,00                                & 295 068,80                                                & 103,00                                        & 105,40              & 108,30                                  \\
2~020,00                                                                                                                                      & 13 459 802,6                               & 2 245,00                             & 196,00                                      & 10,00                               & 144 197,10                                                & 104,80                                        & 107,50              & 117,90                                  \\
2~021,00                                                                                                                                      & 15 000 060,4                               & 2 058,00                             & 196,00                                      & 10,00                               & 306 140,40                                                & 113,30                                        & 108,40              & 115,10                                  \\
2~022,00                                                                                                                                      & 19 154 536,7                               & 2 137,00                             & 246,00                                      & 11,00                               & 304 877,80                                                & 106,60                                        & 120,30              & 112,50                                  \\
2~023,00                                                                                                                                      & 24 895 989,6                               & 2 154,00                             & 296,00                                      & 15,00                               & 375 046,20                                                & 101,20                                        & 120,30              & 125,20                                  \\
Note - compiled by the authors based on sources [\href{https://taldau.stat.gov.kz/kk/Search/SearchByKeyWord}{https://taldau.stat.gov.kz}; \href{https://stat.gov.kz/ru/region/almaty/}{https://stat.gov.kz}] &                                            &                                      &                                             &                                     &                                                           &                                               &                     &                                         
\end{tblr}
}
\end{table}

Analysis of the correlation between various economic and transport
indicators will allow us to determine the relationship between them and
assess the impact of one indicator on another. In this case, we consider
several variables: gross regional product (GRP), the number of passenger
buses, trolleybuses, the number of cars, income from passenger
transportation, transportation tariff indices, consumer price index and
investment in fixed capital.

{\bfseries Table 2 - Correlation model}
\begin{table}[H]
\centering
\resizebox{\linewidth}{!} \\
Number of passenger buses,             & 0,049                 &                               &                                      &                                           &                                              &                                           &                 \\
Number of passenger trolleybuses       & 0,674                 & -0,433                        &                                      &                                           &                                              &                                           &                 \\
Number of passenger carriages, meter   & 0,875                 & -0,051                        & 0,521                                &                                           &                                              &                                           &                 \\
Revenues from passenger transportation & 0,918                 & 0,214                         & 0,552                                & 0,774                                     &                                              &                                           &                 \\
Transportation tariff indices          & -0,190                & 0,279                         & -0,375                               & -0,066                                    & -0,161                                       &                                           &                 \\
CPI rates, \%                          & 0,711                 & -0,141                        & 0,538                                & 0,566                                     & 0,545                                        & 0,113                                     &                 \\
Investments in fixed capital           & 0,847                 & -0,109                        & 0,530                                & 0,788                                     & 0,730                                        & -0,188                                    & 0,499           \\
Note - Compiled by the authors         &                       &                               &                                      &                                           &                                              &                                           &                 
\end{tblr}
}
\end{table}

\begin{multicols}{2}
\emph{1. Correlation between GRP and other variables}.

GRP shows a strong positive correlation with passenger transportation
revenues (\emph{R}=0.918). This shows that the growth of economic
activity in Almaty is closely related to the increase in revenues in the
transport sector. GDP has a significant impact on the development of
transport infrastructure, especially passenger transportation.

GRP also shows a strong positive correlation with the number of metro
passenger cars (\emph{R}=0.875) and investment in fixed assets
(\emph{R}=0.847). This confirms that economic growth affects the
development of infrastructure and the modernization of the transport
sector, which contributes to an increase in the number of transport
units, especially the metro.

In particular, Yang, Lin (2024) identified the criteria and favorable
conditions for the construction of metro stations in cities using the
example of the northern regions of China {[}9{]}. In order to improve
the quality of metro construction, a system of engineering prefabricated
and waterproofing structures is proposed. Such innovative engineering
technologies of prefabricated metro construction contribute to the
growth of quality, productivity and safety for the construction of
public metro stations in cities.

\emph{2. Correlation between the number of passenger buses, trolleybuses
and metro cars.}

The number of passenger buses has a weak positive correlation with the
number of passenger trolleybuses (R = 0.674), which may indicate that an
increase in one mode of transport is associated with an increase in
another mode of transport. However, this correlation is not strong
enough, indicating differences in the demand and use of these modes of
transport.

The number of passenger buses and trolleybuses is negatively correlated
with the number of passenger metro cars (R = -0.051 and R = -0.433).
This may mean that the increase in the number of buses and trolleybuses
is not always related to the increase in the number of metro cars, and
may also be related to different passenger needs depending on the type
of transport network. Giagnorio, Börjesson, D' Alfonso
(2024) examine the degree of electrification of city buses using
Stockholm as an example. They found that optimization of fares and
pricing policy of the bus fleet has a positive effect on improving the
well-being of city residents {[}10{]}.

\emph{3. Fare Index and CPI.}

The fare index shows a weak negative correlation with passenger revenue
(\emph{R} = -0.161) and the number of passenger buses (\emph{R} =
-0.190). This suggests that the growth of transport costs may have some
negative impact on total passenger revenue and demand for buses.
However, the correlation with the number of buses and trolleybuses
remains small (\emph{R} = -0.141 and \emph{R} = -0.141), indicating that
there is no direct relationship between inflation and the dynamics of
the number of vehicles.

\emph{4. Fixed Capital Investments.}

Fixed capital investments have a strong positive correlation with
passenger revenues (\emph{R} = 0.788) and GRP (\emph{R} = 0.847),
confirming the importance of investment in transport infrastructure for
income growth in this sector and contributing to economic development.

Thus, the correlation analysis showed that factors such as economic
activity (GRP), transport infrastructure (number of passenger cars in
the metro) and investment in fixed capital have the greatest impact on
passenger transportation revenues in Almaty. These results highlight the
importance of an integrated approach to the development of the transport
sector, including the modernization of existing facilities and increased
investment. On the other hand, tariff policy, although it affects
transportation, does not have such a pronounced relationship with
passenger transportation revenues, which is confirmed by a negative
correlation with a number of variables. The data obtained in the
correlation analysis can be used to improve the transport system and
optimize tariff policy in Almaty, develop effective strategies to
increase profitability and improve the quality of passenger
transportation. For example, Hluško, Stanek, Ďurček, Kusendová (2024)
determined the location and activity of public transport in Bratislava
at different times of the day {[}11{]}. Thus, the authors modeled how
convenient and accessible it is for certain categories of city residents
based on the "UPTS" transport route. The peculiarity here is that the
"UPTS" route has a high level of occupancy in the morning and evening.
This is of particular concern to women, pensioners and children. Next,
we will conduct a regression analysis. The regression equation looks
like this:
\end{multicols}

\begin{equation*}
    Y = -44201719 + 274 \cdot X_1 + 19515 \cdot X_2 + 447799 \cdot X_3 
    + 16.79 \cdot X_4 - 36972 \cdot X_5 + 272964 \cdot X_6 + 168181 \cdot X_7
\end{equation*}

The results of the regression analysis are presented in Table 3.

\begin{table}[H]
\caption*{Table 3 - Model of coefficients}
\centering
\begin{tblr}{
  row{1} = {c},
  cell{2}{2} = {c},
  cell{2}{3} = {c},
  cell{2}{4} = {c},
  cell{2}{5} = {c},
  cell{2}{6} = {c},
  cell{3}{2} = {c},
  cell{3}{3} = {c},
  cell{3}{4} = {c},
  cell{3}{5} = {c},
  cell{3}{6} = {c},
  cell{4}{2} = {c},
  cell{4}{3} = {c},
  cell{4}{4} = {c},
  cell{4}{5} = {c},
  cell{4}{6} = {c},
  cell{5}{2} = {c},
  cell{5}{3} = {c},
  cell{5}{4} = {c},
  cell{5}{5} = {c},
  cell{5}{6} = {c},
  cell{6}{2} = {c},
  cell{6}{3} = {c},
  cell{6}{4} = {c},
  cell{6}{5} = {c},
  cell{6}{6} = {c},
  cell{7}{2} = {c},
  cell{7}{3} = {c},
  cell{7}{4} = {c},
  cell{7}{5} = {c},
  cell{7}{6} = {c},
  cell{8}{2} = {c},
  cell{8}{3} = {c},
  cell{8}{4} = {c},
  cell{8}{5} = {c},
  cell{8}{6} = {c},
  cell{9}{2} = {c},
  cell{9}{3} = {c},
  cell{9}{4} = {c},
  cell{9}{5} = {c},
  cell{9}{6} = {c},
  cell{10}{1} = {c=6}{},
  hlines,
  vlines,
}
\textbf{Term}                          & \textbf{Coef} & \textbf{SE Coef} & \textbf{T-Value} & \textbf{P-Value} & \textbf{VIF} \\
Constant                               & -44201719     & 9183958          & -4,81            & 0,003            &              \\
Number of passenger buses,             & 274           & 123              & 2,22             & 0,068            & 2,37         \\
Number of passenger trolleybuses       & 19515         & 9460             & 2,06             & 0,085            & 3,00         \\
Number of passenger carriages, meter   & 447799        & 164063           & 2,73             & 0,034            & 3,78         \\
Revenues from passenger transportation & 16,79         & 6,80             & 2,47             & 0,048            & 5,35         \\
Transportation tariff indices          & -36972        & 23086            & -1,60            & 0,160            & 1,51         \\
CPI rates, \%                          & 272964        & 76534            & 3,57             & 0,012            & 2,04         \\
Investments in fixed capital           & 168181        & 60758            & 2,77             & 0,033            & 3,18         \\
Note - Compiled by the authors         &               &                  &                  &                  &              
\end{tblr}
\end{table}

\begin{multicols}{2}
\emph{Evaluation of the significance of factors}.

1) The number of passenger buses (thousand units). The coefficient of
the number of buses is 274, which indicates that an increase in the
number of buses by 1 thousand units is associated with an increase in
passenger transportation revenue by 274 thousand tenge, all other things
being equal. Everything else is the same. However, the \emph{p-value}
(0.068) is greater than 0.05, so this factor is not statistically
significant at the 5\% level. The \emph{t-value} (2.22) indicates low
statistical significance, which can be significant at the 10\% level.

2) The coefficient of the number of trolleybuses is 19.515, which
indicates an increase in passenger transportation revenue by 19.515
thousand tenge and an increase in the number of trolleybuses by 1 unit.
However, the \emph{p-value} (0.085) is greater than the threshold of
0.05, which indicates that 5\% is statistically insignificant.
\emph{t-value} (2.06) shows moderate significance of this parameter.

3) The coefficient of the number of metro cars is 447,799, which means
that an increase in the number of vehicles by one unit is associated
with an increase in passenger transportation revenue by 447,799 thousand
tenge. \emph{P-value} (0.034) indicates the statistical significance of
this parameter at the 5\% level, therefore, it is an important factor
affecting transportation revenue.

4) Passenger transportation revenue. The coefficient of passenger
transportation revenue is 16.79, which means an increase in revenue by 1
thousand tenge is associated with an increase in transportation revenue
by 16.79 thousand tenge. \emph{P-value} (0.048) confirms the statistical
significance of this parameter at the 5\% level. This indicates that it
is important to consider current revenues in the process of forecasting
future revenues.

5) Transportation tariff indices. The coefficient for the indices of
transportation tariffs is -36.972, which indicates that as tariffs
increase, transportation revenues decrease. However, the \emph{blood
value} (0.160) significantly exceeds the threshold of 0.05, so this
parameter is statistically insignificant at the 5\% level.

6) CPI (\%). It is a coefficient of 272,964, which indicates that if the
CPI increases by 1\%, passenger transportation revenues will increase to
272,964 thousand tenge. The \emph{t-value} (3.57) and \emph{p-value}
(0.012) confirm the statistical significance of this parameter at the
5\% level.

7) Investment in fixed capital. The coefficient of investment in fixed
capital is 168,181, which means that transportation revenues increase by
168,181 thousand tenge with an increase in investment by 1 thousand
tenge. \emph{T-value} (2.77) and \emph{p-value} (0.033) indicate
statistical significance of this parameter is at the level of 5\%.

Multicollinearity problem. The \emph{VIF} indicator for most variables
is within the acceptable limit (\emph{VIF} \textless5), indicating a low
level of multicollinearity. The exception is the variable "passenger
revenue". This variable is specified on December 5.35, which may
indicate a correlation between these variables.

To assess the quality of the regression model, it is important to
consider several basic statistical indicators: standard error
(\emph{S}), coefficient of determination (\emph{R-sq}), adjusted
coefficient of determination (\emph{R-sq}(adj)) and predicted
coefficient of determination (\emph{R-sq}(pred)). These indicators not
only assess the accuracy of the model, but also check the ability to
generalize to new data (Table 4).
\end{multicols}

\begin{table}[H]
\caption*{Table 4 - Model}
\centering
\begin{tblr}{
  row{1} = {c},
  row{2} = {c},
  cell{3}{1} = {c=4}{},
  hlines,
  vlines,
}
\textbf{S}                     & \textbf{R-sq} & \textbf{R-sq(adj)} & \textbf{R-sq(pred)} \\
991709                         & 98,63\%       & 97,03\%            & 89,38\%             \\
Note - Compiled by the authors &               &                    &                     
\end{tblr}
\end{table}

\begin{multicols}{2}
Here are the obtained results of the model.

1) The standard error for this model is 991.709. This indicator shows
the average deviation from the calculated value of the actual
observation. The smaller the value of the standard error, the more
accurate the model. In this case, the value is very low, which indicates
good quality of the model in terms of forecasting.

2) Determination Ratio (\emph{R-sq}): The decision ratio is 98.63\%,
which is a very high level of explanation of the variability in the
dependent variable model. This means that the model accounts for almost
99\% of the total variability in the data. Such high scores indicate
good quality of the model. This indicates that the selected variable
really affects the dependent variable.

3) The adjusted determination ratio (\emph{R-sq}(adj)) is 97.03\%, which
is a very effective result. The adjusted decision factor considers the
number of independent variables in the model and adjusts the square root
of \emph{R-sq} to avoid over-fitting the model. This indicator confirms
that the model remains very stable even for possible additional
variables.

4) Predicted coefficient of determination (\emph{R-sq}(pred)) its value
is 89.38\%, which indicates that the model has a high generalization
ability with unprecedented new data. This indicator evaluates the degree
to which the model can predict the value of the tested sample and is an
important indicator confirming the stability of the model in real
conditions. A high value of this indicator confirms that the model can
not only describe the current data well, but also predict the behavior
of the dependent variables in future observations.

Based on the analysis of statistical indicators, it can be concluded
that the proposed regression model is very effective for data analysis.
High values \hspace{0pt}\hspace{0pt}of the measurement coefficients
(\emph{R-sq and R-sq}(adj)) and the predicted decision coefficients
\emph{R-sq}(pred) indicate the ability to describe the existing data
well and adequately predict the results in new samples. The standard
errors in the model also indicate high accuracy. Overall, this model can
be considered reliable for use in other studies and practical
applications.
\end{multicols}

\begin{table}[H]
\caption*{Table 5 - Analysis of variance}
\centering
\resizebox{\linewidth}{!}{%
\begin{tblr}{
  row{1} = {c},
  cell{2}{2} = {c},
  cell{2}{3} = {c},
  cell{2}{4} = {c},
  cell{2}{5} = {c},
  cell{2}{6} = {c},
  cell{3}{2} = {c},
  cell{3}{3} = {c},
  cell{3}{4} = {c},
  cell{3}{5} = {c},
  cell{3}{6} = {c},
  cell{4}{2} = {c},
  cell{4}{3} = {c},
  cell{4}{4} = {c},
  cell{4}{5} = {c},
  cell{4}{6} = {c},
  cell{5}{2} = {c},
  cell{5}{3} = {c},
  cell{5}{4} = {c},
  cell{5}{5} = {c},
  cell{5}{6} = {c},
  cell{6}{2} = {c},
  cell{6}{3} = {c},
  cell{6}{4} = {c},
  cell{6}{5} = {c},
  cell{6}{6} = {c},
  cell{7}{2} = {c},
  cell{7}{3} = {c},
  cell{7}{4} = {c},
  cell{7}{5} = {c},
  cell{7}{6} = {c},
  cell{8}{2} = {c},
  cell{8}{3} = {c},
  cell{8}{4} = {c},
  cell{8}{5} = {c},
  cell{8}{6} = {c},
  cell{9}{2} = {c},
  cell{9}{3} = {c},
  cell{9}{4} = {c},
  cell{9}{5} = {c},
  cell{9}{6} = {c},
  cell{10}{2} = {c},
  cell{10}{3} = {c},
  cell{10}{4} = {c},
  cell{10}{5} = {c},
  cell{10}{6} = {c},
  cell{11}{2} = {c},
  cell{11}{3} = {c},
  cell{11}{4} = {c},
  cell{11}{5} = {c},
  cell{11}{6} = {c},
  cell{12}{1} = {c=6}{},
  hlines,
  vlines,
}
\textbf{Source}                        & \textbf{DF} & \textbf{Adj SS} & \textbf{Adj MS} & \textbf{F-Value} & \textbf{P-Value} \\
Regression                             & 7           & 4,24726E+14     & 6,06752E+13     & 61,69            & 0,000            \\
Number of passenger buses,             & 1           & 4,86174E+12     & 4,86174E+12     & 4,94             & 0,068            \\
Number of passenger trolleybuses       & 1           & 4,18512E+12     & 4,18512E+12     & 4,26             & 0,085            \\
Number of passenger carriages, meter   & 1           & 7,32675E+12     & 7,32675E+12     & 7,45             & 0,034            \\
Revenues from passenger transportation & 1           & 6,00035E+12     & 6,00035E+12     & 6,10             & 0,048            \\
Transportation tariff indices          & 1           & 2,52234E+12     & 2,52234E+12     & 2,56             & 0,160            \\
CPI rates, \%                          & 1           & 1,25105E+13     & 1,25105E+13     & 12,72            & 0,012            \\
Investments in fixed capital           & 1           & 7,53559E+12     & 7,53559E+12     & 7,66             & 0,033            \\
Error                                  & 6           & 5,90092E+12     & 9,83487E+11     &                  &                  \\
Total                                  & 13          & 4,30627E+14     &                 &                  &                  \\
Note - Compiled by the authors         &             &                 &                 &                  &                  
\end{tblr}
}
\end{table}

\begin{multicols}{2}
To further assess the importance of the influence of various factors on
passenger transportation revenues in Almaty, we used the analysis of
variance. This method allows us to determine the importance of
independent individual variables such as the number of passenger buses,
trolleybuses, metro cars, passenger transportation revenues, the
transportation tariff index, the consumer price index, and the amount of
investment in fixed capital (Table 5).

1. Overall regression effect. The overall F-statistic value for the
model is 61.69 and the \emph{p-value} is 0.000, indicating that the
overall model is statistically significant. This means that the set of
independent variables significantly affects Almaty' s
passenger transportation revenue.

2. The number of passenger buses is one thousand. The F-value for the
number of buses is 4.94 and the \emph{p-value} is 0.068, which is
slightly higher than the standard significant level of 0.05. This means
that the coefficient of the number of buses is positive, but its effect
on transportation revenue is statistically insignificant at 5\%.
However, given the \emph{p-value} of 0.068, the effect of the number of
buses can be significant at the 10\% level.

3. The number of passenger trolleybuses. The \emph{f-value} for the
number of trolleybuses is 4.26 and the \emph{p-value} is 0.08.
Furthermore, it is not statistically significant at 5\%, but it can be
significant at 10\%. Therefore, the impact of the number of trolleybuses
on transportation revenues should also be the subject of further study.

4. Number of metro cars. The \emph{f-statistic} is -7.45 for the number
of metro cars, and the \emph{p-value} is 0.034, which is statistically
significant at the 5\% level. This confirms that the number of metro
cars has a significant impact on passenger transportation revenues in
Almaty.

5. Passenger transportation revenues, million tenge. The F coefficient
for passenger transportation revenues is 6.10, and the \emph{p-value} is
0.048, which indicates statistical significance of this parameter at the
5\% level. This confirms that passenger transportation revenues in the
previous period had a significant impact on future revenues.

6. Transportation tariff index. The F-statistic is 2.56, and the
\emph{p-value} is 0.160. This value is significantly higher than 0.05
and shows that the fare index is not statistically significant in this
model.

7. Consumer Price Index \%. The \emph{f-value} for CPI is 12.72 and the
\emph{p-value} is 0.012, indicating that the statistical significance of
this indicator was confirmed at the 5\% level. This shows that changes
in the consumer price index have a significant impact on passenger
transportation revenues in Almaty.

8. Fixed Capital Investment. The \emph{f-value} for fixed capital
investment is 7.66 and the p-value is 0.033, indicating that this
parameter is statistically significant at the 5\% level. This confirms
the importance of investment in transport infrastructure to increase
passenger transportation revenues.

{\bfseries Conclusions.} Khademi-Vidra, Nemecz, Bakos (2024) conducted a
case study on public transport operations in Hungary, Budapest. The
authors conducted a survey among passengers using public transport in
electronic form {[}12{]}. The study found that the following factors
influence the growth of the number of passengers: investments in the
quality of public transport, cleanliness, innovation, safety, and
schedule. These results can allow local authorities to improve the
development of urban public transport in the world.

The results of our analysis of variance confirm that the number of metro
passenger cars, passenger revenues, consumer price index, and investment
in fixed capital have a statistically significant effect on
Almaty' s passenger revenues. However, the effect of the
number of passenger buses and trolleybuses is less significant, which
may be due to their smaller role in the entire transport system of the
city during the study. This model is generally important, and its
results can be used to develop new political and economic strategies
aimed at improving transport infrastructure and increasing passenger
revenues.

The study allowed us to conclude that transport infrastructure and
macroeconomic factors have a heterogeneous effect on the gross regional
product in Almaty. Regression analysis conducted from 2010 to 2023
showed that factors such as the number of passenger buses, trolleybuses
and metro cars, consumer price index (CPI), transportation tariffs and
investment in fixed assets significantly affect the revenue generated
from passenger transportation in Almaty. Transport means such as metro
and economic indicators related to tariffs and inflation are the most
important for generating transportation revenue, which is confirmed by
the high value of regression coefficient and the importance of factors.
Multiple linear regression and analysis of variance methods analysis
showed that effective management of transport infrastructure and
economic factors can significantly improve the profitability of
passenger transportation. In particular, improving transport
infrastructure and increasing investment in transport development, i.e.
modernizing cars and expanding transport routes, can be important
strategies to increase revenue. In addition, existing Almaty city
development programs such as the «Almaty Development Program -- 2025»
are aimed at modernizing transport infrastructure and improving the
quality of urban transport. This coincides with the conclusion of the
study.

Improving the efficiency of transport systems is directly related to
increasing revenues from transportation. According to the
city' s strategic documents, transport services should be
further improved, and the investment attractiveness of the sector should
steadily grow. Thus, in the context of city development and improving
the transport system, it is necessary to continue investing in upgrading
infrastructure and vehicles to ensure stable growth in passenger
transportation revenues in Almaty, and also consider the economic
situation, such as the consumer price index and electricity prices.
transport fees.

The prospects for the development of the urban transport system in the
global economy include the introduction of zero-emission buses (Avenali
et al. 2024). For example, Avenali et al. (2024) analyzed the factors
affecting the use of zero-emission buses (ZEB) in cities. During the
quantitative assessment, negative factors were revealed that lead to the
inhibition of the development of "ZEB" {[}13{]}. In particular, these
include: technological, organizational, managerial and economic factors.
And the promotion of institutional, social and environmental indicators
- stimulated the introduction of buses (ZEB).

The results of the study emphasize the need for an integrated approach
to the development of transport infrastructure and macroeconomic
regulation in the city of Almaty. In order to improve the efficiency of
the urban transport system, it is recommended to modernize the existing
infrastructure, optimize the tariff policy and promote the growth of
passenger transportation.

Some experts assess traffic jams in cities that interfere with the
normal movement of public transport (Cantos-Sánchez et al. 2011) The
authors'{} recommendations are aimed at the strategic
development of road infrastructure and optimization of the schedule of
transport modes in cities {[}14{]}.

In addition, stabilizing inflation and increasing the profitability of
the transport sector can further stimulate economic growth. Recently,
significant emphasis has been placed on the development of "Green
Economy" measures in the world. In this regard, the processes of
introducing innovations that reduce emissions into the atmosphere in the
organization of public transport are increasing (Stępniak et al. 2023).
At the same time, maintaining environmental sustainability in the
operation of passenger transport is also supported {[}15{]}.

It should be noted that future research could focus on studying the
nonlinear effects of transport and macroeconomic factors, analyzing the
long-term impact of transport investments, and developing multifactor
models to more accurately forecast GRP.
\end{multicols}

\begin{center}
{\bfseries References}
\end{center}

\begin{references}
1. Sarah B. Gruetzmacher, Clara B. Vaz, Ângela P. Ferreira. Assessing the
sustainable performance of the transport sector in European countries
using alternative Benefit-of-the-Doubt models // Transportation
Research Interdisciplinary Perspectives. -- 2025. - Vol. 29. - Р.
101326. \href{https://doi.org/10.1016/j.trip.2025.101326}{DOI
10.1016/j.trip.2025.101326}

2. Haoyan Peng, Zhao-Dong Xu, Hongfang Lu, Dongmin Xi, Zhiheng Xia, Cen
Yang, Bohong Wang. A review of underground transport infrastructure
monitoring in CCS: Technology and Engineering Practice // Process
Safety and Environmental Protection. - 2024. - Vol. 190. - P. 726-745.
\href{https://doi.org/10.1016/j.psep.2024.08.057}{DOI\\
10.1016/j.psep.2024.08.057}

3. Piotr Mamcarz, Paweł Droździel, Aleksandra Gzik, Iwona Rybicka,
Paulina Droździel. Characteristics of urban transport users and their
level of satisfaction with transport services. A longitudinal study of
passengers in Lublin city in 2018 and 2020// Transportation Research
Procedia. -- 2023. - Vol. 74. - P. 371-378. DOI
10.1016/j.trpro.2023.11.157

4. Kostas Mouratidis, Jonas De Vos, Athena Yiannakou, Ioannis Politis.
Sustainable transport modes, travel satisfaction, and emotions:
Evidence from car-dependent compact cities // Travel Behaviour and
Society. -- 2023. - Vol. 33. - Р. 100613.
\href{https://doi.org/10.1016/j.tbs.2023.100613}{DOI
10.1016/j.tbs.2023.100613}

5. Luigi Castagna, António Lobo, Pierluigi Coppola, António Couto.
Benchmarking the efficiency of European metros from a production
perspective // Research in Transportation Business \& Management. -
2024. - Vol. 53. - Р101102.
\href{https://doi.org/10.1016/j.rtbm.2024.101102}{DOI
10.1016/j.rtbm.2024.101102}

6. Moawiah A. Alnsour. Assessment of risks affecting the operational
activities of the Amman bus rapid transit (BRT) system // Alexandria
Engineering Journal. - 2023. - Vol. 78. - P. 265-280.
\href{https://doi.org/10.1016/j.aej.2023.07.036}{DOI\\
10.1016/j.aej.2023.07.036}

7. Adriana Quezada Larriva, Daniel Orellana, María Laura Guerrero
Balarezo, Javier Andrés García, Galo Cárdenas Villenas, Pablo Osorio
Guerrero. Impact of Quito' s first metro line on the
accessibility to urban opportunities // Journal of Transport
Geography.- 2023. - Vol. 108. - Р. 103548.\\
\href{https://doi.org/10.1016/j.jtrangeo.2023.103548}{DOI.org/10.1016/j.jtrangeo.2023.103548}

8. \href{https://www.researchgate.net/profile/Viswanadha-Giduturi?_tp=eyJjb250ZXh0Ijp7ImZpcnN0UGFnZSI6InB1YmxpY2F0aW9uIiwicGFnZSI6InB1YmxpY2F0aW9uIn19}{Viswanadha
Kumar Giduturi}. Sustainable Urban Mobility: Challenges, Initiatives
and Planning//
\href{https://www.researchgate.net/journal/Current-Urban-Studies-2328-4919?_tp=eyJjb250ZXh0Ijp7ImZpcnN0UGFnZSI6InB1YmxpY2F0aW9uIiwicGFnZSI6InB1YmxpY2F0aW9uIn19}{Current
Urban Studies}.- 2015. - Vol. 3(03).- Р.261-265. DOI
10.4236/cus.2015.33022

9. Xiuren Yang, Fang Lin. Research on prefabricated metro station
structure and key assembly technologies // Tunnelling and Underground
Space Technology.- 2024. - Vol. 153. - Р.106029.
\href{https://doi.org/10.1016/j.tust.2024.106029}{DOI\\
10.1016/j.tust.2024.106029}

10. Mirko Giagnorio, Maria Börjesson, Tiziana~D' Alfonso.
Introducing electric buses in urban areas: Effects on welfare,
pricing, frequency, and public subsidies//
\href{https://www.sciencedirect.com/journal/transportation-research-part-a-policy-and-practice}{Transportation
Research Part A: Policy and Practice}. - 2024. -
Vol.\href{https://www.sciencedirect.com/journal/transportation-research-part-a-policy-and-practice/vol/185/suppl/C}{185}.-
Р.104103. DOI
\href{https://doi.org/10.1016/j.tra.2024.104103}{10.1016/j.tra.2024.104103}

11. Richard Hluško, Richard Stanek, Pavol Ďurček, Dagmar Kusendová. Urban
public transport system accessibility for different groups of
residents: Case of Bratislava city // Case Studies on Transport
Policy.-2024. - Vol. 16. -Р. 101200.
\href{https://doi.org/10.1016/j.cstp.2024.101200}{DOI
10.1016/j.cstp.2024.101200}

12. Anikó Khademi-Vidra, Gábor Nemecz, Izabella Mária Bakos. Satisfaction
measurement in the \\sustainable public transport of Budapest //
Transportation Research Interdisciplinary Perspectives. -2024. - Vol.
23. - Р. 100989. \href{https://doi.org/10.1016/j.trip.2023.100989}{DOI
10.1016/j.trip.2023.100989}

13. Alessandro Avenali, Giuseppe Catalano, Mirko Giagnorio, Giorgio
Matteucci. Factors influencing the adoption of zero-emission buses: A
review-based framework // Renewable and Sustainable Energy Reviews. -
2024. -Vol. 197. - Р. 114388.
\href{https://doi.org/10.1016/j.rser.2024.114388}{DOI
10.1016/j.rser.2024.114388}

14. Pedro~Cantos-Sánchez,~Rafael~Moner-Colonques,~José
J.~Sempere-Monerris,~Óscar~Álvarez-SanJaime. Viability of new road
infrastructure with heterogeneous users// Transportation Research Part
A: Policy and Practice/ 2011. - Vol. 45(5). - Р. 435-450. DOI
10.1016/j.tra.2011.02.003

15. Marcin Stępniak, Konstantinos Gkoumas, Fabio Marques dos Santos,
Monica Grosso, Ferenc Pekár. Recent trends and progress in public
transport innovation in the scope of European research
projects // Transportation Research Procedia. - 2023 -Vol.72. - P.
295-4302. DOI \href{https://doi.org/10.1016/j.trpro.2023.11.342}{10.1016/j.trpro.2023.11.342}
\end{references}

\begin{authorinfo}
\emph{{\bfseries Сведение об авторах}}

Смагулова Ш.А. - доктор экономических наук, профессор, Университет
международного бизнеса им. Кенжегали Сагадиева, Алматы, Казахстан,
e-mail:
\href{mailto:shsmagulova@mail.ru}{\nolinkurl{shsmagulova@mail.ru}};

Омаров A.- доктор технических наук, профессор,Международный
транспортно-гуманитарный университет, Алматы, Казахстан, e-mail:
\href{mailto:Info@mtgu.edu.kz}{\nolinkurl{Info@mtgu.edu.kz}};

Алибекова А.Т. - докторант Университета Международного Бизнеса им.
Кенжегали Сагадиева, Алматы, Казахстан, е-mail:
\href{mailto:ainura_alibekova97@mail.ru}{\nolinkurl{ainura\_alibekova97@mail.ru}};

Омарова Б.А.-кандидат экономических наук, доцент,Международный
транспортно-гуманитарный университет, Алматы, Казахстан, е-mail:
\href{mailto:bota1868@mail.ru}{\nolinkurl{bota1868@mail.ru}}

\emph{{\bfseries Information about the authors}}

Smagulova Sh.- Doctor of Economics, Professor, Kenzhegali Sagadiyev
University of International Business, Almaty, \\Kazakhstan, е-mail:
\href{mailto:shsmagulova@mail.ru}{\nolinkurl{shsmagulova@mail.ru}};

A.Omarov - doctor of technical sciences, Professor, International
university of transport and humanities, Almaty, Kazakhstan, е- mail:
\href{mailto:Info@mtgu.edu.kz}{\nolinkurl{Info@mtgu.edu.kz}};

Alibekova А. - PhD student Kenzhegali Sagadiyev University of
International Business, Almaty, Kazakhstan, е-mail:\\
\href{mailto:ainura_alibekova97@mail.ru}{\nolinkurl{ainura\_alibekova97@mail.ru}};

Omarova B.А. - candidate of Economics, of Economics, associate
professor, International university of transport and humanities,
Almaty,Kazakhstan; e-mail:
\href{mailto:bota1868@mail.ru}{\nolinkurl{bota1868@mail.ru}}
\end{authorinfo}
