%% DONE
\id{IRSTI 14.15.15}{https://doi.org/10.58805/kazutb.v.1.26-794}

\begin{articleheader}
\sectionwithauthors{Zh. Kiizbayeva, A. Turegeldinova, B. Amralinova, Sh. Sarkambayeva, G.S. Mukhanova, A.T. Kazykeshova}{DESIGN THINKING: AN OVERVIEW OF NEEDFINDING IN EDUCATION}

{\bfseries Zh. Kiizbayeva\alink{https://orcid.org/0009-0000-3890-4011}
A. Turegeldinova\textsuperscript{\envelope } \alink{https://orcid.org/0000-0003-1042-1530},
B. Amralinova\alink{https://orcid.org/0000-0003-0716-5265},
Sh. Sarkambayeva\alink{https://orcid.org/0000-0001-8509-3688},
G.S. Mukhanova\alink{https://orcid.org/0000-0003-4985-4018},
A.T. Kazykeshova\alink{https://orcid.org/0000-0002-3143-3680}
\emph{(Аnalytical review)}}
\end{articleheader}

\begin{affiliation}
\emph{\textsuperscript{1} Kazakh National Research Technical University named after K. I. Satpayev, Almaty, Kazakhstan,}

\emph{\textsuperscript{2} D. Serikbayev East Kazakhstan Technical University, Ust-Kamenogorsk, Kazakhstan}

\raggedright \textsuperscript{\envelope }{\em Correspondent-author: \href{mailto:a.turegeldinova@satbayev.university}{\nolinkurl{a.turegeldinova@satbayev.university}}}
\end{affiliation}

Nowadays, modern education is incredibly diverse. Each university or
school uses different successful approaches, such as peer learning,
flipped teaching, building Fab Labs, Makerspases and opening
University-Based Incubators. Despite the many approaches, mindsets,
techniques and tools available in education, there is not a universal
formula for what should be included in the educational infrastructure,
or which tools are the most effective. These questions are "wicked",
complex and unclear. The design thinking approach can help us transition
from knowledge space to the concept space to effectively utilizing it.
This paper provides an overview of several popular approaches and tools
that have proven effective in higher education. Furthermore, this paper
aims to inspire future research, utilizing the design thinking tool
known as "desktop research." Through review and reflection, we aim to
understand the effectiveness of the needfinding approach and its
potential to uphold the integrity of the university and fulfill its
initial mission.

{\bfseries Keywords:} design thinking, education, management, innovation,
needfinding in education

\begin{articleheader}
{\bfseries ДИЗАЙН-ОЙЛАУ: БІЛІМ САЛАСЫНДАҒЫ ҚАЖЕТТІЛІКТЕРГЕ ШОЛУ}

{\bfseries  
\textsuperscript{1}Ж.Е. Киізбаева,  
\textsuperscript{1}А.Ж. Турегельдинова\textsuperscript{\envelope },  
\textsuperscript{1}Б.Б. Амралинова,  
\textsuperscript{1}Ш.Г. Саркамбаева,  
\textsuperscript{1}Г.С. Муханова,  
\textsuperscript{2}А.Т. Казыкешова  
\emph{(Аналитикалық шолу)}}
\end{articleheader}

\begin{affiliation}
\emph{\textsuperscript{1}Қ.И. Сәтбаев атындағы Қазақ Ұлттық Техникалық Зерттеу Университеті, Алматы, Казақстан,}

\emph{\textsuperscript{2}Д.Серікбаев атындағы ШҚТУ,}

\emph{e-mail: a.turegeldinova@satbayev.university}
\end{affiliation}

Қазіргі білім беру жүйесі керемет әртүрлілігімен ерекшеленеді.
Университеттер бірқатар инновациялық тәсілдерді қолданады, мысалы,
бірлесіп оқу, «төңкерілген сынып» әдісі, Fab Labs және Makerspaces құру,
сондай-ақ университет негізінде инкубаторларды дамыту. Әдістердің,
идеялардың, техникалар мен құралдардың молшылығына қарамастан, білім
беру инфрақұрылымының оңтайлы компоненттері мен ең тиімді құралдарын
анықтайтын әмбебап формула әлі де жоқ. Бұл мәселелер «зұлым» проблемалар
ретінде сипатталады --- күрделі, екіұшты және көпқырлы.

Дизайн-ойлау әдіснамасы «білім кеңістігінен» «тұжырымдамалық кеңістікке»
өтуге мүмкіндік беріп, алынған инсайттарды тиімді қолдануды қамтамасыз
етеді. Бұл жұмыста жоғары білім беру саласында тиімділігін дәлелдеген
кеңінен танылған тәсілдер мен құралдардың жан-жақты талдауы ұсынылған.
Сонымен қатар, зерттеу дизайн-ойлау әдістерінің бірі --- «кабинеттік
зерттеу» арқылы болашақ ғылыми зерттеулерге шабыт беруді мақсат етеді.
Осы тәжірибелерді сыни тұрғыдан талдай отырып, жұмыс needfinding
әдісінің тиімділігін және оның университеттердің тұтастығын сақтап,
олардың бастапқы миссиясына сәйкес болу әлеуетін бағалауға бағытталған.

{\bfseries Түйін сөздер:} дизайн-ойлау, білім беру, басқару, инновация,
білім берудегі қажеттіліктерді табу

\begin{articleheader}
{\bfseries ДИЗАЙН-МЫШЛЕНИЕ: ОБЗОР ПОТРЕБНОСТЕЙ В ОБРАЗОВАНИИ}

{\bfseries  
\textsuperscript{1}Ж.Е. Киізбаева,  
\textsuperscript{1}А.Ж. Турегельдинова\textsuperscript{\envelope },  
\textsuperscript{1}Б.Б. Амралинова,  
\textsuperscript{1}Ш.Г. Саркамбаева,  
\textsuperscript{1}Г.С. Муханова,  
\textsuperscript{2}А.Т. Казыкешова  
(\emph{Аналитический обзор})}
\end{articleheader}

\begin{affiliation}
\emph{\textsuperscript{1}Казахский национальный исследовательский технический университет имени К.И. Сатпаева, Алматы, Казахстан,}

\emph{\textsuperscript{2}ВКТУ им.Д.Серикбаева, Усть-Каменогорск, Казахстан,}

\emph{e-mail: \href{mailto:a.turegeldinova@satbayev.university}{\nolinkurl{a.turegeldinova@satbayev.university}}}
\end{affiliation}

Современное образование отличается удивительным разнообразием.
Университеты внедряют широкий спектр инновационных подходов, включая
совместное обучение, перевернутый класс, создание Fab Labs и
Makerspaces, а также развитие университетских инкубаторов. Несмотря на
обилие методов, концепций, техник и инструментов, до сих пор не
существует универсальной формулы, определяющей оптимальные элементы
образовательной инфраструктуры или самые эффективные инструменты. Эти
вопросы представляют собой «злые» проблемы --- сложные, неоднозначные и
многогранные.

Методология дизайн-мышления предлагает путь перехода от «пространства
знаний» к «пространству концепций», обеспечивая эффективное применение
полученных инсайтов. В данной работе представлен всесторонний обзор
широко признанных подходов и инструментов, которые доказали свою
эффективность в высшем образовании. Кроме того, исследование нацелено на
вдохновение будущих научных изысканий с использованием метода
дизайн-мышления, известного как «кабинетное исследование». Критически
анализируя и оценивая данные практики, работа стремится изучить
эффективность подхода needfinding и его потенциал в сохранении
целостности университетов, а также в соответствии с их изначальной
миссией.

{\bfseries Ключевые слова:} дизайн-мышление, образование, менеджмент,
инновации, поиск потребностей в образовании

\begin{multicols}{2}
{\bfseries Introduction.} Design thinking extends the boundaries of
traditional design by offering a universal methodology for solving
``wicked problems'' {[}1{]}. It is an analytical and creative process
that involves experimentation, creating and prototyping models,
collecting feedback, and refining solutions. It is applicable to a
variety of disciplines, including education, business, and engineering
{[}2{]}. Innovation in education is no longer limited to the
introduction of new technologies or teaching methods. It is transforming
into a process of co-creation of value that integrates people, culture
and technology into a single ecosystem, where diversity and differences
become the driving forces of progress {[}3{]}.

\emph{Needfinding} is a fundamental process for developing user-centered
solutions. It involves learning what users cannot clearly express, but
which is critical to a successful outcome {[}4{]}. Empathy is used to
understand the user's context and create solutions that truly solve
their problems. It is important to focus on \emph{hidden or implicit
needs} that are difficult to identify using traditional methods.
Analysis of current solutions and real user behavior helps not only to
clarify the problem, but also to create innovative ideas based on
existing scenarios.

Design thinking consists of three phases: \emph{inspiration, ideation,}
and \emph{implementation}. In this paper, we will fully immerse in
inspiration. The initial stages of design thinking emphasize a deep
understanding of the problem through research and analysis. The key
tools are empathy, studying best practices and conducting desktop
research. This process helps to reveal hidden aspects of the problem,
study successful approaches of others and form a well-founded direction
for further prototyping and testing of solutions {[}5{]}.

In this paper, we will dive into the context of the problem by
understanding the needs and motivations of users. This will allow us to
create a basis for an effective solution based on real data, not
assumptions. Studying best practices and successful cases in similar
areas allows us not only to determine the current level of standards,
but also to be inspired by innovative solutions. This creates a basis
for identifying areas that can be improved. With empathy we will reveal
hidden or insufficiently studied aspects of the problem to provide a
foundation for generating ideas that will truly be based on user needs
and real data {[}6{]}.

{\bfseries Main part.} Learning is a process that requires
\emph{flexibility} and an \emph{eclectic} approach that integrates
elements of behaviorism, cognitivism, and constructivism.
\emph{Behaviorism} views learning as a change in observable behavior
caused by a response to specific stimuli and reinforced by external
influences. \emph{Cognitivism} focuses on mental processes such as
thinking, problem solving, and information processing, paying attention
to how learners perceive, organize, and store knowledge.
\emph{Constructivism} views learning as a process of creating meaning
through personal experience. Learning occurs in a context where
knowledge is not transmitted but constructed by learners based on their
experiences and interactions. Effective learning, however, requires an
integration of approaches that match the cognitive needs of the task and
the knowledge levels of learners. Behavioural strategies are effective
for basic skill acquisition, cognitive methods support problem solving,
and a constructivist approach is indispensable when working with
ill-structured tasks that require critical thinking and independent
interpretation of knowledge. Undoubtedly, modern educational systems
increasingly require a shift away from traditional learning models to
more dynamic, interactive, and human-centered approaches. Therefore,
educational designers should ask not ``Which theory is best?'' but
``\emph{Which theory is most effective in developing the acquisition of
specific tasks by specific students?} {[}7{]}''

At the same time, collaborative learning among students, based on the
exchange of experience and joint problem solving, is becoming a key
element of modern educational practice. Approaches such as
\emph{collaborative learning} and the use of educational spaces
stimulate the social construction of knowledge and contribute to the
formation of sustainable skills of interaction and creativity.
Constructivist and collaborative approaches are linked, creating a
synergy that stimulates the development of critical thinking, creativity
and cooperation skills in students {[}8{]}.

Collaborative learning is relevant in the context of the modern
educational process, as it meets the key challenges of the 21st century.
In an era when learning is becoming increasingly focused on the
development of social and professional competencies, this approach
ensures not only academic success, but also forms critical skills such
as cooperation, mutual support and adaptability. The study by Laal \&
Ghodsi emphasizes the importance of moving from an individualistic and
competitive approach to a cooperative one, which is especially relevant
for creating inclusive and human-centered educational systems. This
helps students realize the importance of mutual support and
responsibility, as well as develop critical thinking and problem-solving
skills through constant interaction and discussion {[}9{]}.

Active collaborative learning, supported by social interactions and
student engagement promotes knowledge sharing and collective discussion
of ideas as stimulates the development of critical thinking, social
responsibility and problem-solving skills, which makes the learning
process deeper and more effective {[}10{]}. Collaborative learning has a
significant positive impact on the development of
students'{} critical thinking skills. It creates an
\emph{environment} in which students actively exchange ideas, discuss
and analyze complex issues, which contributes to their cognitive
development {[}11{]}.

Of course, the environment plays a key role in the success of any
educational process. The environment can be identified in different
manifestations and be physical, digital, hybrid. Often, in higher
education, the environment is considered in a broader sense, which
includes both physical space and social interaction and cultural
aspects. Hira \& Hynes in their study offers a conceptual model based on
three aspects: people --- participants, including students, teachers and
the community, who form unique cultural and educational conditions in
each space; means --- the tools, technologies and materials used, which
are adapted to the goals and context of the educational space; and
activities - a variety of educational events, from prototyping and
experimentation to creative tasks, which are aimed at developing skills
and achieving learning goals, which help create the best educational
experience through \emph{makerspaces} {[}12{]}.

Based on the philosophy of constructionism, makerspaces are unique
learning environments where students use physical objects to co-create
knowledge. Indeed, the culture of makerspaces in higher education is not
limited to access to modern equipment, but represents more: the
integration of space, community, and educational programs {[}13{]}.
These spaces foster social constructivist learning environments where
ideas are generated and developed through interaction, negotiation and
collaboration. The teacher acts as a mentor, blurring the boundaries
between teacher and student. This methodology supports deep immersion in
the educational process, making learning active and personalized, which
is especially important for preparing students for modern challenges
{[}3{]}.\hspace{0pt}

Mersand, in his research on \emph{makerspaces} and \emph{Fablabs}, finds
that such spaces democratize access to tools, ideas, and learning
processes, allowing participants to become not only consumers but also
creators of knowledge {[}14{]}. However, the analysis shows that most
studies focus on practical and technological aspects, leaving issues of
inclusion, impact on educational outcomes, and assessment methodologies
without due attention.

But at the same time, Fablabs and Makerspaces play a key role in
rethinking educational approaches, especially in the context of global
challenges and changes. As unique educational spaces, they develop as
digital skills, as key entrepreneurial competencies of the 21st century.
However, full development of skills is only possible with clearly
structured educational programs aimed at entrepreneurship {[}15{]}.
Often such educational practices are focused on the creation of
artifacts (products) through practical learning, where the emphasis is
placed more on the process and result of creation than on the conscious
development of skills {[}16{]}. Also, such FabLabs are an excellent
example of a global ecosystem, for example, the FabLab Network is a
global network of laboratories, including more than 3,000 laboratories,
uniting students, engineers, designers and entrepreneurs to jointly
solve problems and share knowledge.

However, in a Science for Policy report by the Joint Research Centre,
three unique aspects of makerspaces are highlighted that make them
particularly attractive for educational purposes {[}17{]}. First,
makerspaces bring together traditionally separate disciplines such as
\emph{science, technology, engineering, arts, and mathematics} (STEAM).
This allows for interdisciplinary connections, critical thinking, and
creativity. Second, participants solve practical problems, which helps
them acquire knowledge and make sense of it through real-world
experience. This promotes both deliberate and incidental learning.
Third, makerspaces provide a variety of learning formats, from peer
learning, peer coaching, to individual mentoring and hands-on workshops.

According to the report, by 2034, Makerspaces could evolve into four key
areas: as educational spaces integrated into schools and universities
for hands-on learning; as a methodology focusing on a project-based
approach and solving real-world problems; as communities that bring
together people with different backgrounds to co-create and share
knowledge; and as a vital skill that develops students and professionals
with the creativity, innovative thinking, and entrepreneurial skills
needed to succeed in a dynamic world. Each scenario highlights how
Makerspaces can be more than just spaces, but also a strategy, tool, and
means to achieve educational goals.

Co-curricular activities such as \emph{entrepreneurship competitions},
\emph{mentoring programs}, and \emph{incubators} have a significant
impact on startup activity by providing students with hands-on
experience and opportunities to build social networks {[}18{]}. Modern
university campuses are successfully transformed into
\emph{entrepreneurial ecosystems}. Campuses can serve as modern
\emph{"frontiers"} --- spaces where entrepreneurs can experiment,
leverage resources, and build innovative companies {[}19{]}.

Universities create educational ecosystems that support interactions
between the academic community and small businesses, contributing to
regional economic development. This helps to simultaneously improve the
innovation potential of Small and Medium Enterprises (SMEs) and increase
the employability of graduates through University-Based Incubators
(UBIs) and student internships {[}20{]}.

Furthermore, \emph{peer learning} provides unique cognitive and social
benefits. Through collaborative discussion, assessment and feedback,
students not only strengthen their knowledge but also develop critical
thinking, reasoning and self-reflection skills {[}21{]}. Despite the
active diversity of teaching methods, there is a need to apply
innovative methods of learning support that focus not only on the
analysis of current results, but also on building strategies for future
development. Incorporating both \emph{peer feedback} and \emph{peer
feedforward} into the collaborative learning process significantly
improves the quality of argumentative essays, cognitive assimilation of
material and the development of critical thinking skills. The
peculiarity of peer feedforward is that it helps students focus on
prospects and strategies for achieving goals, and not only on the
current work, which makes the learning process more future-oriented and
productive. This approach is especially effective in online
environments, where students can interact anonymously, minimizing social
biases and increasing the depth of cognitive processing {[}22{]}.

Subsequently, providing students with opportunities to engage in peer
feedback improves their academic performance, and also promotes the
development of self-reflective skills. Peer feedback transforms the role
of the teacher from a "knowledge carrier" to a "facilitator", allowing
students to take responsibility for their learning and become active
participants in the educational process. By teaching students, the
skills to give and receive feedback, it is possible to create a more
dynamic and supportive learning environment, where mistakes are
perceived as opportunities for growth and interactions between students
become the main tool for learning {[}23{]}.

In addition, \emph{peer-to-peer learning} is becoming especially
relevant in the context of increasing student numbers and limited
resources in higher education. It supports the movement towards
interactive and human-centered teaching methods, replacing traditional
lectures with more active and involved approaches. Peer-to-peer learning
is effective for developing metacognitive skills such as
self-reflection, learning management, and autonomous learning. Students,
taking on the role of "teachers", not only learn the material at a
deeper level, but also develop critical thinking, communication skills,
and the organization of their own learning activities. The teacher takes
on the role of a facilitator, guiding students but allowing them to
control the learning process themselves. Peer-to-peer learning meets the
challenges of the 21st century, and this correspondence repeatedly
emerges in various sources {[}24{]}, {[}25{]}, {[}26{]}, {[}27{]}.

At the same time, structured interactions, an active role of course
organizers, and design of materials with clear invitations to
participate are critical to successfully engaging students in online
learning, where Peer-to-peer learning will be very valuable and will
help to properly build an online platform {[}8{]}. Peer-to-peer learning
is particularly effective when integrated as a complement, considering
the needs of students and specific educational contexts. This approach
helps reduce anxiety, improve student engagement in the educational
process, and create learning communities {[}29{]}.

\emph{Communities of practice} are a powerful tool for stimulating
learning through knowledge sharing and collaborative problem solving in
a professional environment. They are formed organically and are based on
social interaction, support and mutual learning between participants.
These communities allow combining formal and informal learning,
integrating practical experience and theoretical knowledge. An important
aspect is their ability to support the development of professional
identity and collective intelligence, which is especially important in
the context of a rapidly changing information space. Hara in his
research emphasizes the need to move away from the traditional approach
to learning based on top-down knowledge transfer in favor of creating
supportive and interactive learning environments {[}30{]}.

Communities of practice are especially valuable in organizations where
it is important to preserve and disseminate \emph{tacit knowledge} such
as professional histories, contextual decisions, and collective
experience. This knowledge is difficult to formalize because it is
transmitted through experience, context, interaction, and observation.
It includes intuition, practical skills, social interaction,
professional "tricks," and deeply rooted understanding of work
processes. Tacit knowledge forms the basis for sustainable competence
growth. It is retained within the community even when members change. It
also helps to develop a deep understanding of the profession that goes
beyond standard training materials. Tacit knowledge becomes a key asset
not only for personal growth but also for creating collective
intelligence, making it central to the development of successful
communities of practice. Tacit knowledge is a central element of
professional competence, but due to its nature, it is difficult to
transmit through traditional educational methods. Tacit knowledge plays
a central role in our ability to understand and act in the world,
despite its inaccessibility to full verbalization. Given that we live in
an era of active digitalization and automation, the very idea that not
all knowledge can be encoded or formalized is key to understanding the
limitations of modern technologies such as artificial intelligence
{[}31{]}. Attempts to formalize tacit knowledge can distort its essence,
since it is linked to context and intuitive perception. It is
transmitted through social interaction, observation, mentoring and
practical activities, which makes it especially important in education
{[}32{]}.

Gafney \& Varma-Nelson in their study \emph{Peer-Led Team Learning}
(PLTL) describe an innovative pedagogical model that integrates
student-centered active learning into the educational process through
specially organized workshops led by students {[}33{]}. An important
feature of PLTL is the role of workshop leaders, who act as equal
partners rather than authority figures, facilitating the creation of an
informal environment for in-depth study of the material. The PLTL
program has proven its effectiveness in more than 100 educational
institutions, including universities, colleges and research centers,
reaching over 20,000 students annually. It allows students to work in
small groups, where they can discuss complex topics, solve problems and
deepen their understanding of key concepts through cooperative efforts.

PLTL emerged as a response to the need to improve student engagement in
STEM (Science, Technology, Engineering, and Mathematics) subjects. Its
implementation has shown that peer-based models can improve academic
outcomes and create a culture of collaborative learning. An important
element is the involvement of peer leaders who guide groups of students,
helping them to solve problems together and deepen their understanding
of the material. The supportive environment of the workshops helps
students to participate in learning without fear of failure, which
increases their confidence and motivation, which contributes to an
inclusive environment {[}34{]}.

Certainly, peer learning is effective through two key formats:
cooperative learning and \emph{peer tutoring}. Successful peer learning
requires a clear structure for interactions that teachers create.
Principles such as positive interdependence, individual responsibility,
and group engagement ensure constructive social interactions and promote
deeper learning. Peer learning techniques such as Jigsaw, Peer Tutoring,
Constructive Controversy, Reciprocal Teaching, Think-Pair-Share,
Collaborative Learning Groups, and Peer Assessment, as well as
approaches such as Structured Academic Controversy and Numbered Heads
Together, promote active student engagement by encouraging
collaboration, critical thinking, and individual responsibility. Methods
such as Learning Together, Team-Assisted Individualization, and Group
Investigation further deepen understanding through collaborative problem
solving and group reflection. Together, these strategies create a
dynamic and inclusive learning environment, enabling students to deepen
their knowledge, develop key interpersonal skills, and confidently
contribute to collective learning outcomes {[}27{]}.

Moreover, in the context of modern education, focused on the development
of social and professional competencies, the integration of
\emph{service learning} is becoming extremely relevant. This approach
not only combines theoretical training with practical experience, but
also helps to strengthen the connection of universities with
communities, develop students'{} social responsibility
and develop skills for solving real problems. Service learning is
especially in demand in the context of human-centered educational
systems, where the emphasis is on the individual needs of students and
their role as active participants in social change. Service learning
offers an effective model for integrating theory and practice, promoting
the development of students'{} skills, their social
responsibility and partnerships between universities and communities
{[}35{]}.

For example, in research universities, service-learning is not only a
pedagogical approach but also a strategy that integrates teaching,
research and service to society through integration into the mission of
the university and the stimulation of research activities based on
interaction with society {[}36{]}. However, there are significant
pedagogical, political and institutional limitations to service learning
in higher education {[}37{]}.

Criticism of service-learning highlights the need to overcome the
imbalance between universities and communities, ensure sustainability of
projects and improve student training. It is important to consider the
needs and perspectives of communities so that service-learning
partnerships become truly mutually beneficial {[}38{]}.

Also, universities, in an effort to support the development of their
strategic initiatives and global challenges, resort to methodologies
such as \emph{project-based learning} (PBL) and \emph{challenge-based
learning} (CBL), which help develop transversal skills. CBL is a
promising educational approach in higher education aimed at connecting
theoretical knowledge with practical skills through solving real
sociotechnical problems, involving students in interdisciplinary
projects with the participation of academic and external actors
{[}39{]}. CBL actively supports students' active participation in the
learning process. Students take responsibility for choosing and solving
a problem, which promotes their self-organization and independence. The
role of the teacher changes: he or she becomes a facilitator who guides
the process rather than transmits knowledge. CBL helps to unite
different disciplines in solving complex, interdisciplinary challenges,
such as sustainability, health, or technology. Students work in groups
with diverse backgrounds, which improves communication and co-creation
skills. However, despite the popularity of CBL, the methodology is often
applied without a clear theoretical basis {[}40{]}. However, its ability
to connect academic learning with practical problems makes it a
promising tool for preparing students for today' s global
challenges. CBL represents an evolution from problem-based learning,
where the focus shifts to complex, interdisciplinary problems that
require the participation of students, teachers and external
stakeholders. A key element of CBL is not only learning through solving
social and technological problems, but also the need for a systems
approach {[}41{]}. Therefore, this methodology is often combined with
other, more systematic approaches. For example, Charosky et. all in
their study demonstrates the effectiveness of using Challenge-Based
Education in combination with the design thinking methodology, which
actively stimulates innovative thinking in students {[}42{]}.

The world is changing rapidly and a lesson in a classic lecture format,
where the teacher delivers a monologue, is almost of no value anymore.
The pandemic has shown that methods such as \emph{flipped teaching},
especially in online and blended formats, are ideally replacing
traditional lectures with active and student-centered classes.
Innovations such as e-flip and hyflex demonstrate its adaptability and
potential for expanded use in the future. Flipped teaching changes the
role of the teacher from a "sage on stage" to a "mentor and
facilitator", increasing student responsibility for their learning and
improving results {[}43{]}. The positive impact of the flipped approach
is not only on the availability of materials, self-organization, but
also on the independent pace of study. Which is especially important for
a more customized and individual approach for each student {[}44{]}. The
method is especially useful for disciplines that require a deep
understanding of theory and its practical application, such as STEM. But
despite all the advantages of flipped classrooms, it is worth
considering that students are often not motivated enough to complete
independent study {[}45{]}.

The literature is full of successful examples, approaches, practices and
methodologies for creating a better educational experience, but
nevertheless we cannot take all of them and implement them in one
educational institution. There is no coherence between them, there are
no rules of the game, and we cannot implement inconsistent elements in a
single university ecosystem. Although there is a tendency to oppose the
traditional university model --- \emph{unbundling}. This concept means
dividing traditional university functions, such as teaching, research,
assessment, certification and student support, into separate services
that can be provided by different organizations or platforms.
Competition forces universities to follow this path, but the process
must be carefully adapted to avoid undermining the fundamental goals of
education. Unbundling can lead to \emph{"hyperporosity"} of university
boundaries, where the connection between them and society becomes so
strong that the space for long-term academic research that is not
focused on immediate results disappears. {[}46{]}.

{\bfseries Conclusion.} Disruptive educational institutions have moved away
from traditional teaching and enable students to become active
participants in their learning, develop key 21st century skills, and
prepare for the challenges of a rapidly changing world. Design thinking
is becoming the foundation for developing 21st century skills, including
critical thinking, creative problem solving, and collaborative
interaction. It integrates technology and real-world problems into
educational processes, giving students the opportunity to develop
metacognitive skills and adapt to the complex challenges of the future
{[}47{]}.

Knowledge creation spaces can be divided into social, cognitive, and
structural factors {[}48{]}. Design thinking is particularly successful
in addressing social and cognitive aspects, creating a trusting
environment for idea sharing, analysis, and synthesis, leading to
collective knowledge creation. This process involves social interaction,
external knowledge adaptation, digital communication, and application in
practice.

In this paper, we reviewed existing concepts and examples, available
knowledge to expand our understanding and experience. Razzouk \& Shute
confirm in their research that design thinking is not just a tool, but a
holistic way of thinking that transforms uncertain tasks into structured
possibilities by creating a relationship between the \emph{knowledge
space} and the \emph{concept space} {[}2{]}. The knowledge space is a
collection of all available knowledge that already exists at the start
of the design. The knowledge in this space includes both scientific
facts and practical information accumulated through experience. An
important feature is that it is limited only to what is already known,
which emphasizes the need to involve experts and study existing data.

Design thinking is based on an iterative process of moving from
\emph{creative concepts} to \emph{validated knowledge}. This process
involves ideation, refinement, and empirical testing, which transforms
initial concepts into feasible solutions. Design thinking combines
\emph{empirical} and \emph{interpretive} approaches to solve complex
problems, integrating creativity and practical knowledge {[}49{]}.

Design thinking is based on the transition from the Concept Space to the
knowledge space. This process involves the transformation of concepts
into tested and implementable knowledge. Design begins with the
formation of concepts, which are gradually refined and tested until they
become part of the existing knowledge space. The iterative process
involves expanding the concept space through experimentation,
prototyping, and feedback, which contributes to the continuous growth of
the knowledge space. In the future research, we plan to further expand
our knowledge space by engaging experts in the field of education and
collecting information from key stakeholders. We will move on to the
concept space, where we will consider not yet tested and true ideas and
concepts based on our knowledge space expanded by the conducted
research.

This research was funded by the Committee of Science of the Ministry of
Science and Higher Education of the Republic of Kazakhstan under the
Grant No. BR27198643 -- ``Development of digital competences of human
capital in industry and logistics through cluster collaboration of
science, education and industry''.
\end{multicols}

\begin{center}
{\bfseries References}
\end{center}

\begin{references}
1.Melles G., Howard Z., \& Thompson-Whiteside S. (2012). Teaching design
thinking: Expanding horizons in design education// Procedia-Social and
Behavioral Sciencesю-2012.-Vol.31.-P.162-166. DOI\\
10.1016/j.sbspro.2011.12.035

2.Razzouk R., \& Shute, V. (2012). What is design thinking and why is it
important?// Review of educational research,- 2012.-Vol. 82(3).-
P.330-348. DOI /10.3102/0034654312457429

3.Taricani E. Design Thinking and Innovation in Learning// Emerald
Publishing Limited.-2021.-122 p. ISBN 978-1-80071-109-9

4.Ericson Å., Bergström M., Larsson A. C., \& Törlind P. Design thinking
challenges in education// In International Conference on Engineering
Design:/- 2009.-Vol.10.-P.89-100

5.Lee J. H., Ostwald, M. J., \& Gu, N. (2020). Design thinking:
creativity, collaboration and culture.- Switzerland Springer.- 2020.-
245 p. ISBN 9783030565572 DOI 10.1007/978-3-030-56558-9

6.Melles, G. (Ed.). (2020). Design thinking in higher education:
Interdisciplinary encounters. Springer Nature Singapore/- 2021.- ISBN
9811557829, 9789811557828

7. Ertmer P. A., Newby T. J. Behaviorism, cognitivism, constructivism:
Comparing critical features from an instructional design
perspective.//Performance improvement quarterly.-1993.-Vol. 6(4), 50-72.
~DOI 10.1111/j.1937-8327.1993.tb00605.x

8.Supena I., Darmuki A.,Hariyadi A. The Influence of 4C (Constructive,
Critical, Creativity, Collaborative) Learning Model on
Students'{} Learning Outcomes//International Journal of
Instruction.- 2021.-Vol.14(3).- P. 873-892. DOI 10.29333/iji.2021.14351a

9. Laal M., Ghodsi S.M. Benefits of collaborative learning
//Procedia-social and behavioral sciences.-2012.- Vol.31.-P. 486-490.
DOI 10.1016/j.sbspro.2011.12.091

10.Qureshi M.A., Khaskheli A., Qureshi J.A., Raza S.A.,Yousufi S.Q.
Factors affecting students' learning performance through collaborative
learning and engagement//Interactive Learning Environments.-
2021.-Vol.31(4).- P.2371-2391. DOI 10.1080/10494820.2021.1884886

11.Warsah I., Morganna R., Uyun M., Afandi M.,Hamengkubuwono H. (2021).
The impact of collaborative learning on learners' critical thinking
skills//International Journal of Instruction.-2021.- Vol.14(2).-
P.443-460. DOI
\href{http://dx.doi.org/10.29333/iji.2021.14225a}{10.29333/iji.2021.14225a}

12.Hira A.,Hynes M. M. People, means, and activities: A conceptual
framework for realizing the \\educational potential of makerspaces
//Education Research International.- 2018. Vol.(1).-P.1-10\\
\href{https://doi.org/10.1155/2018/6923617}{DOI 10.1155/2018/6923617}

13.Wilczynski V., Adrezin R. Higher education makerspaces and
engineering education //In ASME \\International Mechanical Engineering
Congress and Exposition.-2016 DOI
\href{http://dx.doi.org/10.1115/IMECE2016-68048}{10.1115/IMECE2016-68048}

14 .Mersand S. The state of makerspace research: A review of the
literature//TechTrends.-2021.-Vol. 65(2).- P.174-186.
DOI:\href{http://dx.doi.org/10.1007/s11528-020-00566-5}{10.1007/s11528-020-00566-5}

15. Rayna T., Striukova L. Fostering skills for the 21st century: The
role of Fab labs and makerspaces //Technological Forecasting and Social
Change.-2021.-Vol.164:120391.

DOI 10.1016/j.techfore.2020.120391

16.Ioannou A., Miliou O., Adamou M., Kitsis A., Timotheou S., Mavri A.
Understanding practicing

and assessment of 21st-century skills for learners in makerspaces and
FabLabs// Education and Information Technologies.-2024.- DOI
\href{https://doi.org/10.1007/s10639-024-13178-w}{10.1007/s10639-024-13178-w}

17. Vuorikari R., Ferrari A., Punie Y. Makerspaces for Education and
Training. Luxembourg: Publications Office of the European Union, 2019.
ISBN 978-92-76-09032-8 I DOI 10.2760/946996

18.Morris M.H., Shirokova G.,Tsukanova T. Student entrepreneurship and
the university ecosystem: A multi-country empirical
exploration//European Journal of International
Management.-2017.-Vol.11(1). - P.65-85. DOI 10.1504/EJIM.2017.081251

19.Miller D. J., Acs Z. J. The campus as entrepreneurial ecosystem: the
University of Chicago.// Small Business Economics. 2017-Vol.49(1).-
P.75-95.
DOI:\href{https://link.springer.com/article/10.1007/s11187-017-9868-4}{10.1007/s11187-017-9868-4}

20. Piterou A., Birch C. The role of Higher Education Institutions in
supporting innovation in SMEs: university-based incubators and student
internships as knowledge transfer tools.//Journal of Innovation
Impact.-2016.-Vol. 7(1).-P.72-79

21.Noroozi O., De Wever B.T he power of peer learning: Fostering
students' learning processes and outcomes/// Springer Nature. -2023.-
392 p.ISBN9783031294112, 9783031294105

22. Latifi S., Noroozi O.,Talaee E. Peer feedback or peer feedforward?
Enhancing students' argumentative peer learning processes and
outcomes//British Journal of Educational Technology.-2021.-Vol.
52(2).-P.768-784. \href{https://doi.org/10.1111/bjet.13054}{DOI
10.1111/bjet.13054}

23.Sackstein S. Peer feedback in the classroom: Empowering students to
be the experts. //Ascd. 2017.- 134 p. ISBN 978-1-4166-2366-3

24.Assinder W. Peer teaching, peer learning: one model.// ELT
Journal.-Vol.45(3).-P.218-229. DOI \\10.1093/elt/45.3.218

25.Guldberg K. Adult learners and professional development: peer‐to‐peer
learning in a networked \\community// International Journal of Lifelong
Education.-2008.-Vol. 27(1).-P.35-49.
\href{https://doi.org/10.1080/02601370701803591}{DOI\\
10.1080/02601370701803591}

26.Alexander B.J., Lindow L.E., Schock M.D. Measuring the impact of
cooperative learning exercises on student perceptions of peer-to-peer
learning: A case study//The Journal of Physician Assistant
Education.-2008.-Vol.19(3). - P.18-25. DOI
\href{http://dx.doi.org/10.1097/01367895-200819030-00005}{10.1097/01367895-200819030-00005}

27. Topping K., Buchs C., Duran D., Van Keer H. Effective peer learning:
From principles to practical implementation// Routledge.2017.- 192 p.
ISBN 978-1-13-890649-5
DOI \href{http://dx.doi.org/10.4324/9781315695471}{10.4324/9781315695471}

28.Ahn J., Weng C., Butler, B. S. The dynamics of open, peer-to-peer
learning: what factors influence participation in the P2P University?.
In 2013 46th Hawaii International Conference on System Sciences (pp.
3098-3107). DOI 10.1109/HICSS.2013.515

29.Stigmar M. Peer-to-peer teaching in higher education: A critical
literature review// Mentoring \& \\Tutoring: partnership in
learning/-2016.-Vol. 24(2).- P.124-136.
DOI 10.1080/13611267.2016.1178963

30.HaraN. Communities of practice: Fostering peer-to-peer learning and
informal knowledge sharing in the work place//Springer Science \&
Business Media.-2008-138 p.
DOI \href{http://dx.doi.org/10.1007/978-3-540-85424-1}{10.1007/978-3-540-85424-1}
ISBN 978-3-540-85423

31. Howells J. Tacit knowledge// Technology analysis \& strategic
management.-1996.-P.91-106.
\href{https://doi.org/10.1080/09537329608524237}{DOI
10.1080/09537329608524237}

32.Lejeune M. Tacit knowledge: Revisiting the epistemology of knowledge
//McGill Journal of Education.-2011.-Vol. 46(1).- P.91-105.
DOI:\href{http://dx.doi.org/10.7202/1005671ar}{10.7202/1005671ar}

33.Gafney L.,Varma-NelsonP. Peer-led team learning: Evaluation,
dissemination and institutionalization of a college level
initiative//Springer Science \& Business Media.-2008.-155 p.
DOI \href{http://dx.doi.org/10.1007/978-1-4020-6186-8}{10.1007/978-1-4020-6186-8}

34.Gosser D. K., Gosser D. K. Peer-led team learning: A
guidebook//Pearson College Div.-~2000- 133 p. ISBN-13 ‏ ~978-0130288059

35.Bringle R.G., Hatcher J.A. Implementing service learning in higher
education//The Journal of Higher Education.-1996.-Vol. 67(2).- P.
221-239.

36.Furco A. Advancing service‐learning at research universities// New
directions for higher education.- 2001.- Vol.114.- P.67-78. DOI
\href{http://dx.doi.org/10.1002/he.15}{10.1002/he.15}

37. Butin D. W. The limits of service-learning in higher education// The
review of higher education.-2006.-Vol. 29(4).- P.473-498. DOI
\href{http://dx.doi.org/10.1353/rhe.2006.0025}{10.1353/rhe.2006.0025}

38.Blouin D.D., Perry E.M. Whom does service learning really serve?
Community-based organizations'{} perspectives on service
learning// Teaching Sociology.-2009.-Vol. 37(2).-P.120-135.
DOI \\10.1177/0092055X0903700201

39. Gallagher S.E., Savage T. Challenge-based learning in higher
education: an exploratory literature review//Teaching in Higher
Education.- 2023.-Vol. 28(6). - P.1135-1157.
DOI \\10.1080/13562517.2020.1863354

40.Leijon M.,Gudmundsson, P., Staaf P,Christersson C. Challenge based
learning in higher education - A systematic literature review
//Innovations in education and teaching international.
-2022.-Vol.59(9).- P.609-618. DOI
\href{http://dx.doi.org/10.1080/14703297.2021.1892503}{10.1080/14703297.2021.1892503}

41.Malmqvist J., Rådberg K.K., Lundqvist U.Comparative analysis of
challenge-based learning experiences. In Proceedings of the 11th
International CDIO Conference, Chengdu University of Information
Technology, Chengdu, Sichuan, PR China. 2015.- P. 87-94

42. Charosky G., Leveratto L., Hassi L., Papageorgiou K.,Ramos-Castro
J., Bragós, R. Challenge based education: an approach to innovation
through multidisciplinary teams of students using Design Thinking. In
2018 XIII Technologies Applied to Electronics Teaching Conference (TAEE)
DOI \\\href{https://doi.org/10.1109/TAEE.2018.8476051}{10.1109/TAEE.2018.8476051}

43.Gopalan C., Daughrity, S., Hackmann E. The past, the present, and the
future of flipped teaching. //Advances in physiology education.-
2022.-Vol. 46(2).- P.331-334.
\href{https://doi.org/10.1152/advan.00016.2022}{DOI
10.1152/advan.00016.2022}

44. Newman G., Kim J. H., Lee R. J., Brown B.A., Huston S.The perceived
effects of flipped teaching on knowledge acquisition// Journal of
Effective Teaching. -2016.-Vol.16(1). - P.52-71.

45.Yeung K. Making `the flip'work: Barriers to and implementation
strategies for introducing flipped teaching methods into traditional
higher education courses//New Directions in the Teaching of Natural
Sciences. -2014.-Vol.10. -P. 59-63. DOI
\href{https://doi.org/10.29311/ndtps.v0i10.518}{10.29311/ndtps.v0i10.518}

46.McCowan T. Higher education, unbundling, and the end of the
university as we know it //Oxford Review of Education.-2017.-Vol.
43(6).- P.733-748.
\href{https://doi.org/10.1080/03054985.2017.1343712}{DOI
10.1080/03054985.2017.1343712}

47.Koh J H.L., Chai C.S.,Wong B., Hong H.Y., KohJ.H.L., Chai C.S., Hong
H. Y. Design thinking and education // Springer Singapore.-2015.ISBN
978-981-287-443-6

48.Davis J.,Docherty C.A., Dowling K. Design thinking and innovation:
Synthesising concepts of knowledge co-creation in spaces of professional
development//The Design Journal.-2016.-Vol. 19(1).- P.117-139. DOI
\href{http://dx.doi.org/10.1080/14606925.2016.1109205}{10.1080/14606925.2016.1109205}

49.Rylander A. Design thinking as knowledge work: Epistemological
foundations and practical implications // Design Management
Journal.-2009.-Vol.~4(1).-P.7-19.
\href{https://doi.org/10.1111/j.1942-5074.2009.00003.x}{DOI
10.1111/j.1942-5074.2009.00003.x}
\end{references}

\begin{authorinfo}
\emph{{\bfseries Information about the authors}}

Kiizbayeva Zh. - PhD Student, senior lecturer, Satbayev University,
Almaty, Kazakhstan, e-mail: \\z.kiizbayeva@satbayev.university;

Turegeldinova А. - Candidate of Economic Sciences, PhD, associate
professor, Satbayev University, Almaty, Kazakhstan, e-mail:
a.turegeldinova@satbayev.university;

Amralinova B. - PhD, associate professor, Satbayev University,
Almaty, Kazakhstan, e-mail: b.amralinova@satbayev.university;

Sarkambayeva Sh. - PhD, associate professor, Satbayev
University, Almaty, Kazakhstan, e-mail:\\
\href{mailto:sh.sarkambayeva@satbayev.university}{\nolinkurl{sh.sarkambayeva@satbayev.university}};

Mukhanova G. - candidate of technical science, associate
professor, Satbayev University, Almaty, Kazakhstan, e-mail:\\
g.mukhanova@satbayev.university;

Kazykeshova A. - PhD, D.Serikbayev East Kazakhstan technical university,
e-mail: \href{mailto:zh.serikbayeva@almau.edu.kz}{AKazykeshova@ektu.kz}

\emph{{\bfseries Сведения об авторах}}

Киізбаева Ж.Е. - PhD докторант, старший преподаватель, Сатпаев
Университет, Алматы, Казахстан, e-mail:\\
z.kiizbayeva@satbayev.university;

Турегельдинова А.Ж. - к.э.н., PhD, ассоц. профессор, Сатпаев
Университет, Алматы, Казахстан, e-mail:\\
a.turegeldinova@satbayev.university;

Амралинова Б.Б. - PhD, ассоц. профессор, Сатпаев Университет,
Алматы, Казахстан, e-mail: \\b.amralinova@satbayev.university;

Саркамбаева Ш.Г. - PhD, Сәтбаев университеті, Алматы, Қазақстан, e-mail:
sh.sarkambayeva@satbayev.university;

Муханова Г.С. - к.т.н., доцент, Сәтбаев университеті, Алматы, Қазақстан,
e-mail:g.mukhanova@satbayev.university;

Казыкешова А.Т. - PhD доктор, ВКТУ им. Д. Серикбаева, e-mail:
\href{mailto:zh.serikbayeva@almau.edu.kz}{AKazykeshova@ektu.kz}
\end{authorinfo}
