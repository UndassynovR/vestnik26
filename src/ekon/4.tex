\id{ҒТАМР 06.77.71}{}

\begin{articleheader}
{\bfseries ҚАЗАҚСТАННЫҢ АУЫЛ ШАРУАШЫЛЫҒЫНДАҒЫ ИНВЕСТИЦИЯЛАР ЖӘНЕ ЕҢБЕК
ӨНІМДІЛІГІ: ЖАҒДАЙЫ, МӘСЕЛЕЛЕРІ МЕН ШЕШІМДЕРІ}

{\bfseries \textsuperscript{1}А.К. Байдаков\authorid,
\textsuperscript{2}Н.К. Кучукова\authorid,
\textsuperscript{1}Р.С. Беспаева\authorid,
\textsuperscript{1}Ж.С. Булхаирова\authorid,
\textsuperscript{3}Б.А. Жуматаева\authorid}
\end{articleheader}

\begin{affiliation}
\emph{\textsuperscript{1}С.Сейфуллин атындағы Қазақ агротехникалық
зерттеу университеті, Астана, Қазақстан,}

\emph{\textsuperscript{2}Л.Н. Гумилев атындағы Еуразия ұлттық
университеті, Астана, Қазақстан,}

\emph{\textsuperscript{3}Қ. Құлажанов атындағы Қазақ технология және
бизнес университеті, Астана, Қазақстан}

\raggedright {\bfseries \textsuperscript{\envelope }}Корреспондент-автор: \href{mailto:a_baidakov@mail.ru}{\nolinkurl{a\_baidakov@mail.ru}}
\end{affiliation}

Жұмыс аграрлық сектордың қазіргі заманғы сын-қатерлері жағдайында ерекше
өзекті болып отырған Қазақстан ауыл шаруашылығының негізгі капиталына
салынған инвестициялардың жай-күйі мен динамикасын және олардың еңбек
өнімділігіне әсерін зерттеуге бағытталған. Енгізіліп жатқан бірқатар
мемлекеттік бағдарламалар мен ауыл шаруашылығына бағытталған қомақты
инвестицияларға қарамастан, еңбек өнімділігінің төмендігі терең талдауды
қажет ететін маңызды мәселе болып отыр. Зерттеудің мақсаты Қазақстан
ауыл шаруашылығының негізгі капиталына инвестицияларды жүйелі талдау
және олардың еңбек өнімділігіне әсерін бағалау болып табылады. Мақалада
мемлекеттік бағдарламаларды іске асырудың тиімділігіне және олардың ауыл
шаруашылығын дамытуға қосқан үлесіне әсер ететін негізгі факторлар да
қарастырылады. Зерттеу гипотезасы -- Қазақстанның ауыл шаруашылығының
негізгі капиталына инвестициялар көлемінің ұлғаюы еңбек өнімділігіне оң
әсер етеді, дегенмен бұл ықпалдың тиімділігі мемлекеттік бағдарламаларды
жоспарлау мен іске асыру сапасына және ауыл шаруашылығы тауарын
өндірушілердің іскерлік белсенділік деңгейіне де байланысты болады.
Зерттеу нәтижелері Қазақстанның азық-түлік қауіпсіздігі мен тұрақты
экономикалық өсуін қамтамасыз ету үшін еңбек өнімділігін және аграрлық
сектордағы инвестицияларды пайдалану тиімділігін арттыруда кешенді бағыт
ұстану қажеттігін көрсетеді.

{\bfseries Түйін сөздер:} инвестициялар, ауыл шаруашылығы, еңбек
өнімділігі, мемлекеттік қолдау, тиімділік, аграрлық сектор.

\begin{articleheader}
{\bfseries ИНВЕСТИЦИИ И ПРОИЗВОДИТЕЛЬНОСТЬ ТРУДА В СЕЛЬСКОМ ХОЗЯЙСТВЕ
КАЗАХСТАНА: СОСТОЯНИЕ, ПРОБЛЕМЫ И РЕШЕНИЯ}

{\bfseries \textsuperscript{1}А.К. Байдаков\textsuperscript{\envelope },
\textsuperscript{2}Н.К. Кучукова,
\textsuperscript{1}Р.С Беспаева,
\textsuperscript{1}Ж.С.Булхаирова,
\textsuperscript{3}Б.А. Жуматаева}
\end{articleheader}

\begin{affiliation}
\emph{\textsuperscript{1}Казахский агротехнический исследовательский
университет}

\emph{им. С. Сейфуллина, Астана, Казахстан,}

\emph{\textsuperscript{2} Евразийский национальный университет им. Л.Н.
Гумилева, Астана, Казахстан,}

\emph{\textsuperscript{3}Казахский университет технологии и бизнеса им.
К.Кулажанова, Астана, Казахстан,}

\emph{e-mail: a\_baidakov@mail.ru}
\end{affiliation}

Данная работа направлена на исследование состояния и динамики инвестиций
в основной капитал сельского хозяйства Казахстана и их влияния на
производительность труда, что становится особенно актуальным в условиях
современных вызовов аграрного сектора. Несмотря на ряд внедряемых
государственных программ и значительные инвестиции, которые вливаются в
сельское хозяйство, низкая производительность труда остается серьезной
проблемой, требующей глубокого анализа. Целью исследования является
системный анализ инвестиций в основной капитал сельского хозяйства
Казахстана и оценка их влияния на производительность труда. В статье
также рассматриваются ключевые факторы, влияющие на эффективность
реализации государственных программ и их вклад в развитие сельского
хозяйства. Гипотеза исследования заключается в том, что увеличение
объемов инвестиций в основной капитал сельского хозяйства Казахстана
положительно сказывается на производительности труда, при этом
эффективность этого влияния зависит от качества планирования и
реализации государственных программ и уровня деловой активности самих
сельхозтоваропроизводителей. Результаты исследования подчеркивают
необходимость комплексного подхода к повышению производительности труда
и эффективности использования инвестиций в аграрном секторе для
обеспечения продовольственной безопасности и устойчивого экономического
роста Казахстана.

{\bfseries Ключевые слова:} инвестиции, сельское хозяйство,
производительность труда, государственная поддержка, эффективность,
аграрный сектор.

\begin{articleheader}
{\bfseries INVESTMENTS AND LABOR PRODUCTIVITY IN AGRICULTURE OF KAZAKHSTAN:
STATUS, PROBLEMS AND SOLUTIONS}

{\bfseries \textsuperscript{1}A.K. Baidakov\textsuperscript{\envelope },
\textsuperscript{2}N.K. Kuchukova,
\textsuperscript{1}R.S. Bespayeva,
\textsuperscript{1}Zh.S. Bulkhairova,
\textsuperscript{3}B.A. Zhumataeva}
\end{articleheader}

\begin{affiliation}
\emph{\textsuperscript{1}S. Seifullin Kazakh Agro Technical Research
University, Astana, Kazakhstan,}

\emph{\textsuperscript{2}L.N. Gumilyov Eurasian National University,
Astana, Kazakhstan,}

\emph{\textsuperscript{3}K. Kulazhanov Kazakh University of Technology
and Business, Astana, Kazakhstan,}

\emph{e-mail: a\_baidakov@mail.ru}
\end{affiliation}

This work is aimed at studying the state and dynamics of investments in
fixed assets of agriculture in Kazakhstan and their impact on labor
productivity, which is becoming especially relevant in the context of
modern challenges of the agricultural sector. Despite a number of
government programs being implemented and significant investments being
poured into agriculture, low labor productivity remains a serious
problem that requires in-depth analysis. The purpose of the study is a
systematic analysis of investments in fixed assets of agriculture in
Kazakhstan and an assessment of their impact on labor productivity. The
article also examines the key factors influencing the effectiveness of
the implementation of government programs and their contribution to the
development of agriculture. The hypothesis of the study is that an
increase in investments in fixed assets of agriculture in Kazakhstan has
a positive effect on labor productivity, while the effectiveness of this
influence depends on the quality of planning and implementation of
government programs and the level of business activity of agricultural
producers themselves. The results of the study emphasize the need for an
integrated approach to improving labor productivity and the efficiency
of using investments in the agricultural sector to ensure food security
and sustainable economic growth in Kazakhstan.

{\bfseries Keywords:} investments, agriculture, labor productivity,
government support, efficiency, agricultural sector.

\begin{multicols}{2}
{\bfseries Кіріспе.} Қазақстанның ауыл, орман және балық шаруашылығының
(ауыл шаруашылығы) негізгі капиталына инвестициялардың жай-күйі мен
динамикасын зерттеу тақырыбын таңдау елдің аграрлық секторының алдында
тұрған өзекті сын-қатерлерге байланысты. 2000 жылдан бастап ауыл
шаруашылығын дамытуға бағытталған оннан астам ірі мемлекеттік
бағдарламалардың жүзеге асырылуына қарамастан, негізгі капиталға
инвестицияларды жандандыру мен еңбек өнімділігінің төмендігі мәселесі
сақталып отыр және мұқият талдауды қажет етеді. Мемлекет Басшысы
Қасым-Жомарт Тоқаетың 2024 жылғы 2 қыркүйектегі Қазақстан халқына
Жолдауында: «Агроөнеркәсіп кешеніне инвестиция тарту -- өте маңызды
міндет. Алайда агроөнеркәсіпке бөлінетін бүкіл ақшаның 70 пайызы --
мемлекет қаржысы. Бұл салаға коммерциялық банктердің қаражатын да тарту
қажет» деп көрсетілген {[}1{]}.

Саланың ел экономикасына маңыздылығын қоса алғанда климаттың өзгеруі,
технологияларды жаңғырту қажеттілігі және халықтың тұрақты өсуі сияқты
өзекті мәселелермен байланысты соңғы жылдары бұл тақырыпты зерттеуге
елімізде де, шет мемлекеттерде де қызығушылық еселеп артып отыр.
Аймурзина Б.Т., Каменова М.Ж., Бектенова Д.Ч. зерттеуінде еліміздегі
нақты статистикалық деректер негізінде өндіріске бағытталған
инвестициялар мен өсімдік шаруашылығы өндірісінің арту қарқына арасында
оң, күшті байланыс бар екендігі анықталған. Зерттеушілер ауыл
шаруашылығының тұрақты дамуының теріс факторлары арасында заманауи ауыл
шаруашылығы техникасының болмауын, негізгі құралдардың тозуын,
инвестициялардың шектеулі ағынын атап өтеді {[}2{]}. Мұндай байланысты
шетелдік зерттеушілердің де еңбектері дәлелдейді. Will Martin
технологияларға бағытталған инвестициялардың ауыл шаруашылығындағы еңбек
өнімділігіне әсерін қарастырып, авторлар ғылыми зерттеулер мен
әзірлемелерге, сондай-ақ жаңа технологияларға инвестициялар еңбек
өнімділігін айтарлықтай арттырады және аграрлық сектордағы тұрақты
экономикалық өсуге ықпал етеді деген қорытындыға келеді {[}3{]}.
Дегенмен, заманауи шетелдік зерттеулер әртүрлі елдерде инвестициялардың
ауыл шаруашылығындағы еңбек өнімділігіне әсері аграрлық экономиканың
жағдайына байланысты айтарлықтай өзгеруі мүмкін екендігін көрсетеді. АҚШ
пен Еуропа елдері сияқты жоғары дамыған аграрлық секторлары бар елдерде
өнімділіктің жоғарылауына автоматтандыру және инновациялық
технологияларды енгізу айтарлықтай әсер етеді. Мысалы, генетикалық
инновациялар, жасанды интеллект арқылы ұшқышсыз ұшу аппараттарын
(дрондарды) және GPS басқару жүйелерін пайдалану. Ал дамушы елдерде
негізгі инфрақұрылымға (жолдар, тасымалдау мен сақтау, суару жүйелеріне)
инвестициялау, сондай-ақ сала жұмысшыларының біліктілігін арттыру,
қаржыландыру мен микронесиелерге қол жетімділікті жақсарту
бағдарламалары еңбек өнімділігіне әсер ететін маңызды факторлар болып
табылады {[}4{]}. Елімізде ұшқышсыз ұшу аппараттарын, GPS басқару
жүйелерін қолдану шектеулі, оған тек ірі шаруашылықтардың ғана
мүмкіндігі бар. Шын мәнінде, көптеген жаңа ауылшаруашылық
технологияларының, соның ішінде техника мен басқару әдістерінің айқын
артықшылықтарына қарамастан, фермерлер оларды іске асырмайды немесе
енгізу және кеңейту процесін бастау үшін көп уақыт қажет. Бұл мәселе
дамушы елдердің көпшілігіне қатысты: зерттеулер жергілікті талаптар мен
сипаттамаларға бейімделу үшін жаңа технологияны өзгерту керек екенін
көрсетеді. Ең бастысы, жаңа технологияның бағасы алдыңғы
технологиялармен салыстырғанда бәсекеге қабілетті болуы керек {[}5{]}.

Отандық зерттеушілердің осы бағыттағы зерттеулерінде инфрақұрылымды
жақсартуға, заманауи технологиялар мен инновацияларға қол жеткізудегі
инвестициялардың рөліне назар аударылады. Калыкова Б.Б., Мадиев Г,
Бекбосынова А.Б. зерттеуінің негізгі аспектісі кәсіпорындардың
инновациялық және инвестициялық белсенділігін тежейтін факторларды
анықтау болып табылады. Авторлар ауыл шаруашылығындағы еңбек өнімділігін
арттыру үшін инвестицияларды тартуға жағдай жасау қажеттігін атап
көрсетеді. Олар қаржылық реттеуде, салық және сақтандыру жүйелерінде
жаңа тәсілдерді әзірлеуді қоса алғанда, кәсіпкерлік белсенділікті
ынталандыру үшін қажетті макроэкономикалық ортаны қалыптастыру жолдарын
ұсынады {[}6{]}. Ж.К. Карымсакова, У.К. Керимованың зерттеулерінде
республиканың ауыл шаруашылығында жоғары нәтижелерге қол жеткізуді, оның
әлеуетін арттыруды тежейтін негізгі себептер көрсетілген: техникалық
жарақтандырудың төмендігі, табиғи ресурстарды, атап айтқанда жер, су
ресурстарын тиімсіз пайдалану, өндірілген өнімді сақтаудың,
тасымалдаудың және өткізудің қажетті жүйесінің болмауы, инвестициялар
салу үшін осы саланың тартымсыздығы {[}7{]}. Кучукова Н.К., Рамазанова
Ш.Ш. зерттеуінде агроөнеркәсіптік кешенді қаржылық қамтамасыз ету
мәселелері жан-жақты зерттеліп, авторлар агроөнеркәсіп кешені (АӨК)
субъектілерінің қаржылық қызметтерге қол жетімділігінің неғұрлым маңызды
тежеуші факторлары ретінде еңбекке қабілетті халықтың ауылдық жерлерден
кетуі, кепіл базасының болмауы немесе жетіспеушілігі, сондай-ақ шағын
АӨК субъектілерінің төлем қабілеттілігіне теріс әсер ететін ауылдық
жерлердегі табыстың жеткілікті төмен деңгейін атап өтеді {[}8{]}.

Кейбір зерттеушілер саланың дамуын Біріккен Ұлттар Ұйымына (БҰҰ) мүше
мемлекеттер 2015 жылы қабылдаған 17 мақсатты 2030 жылға дейін жүзеге
асыруды көздейтін Тұрақты даму мақсаттарымен (ТДМ) байланыстырып, АӨК-ті
тұрақты дамыту мәселелері қатарында ауылшаруашылық жерлерінің қатты
тозуы, экологиялық және су проблемаларымен қатар физикалық және
моральдық тұрғыдан ескірген жабдықтар мен технологияларды пайдалануды да
атап өтеді {[}9{]}. Ел халқының 40\%-ға жуығы шоғырланған ауыл
шаруашылығын дамыту мен қолдау: 1 ТДМ «Кедейлікті жою», 6 ТДМ «Таза су
және санитария», 8 ТДМ «Лайықты жұмыс орны мен экономикалық өсу», 9 ТДМ
«Инновация мен инфрақұрылым», 11 ТДМ «Тұрақты қалалар мен елді
мекендер», 13 ТДМ «Климаттың өзгеруімен күрес» және басқа да мақсаттарды
жүзеге асырумен де тығыз байланысты. Ауыл шаруашылығын қоса алғанда
басқа барлық бағыттарды (2024 жылғы 1 қаңтардан бастап тоқтатылған 9
Ұлттық жоба) қамтитын ұлттық жобаларды қомақты қаржыландыруға
қарамастан, Қазақстан 2024 жылғы ТДМ жүзеге асыру туралы Баяндаманың
қорытындысы бойынша жарияланған рейтингте 167 ел арасында 71,1 балл
жинап, 66-шы орын деңгейіндегі өткен жылғы өз орнын сақтап қалғанымен,
2021 ж. 59-орынмен салыстырғанда көп шегініс жасаған {[}10{]}. ЕО
мемлекеттері ТДМ мақсаттарын жүзеге асыруда көшбасшы орындарды иеленіп,
қазіргі уақытта еңбек өнімділігін жоғарылату мәселесі ауыл шаруашылығы
экономикасындағы жасыл технологиялармен, қоршаған ортаны қорғаумен,
парниктік газдардың көлемін қысқартумен байланыстырылады. Осы бағыттағы
зерттеулер бойынша 2004-2021 жж. аралығында ЕО мемлекеттері ауыл
шаруашылығында тиімсіз экономикалық өсім көрсеткен: сыртқы ортаға келген
зиян экономикалық өсімнен алынған нәтижеден асып кеткен {[}11{]}.
Дамыған елдерде экономикалық өсу қоршаған ортаны сақтаумен қатар жүруі
керек: экономикадағы жетістіктер экожүйелердің нашарлауын туындатпауы
тиісті. Бұл тұжырымдама жүзеге асырылатын негізгі салалардың бірі - ауыл
шаруашылығы. Мұнда экологиялық жағдайды жақсартуға және өндіріс
тиімділігін арттыруға бағытталған "жасыл" технологияларға инвестициялар
белсенді түрде ынталандырылады. Мемлекеттер мен халықаралық ұйымдар бұл
бастамаларды субсидиялар, гранттар және салықтық жеңілдіктер
бағдарламалары арқылы қолдайды, бұл фермерлер мен агроөнеркәсіптік
кешендерді экологиялық таза технологияларға көшуге ынталандырады. Бұл
халықтың өмір сүру сапасы мен денсаулығын жақсартуға ғана емес, сонымен
қатар "жасыл" жұмыс орындарын құруға және экологиялық таза нарықтарды
дамытуға жаңа мүмкіндіктер ашады {[}12{]}.

Бұл зерттеулер Қазақстанның аграрлық секторындағы инвестициялық қызмет
пен оның өндірісті дамытуға ықпалын одан әрі зерделеу қажеттігін
көрсетеді. Атап айтқанда, қазіргі заманғы технологияларды енгізуге,
сондай-ақ мемлекеттік қолдаудың инвестициялар тиімділігіне, ал олардың
өз кезегінде еңбек өнімділігіне әсерін бағалауға көбірек көңіл бөлу
қажет. Сонымен бірге, қазіргі уақытты барлық мемлекеттер үшін, әсіресе
Қазақстанда климаттық өзгерістер - еңбек өнімділігіне әсер ететін ең
маңызды факторлардың бірі болып отыр. Ауыл шаруашылығы климаттың
өзгеруіне барған сайын сезімтал болып келеді, бұл ауыл шаруашылығын
тұрақты дамытуға қомақты қосымша инвестицияларды қажет етеді.
Мемлекетіміз климаттың өзгеруіне қарсы күрес жөніндегі жаһандық
бастамаларға белсенді қатысып, парниктік газдар шығарындыларын азайтуға
бағытталған келісімдерге қол қойды. Ауыл шаруашылығында "жасыл"
технологияларды енгізу осы міндеттемелерді орындаудың маңызды бөлігі
болып табылады.

Зерттеудің өзектілігі ғылыми қоғамдастықтың қызығушылығымен ғана емес,
сонымен бірге оның экономикалық саясатта практикалық қолдану үшін
маңыздылығымен де анықталады. Қазақстан Республикасының Стратегиялық
жоспарлау және реформалар агенттігі Ұлттық статистика бюросының (ҚР СЖРА
ҰСБ) деректері бойынша 2024 жылдың басына ел тұрғындарының 37\%-ы (7520
мың адам) ауылдық жерлерде тұратынын және жұмыспен қамтылған
тұрғындардың 12\%-ға жуығы (1061 мың адам) осы салада еңбек ететінін
ескерсек, аграрлық сектордағы өнімділіктің артуы тұрақты экономикалық
өсудің негізгі факторы ретінде бағаланады. Жұмыскерлер санының кемуіне
қарамастан ауыл шаруашылығы жұмыспен қамтылғандардың ең көп үлесі
шоғырланған сауда саласынан кейінгі салалардың қатарында: сауда
саласында жұмыспен қамтылғандардың 16,8\%-ы еңбек етсе, келесі орындарда
үлестері шамалас білім беру саласы (13\%), өнеркәсіп (12,6\%) және ауыл
шаруашылығы (11,5\%) үлесті алады {[}13{]}.

Зерттеудің мақсаты Қазақстан ауыл шаруашылығының негізгі капиталына
инвестицияларды жүйелі талдау және олардың еңбек өнімділігіне әсерін
бағалау болып табылады. Зерттеу гипотезасы -- Қазақстанның ауыл
шаруашылығының негізгі капиталына инвестициялар көлемінің ұлғаюы еңбек
өнімділігіне оң әсер етеді, дегенмен бұл ықпалдың тиімділігі мемлекеттік
бағдарламаларды жоспарлау мен іске асыру сапасына және ауыл шаруашылығы
тауарын өндірушілердің іскерлік белсенділік деңгейіне де байланысты
болады. Зерттеу нәтижелері Қазақстанның азық-түлік қауіпсіздігі мен
тұрақты экономикалық өсуін қамтамасыз ету үшін еңбек өнімділігін және
аграрлық сектордағы инвестицияларды пайдалану тиімділігін арттыруда
кешенді бағыт ұстану қажеттігін көрсетеді. Зерттеудің маңыздылығы
инвестициялық ахуалды жақсартуға және Қазақстанның ауылдық аудандарында
еңбек өнімділігі мен халықтың өмір сүру деңгейін арттыруға ықпал ететін
практикалық ұсыныстарды әзірлеумен байланысты.

{\bfseries Материалдар мен әдістер.} Зерттеудің негізгі мәселесі
Қазақстанның ауыл шаруашылығында инвестицияларды тиімді бағыттау мен
пайдалануға қандай факторлар кедергі келтіреді және бұл еңбек
өнімділігіне қалай әсер ететінін анықтауда. Мәселені зерттеу келесі
кезеңдерді қамтыды:

1) деректерді жинау: ауыл шаруашылығының жағдайы, инвестициялар мен
еңбек өнімділігінің динамикасы туралы статистикалық деректерді зерттеу,
әдебиетке шолу жасау;

2) ағымдағы жағдайды талдау: соңғы 5-10 жылдағы ауыл шаруашылығының
негізгі капиталына инвестициялар, еңбек өнімділігі және басқа да сапалық
көреткіштердің динамикасын бағалау;

3) мәселелерді анықтау: инвестициялардың тиімділігі мен еңбек
өнімділігіне әсер ететін факторларды талдау;

4) ұсыныстар мен қорытындылар әзірлеу.

Зерттеу материалдары ретінде ҚР СЖРА ҰСБ статистикалық деректері,
Қазақстан Республикасы Ауыл шаруашылығы министрлігінің есептері мен
ақпараттық материалдары, ҚР даму стратегиялары мен ұлттық жобаларда
келтірілген деректер, даму институттары мен зерттеу ұйымдарының
талдамалық материалдары, Қазақстанның жетекші ғалымдарының
монографиялары мен мақалаларының материалдары, өзіндік ғылыми зерттеулер
мен алынған нәтижелердің деректері қолданылды.

Зерттеудің негізі жүйелік тәсіл, танымның диалектикалық әдісі, логикалық
талдау және синтез, ғылыми абстракция, сараптамалық бағалау әдістері
және салыстырмалы талдау, зерттеудің экономикалық-статистикалық әдістері
саналады.

{\bfseries Нәтижелер мен талқылау.} 1-кесте келтірілген ҚР СЖРА ҰСБ
деректері бойынша Қазақстан Республикасы бойынша 2015-2023 жж.
аралығында негізгі капиталға инвестициялар тұрақты өсім көрсетті, 7025
миллиард теңгеден 17649 миллиард теңгеге дейін артты. Дегенмен оның ЖІӨ
қатынасы төмендеп, 2023 жылы ол 14,7\% құраған. Бұл өсімнің жыл сайынғы
орташа көрсеткіші 10\% деңгейінде. Ауыл шаруашылығының негізгі
капиталына инвестициялар мемлекеттік бағдарламалардың жүзеге асырылуына
байланысты да жоғары қарқынмен өсіп, 2015 жылғы 164 миллиард теңгеден
904 миллиард теңгеге жеткенімен (саланың жалпы өніміне қатынасы 11,8\%
болғанымен, оның ЖІӨ-ге қатынасы тарихи 1\%-ға жеткен жоқ ), оның үлес
салмағы жалпы инвестицияардан 5,1\% шамасында төмен күйінде қалып отыр.
Ауыл шаруашылығына инвестициялардың ЖІӨ-ге пайыздық деңгейі көптеген
факторларға, соның ішінде елдің және оның ауылшаруашылық секторының даму
деңгейіне байланысты болғанымен, Дүниежүзілік банк 2021 ж. есебінде
дамушы елдерде еңбек өнімділігінің өсуін және сектордың тұрақты дамуын
қамтамасыз ету үшін ауыл шаруашылығына ЖІӨ-нің 5\% дан жоғары, дамыған
аграрлық секторлары бар елдерде ауыл шаруашылығын инновациялау мен
жаңғыртуды қолдау үшін ЖІӨ-нің 1-3\% шамасында инвестициялау жеткілікті
деп санайды {[}14{]}. Сонымен бірге, ауыл шаруашылығына бағытталған
инвестициялардың арту қарқындарында жоғары ауытқу, құбылмалық байқалады.
Ауыл шаруашылығының негізгі капиталына инвестициялардың арту қарқыны
экономикалық жағдайды, мемлекеттік саясатты, климаттық жағдайларды және
қаржыландырудың қолжетімділігін қоса алғанда, бірқатар факторларға
байланысты 5-тен 55\%-ға дейін ауытқиды. Бұл ауытқулар секторға
инвестициялардың қарқынды бағытталуын көрсеткенмен, сыртқы және ішкі
тәуекелдерге жоғары бейімділік пен сын-қатерлерге тұрақсыздықты
айқындайды.
\end{multicols}

\begin{table}[H]
\caption*{1-кесте - 2015-2023 жж. негізгі капиталға инвестициялар: Қазақстан және оның ауыл шаруашылығы}
\centering
\begin{tabular}{|lccccc|}
\hline
\multicolumn{1}{|l|}{\multirow{2}{*}{Жылдар}} &
  \multicolumn{1}{p{0.2\textwidth}|}{\multirow{2}{=}{ҚР бойынша негізгі капиталға инвестициялар, млрд. теңге}} &
  \multicolumn{2}{p{0.25\textwidth}|}{оның ішінде ауыл шаруашылығының негізгі капиталына инвестициялар} &
  \multicolumn{2}{p{0.25\textwidth}|}{Тізбекті өсу қарқыны (ағымдағы бағада), \%} \\ \cline{3-6} 
\multicolumn{1}{|l|}{} &
  \multicolumn{1}{l|}{} &
  \multicolumn{1}{l|}{млрд. теңге} &
  \multicolumn{1}{p{0.1\textwidth}|}{жалпыдан үлесі, \%} &
  \multicolumn{1}{p{0.12\textwidth}|}{республика бойынша} &
  \multicolumn{1}{p{0.15\textwidth}|}{ауыл шаруашылығы} \\ \hline
\multicolumn{1}{|l|}{2015} & \multicolumn{1}{c|}{7025}  & \multicolumn{1}{c|}{164} & \multicolumn{1}{c|}{2,3} & \multicolumn{1}{c|}{-}     & -     \\ \hline
\multicolumn{1}{|l|}{2016} & \multicolumn{1}{c|}{7762}  & \multicolumn{1}{c|}{254} & \multicolumn{1}{c|}{3,3} & \multicolumn{1}{c|}{110,5} & 154,9 \\ \hline
\multicolumn{1}{|l|}{2017} & \multicolumn{1}{c|}{8771}  & \multicolumn{1}{c|}{348} & \multicolumn{1}{c|}{4,0} & \multicolumn{1}{c|}{113,0} & 137   \\ \hline
\multicolumn{1}{|l|}{2018} & \multicolumn{1}{c|}{11179} & \multicolumn{1}{c|}{365} & \multicolumn{1}{c|}{3,3} & \multicolumn{1}{c|}{127,5} & 104,9 \\ \hline
\multicolumn{1}{|l|}{2019} & \multicolumn{1}{c|}{12577} & \multicolumn{1}{c|}{495} & \multicolumn{1}{c|}{3,9} & \multicolumn{1}{c|}{112,5} & 135,6 \\ \hline
\multicolumn{1}{|l|}{2020} & \multicolumn{1}{c|}{12270} & \multicolumn{1}{c|}{573} & \multicolumn{1}{c|}{4,7} & \multicolumn{1}{c|}{97,6}  & 115,8 \\ \hline
\multicolumn{1}{|l|}{2021} & \multicolumn{1}{c|}{13240} & \multicolumn{1}{c|}{772} & \multicolumn{1}{c|}{5,8} & \multicolumn{1}{c|}{107,9} & 134,7 \\ \hline
\multicolumn{1}{|l|}{2022} & \multicolumn{1}{c|}{15251} & \multicolumn{1}{c|}{850} & \multicolumn{1}{c|}{5,6} & \multicolumn{1}{c|}{115,2} & 110,1 \\ \hline
\multicolumn{1}{|l|}{2023} & \multicolumn{1}{c|}{17649} & \multicolumn{1}{c|}{904} & \multicolumn{1}{c|}{5,1} & \multicolumn{1}{c|}{115,7} & 106,4 \\ \hline
\multicolumn{6}{|l|}{\textit{Ескерту – ҚР СЖРА ҰСБ {[}13{]} деректері негізінде авторлармен есептелген.}}                                          \\ \hline
\end{tabular}
\end{table}

\begin{multicols}{2}
Салада тұрақты дамуға қол жеткізу және ауыл шаруашылығының инвестициялық
тартымдылығын арттыру үшін бар мәселелерді жою және тиімді қолдау
стратегияларын енгізу қажет. Осы тұстағы ең маңызды мәселенің бірі --
агроөнеркәсіптік кешенді дамытудың мемлекеттік бағдарламаларының өз
мақсаттарына жетпей, аяқсыз қалуы. Сонымен қатар, субсидиялау ережелері
жиі өзгереді, мысалы, соңғы үш жылда олар 40-ден астам рет, көбінесе
теріс әсер ететін бағытқа өзгертілген (қолдау көлемін қысқарту, кейбір
бағыттарды алып тастау және мемлекеттік көмек алу критерийлерін
қатаңдату). Мұндай шарттарда ұсақ шаруалар тұрмақ ірі агробизнес
өкілдері үшін де ұзақ мерзімді инвестициялық саясатты жүзеге асыру,
тұрақтылықты сақтау, өндірістік шығындарды төмендету, сондай-ақ несиелік
міндеттемелерді уақытында орындау өте қиын.

Мысалы, тәуелсіздік жылдары агроөнеркәсіптік кешенді дамытуға тікелей
бағытталған онға жуық ірі мемлекеттік бағдарламалардың жүргізілуіне
қарамастан, олардың негізгі мақсаттарына қол жеткізілген жоқ. Атап
айтқанда:

1. ҚР Үкіметінің 12.07.2018 ж. қаулысымен бекітілген «Қазақстан
Республикасының агроөнеркәсіптік кешенін дамытудың 2017 -- 2021 жылдарға
арналған мемлекеттік бағдарламасына» сәйкес агроөнеркәсіптік кешенде
еңбек өнімділігін 2015 жылғы 1,2 млн. теңгеден 3,7 млн. теңгеге дейін,
өңделген өнім экспортын 945 млн. теңгеден 2400 млн. Ақш долларына дейін
ұлғайту көзделген {[}15{]}. Бағдарлама 2017 жылды қамтуына қарамастан,
бір жылдан кейін бекітілген. Бағдарламаға жалпы 2774,6 млрд. теңге
көзделген (республикалық бюджет -- 1740,1 млрд. теңге не 63\%,
жергілікті бюджет -- 768,2 млрд. теңге не 28 \%, басқа көздер -- 266,3
млрд. теңге не 9\% ). Дегенмен, бағдарламаның негізгі мақсаттарына қол
жеткізілген жоқ, осы 5 жылдық бағдарлама аяқталмай тұрып ҚР Ауыл
шаруашылығы министрлігі «Қазақстан Республикасының агроөнеркәсіптік
кешенін дамыту жөніндегі 2021 -- 2025 жылдарға арналған ұлттық жобаны»
бекітеді (ол жоба да 01.01.2024 ж. бастап күшін жойған).

Осы бағдарламаларға дейін де 2010-2014 жж. агроөнеркәсіптік кешенді
дамыту жөніндегі бағдарлама аяқталудан 2 жыл бұрын (2013 ж. басында),
2013-2020 жж. арналған «Агробизнес-2020» бағдарламасы да 2017 жылы күшін
жойған болатын. Олар да өз кезегінде соңына дейін жүзеге асырылмаған
келесі кезекті бағдарламалармен алмастырылған.

2. 01.01.2024 ж. бастап күшін жойған «Қазақстан Республикасының
агроөнеркәсіптік кешенін дамыту жөніндегі 2021 -- 2025 жылдарға арналған
ұлттық жобаға» сәйкес агроөнеркәсіптік кешенде 2025 жылға дейін еңбек
өнімділігін 6,2 млн. теңгеге дейін өсіру (2,5 есе), сала өнімі экспортын
6,6 млрд. АҚШ долларына жеткізу, 7 ірі экожүйе қалыптастыру және
инвестициялық жобалар есебінен 1 млн. ауыл тұрғынының табысын
жоғарылатуды тұрақты түрде қамтамасыз етуді және басқа да стратегиялық
мақсаттар көзделген {[}16{]}. Бағдарламаға жалпы 6807,3 млрд. теңге
көзделген (бюджет қаражаттары -- 2707,3 млрд. теңге не 40\%, бюджеттен
тыс қаражат -- 4100,0 млрд. теңге не 60\% ). Дегенмен, ҚР Үкіметінің
22.09.2023 жылғы қаулысымен 2021 жылы 12 қазанда бекітілген, 2025 жылға
дейін ел тұрғындарының өмір сүру деңгейі мен сапасын жоғарылатуға
бағытталған 9 Ұлттық жоба (құны 48 трлн. теңге, шамамен 110 млрд. АҚШ
долларын құрайтын денсаулық сақтау, білім беру, кәсіпкерлікті дамыту,
агроөнеркәсіптік кешенді дамыту және басқа да бағыттарды қамтитын 10
жоба) 2024 жылдан бастап күштерін жойды. 2022 жылғы республикалық
бюджеттің орындалуы туралы Есепте: Ұлттық жобалар өте қомақты, оларға
алдыңғы бағдарламалардағы шаралар кемшіліктері түзетілмей көшірілген;
2022 жыл қорытындысы бойынша 9 Ұлттық жобадан 7 жобаны жүзеге асыру
тиімділігі төмен, жобаларды жүзеге асыру мақсаттары көзделген
мерзімдерде орындалмайтындығы нақты айтылған, оның ішінде
агроөнеркәсіптік кешен де бар {[}17{]}. Ұлттық жобаларға бағытталған
бюджет ресурстарын тиімсіз пайдалану орын алып, бұл Жалпыұлттық
мақсаттарға қол жеткізуге теріс әсер етіп отыр. Мәселелерге жобалардың
жеткіліксіз пысықталуы, олардың кейінгі бақылаусыз мерзімінен бұрын
аяқталуы және мониторингтің болмауы жатады, бұл қажетсіз нәтижелерге
және іске асырылып жатқан бастамалардың күмәнді тиімділігіне әкелді.

2024 жылы 17 бағытты қамтитын жаңа «Қазақстан Республикасының 2029 жылға
дейінгі ұлттық даму жоспары» бекітілді (2025 жылға дейінгі даму жоспары
күшін жойды). Құжатқа алдыңғы 2025 жылға дейінгі даму жоспарына енген,
оның ішінде 2021 жылы қабылданып, күшін жойған Ұлттық жобалардағы
міндеттер де енген. Даму жоспарының «Агроөнеркәсітік кешен» бағытында
2029 жылға дейінгі Қазақстанда АӨК дамыту үш іргелі міндетке қол
жеткізуге бағытталған: әлеуметтік маңызы бар тауарлардың барлық түрлері
бойынша азық-түлік қауіпсіздігін қамтамасыз ету, мал шаруашылығы
өнімдерінің экспорттық әлеуетін жоғарылату және ауыл шаруашылығы
өнімдерін қайта өңдеу, сондай-ақ нәтижелі жұмыспен қамтуды және
жұмыскерлердің лайықты әл-ауқатын қамтамасыз ету {[}18{]}.

Осылайша, қазіргі уақытта елімізде агроөнеркәсіптік кешенді дамытудың
жеке бағдарламасы жоқ: 2029 жылға дейін нақты салаларды дамыту туралы 15
тұжырымдама қатарында агроөнеркәсіптік кешенді дамытуға бағытталған
«Қазақстан Республикасының агроөнеркәсіптік кешенін дамытудың 2021 --
2030 жылдарға арналған тұжырымдамасында» 20 нысаналы индикатор
қамтылған, атап айтқанда зерттеу тақырыбына қатысты:

1-индикатор: ауыл шаруашылығында еңбек өнімділігін 2020 жылмен
салыстырғанда 3 есеге арттыру;

3-индикатор: ауыл шаруашылығы техникасын жыл сайынғы жаңарту деңгейін
қазіргі 4\% дан 2030 жылы 10\%-ға жеткізу;

11-индикатор: ауыл шаруашылығының негізгі капиталына тартылған
инвестиция көлемін 4,2 есеге ұлғайту көзделген {[}19{]}.

Бірақ бұл құжаттарда осыған дейінгі жүзеге асырылған мемлекеттік
бағдарламалар мен шаралардың нәтижелеріне сыни талдау жүргізілмеген,
орын алған кемшіліктер есепке алынбаған.

Қазақстанда құнарлы жерлер мен азық-түлікке сұранысы жоғары елдерге
географиялық жақындығының арқасында осы саланың айтарлықтай әлеуеті бар
екендігі белгілі. Алайда, бұл әлеует негізінен пайдаланылмаған күйінде
қалып отыр. Ауыл шаруашығы жалпы өнімінің құндық өлшемде еселеп өсуіне
қарамастан, ауыл шаруашылығының ЖІӨ-дегі үлес салмағы соңғы он жылда
тұрақты 4-5\% шамасында. Бұл көрсеткіш 2022 жылы Өзбекстанда 25,2\%,
Қырғызстанда 13,7\%, Украинада 9,3\%, Беларусьта 7,7\% құраған {[}20{]}.
Жалпы тұрғындардың 37\% мен еңбек ресурстарының 12\% шоғырланған салады
ЖІӨ-нің 5\% көлемінің өндірілуі саладағы еңбек өнімділігінің абсолютті
шамада өсуіне қарамастан, салыстырмалы түрде өте төмен екендігін
көрсетеді (2-кесте).
\end{multicols}

\begin{table}[H]
\caption*{2-кесте - ҚР бойынша ауылшаруашылының жалпы өнімі мен еңбек өнімділігінің көрсеткіштері}
\centering
\resizebox{\textwidth}{!}{%
\begin{tabular}{|lllllrrr|}
\hline
\multicolumn{1}{|l|}{\multirow{3}{*}{Жылдар}} &
  \multicolumn{1}{p{0.15\textwidth}|}{\multirow{3}{=}{Ауыл шаруашылығының жалпы өнімі, млн. теңге}} &
  \multicolumn{1}{l|}{\multirow{3}{*}{\begin{tabular}[c]{@{}l@{}}Физикалық  \\   көлем  \\   индексі,  \\   алдыңғы  \\   жылға \%\end{tabular}}} &
  \multicolumn{2}{l|}{оның ішінде} &
  \multicolumn{1}{p{0.13\textwidth}|}{\multirow{3}{=}{Ауыл шаруашылығының ЖІӨ-дегі үлесі, \%}} &
  \multicolumn{2}{p{0.15\textwidth}|}{\multirow{2}{=}{Еңбек өнімділігі}} \\ \cline{4-5}
\multicolumn{1}{|l|}{} &
  \multicolumn{1}{l|}{} &
  \multicolumn{1}{l|}{} &
  \multicolumn{1}{l|}{\multirow{2}{*}{\begin{tabular}[c]{@{}l@{}}өсімдік  \\   шаруашылығы\end{tabular}}} &
  \multicolumn{1}{l|}{\multirow{2}{*}{\begin{tabular}[c]{@{}l@{}}мал  \\   шаруашылығы\end{tabular}}} &
  \multicolumn{1}{l|}{} &
  \multicolumn{2}{l|}{} \\ \cline{7-8} 
\multicolumn{1}{|l|}{} &
  \multicolumn{1}{l|}{} &
  \multicolumn{1}{l|}{} &
  \multicolumn{1}{l|}{} &
  \multicolumn{1}{l|}{} &
  \multicolumn{1}{l|}{} &
  \multicolumn{1}{l|}{} &
  \multicolumn{1}{l|}{} \\
\multicolumn{1}{|l|}{} &
  \multicolumn{1}{l|}{} &
  \multicolumn{1}{l|}{} &
  \multicolumn{1}{l|}{} &
  \multicolumn{1}{l|}{} &
  \multicolumn{1}{l|}{} &
  \multicolumn{1}{p{0.06\textwidth}|}{мың теңге} &
  \multicolumn{1}{p{0.06\textwidth}|}{АҚШ долл.} \\ \hline
\multicolumn{1}{|c|}{2015} &
  \multicolumn{1}{c|}{3 307 010} &
  \multicolumn{1}{c|}{103,4} &
  \multicolumn{1}{c|}{104,0} &
  \multicolumn{1}{c|}{102,7} &
  \multicolumn{1}{c|}{4,7} &
  \multicolumn{1}{c|}{1242} &
  5619 \\ \hline
\multicolumn{1}{|c|}{2016} &
  \multicolumn{1}{c|}{3 684 393} &
  \multicolumn{1}{c|}{105,4} &
  \multicolumn{1}{c|}{107,5} &
  \multicolumn{1}{c|}{102,8} &
  \multicolumn{1}{c|}{4,6} &
  \multicolumn{1}{c|}{1402} &
  4100 \\ \hline
\multicolumn{1}{|c|}{2017} &
  \multicolumn{1}{c|}{4 070 917} &
  \multicolumn{1}{c|}{103,0} &
  \multicolumn{1}{c|}{102,2} &
  \multicolumn{1}{c|}{103,9} &
  \multicolumn{1}{c|}{4,5} &
  \multicolumn{1}{c|}{1736} &
  5325 \\ \hline
\multicolumn{1}{|c|}{2018} &
  \multicolumn{1}{c|}{4 474 088} &
  \multicolumn{1}{c|}{103,5} &
  \multicolumn{1}{c|}{103,2} &
  \multicolumn{1}{c|}{103,9} &
  \multicolumn{1}{c|}{4,4} &
  \multicolumn{1}{c|}{2077} &
  6037 \\ \hline
\multicolumn{1}{|c|}{2019} &
  \multicolumn{1}{c|}{5 151 163} &
  \multicolumn{1}{c|}{99,9} &
  \multicolumn{1}{c|}{96,4} &
  \multicolumn{1}{c|}{104,0} &
  \multicolumn{1}{c|}{4,5} &
  \multicolumn{1}{c|}{2466} &
  6455 \\ \hline
\multicolumn{1}{|c|}{2020} &
  \multicolumn{1}{c|}{6 334 669} &
  \multicolumn{1}{c|}{105,7} &
  \multicolumn{1}{c|}{107,8} &
  \multicolumn{1}{c|}{103,1} &
  \multicolumn{1}{c|}{5,4} &
  \multicolumn{1}{c|}{3005} &
  7276 \\ \hline
\multicolumn{1}{|c|}{2021} &
  \multicolumn{1}{c|}{7 515 434} &
  \multicolumn{1}{c|}{97,7} &
  \multicolumn{1}{c|}{93,4} &
  \multicolumn{1}{c|}{103,6} &
  \multicolumn{1}{c|}{5,4} &
  \multicolumn{1}{c|}{3351} &
  7866 \\ \hline
\multicolumn{1}{|c|}{2022} &
  \multicolumn{1}{c|}{8 407 512} &
  \multicolumn{1}{c|}{109,1} &
  \multicolumn{1}{c|}{115,1} &
  \multicolumn{1}{c|}{100,8} &
  \multicolumn{1}{c|}{5,2} &
  \multicolumn{1}{c|}{4608} &
  10017 \\ \hline
\multicolumn{1}{|c|}{2023} &
  \multicolumn{1}{c|}{7 625 151} &
  \multicolumn{1}{c|}{91,7} &
  \multicolumn{1}{c|}{85,9} &
  \multicolumn{1}{c|}{104,5} &
  \multicolumn{1}{c|}{4,3} &
  \multicolumn{1}{c|}{4378} &
  9601 \\ \hline
\multicolumn{8}{|l|}{\textit{Ескерту – ҚР СЖРА ҰСБ {[}13{]} деректері негізінде авторлармен есептелген.}} \\ \hline
\end{tabular}}
\end{table}

\begin{multicols}{2}
2-кестедегі деректер бойынша ауыл шаруашылығының жалпы өнімі құндық
өлшемде еселеп өсуіне қарамастан, физикалық көлем индексі негізінен
3-5\% аралығында, 2022 жылы ғана ең жоғары 9,1\% (қолайсыз болған 2021
жылмен салыстырғанда) құрайды. Ауыл шаруашылығы өнімінің жалпы көлемі
құндық өлшемде 2022 жылға дейін тұрақты өсуді көрсетті, осы жылы ол ең
жоғары 8407 миллиард теңгеге жетті. 2023 жыл отандық ауыл шаруашылығы
үшін өте қиын жыл болды (қолайсыз климаттық жағдай, негізгі астық
дақылдарының бағасының бірден төмендеуі, ресейлік астықтың заңсыз
тасымалдануы) және оның салдары алдағы екі-үш жыл ішінде қазақстандық
фермерлердің әл-ауқатында сезілетін болады. Ауа-райының қолайсыздығына
байланысты астық және майлы дақылдардың өнімділігі айтарлықтай
төмендеді: жаз мезгіліндегі құрғақшылық, сондай-ақ өткен тамыз-қыркүйек
айларында қатты жаңбыр әсерінен астық және бұршақ дақылдарының
өнімділігі 2022 жылғы 13,8 ц/га-дан 2023 жылы 10,3 ц/га дейін, ал
өндіріс көлемі 22,8\%, майлы дақылдар 31\% кеміген. Cол себепті өсімдік
шаруашылығындағы өндірістің физикалық көлемі бірден 14,1\%-ға төмендеп,
мал шаруашылығының өндіріс көлемі 4,5\% артуына қарамастан, осы фактор
ауыл шаруашылығының жалпы өнімі 2022 жылмен салыстырғанда құндық өлшемде
782 млрд. теңгеге (7\%), ал физикалық шамада 8,3\% кемуіне әсер етті.
Өндірістің мұндай жоғары кему қарқыны соңғы 10 жылда байқалмаған, ол
Қазақстандағы өсімдік шаруашылығының климаттық жағдайларға жоғары
тәуелділігімен байланысты. Мұндай қолайсыз жағдай 2019, 2021 жылдары да
орын алған. Сол себепті 2022 жылға дейін тұрақты өсім көрсеткен еңбек
өнімділігі, 2023 жылы 230 мың теңгеге немесе 5\%-ға төмендеген. Бұл
табиғи-климаттық жағдайға тәуелділік пен климаттық өзгерістер --
елімізде ауыл шаруашылығында еңбек өнімділігіне әсер ететін ең маңызды
факторлардың бірі екендігін дәлелдейді. Бұл жағдай ауыл шаруашылығын
тұрақты дамытуға қомақты қосымша инвестицияларды қажет етеді: тамшылатып
суару жүйелері, дақылдардың төзімді сорттары және жерді басқарудың
инновациялық әдістері сияқты тұрақты агротехнологияларға инвестициялар;
сонымен қатар, климаттық тәуекелдерден қорғау үшін инфрақұрылымды дамыту
(мысалы, суару жүйелері мен қорғаныс құрылыстары); жасыл
агротехнологияға инвестиция салу климаттың өзгеруінің жағымсыз
әсерлерімен күресуге ғана емес, сонымен қатар өнімділіктің ұзақ мерзімді
өсуіне және азық-түліктік қауіпсіздікке ықпал етеді. Бұл мемлекеттік
қолдаусыз жүзеге асыру мүмкін емес шаралар, себебі қаржылық
мүмкіндіктері шектеулі фермерлер қымбат инновациялық технологиялар мен
тұрақты агротехникалық шешімдерге өз бетінше инвестиция сала алмайды.

Ауыл шаруашылығында еңбек өнімділігінің тұрақты түрде жоғарылауына
қарамастан, оның деңгейі өте төмен деп саналады. Бұл Қазақстан
экономикасындағы құрылымдық мәселе болып саналады. Жұмыспен
қамтылғандардың 11,9\%-ы еңбек ететін салада экономика секторлары
арасында еңбек өнімділігі де, тиісінше жалақы да ең төмені (3-кесте).
\end{multicols}

\begin{table}[H]
\caption*{3-кесте - Экономика секторлары бойынша еңбек өнімділігі мен жалақы қатынасы, 2023 жыл}
\centering
\resizebox{\textwidth}{!}} &
  \multicolumn{1}{p{0.12\textwidth}|}{\multirow{2}{=}{Еңбек өнімділігі, мың теңге}} &
  \multicolumn{1}{l|}{\multirow{2}{*}{\begin{tabular}[c]{@{}l@{}}Орташа  \\   айлық жалақы,  \\   теңге\end{tabular}}} &
  \multicolumn{2}{l|}{\begin{tabular}[c]{@{}l@{}}Экономика бойынша орташаға  \\   қатынасы, \%\end{tabular}} \\ \cline{5-6} 
\multicolumn{1}{|l|}{} &
  \multicolumn{1}{l|}{} &
  \multicolumn{1}{l|}{} &
  \multicolumn{1}{l|}{} &
  \multicolumn{1}{l|}{} &
  \multicolumn{1}{l|}{} \\
\multicolumn{1}{|l|}{} &
  \multicolumn{1}{l|}{} &
  \multicolumn{1}{l|}{} &
  \multicolumn{1}{l|}{} &
  \multicolumn{1}{l|}{еңбек өнімділігі} &
  \multicolumn{1}{l|}{орташа жалақы} \\ \hline
\multicolumn{1}{|l|}{Экономика бойынша} &
  \multicolumn{1}{c|}{100,0} &
  \multicolumn{1}{c|}{11407} &
  \multicolumn{1}{c|}{364295} &
  \multicolumn{1}{c|}{100,0} &
  100,0 \\ \hline
\multicolumn{1}{|l|}{1.Тауар өндірісі} &
  \multicolumn{1}{c|}{31,3} &
  \multicolumn{1}{c|}{-} &
  \multicolumn{1}{c|}{-} &
  \multicolumn{1}{c|}{-} &
  - \\ \hline
\multicolumn{1}{|p{0.3\textwidth}|}{1.1. Ауыл, орман және балық шаруашылығы} &
  \multicolumn{1}{c|}{11,9} &
  \multicolumn{1}{c|}{4378} &
  \multicolumn{1}{c|}{222532} &
  \multicolumn{1}{c|}{38,4} &
  61,1 \\ \hline
\multicolumn{1}{|l|}{1.2. Өнеркәсіп:} &
  \multicolumn{1}{c|}{12,3} &
  \multicolumn{1}{c|}{28544} &
  \multicolumn{1}{c|}{496733} &
  \multicolumn{1}{c|}{2,5 есе} &
  136,4 \\ \hline
\multicolumn{1}{|l|}{- тау-кен өндірісі} &
  \multicolumn{1}{c|}{3,1} &
  \multicolumn{1}{c|}{55371} &
  \multicolumn{1}{c|}{771048} &
  \multicolumn{1}{c|}{4,8 есе} &
  211,7 \\ \hline
\multicolumn{1}{|l|}{- өңдеу өнеркәсібі} &
  \multicolumn{1}{c|}{6,7} &
  \multicolumn{1}{c|}{24238} &
  \multicolumn{1}{c|}{414388} &
  \multicolumn{1}{c|}{212,5} &
  113,8 \\ \hline
\multicolumn{1}{|l|}{1.3. Құрылыс} &
  \multicolumn{1}{c|}{7,1} &
  \multicolumn{1}{c|}{10464} &
  \multicolumn{1}{c|}{477821} &
  \multicolumn{1}{c|}{91,7} &
  131,2 \\ \hline
\multicolumn{1}{|l|}{2. Қызмет көрсету} &
  \multicolumn{1}{c|}{68,7} &
  \multicolumn{1}{c|}{-} &
  \multicolumn{1}{c|}{-} &
  \multicolumn{1}{c|}{-} &
  - \\ \hline
\multicolumn{1}{|l|}{- көтерме және бөлшек сауда} &
  \multicolumn{1}{c|}{16,7} &
  \multicolumn{1}{c|}{14399} &
  \multicolumn{1}{c|}{319218} &
  \multicolumn{1}{c|}{126,2} &
  87,6 \\ \hline
\multicolumn{1}{|l|}{- көлік және қойма} &
  \multicolumn{1}{c|}{7,1} &
  \multicolumn{1}{c|}{10425} &
  \multicolumn{1}{c|}{465666} &
  \multicolumn{1}{c|}{91,4} &
  127,8 \\ \hline
\multicolumn{1}{|l|}{- ақпарат және байланыс} &
  \multicolumn{1}{c|}{2,1} &
  \multicolumn{1}{c|}{13821} &
  \multicolumn{1}{c|}{588205} &
  \multicolumn{1}{c|}{121,2} &
  161,5 \\ \hline
\multicolumn{1}{|l|}{- қаржы және сақтандыру} &
  \multicolumn{1}{c|}{2,2} &
  \multicolumn{1}{c|}{19214} &
  \multicolumn{1}{c|}{690772} &
  \multicolumn{1}{c|}{168,4} &
  189,6 \\ \hline
\multicolumn{1}{|l|}{- ғылыми-техникалық қызмет} &
  \multicolumn{1}{c|}{2,9} &
  \multicolumn{1}{c|}{15803} &
  \multicolumn{1}{c|}{517028} &
  \multicolumn{1}{c|}{138,5} &
  141,9 \\ \hline
\multicolumn{1}{|l|}{- мемлекеттік басқару, қорғаныс} &
  \multicolumn{1}{c|}{5,8} &
  \multicolumn{1}{c|}{4753} &
  \multicolumn{1}{c|}{313769} &
  \multicolumn{1}{c|}{41,7} &
  86,1 \\ \hline
\multicolumn{1}{|l|}{- білім беру} &
  \multicolumn{1}{c|}{13} &
  \multicolumn{1}{c|}{4742} &
  \multicolumn{1}{c|}{281991} &
  \multicolumn{1}{c|}{41,6} &
  77,4 \\ \hline
\multicolumn{1}{|l|}{- денсаулық сақтау} &
  \multicolumn{1}{c|}{6,4} &
  \multicolumn{1}{c|}{6048} &
  \multicolumn{1}{c|}{292730} &
  \multicolumn{1}{c|}{53} &
  80,4 \\ \hline
\multicolumn{6}{|l|}{\textit{Ескерту - ҚР СЖРА ҰСБ {[}13{]} деректері негізінде авторлармен есептелген.}} \\ \hline
\end{tabular}%
}
\end{table}

\begin{multicols}{2}
ҚР СЖРА ҰСБ 2023 жылғы деректерін талдау еңбек өнімділігіне қатысты
келесідей құрылымдық мәселелерді көрсетеді: елімізде орташа еңбек
өнімділігі (11406 мың теңге) тау-кен өндірісі (55371 мың теңге), өңдеу
өнеркәсібі (24238 мың теңге), қаржылық және сақтандыру қызметі (19214
мың теңге) секілді еңбек өнімділігі ең жоғары салалардан 20-60\% шамасын
құрайды; ал жұмыскерлердің 40\%-ға жуығы еңбек өнімділігі еліміз бойынша
орташа деңгейден төмен секторларда еңбек етеді, оның ішінде білім беру
саласы (13\%), ауыл шаруашылығы (11,9\%), мемлекеттік басқару (5,8\%)
және басқа секторлар (10\%). Еңбек өнімділігі ең жоғары өнеркәсіп саласы
ең өнімді жұмыс орындары және инвестициялардың ең жоғары деңгейлерін
қамтамасыз еткенімен, бірақ олардың жалпы жұмыспен қамтудағы үлесі аз
(12,3 \% шамасында).

Ауыл шаруашылығы мен экономиканың басқа өнімді секторлары арасындағы
жалақы деңгейінің айырмашылығы да айтарлықтай байқалады. 2023 жылы ауыл
шаруашылығындағы орташа айлық атаулы жалақы ең төменгі көрсеткіш -
222532 теңгені құрайды, бұл өңдеу өнеркәсібіне (414388 теңге) және
тау-кен өндіру секторына (771048 теңге) қарағанда еселеп төмен. Оның
басты себебі -- еңбек өнімділігінің төмен болуы. Жұмыспен
қамтылғандардан 3,1\% ғана жұмыскерлер еңбек ететін, еңбек өнімділігі
мен жалақы деңгейі ең жоғары тау-кен өнеркәсібі экономика бойынша еңбек
өнімділігі мен табыстың жоғарылауына елеулі ықпал ете алмайды. Сол
себепті елдің тұрақты экономикалық өсуі мен дамуы үшін ауыл шаруашылығы
сияқты шикізаттық емес секторлардағы өнімділікті арттыруға назар аудару
өте маңызды. Қазақстан ауыл шаруашылығындағы көрсеткіштерді жақсартпай,
жалпы экономикадағы еңбек өнімділігінің артуына қол жеткізе алмайды.

Қазақстанның аграрлық секторындағы еңбек өнімділігі экономиканың басқа
секторларынан ғана емес, дамыған мемлекеттер деңгейінен де еселеп төмен
(4-кесте).
\end{multicols}

\begin{table}[H]
\caption*{4-кесте - Мемлекеттер бойынша еңбек өнімділігінің көрсеткіштері, ағымдағы бағада, 2022 жыл}
\centering
\begin{tabular}{|lccccc|}
\hline
\multicolumn{1}{|l|}{\multirow{2}{*}{Мемлекеттер}} &
  \multicolumn{2}{p{0.2\textwidth}|}{Ауыл шаруашылығындағы еңбек өнімділігі} &
  \multicolumn{2}{p{0.2\textwidth}|}{Өңдеу өнеркәсібіндегі еңбек өнімділігі} &
  \multicolumn{1}{p{0.2\textwidth}|}{\multirow{2}{=}{Өңдеу өнеркәсібінің ауыл шаруашылығына қатынасы, \%}} \\ \cline{2-5}
\multicolumn{1}{|l|}{} &
  \multicolumn{1}{p{0.1\textwidth}|}{мың АҚШ долл.} &
  \multicolumn{1}{p{0.1\textwidth}|}{max шамаға қатынасы, \%} &
  \multicolumn{1}{p{0.1\textwidth}|}{мың АҚШ долл.} &
  \multicolumn{1}{p{0.1\textwidth}|}{max шамаға қатынасы, \%} &
  \multicolumn{1}{l|}{} \\ \hline
\multicolumn{1}{|l|}{Канада}    & \multicolumn{1}{c|}{117} & \multicolumn{1}{c|}{100,0} & \multicolumn{1}{c|}{134} & \multicolumn{1}{c|}{34,1}  & 114,5   \\ \hline
\multicolumn{1}{|l|}{АҚШ}       & \multicolumn{1}{c|}{100} & \multicolumn{1}{c|}{85,5}  & \multicolumn{1}{c|}{392} & \multicolumn{1}{c|}{100,0} & 3,9 есе \\ \hline
\multicolumn{1}{|l|}{Қазақстан} & \multicolumn{1}{c|}{10}  & \multicolumn{1}{c|}{8,5}   & \multicolumn{1}{c|}{53}  & \multicolumn{1}{c|}{13,5}  & 5,3 есе \\ \hline
\multicolumn{6}{|l|}{\textit{Ескерту – {[}13,20, 21{]} дереккөдер негізінде авторлармен есептелген.}}                                                     \\ \hline
\end{tabular}%
\end{table}

\begin{multicols}{2}
Қазақстанның ауыл шаруашылығындағы еңбек өнімділігі 10 мың АҚШ доллары
деңгейінде, бұл климаттық және географиялық шарттары салыстыруға келетін
Канадада қол жеткізілген максималды көрсеткіштің 8,5\%-ын құрайды
(немесе 11,7 есе төмен). Ал елімізде өңдеу өнеркәсібіндегі орташа еңбек
өнімділігі 53 мың АҚШ долларын құрайды, оның деңгейі ауыл шаруашылығының
көрсеткішінен 5,3 есе жоғары. ҚР СЖРА ҰСБ деректері бойынша тау-кен
өндірісінде еңбек өнімділігі ауыл шаруашылығымен салыстырғанда 12,6 есе
жоғары. Мұндай жағдай жалақы деңгейінде де осындай орасан үлкен
айырмашылықты туындатады: 2023 жылы елімізде орташа жалақы 364295
теңгені құраған, оның ішінде экономика секторлары бойынша ең төменгі
жалақы ауыл шаруашылығында -- 222532 теңге (орташа деңгейден 39\%
төмен), ең жоғары жалақы тау-кен өндірісінде 771048 теңгені құраған
{[}13{]}. Сол себепті сарапшылар Қазақстанда ауыл шаруашылығын одан әрі
дамыту үшін үкімет кадр мәселесін емес, саладағы төмен еңбек өнімділігі
мәселесін шешу қажет деп санайды. Олардың пікірінше өндірістің қазіргі
көлемі мен құрылымында ауылдың кадрлық әлеуеті жеткілікті: ЖІӨ-нің 5\%
көлемі өндірілетін салада ел тұрғындардың 37\%-ы мен еңбек ресурстарының
12\%-ы шоғырланған {[}20{]}. Өнімділікті арттыру мемлекет тарапынан да,
өндірушілер мен фермерлер тарапынан да кешенді тәсілді талап ететін
аграрлық саясаттың негізгі бағыттарының біріне айналуы тиіс.

Дүниежүзілік банк пен Халықаралық валюта қоры Қазақстан экономикасы
бойынша жасаған ауқымды баяндамасында инвестицияларды бірден арттыру
есебінен Қазақстан экономикасын екі еселеу мүмкін еместігін,
экономиканың сапалы дамуында басты рольді өз кезегінде жеке
инвестициялардың артуына әкелетін жалпы факторлық еңбек өнімділігінің
жоғарылауы алатындығын үнемі атап өтеді. Осы баяндамада елімізде
факторлық еңбек өнімділігі соңғы 10 жылда өзгермегендігін, оның басты
себебі ретінде -- экономиканы жекешелендіріп, толыққанды нарықтық
экономикаға көшу қажеттігін атайды {[}22{]}.

Зерттеу нәтижелері бойынша ауыл шаруашылығындағы еңбек өнімділігінің
төмендігін келесідей негізгі факторлармен түсіндіруге болады:

1) ауыл шаруашылығындағы еңбек ресурстарының жалпы саны «жеткілікті»
сияқты болып көрінгенімен, еңбек ресурстарының құрамының қартаюы орын
алған және объективті себептерге байланысты аграрлық секторда жұмыс
істеуге дайын жас мамандардың жетіспеушілігі бар. 2023 ж. деректер
бойынша ауыл шаруашылығында жұмыспен қамтылғандар арасында 55 жастан
асқандардың үлес салмағы республика бойынша орташадан елеулі жоғары:
55-65 жас аралығындағы жұмыспен қамтылғандардың 17,5\%-ы, 65 жастан
асқандардың 61,4\%-ы ауыл шаруашылығында еңбек етеді. Сонымен бірге,
республика бойынша жұмыспен қамтылғандар санатындағы еңбекке жарамды
жастан асқандар саны 249907 адам болса, оның ішінде 101349 адам ауыл
шаруашылығында жұмыспен қамтылған (40,7\%); республика бойынша орташа
2,8\% жағдайында аграрлық салада жұмыспен қамтылғандардың 9,4\%-ы
еңбекке жарамды жастан асқандар {[}13{]}. Мемлекет тарапынан ауылдық
аймақтарды дамытуға, жас мамандарды ауылға тартумен байланысты бірнеше
арнайы бағдарламалардың жүзеге асырылуына және білікті кадрларды
даярлауға ықпал ету үшін ауыл шаруашылығы мамандықтары бойынша оқыту
үшін айтарлықтай білім беру гранттарын бөлуіне қарамастан орын алған
жағдай өзгеріп отырған жоқ. Бұл құбылыс тұрақты табыс көзі мен
әлеуметтік мәртебенің болмауы, сондай-ақ мансаптық өсудің шектеулі
мүмкіндіктері, аграрлық саладағы жұмысты қызықты әрі перспективалы ете
алатын заманауи технологиялардың болмауына байланысты -- ауыл
шаруашылығына кәсіп пен өмір салты ретінде қызығушылықтың төмендеуіне
байланысты жастардың көбінесе ата-аналарының фермерлік ісін
жалғастырмауына, жас мамандардың жоғары ақы төленетін және әлеуметтік
тартымды салаларды артық көріп, аграрлық салада жұмыс істеуден бас
тартуына әкеліп отыр;

2) ауылшаруашылық дақылдары мен мал шаруашылығындағы төмен өнімділік.
Табиғи-климаттық факторларға жоғары тәуелділікке, сонымен бірге егістік
алқаптарының диверсификацияланбауы, тұқым шаруашылығының жағдайына,
жердің құнарлығын жоғарылату жұмыстарының жүргізілмеуіне, ылғал сақтау
технологияларының сақталмауына, төмен техникалық жарақтандыруға
байланысты негізгі дақылдардың өнімділігінің төмендегі, мысалы, астық
дақылдарының өнімділігі 10-13 ц/га шамасында ғана; азықтандырудың
талаптарға сәйкес болмауы, асыл тұқымды мал басының төмен болуы,
зоотехникалық талаптардың сақталмауына байланысты 1 сауын сиырдың
өнімділігі өте төмен - 2500 кг. шамасында. Бұл дамыған шет мемлекеттер
тұрмақ, Ресей мен Беларусь мемлекеттері қол жеткізген деңгейден екі
еседен астам төмен;

3) ауылшаруашылық тауар өндірушілерінің құрылымдық ерекшеліктері,
өндірістің еңбек өнімділігін арттыруға материалдық, қаржылық мүмкіндігі
жоқ ұсақ тауарлы шаруашылықтарда шоғырлануы: ҚР СЖРА ҰСБ деректері 2023
жылы ауыл шаруашылығының жалпы өнімі құрылымында ауылшаруашылық
кәсіпорындарының үлесі 28\% болса, бұл көрсеткіш өсімдік шаруашылығында
30\%, мал шаруашылығында 26\% құрайды. Ауылшаруашылығы өнімінің 38\%-н
жұртшылық шаруашылықтары өндірсе, оның үлесі өсімдік шаруашылығында
26\%, ал мал шаруашылығында 56\% құрайды. Өндірістің қалған бөлігі
фермерлік және шаруа қожалықтарында өндіріледі;

4) ескірген, өнімділігі төмен, еңбек және күту шығындары жоғары
технологияларды пайдалану және сала субъектілерінің инновациялық
белсенділігінің төмендігі. Қазақстан Республикасының 2029 жылға дейінгі
ұлттық даму жоспарының өзінде осы мәселе аталып өтеді: тракторлар -
80\%, комбайндар - 72\% тозған, нормативті 8-10\% жағдайында техниканы
жаңартудың орташа қарқыны 4\%. Ескірген техниканы қолданудың еңбек
өнімділігінің төмендігі, технологиялық мерзімдерді сақтауға
мүмкіншіліктің болмауы, материалдық шығындардың жоғары болу секілді
кемшіліктерімен қатар, мысалы, ескірген комбайндар астық дақылдарын
жинауда өнімнің 20-25\%-на дейін жоғалтуға әкеледі. 2029 жылға дейінгі
ұлттық даму жоспарында техниканы жаңартудың нормативті деңгейіне 2030
жылы қол жеткізу жоспарланған. Елімізде 65 мың трактор мен 43 мың
комбайн тапшылығы бар, техниканы жаңартудың қазіргі қарқыны сақталса
(4-4,5\%), жетіспеушілікті толықтыруға 14-15 жыл уақыт кетеді. Сонымен
бірге, жаңа ауыл шаруашылығы техникасын сатып алу үшін кәдеге жарату
алымының жоғары мөлшерлемелері фермерлердің автопарктерді жаңарту
мүмкіндігін айтарлықтай шектеп, бұл оларды ескірген тракторлар мен
комбайндарды пайдалануды жалғастыруға мәжбүр етіп отыр; ал мемлекеттің
осы шарадан бас тартуы техниканы жаңғыртуды ынталандырып, нәтижесінде
аграрлық сектордағы өнімділікті арттыруға әсер ете алар еді;

5) аграрлық нарық субъектілерінің қаржыландыруға қолжетімділігінің
жеткіліксіздігі (екінші деңгейлі банктердің несие портфеліндегі шағын
ауыл шаруашылығы кәсіпорындарының үлесі небәрі 5\% шамасында), өйткені:

- ауыл шаруашылығы кәсіпорындарының жартысынан астамы тиімсіз қызмет
етуде;

-екінші деңгейдегі банктер экономиканың осы секторына жоғары
тәуекелдерге байланысты несие беруге құлықсыз;

-агробизнесті қаржыландыруды ауылдық жерлердегі қалаларға қарағанда
әлдеқайда жоғары операциялық (транзакциялық) шығындар шектейді;

-ауыл шаруашылығы кәсіпорындарындағы дұрыс жолға қойылмаған бухгалтерлік
есеп пен статистика қарыз алушының несиелік қабілетін бағалауды
қиындатады, бұл оларды қаржыландыруға үлкен кедергі болып табылады және
т.б.

6) ауыл шаруашылығының негізгі капиталына инвестициялар тартуды
ынталандырудың төмендігі: оның басты себептері өндірістің төмен
рентабельділігі (зиян) себепті инвестициялардың қайтарымының төмендігі,
меншікті қаражаттардың жетіспеушілігі, міндеттемелер мен қаржыландыруға
қол жетімділіктің төмендігі, сонымен бірге инвестицияларды
қаржыландырудың ең маңызды ішкі көзі ретінде -- амортизацияның өзінің
экономикалық функциясын орындамауы. Бұл кедергілердің ауылшаруашылық
тауарөндірушілерінің негізгі бөлігін құрайтын ұсақ шаруашылықтар үшін
ықпалы үлкен;

7) негізгі капиталға инвестицияның жетіспеушілігі аясында экологиялық
тұрақты ауыл шаруашылығын дамытуға, "жасыл" технологияларға инвестиция
тарту көлемі өте төмен, фермерлердің мұндай әдістердің артықшылықтары
туралы хабардарлығының төмендігі, сондай-ақ жоғары бастапқы шығындарға
байланысты экологиялық таза егіншілікке ауысудың қиындығы, органикалық
өнімді өңдеу және тарату үшін тиімді инфрақұрылымның жетіспеушілігі
қосымша кедергілер тудырады. Су ресурстарын тұрақты пайдалану, әсіресе
құрғақшылық пен климаттың өзгеруі жағдайында, негізгі экологиялық
мәселелердің бірі болып отыр.

{\bfseries Қорытынды.} Зерттеу нәтижелері Қазақстанның ауыл
шаруашылығындағы инвестициялық белсенділік деңгейі еңбек өнімділігін
экономикадағы орташа деңгейге дейін арттыру үшін қажетті шарттарға
сәйкес келмейді деген болжамдарды растайды. Зерттеу нәтижелері ауыл
шаруашылығындағы төмен еңбек өнімділігі мәселесі тұрақты сипатқа ие және
оны жоғарылату кешенді тәсілді қажет ететінін көрсетеді. Атап айтқанда:

1. Көптеген мемлекеттік бағдарламалар мен бастамалардың жүзеге
асырылуына қарамастан, Қазақстанның ауыл шаруашылығына инвестициялар
көлемі экономикаға инвестициялардың жалпы көлемінің небәрі 5\%
деңгейінде қалып отыр, бұл сектордың өзекті мәселелерін шешу үшін
қажетті мөлшерден төмен болғанымен, осы инвестицияларды тартудың өзі
тиісті нәтиже берген жоқ. Бұл жаңа даму жоспарларын дайындауда осыған
дейінгі нәтижелер мен орын алған кемшіліктерді сыни талдаудың болмауына,
қателіктерді елемеуге де байланысты. Жоспарлаудағы мұндай кемшіліктер
аграрлық сектордың нақты қажеттіліктерін нақты ескеретін және оның
тұрақты дамуына ықпал ететін тиімді стратегияларды әзірлеуге кедергі
келтіреді. Нақты өзгерістерге қол жеткізу үшін жаңа мақсаттар қою ғана
емес, сонымен қатар қателіктердің қайталануын болдырмау үшін алдыңғы
әрекеттерді мұқият талдау, жаңа шараларды үнемі бақылау мен бағалау
тетіктерін ендіру қажет. Мемлекетіміз климаттың өзгеруіне қарсы күрес
жөніндегі міндеттемелерді орындаудың маңызды бөлігі ретінде ауыл
шаруашылығында "жасыл" технологияларды енгізу шараларын мемлекеттік
бағдарламалар негізінде жүзеге асыру мен бақылауға алуы тиісті. Бұл
еңбек өнімділігін арттыруға, халықтың өмір сүру деңгейін жақсартуға,
сапалы өнімдірді өндіруге және экономика секторлары арасындағы
алшақтықты қысқартуға ықпал ететін ауыл шаруашылығының серпінді және
тұрақты моделін құруға мүмкіндік береді.

2. Тікелей субсидиялауға бағытталған мемлекеттік қолдаудың ағымдағы
нысаны елеулі қайта қарауды талап етеді. Аграрлық секторда кәсіпкерлік
белсенділіктің төмендігі мәселесі бар: ауыл шаруашылығы өндірушілерінің
басым бөлігі мемлекеттік көмекке тәуелді, қандай жағдайда да мемлекеттік
көмек алуға бейімделген. Сол себепті субсидияларды беру кезінде олардың
мақсатты пайдаланылуына жауапкершілікті арттыру қажет. Әрбір субсидия
алушыға қаржылық көмекті тиімді әрі дұрыс пайдалану үшін нақты міндеттер
мен бақылау механизмдері енгізілуі тиіс. Осыған орай, субсидиялардың тек
аграрлық сектордағы кәсіпкерлік белсенділікті ынталандыруға,
инновациялық жобаларды қолдауға және еңбек өнімділігін арттыруға
жұмсалуы қамтамасыз етілуі тиіс. Мұндай шаралар мемлекеттік қолдаудың
тиімділігін арттырып, ресурстардың дұрыс әрі мақсатты пайдаланылуын
қамтамасыз етеді.

3. Қаржы ресурстары мен мемлекеттік бағдарламалардың болуына қарамастан,
ауыл шаруашылығына инвестициялар еңбек өнімділігінің күтілетін өсуіне
әкелген жоқ, оның деңгейі дамыған елдермен салыстырмағанда, еліміздегі
өңдеу, тау-кен өндірісінен 5-12 есе төмен. Бұл қазіргі заманғы
технологияларды жаңғырту мен қолданудың жеткіліксіздігіне де тікелей
байланысты, бұл ауыл шаруашылығы техникасының 80\%-ға дейін жететін
тозуының жоғары деңгейінен көрінеді. Аграрлық секторда өнімділікті
арттыру үшін инновациялық технологияларды енгізу және өндірістік
процестерді оңтайландыру өте маңызды. Сол себепті экономиканы
әртараптандырып қана қоймай, оның сыртқы күйзелістерге төзімділігін
арттыратын өңдеу өнеркәсібі және ауыл шаруашылығы сияқты шикізаттық емес
секторларды дамытуға назар аудара отырып, мақсатты инвестициялық
стратегияны әзірлеп, іске асыруы қажет.

4. Нақты осы уақытта аграрлық сектор мен ауыл шаруашылығы өнімдерін
қайта өңдеу арасындағы байланысты реттеу маңызды аспект болып табылады.
Мемлекеттік субсидиялар шикізат өндірушілерді тікелей қолдауға емес,
қайта өңдеуді, инфрақұрылымды және адами капиталды дамытуға қатысты
жобаларға бағытталуы керек. Осыған дейінгі басқа да зерттеулер
көрсеткендей, ғылыми зерттеулерге, су ресурстарын басқаруға және
инфрақұрылымға инвестициялар жоғары экономикалық қайтарымға ие және ауыл
шаруашылығының жалпы жағдайын жақсартуға ықпал етеді, бұл ауылдық
жерлердің тұрақты дамуын қамтамасыз етеді.

5. Еңбек ресурстарының жалпы саны «жеткілікті» болып көрінгенмен еңбек
ресурстарының құрамының қартаюы орын алған және объективті себептерге
байланысты аграрлық секторда жұмыс істеуге дайын жас мамандардың
жетіспеушілігі бар. Бұл құбылыс тұрақты табыс көзі мен әлеуметтік
мәртебенің болмауы, сондай-ақ мансаптық өсудің шектеулі мүмкіндіктері,
аграрлық саладағы жұмысты қызықты әрі перспективалы ете алатын заманауи
технологиялардың болмауына орай -- ауыл шаруашылығына кәсіп пен өмір
салты ретінде қызығушылықтың төмендеуіне байланысты жастардың көбінесе
ата-аналарының фермерлік ісін жалғастырмауына, жас мамандардың жоғары
ақы төленетін және әлеуметтік тартымды салаларды артық көріп, аграрлық
салада жұмыс істеуден бас тартуына әкеліп отыр. Осы тұста ауыл
шаруашылығы жұмыскерлерінің еңбек шарттары мен біліктілігін арттыруға
инвестициялау экономиканың әртүрлі секторлары арасындағы еңбек
өнімділігі мен жалақы алшақтықтарын жоюдың негізгі факторы болып
табылады. Бұл жұмыс істеп жүрген кадрлардың дағдыларын жетілдіріп қана
қоймай, жастарды аграрлық секторға тартуға мүмкіндік береді. Кәсіптік
оқыту және қайта даярлау бағдарламалары қазіргі заманғы аграрлық
сектордың сын-қатерлерімен күресуге қабілетті білікті мамандарды
даярлауды қамтамасыз ету үшін қазіргі заманғы талаптар мен
технологияларға бейімделуі тиіс.

Зерттеу нәтижелері Қазақстанның ауыл шаруашылығындағы инвестициялық
ахуалды жақсартудың аса маңызды қажеттілігін көрсетеді. Еңбек
өнімділігінің тұрақты өсуіне қол жеткізу үшін инвестициялар көлемін
ұлғайту ғана емес, сонымен қатар заманауи технологияларды енгізу,
фермерлік шаруашылықтарды қолдау және тиісті инфрақұрылымды дамыту
қажет. Зерттеу барысында әзірленген ұсыныстар мемлекеттік саясат
деңгейінде де, аграрлық саладағы мамандар үшін де пайдалануға негіз бола
алады.

Осы бағыттағы болашақ зерттеулерде "жасыл" технологиялар мен тұрақты
шаруашылық жүргізу әдістерін енгізуге, сондай-ақ фермерлердің білім
беру, біліктілік және қаржыландырудың қолжетімділігі сияқты
инновацияларға инвестициялауға дайындығына әсер ететін факторларды
зерттеуге ерекше назар аудару қажет. Бұл зерттеулер инвестицияларды
пайдалану тиімділігін арттыру және Қазақстанның ауыл шаруашылығын
тұрақты дамыту үшін неғұрлым дәл ұсынымдарды әзірлеуге көмектеседі.

\emph{{\bfseries Қаржыландыру.}} \emph{Мақала Қазақстан Республикасы Ауыл
шаруашылығы министрлігінің «Агроөнеркәсіптік кешенді тұрақты дамыту.
Ауылдық аумақтарды тұрақты дамыту» бағыты бойынша 2024-2026 жылдарға
арналған бағдарламалық-нысаналы қаржыландырудың «Аграрлық өндірістің
ресурстық әлеуетін пайдалану тиімділігін арттырудың
ұйымдастырушылық-экономикалық шараларын әзірлеу» бағдарламасы
(BR22886885) шеңберінде дайындалған.}
\end{multicols}

\begin{center}
{\bfseries Әдебиеттер}
\end{center}

\begin{references}
1. «Әділетті Қазақстан: заң мен тәртіп, экономикалық өсім, қоғамдық
оптимизм» Мемлекет басшысы Қасым-Жомарт Тоқаевтың 2024 жылғы 2
қыркүйектегі Қазақстан халқына Жолдауы. {[}Электрондық ресурс{]}.-URL:
\href{https://kaz.zakon.kz/kogam-tynysy/6048484\%20-\%2018.12.2024}{https://kaz.zakon.kz/kogam-tynysy/6048484
- Жүгінген күні 18.12.2024}

2. Аймурзина Б.Т., Каменова М.Ж., Бектенова Д.Ч.
Инновационно-инвестиционная деятельность в растениеводстве Казахстана на
примере зерновой отрасли: современные подходы // Проблемы агрорынка. -
2024.-№ 2.- С. 89-99. DOI
\href{https://doi.org/10.46666/2024-2.2708-9991.07}{10.46666/2024-2.2708-9991.07}

3. Will Martin. Economic growth, convergence, and agricultural economics
// Agricultural Economics. - 2019. - № 50. - Р. 7 - 27. DOI
10.1111/agec.12528.

4. Robert Finger,~Carola Grebitus,~Arne Henningsen. ~Replications in
agricultural economics //Applied Economic Perspectives and Policy.-
2023. - № 45(3).- Р.1258 - 1274.

DOI \href{https://doi.org/10.1002/aepp.13386}{10.1002/aepp.13386}

5. Mottaleb Khondoker. Perception and adoption of a new agricultural
technology: Evidence from a developing country // Technology in Society.
- 2018. -№ 55.- Р. 126 - 135.

DOI
\href{http://dx.doi.org/10.1016/j.techsoc.2018.07.007}{10.1016/j.techsoc.2018.07.007}

6. Калыкова Б.Б., Мадиев Г., Бекбосынова А.Б. Макроэкономические
регуляторы функционирования агропромышленного комплекса Республики
Казахстан: ключевые направления // Проблемы агрорынка. - 2024. - № 2. -
C.40 - 52. ~ DOI
\href{https://doi.org/10.46666/2024-2.2708-9991.03}{10.46666/2024-2.2708-9991.03}

7. Карымсакова Ж.К., Керимова У.К., Deliana Y. Инновационное развитие
АПК: проблемы и стратегия их решения // ~Проблемы агрорынка. - 2024.- №
2.- С. 14-24. \href{https://doi.org/10.46666/2024-2.2708-9991.01}{DOI
10.46666/2024-2.2708-9991.01}

8.Кучукова Н.К., Рамазанова Ш.Ш. Финансовое обеспечение
агропромышленного комплекса в условиях модернизации экономики
Казахстана. Монография / Евразийский национальный университет им.
Л.Н.Гумилева. - Астана. - 2023.-192 с.

9.Tokbergenova A, Kiyassova L, Kairova S. Sustainable Development
Agriculture in the Republic of Kazakhstan // Polish Journal of
Environmental Studies.- 2018.- № 27(5).- P.1923-1933.
DOI10.15244/pjoes/78617.

10.Sustainable Development Report {[}Electronic resource{]}. -

URL: \url{https://dashboards.sdgindex.org} - Date of access: 08.08.2024.

11. Bo Peng, Rasa Melnikiene, Tomas Balezentis, Giulio Paolo Agnusdei.
\href{https://ideas.repec.org/a/spr/agfoec/v12y2024i1d10.1186_s40100-024-00321-x.html}{Structural
dynamics and sustainability in the agricultural sector: the case of the
European Union} //
\href{https://ideas.repec.org/s/spr/agfoec.html}{Agricultural and Food
Economics}. -2024. - № 12(1).- P. 1-27. DOI 10.1186/s40100-024-00321-x

12. Lajos Baráth, Imre Fertő.
\href{https://ideas.repec.org/a/bla/jageco/v75y2024i1p404-424.html}{The
relationship between the ecologisation of farms and total factor
productivity: A continuous treatment analysis} //
\href{https://ideas.repec.org/s/bla/jageco.html}{Journal of Agricultural
Economics}. --2023.-- № 75(1).- P. 404-424. DOI
\href{http://dx.doi.org/10.1111/1477-9552.12563}{10.1111/1477-9552.12563}

13. Қазақстан Республикасының Стратегиялық жоспарлау және реформалар
агенттігі Ұлттық статистика бюросының ресми статистикалық деректері.
https://stat.gov.kz.

14{\bfseries .} World Bank. The role of investments in agriculture in
enhancing productivity\emph{.} World Development Report 2021\emph{.}
{[}Electronic resource{]}. -

URL: \url{https://www.worldbank.org/en/publication/wdr2021} - Date of
access: 10.10.2024.

15. «Қазақстан Республикасының агроөнеркәсіптік кешенін дамытудың 2017 -
2021 жылдарға арналған мемлекеттік бағдарламасын бекіту туралы»
Қазақстан Республикасы Үкіметінің 2018 жылғы 12 шілдедегі № 423 қаулысы.
{[}Электрондық ресурс{]}.- URL:
\url{https://adilet.zan.kz/rus/docs/P1800000423} - Kіру күні:
10.09.2024.

16. «Қазақстан Республикасының агроөнеркәсіптік кешенін дамыту жөніндегі
2021 - 2025 жылдарға арналған ұлттық жобаны бекіту туралы» Қазақстан
Республикасы Үкіметінің 2021 жылғы 12 қазандағы № 732 қаулысы.
{[}Электрондық ресурс{]}.--URL:
https://adilet.zan.kz/kaz/docs/P2100000732 - Kіру күні: 11.09.2024.

17. Краткое заключение к отчету правительства Республики Казахстан об
исполнении республиканского бюджета за 2022 год. {[}Электронный
ресурс{]}. - URL:
\url{https://www.gov.kz/uploads/2023/5/18/f1f30399b1cf29f090a2a7a39} -
Дата обращения:

11.09.2024.

18. «Қазақстан Республикасының 2029 жылға дейінгі ұлттық даму жоспарын
бекіту туралы» ҚР Президентінің 2024 жылғы 30 шілдедегі\\
№ 611 Жарлығы. {[}Электрондық ресурс{]}.--URL:
https://adilet.zan.kz/kaz/docs/ - Kіру күні: 11.09.2024.

19. «Қазақстан Республикасының агроөнеркәсіптік кешенін дамытудың
2021-2030 жылдарға арналған тұжырымдамасын бекіту туралы» Қазақстан
Республикасы Үкіметінің 2021 жылғы 30 желтоқсандағы № 960 қаулысы.
{[}Электрондық ресурс - URL: https://adilet.zan.kz/kaz/docs/P2100000960.
- Kіру күні: 10.10.2024.

20. Калдаров С, Темирханов М. Обзор развития сельского хозяйства в
Казахстане. Аналитический центр Halyk Finance. 2023. Обзор.
{[}Электронный ресурс{]}.- URL:
\url{https://halykfinance.kz/download/files/analytics/AC_agriculture_development.pdf}
- Дата обращения: 02.09.2024.

21. Ахмедьярова А. Производительность труда в Казахстане: доминирование
горнодобывающего сектора и сырьевых продуктов в промышленности
{[}Электронный ресурс{]}. -- URL:
https://halykfinance.kz/download/files/analytics/ac\_labor2.pdf - Дата
обращения: 04.09.2024.

22. Всемирный банк. Формирование завтрашнего дня: реформы для
долгосрочного процветания. Доклад об экономике Казахстана. 2024.
{[}Электронный ресурс{]}. -- URL:
https://documents1.worldbank.org/curated/en.pdf - Дата обращения:
04.09.2024.
\end{references}

\begin{center}
{\bfseries References}
\end{center}

\begin{references}
1. «Әdіlettі Қazaқstan: zaң men tәrtіp, jekonomikalyқ өsіm, қoғamdyқ
optimizm» Memleket basshysy Қasym-Zhomart Toқaevtyң 2024 zhylғy 2
қyrkүjektegі Қazaқstan halқyna Zholdauy. {[}Jelektrondyқ resurs{]}.-URL:
https://kaz.zakon.kz/kogam-tynysy/6048484 - Zhүgіngen kүnі 18.12.2024
\href{https://kaz.zakon.kz/kogam-tynysy/6048484.\%20\%20\%20\%5bin}{{[}in}
Kazakh.{]}

2. Ajmurzina B.T., Kamenova M.Zh., Bektenova D.Ch.
Innovacionno-investicionnaja
dejatel' nost'{} v rastenievodstve
Kazahstana na primere zernovoj otrasli: sovremennye podhody // Problemy
agrorynka. - 2024.-№ 2.- S. 89-99. DOI
10.46666/2024-2.2708-9991.07.{[}in Russian{]}

3. Will Martin. Economic growth, convergence, and agricultural economics
// Agricultural Economics. - 2019. - № 50. - Р. 7 - 27. DOI
10.1111/agec.12528.

4.Robert Finger,~Carola Grebitus,~Arne Henningsen. ~Replications in
agricultural economics //Applied Economic Perspectives and Policy.-
2023. - № 45(3).- Р.1258 - 1274.

DOI \href{https://doi.org/10.1002/aepp.13386}{10.1002/aepp.13386}

5. Mottaleb Khondoker. Perception and adoption of a new agricultural
technology: Evidence from a developing country // Technology in Society.
- 2018. -№ 55.- Р. 126 - 135.

DOI
\href{http://dx.doi.org/10.1016/j.techsoc.2018.07.007}{10.1016/j.techsoc.2018.07.007}

6. Kalykova B.B., Madiev G., Bekbosynova A.B. Makrojekonomicheskie
reguljatory funkcionirovanija agropromyshlennogo kompleksa Respubliki
Kazahstan: kljuchevye napravlenija // Problemy agrorynka. - 2024. - № 2.
- C.40 - 52. DOI 10.46666/2024-2.2708-9991.03.{[}in Russian{]}

7. Karymsakova Zh.K., Kerimova U.K., Deliana Y. Innovacionnoe razvitie
APK: problemy i strategija ih reshenija // Problemy agrorynka. - 2024.-
№ 2.- S. 14-24. DOI 10.46666/2024-2.2708-9991.01.{[}in Russian{]}

8.Kuchukova N.K., Ramazanova Sh.Sh. Finansovoe obespechenie
agropromyshlennogo kompleksa v uslovijah modernizacii jekonomiki
Kazahstana. Monografija / Evrazijskij nacional' nyj
universitet im. L.N.Gumileva. - Astana. - 2023.-192 s. {[}in Russian{]}

9.Tokbergenova A, Kiyassova L, Kairova S. Sustainable Development
Agriculture in the Republic of Kazakhstan // Polish Journal of
Environmental Studies.- 2018.- № 27(5).- P.1923-1933.
DOI10.15244/pjoes/78617.

10.Sustainable Development Report {[}Electronic resource{]}. -

URL: \url{https://dashboards.sdgindex.org} - Date of access: 08.08.2024.

11. Bo Peng, Rasa Melnikiene, Tomas Balezentis, Giulio Paolo Agnusdei.
\href{https://ideas.repec.org/a/spr/agfoec/v12y2024i1d10.1186_s40100-024-00321-x.html}{Structural
dynamics and sustainability in the agricultural sector: the case of the
European Union} //
\href{https://ideas.repec.org/s/spr/agfoec.html}{Agricultural and Food
Economics}. -2024. - № 12(1).- P. 1-27. DOI 10.1186/s40100-024-00321-x

12. Lajos Baráth, Imre Fertő.
\href{https://ideas.repec.org/a/bla/jageco/v75y2024i1p404-424.html}{The
relationship between the ecologisation of farms and total factor
productivity: A continuous treatment analysis} //
\href{https://ideas.repec.org/s/bla/jageco.html}{Journal of Agricultural
Economics}. --2023.-- № 75(1).- P. 404-424. DOI
\href{http://dx.doi.org/10.1111/1477-9552.12563}{10.1111/1477-9552.12563}

13. Қazaқstan Respublikasynyң Strategijalyқ zhosparlau zhәne reformalar
agenttіgі Ұlttyқ statistika bjurosynyң resmi statistikalyқ derekterі.
https://stat.gov.kz.
\href{https://kaz.zakon.kz/kogam-tynysy/6048484.\%20\%20\%20\%5bin}{{[}in}
Kazakh.{]}

14. World Bank. The role of investments in agriculture in enhancing
productivity. World Development Report 2021. {[}Electronic resource{]}.
\href{https://kaz.zakon.kz/kogam-tynysy/6048484.\%20\%20\%20\%5bin}{{[}in}
Kazakh.{]}

URL: https://www.worldbank.org/en/publication/wdr2021 - Date of access:
10.10.2024.

15. «Қazaқstan Respublikasynyң agroөnerkәsіptіk keshenіn damytudyң 2017
- 2021 zhyldarғa arnalғan memlekettіk baғdarlamasyn bekіtu turaly»
Қazaқstan Respublikasy Үkіmetіnің 2018 zhylғy 12 shіldedegі № 423
қaulysy. {[}Jelektrondyқ resurs{]}.- URL:
https://adilet.zan.kz/rus/docs/P1800000423 - Kіru kүnі: 10.09.2024.
\href{https://kaz.zakon.kz/kogam-tynysy/6048484.\%20\%20\%20\%5bin}{{[}in}
Kazakh.{]}

16. «Қazaқstan Respublikasynyң agroөnerkәsіptіk keshenіn damytu
zhөnіndegі 2021 - 2025 zhyldarғa arnalғan ұlttyқ zhobany bekіtu turaly»
Қazaқstan Respublikasy Үkіmetіnің 2021 zhylғy 12 қazandaғy № 732
қaulysy. {[}Jelektrondyқ resurs{]}.--URL:
https://adilet.zan.kz/kaz/docs/P2100000732 - Kіru kүnі: 11.09.2024.
\href{https://kaz.zakon.kz/kogam-tynysy/6048484.\%20\%20\%20\%5bin}{{[}in}
Kazakh.{]}

17. Kratkoe zakljuchenie k otchetu pravitel' stva
Respubliki Kazahstan ob ispolnenii respublikanskogo bjudzheta za 2022
god. {[}Jelektronnyj resurs{]}. - URL:
https://www.gov.kz/uploads/2023/5/18/f1f30399b1cf29f090a2a7a39 - Data
obrashhenija:

11.09.2024. {[}in Russian{]}

18. «Қazaқstan Respublikasynyң 2029 zhylғa dejіngі ұlttyқ damu zhosparyn
bekіtu turaly» ҚR Prezidentіnің 2024 zhylғy 30 shіldedegі№ 611 Zharlyғy.
{[}Jelektrondyқ resurs{]}.--URL: https://adilet.zan.kz/kaz/docs/ - Kіru
kүnі: 11.09.2024.
\href{https://kaz.zakon.kz/kogam-tynysy/6048484.\%20\%20\%20\%5bin}{{[}in}
Kazakh.{]}

19. «Қazaқstan Respublikasynyң agroөnerkәsіptіk keshenіn damytudyң
2021-2030 zhyldarғa arnalғan tұzhyrymdamasyn bekіtu turaly» Қazaқstan
Respublikasy Үkіmetіnің 2021 zhylғy 30 zheltoқsandaғy № 960 қaulysy.
{[}Jelektrondyқ resurs - URL:
https://adilet.zan.kz/kaz/docs/P2100000960. - Kіru kүnі: 10.10.2024.
\href{https://kaz.zakon.kz/kogam-tynysy/6048484.\%20\%20\%20\%5bin}{{[}in}
Kazakh.{]}

20. Kaldarov S, Temirhanov M. Obzor razvitija sel' skogo
hozjajstva v Kazahstane. Analiticheskij centr Halyk Finance. 2023.
Obzor. {[}Jelektronnyj resurs{]}.- URL:
https://halykfinance.kz/download/files/analytics/AC\_agriculture\_development.pdf
- Data obrashhenija: 02.09.2024. {[}in Russian{]}

21. Ahmed' jarova A.
Proizvoditel' nost'{} truda v Kazahstane:
dominirovanie gornodobyvajushhego sektora i syr' evyh
produktov v promyshlennosti {[}Jelektronnyj resurs{]}. -- URL:
https://halykfinance.kz/download/files/analytics/ac\_labor2.pdf - Data
obrashhenija: 04.09.2024. {[}in Russian{]}

22. Vsemirnyj bank. Formirovanie zavtrashnego dnja: reformy dlja
dolgosrochnogo procvetanija. Doklad ob jekonomike Kazahstana. 2024.
{[}Jelektronnyj resurs{]}.- URL:
https://documents1.worldbank.org/curated/en.pdf - Data obrashhenija:
04.09.2024. {[}in Russian{]}
\end{references}

\begin{authorinfo}
\emph{{\bfseries Авторлар туралы мәліметтер}}

Байдаков А.К.- экономика ғылымдарының кандидаты, «Есеп және қаржы»
кафедрасының қауымдастырылған профессоры, С.Сейфуллин атындағы Қазақ
агротехникалық зерттеу университеті, Астана, Қазақстан, e-mail:
a\_baidakov@mail.ru;

Кучукова Н.К.- экономика ғылымдарының докторы, «Қаржы» кафедрасының
профессоры, Л.Н. Гумилев атындағы Еуразия ұлттық университеті, Астана,
Қазақстан, e-mail: nkuchukova@mail.ru;

Беспаева Р.С.- PhD докторы, «Менеджмент және маркетинг» кафедрасының
қауымдастырылған профессоры, С.Сейфуллин атындағы Қазақ агротехникалық
зерттеу университеті, Астана, Қазақстан, e-mail: brs\_@mail.ru;

Булхаирова Ж.С.-PhD докторы, «Экономика» кафедрасының қауымдастырылған
профессоры, С.Сейфуллин атындағы Қазақ агротехникалық зерттеу
университеті, Астана, Қазақстан, e-mail: honeyzhu@mail.ru;

Жуматаева Б.А.- PhD докторы, «Есеп және қаржы» кафедрасының меңгерушісі,
Қ.Құлажанов атындағы қазақ Технология және бизнес университеті, Астана,
Қазақстан, e-mail: bahyt\_jumataeva@mail.ru.

\emph{{\bfseries Information about the authors}}

Baidakov А.К.- candidate of economic sciences, Associate Professor of
the Department of Accounting and Finance, S.Seifullin Kazakh Agro
Technical Research University, Astana, Kazakhstan; e-mail:
a\_baidakov@mail.ru;

Kuchukova N.K.- doctor of economic sciences, Professor of the Department
of Finance, L.N. Gumilyov Eurasian National University, Astana,
Kazakhstan, e-mail: nkuchukova@mail.ru;

Bespaeva R.S.- PhD, Associate Professor of the Department of Management
and Marketing, S. Seifullin Kazakh Agro Technical Research University,
Astana, Kazakhstan, e-mail:brs\_@mail.ru;

Bulkhairova Zh.S. -PhD, Associate Professor of the Department of
Economics, S. Seifullin Kazakh Agro Technical Research University,
Astana, Kazakhstan, e-mail: honeyzhu@mail.ru;

Zhumataeva B.A.- PhD, head of the Department of Accounting and finance,
K.Kulazhanov Kazakh University of Technology and Business, Astana,
Kazakhstan, e-mail: bahyt\_jumataeva@mail.ru.
\end{authorinfo}
