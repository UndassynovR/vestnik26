\id{МРНТИ 82.33.13}{https://doi.org/10.58805/kazutb.v.1.26-750}

\begin{articleheader}
\sectionwithauthors{А.Н. Рыстамбаева, М.Р. Сихимбаев}{ЦИФРОВИЗАЦИЯ БИЗНЕС-ПРОЦЕССОВ ГОСУДАРСТВЕННОГО УПРАВЛЕНИЯ В КАЗАХСТАНЕ С УЧЕТОМ МЕЖДУНАРОДНОГО ОПЫТА}

{\bfseries  
А.Н. Рыстамбаева\textsuperscript{\envelope } \authorid,
М.Р. Сихимбаев\authorid}
\end{articleheader}

\begin{affiliation}
\emph{Карагандинский университет Казпотребсоюза, Караганда, Казахстан,}

\raggedright \textsuperscript{\envelope }{\em Корреспондент-автор: \href{mailto:aiga_das@mail.ru}{\nolinkurl{aiga\_das@mail.ru}}}
\end{affiliation}

Данная статья посвящена исследованию процессов цифровизации
бизнес-процессов органов госуправления в Республике Казахстан с акцентом
на адаптацию международного опыта. В работе подчеркивается важность
внедрения передовых технологий и цифровых решений для повышения
эффективности предоставления госуслуг, прозрачности и взаимодействия
государства с гражданами. Приведен анализ ключевых показателей индекса
развития электронного правительства (EGDI), включающий показатели OSI,
TII и HCI, позволяющие оценить прогресс Казахстана в сравнении с другими
странами.

Особое внимание уделено изучению практик цифрового управления таких
стран, как Эстония, Южная Корея и Сингапур, способствуя формированию
рекомендаций по улучшению текущих подходов в Казахстане. Выявлены
основные достижения, такие как развитие электронного правительства
(eGov) и цифровой инфраструктуры, а также барьеры, включая цифровое
неравенство и недостаточную интеграцию решений в регионах.
Представленные результаты исследования подчеркивают важность
стратегического планирования и оптимизации бизнес-процессов для
достижения устойчивого цифрового управления в Казахстане.

Целью исследования является изучение цифровых бизнес-процессов органов
госуправления Казахстана с учетом международного опыта, выявления
ключевых достижений и барьеров, а также предложений по дальнейшему
развитию, уделив особое внимание анализу показателей индекса развития
электронного правительства (EGDI) и его компонентов, позволяя объективно
оценить прогресс Казахстана в сравнении с другими странами.

Результаты подчеркивают стратегическую важность цифровизации для
модернизации госуправления в Казахстане, создания прозрачной и
эффективной системы госуслуг и укрепления позиций страны на
международной арене.

Вывод. Казахстан демонстрирует значительные успехи в цифровизации
госуслуг, что подтверждается его высокими позициями в международных
рейтингах, таких как EGDI (Индекс развития электронного правительства).
В 2022 году Казахстан занял 28-е место в мире с индексом EGDI 0,8628,
что ставит его в категорию стран с очень высоким уровнем EGDI.

{\bfseries Ключевые слова:} цифровизация, бизнес-процесс, госуправление,
госуслуги, электронное правительство, интеграция.

\begin{articleheader}
{\bfseries ХАЛЫҚАРАЛЫҚ ТӘЖІРИБЕНІ ЕСКЕРЕ ОТЫРЫП, ҚАЗАҚСТАНДАҒЫ МЕМЛЕКЕТТІК БАСҚАРУДЫҢ БИЗНЕС-ПРОЦЕСТЕРІН ЦИФРЛАНДЫРУ}

{\bfseries  
А.Н. Рыстамбаева\textsuperscript{\envelope },  
М.Р. Сихимбаев}
\end{articleheader}

\begin{affiliation}
\emph{Қазтұтынуодағы Қарағанды университеті, Қарағанды, Қазақстан}

\emph{e-mail: \href{mailto:aiga_das@mail.ru}{\nolinkurl{aiga\_das@mail.ru}}}
\end{affiliation}

Бұл мақала халықаралық тәжірибені бейімдеуге баса назар аудара отырып,
Қазақстан Республикасындағы мемлекеттік басқару органдарының
бизнес-процестерін цифрландыру процестерін зерттеуге арналған. Жұмыста
Мемлекеттік қызметтер көрсетудің тиімділігін, мемлекеттің азаматтармен
ашықтығы мен өзара іс-қимылын арттыру үшін озық технологиялар мен
цифрлық шешімдерді енгізудің маңыздылығы атап көрсетілген. Қазақстанның
басқа елдермен салыстырғанда ілгерілеуін бағалауға мүмкіндік беретін
OSI, TII және HCI көрсеткіштерін қамтитын электрондық үкіметтің даму
индексінің (EGDI) негізгі көрсеткіштеріне талдау келтірілген.
Қазақстандағы ағымдағы тәсілдерді жақсарту бойынша ұсынымдарды
қалыптастыруға ықпал ете отырып, Эстония, Оңтүстік Корея және Сингапур
сияқты елдердің цифрлық басқару тәжірибелерін зерделеуге ерекше назар
аударылды. Электрондық үкіметтің (eGov) және цифрлық инфрақұрылымның
дамуы, сондай-ақ Цифрлық теңсіздікті және өңірлердегі шешімдердің
жеткіліксіз интеграциясын қоса алғанда, кедергілер сияқты негізгі
жетістіктер анықталды. Ұсынылған зерттеу нәтижелері Қазақстанда тұрақты
цифрлық басқаруға қол жеткізу үшін бизнес-процестерді Стратегиялық
жоспарлау мен оңтайландырудың маңыздылығын көрсетеді.

Зерттеудің мақсаты халықаралық тәжірибені ескере отырып, Қазақстанның
мемлекеттік басқару органдарының Цифрлық бизнес-процестерін зерделеу,
негізгі жетістіктер мен кедергілерді анықтау, сондай-ақ электрондық
үкіметтің даму индексінің (EGDI) көрсеткіштерін және оның компоненттерін
талдауға ерекше назар аудара отырып, Қазақстанның басқа елдермен
салыстырғанда ілгерілеуін объективті бағалауға мүмкіндік бере отырып,
одан әрі дамыту жөніндегі ұсыныстарды зерделеу болып табылады.

Нәтижелер Қазақстандағы мемлекеттік басқаруды жаңғырту, мемлекеттік
қызметтердің ашық және тиімді жүйесін құру және елдің халықаралық
аренадағы ұстанымын нығайту үшін цифрландырудың стратегиялық
маңыздылығын көрсетеді.

Қорытынды. Қазақстан мемлекеттік қызметті цифрландыруда айтарлықтай
табыстар көрсетіп отыр, бұл оның EGDI (электрондық үкіметтің даму
индексі) сияқты халықаралық рейтингтердегі жоғары позицияларымен
расталады. 2022 жылы Қазақстан EGDI индексі 0,8628 болатын әлемде 28-ші
орынға ие болды, бұл оны EGDI деңгейі өте жоғары елдер санатына
жатқызады.

{\bfseries Түйін сөздер:} цифрландыру, бизнес-процесс, мемлекеттік басқару,
мемлекеттік қызметтер, электрондық үкімет, интеграция.

\begin{articleheader}
{\bfseries DIGITALIZATION OF BUSINESS PROCESSES OF PUBLIC ADMINISTRATION IN KAZAKHSTAN, TAKING INTO ACCOUNT INTERNATIONAL EXPERIENCE}

{\bfseries  
A.N. Rystambayeva\textsuperscript{\envelope },  
M.R. Sikhimbayev}
\end{articleheader}

\begin{affiliation}
\emph{Karaganda University of Kazpotrebsoyuz, Karaganda, Kazakhstan,}

\emph{e-mail: \href{mailto:aiga_das@mail.ru}{\nolinkurl{aiga\_das@mail.ru}}}
\end{affiliation}

This article is devoted to the study of the processes of digitalization
of business processes of Public Administration bodies in the Republic of
Kazakhstan with an emphasis on adapting international experience. The
paper emphasizes the importance of introducing advanced technologies and
digital solutions to improve the efficiency of public services,
transparency and interaction of the state with citizens. An analysis of
the main indicators of the e-Government Development Index (EGDI), which
includes OSI, TII and HCI indicators, which allows us to assess the
progress of Kazakhstan in comparison with other countries, is given.
Special attention was paid to the study of digital management practices
in countries such as Estonia, South Korea and Singapore, contributing to
the formation of recommendations for improving current approaches in
Kazakhstan. Key achievements have been identified, such as the
development of e-government (eGov) and digital infrastructure, as well
as barriers, including digital inequality and insufficient integration
of solutions in the regions. The results of the presented study
demonstrate the importance of Strategic Planning and optimization of
business processes to achieve sustainable digital management in
Kazakhstan.

The purpose of the study is to study the digital business processes of
public administration in Kazakhstan, taking into account international
experience, identifying key achievements and barriers, as well as
proposals for further development, paying special attention to the
analysis of the e-government Development Index (EGDI) and its
components, allowing an objective assessment of
Kazakhstan' s progress in comparison with other
countries.

The results highlight the strategic importance of digitalization for
modernizing public administration in Kazakhstan, creating a transparent
and efficient system of public services, and strengthening the
country' s position in the international arena.

Conclusion. Kazakhstan has demonstrated significant success in
digitalizing public services, which is confirmed by its high positions
in international rankings such as EGDI (E-Government Development Index).
In 2022, Kazakhstan ranked 28th in the world with an EGDI index of
0.8628, which puts it in the category of countries with very high EGDI
levels.

{\bfseries Keywords:} digitalization, business process, Public
Administration, public services, e-government, integration.

\begin{multicols}{2}
{\bfseries Введение.} В условиях стремительного технологического прогресса
и глобальной цифровой трансформации госуправление сталкивается с
необходимостью адаптации к новым реалиям, где бизнес-процессы в органах
госуправления стали важным фактором повышения эффективности
предоставления услуг, прозрачности и взаимодействия государства с
гражданами.

Для Республики Казахстан цифровизация госсектора является одним из
ключевых стратегических задач, направленных на достижение устойчивого
экономического роста, улучшение качества госуслуг и повышение
конкурентоспособности страны на международной арене. С момента запуска
программы Цифровой Казахстан в 2018 году страна демонстрирует
значительный прогресс в реализации цифровых инициатив, результатом
которого стала интеграция электронного правительства (eGov),
направленного на упрощение доступа граждан к госуслугам. Однако,
несмотря на достигнутые успехи, Казахстан сталкивается с рядом вызовов,
включая низкий уровень цифровой грамотности населения, недостаточное
использование облачных технологий и ограниченную интеграцию современных
решений в регионах.

Введение международного опыта, таких как практики цифрового управления в
Эстонии, Южной Корее и Сингапуре, открывает возможности для адаптации
передовых решений, направленных на создание более эффективного и
устойчивого госуправления. Настоящее исследование акцентирует внимание
на важности интеграции инновационных технологий, оптимизации процессов и
взаимодействия между госорганами и обществом для достижения
стратегических целей цифровизации.

{\bfseries Материалы и методы.} В эпоху стремительного технологического
развития, цифровизация и формирование цифровой культуры приобретают
особую значимость с целью обеспечения благосостояния граждан и
социально-экономической стабильности государства {[}1{]}. Цифровая
экономика (Digital Economy) -- это экономика, основанная на
использовании цифровых компьютерных технологий. Термины, такие как
интернет-экономика, новая экономика и веб-экономика, также применяются к
данному явлению. Томас Месенбург выделяет три основные составляющие этой
концепции:

- поддерживающая инфраструктура, включающая в себя аппаратное и
программное обеспечение;

- телекоммуникации и сети;

- электронный бизнес, охватывающий любые процессы, проводимые
организацией через компьютерные сети, электронная коммерция (Frey, 2017)
{[}2{]}.

Хотя термин цифровая экономика не преобладает в научной литературе, его
чаще можно встретить в официальных дефинициях многих стран. Например, во
Франции существует Министерство цифровой экономики (Fung, 2016) {[}3{]}.
На данный момент категория электронная экономика включает в себя две
составляющие:

- интернет-экономику (среду для осуществления электронного бизнеса);

- цифровую экономику, где происходит производство, обмен, распределение
и потребление электронных товаров, а также проведение расчетов с
использованием электронных денег (Davis, 2014) {[}4{]}.

Полянин А.В. разделяет это мнение, утверждая, что создание площадок для
взаимодействия государства с представителями отраслей способствует
оперативности принятия решений по ключевым вопросам цифрового развития.
Цифровая грамотность, желание пробовать новые методы решения проблем,
риск, экспериментирование и создание ценных социальных связей и
бизнес-партнерств становятся неотъемлемыми атрибутами успеха граждан и
компаний {[}5{]}. В рамках концепции госпрограммы Цифровой Казахстан
подчеркивается, что цифровая экономика переживает быстрый рост,
стимулируя ускоренное внедрение инноваций и их широкое использование в
других секторах экономики {[}6{]}. Цифровые технологии проникают во все
сферы жизни, изменяя экономические и организационные процессы, а также
способы взаимодействия между поставщиками и потребителями товаров и
услуг. Суть и цель платформизации в цифровой трансформации могут быть
объяснены объективными предпосылками, стимулирующими активное создание,
выбор и использование цифровых платформ {[}7{]}. Практическое применение
цифровых технологий для предоставления госуслуг в разных странах мира
привело к формированию разных моделей электронного правительства (далее
-- ЭП), каждая из которых имеет свои отличия и на формирование которых
оказывают определенные факторы {[}8{]}.

Понятие электронного участия, по методологии исследования ООН,
выражается в возможности оказывать влияние на формирование и принятие
государственных решений, разработку госуслуг и нормативно-правовых
актов. Для реализации этого принципа в республике активно используется
проект Открытое правительство, включающий разделы: Открытые данные,
Открытый диалог, Открытые НПА (нормативно-правовые акты), Открытые
бюджеты и Оценка эффективности деятельности ГО (гражданского общества).

Процесс внедрения цифровых технологий для предоставления госуслуг
охватил все страны мира, приводя к созданию различных моделей ЭП.
Изначально все страны, приступившие к внедрению цифровых технологий,
стремились сократить использование бумажной документации, что привело к
созданию индивидуальных электронных систем документооборота для каждого
ведомства.

Сабденов, Р., Абдрахманова, Г., Жолдыбалина, А., \& Урпекова, А. (2024)
анализируют этапы внедрения электронного правительства в Казахстане,
выделяя ключевые проблемы, такие как недостаточная цифровая грамотность
и слабая инфраструктура, а также предложения по улучшению доступа к
цифровым услугам {[}9{]}. В статье Казиевой А.Н. (2022) рассматриваются
примеры успешной реализации цифровых решений в госуправлении, делая
акцент на взаимодействии между государством и гражданами через
онлайн-платформы с учетом зарубежного опыта {[}10{]}. Исследование Ғалы
Ә.Ғ., Бажаева Н.А. (2024) посвящено влиянию электронного правительства
на прозрачность управления. Особое внимание уделено проблемам факторам
влияния на развитие госуслуг и необходимости повышения доверия граждан
{[}11{]}. Мусина Г.С. (2023) подчеркивает значимость цифровизации для
устранения региональных диспропорций, внедрение цифровых технологий в
жизнь общества побуждая пересмотреть ход многих процессов, в том числе к
цифровому госуправлению {[}12{]}. По мнению Рахметулиной Ж.Б., Кариповой
А.Т., потребуется не просто модернизировать устаревшие системы,
разрабатывать архитектуры микросервисов, обновлять инструменты для
цифровизации своего бизнеса, но и создавать технологические комплексы
для развития, к примеру, нового бизнеса, продвижения инфраструктуры
{[}13{]}.

Проведенный обзор литературы и анализ источников, посвященных
цифровизации госуслуг в Казахстане, подтверждает, что данный процесс
является важным инструментом повышения эффективности госуправления,
способствуя упрощению взаимодействия граждан и госорганов, улучшению
прозрачности, снижению коррупционных рисков, а также оптимизации
административных процессов. Дальнейшая цифровизация госуслуг в
Казахстане, базирующаяся на стратегиях устойчивого развития и опыте
ведущих стран, является важным шагом на пути к повышению качества жизни
граждан, увеличению прозрачности госуправления и ускорению
социально-экономического развития {[}14{]}. Для изучения процессов
цифровизации госуслуг в Казахстане и их влияния на эффективность
госуправления использовались такие методы как:

- анализ документов и нормативно-правовой базы;

- сравнительный анализ;

- количественный анализ - проведен анализ данных индекса EGDI
(E-Government Development Index) и его компонентов (OSI, HCI, TII) за
период 2016-2024 годов. Использовались статистические методы, такие как
расчет динамики, процентных изменений и корреляционного анализа.
\end{multicols}

\begin{table}[H]
\caption*{Таблица 1 - Страны-лидеры по развитию электронного правительства, 2022 год}
\centering
\begin{tblr}{
  row{1} = {c},
  cell{2}{2} = {c},
  cell{2}{3} = {c},
  cell{2}{4} = {c},
  cell{3}{2} = {c},
  cell{3}{3} = {c},
  cell{3}{4} = {c},
  cell{4}{2} = {c},
  cell{4}{3} = {c},
  cell{4}{4} = {c},
  cell{5}{2} = {c},
  cell{5}{3} = {c},
  cell{5}{4} = {c},
  cell{6}{2} = {c},
  cell{6}{3} = {c},
  cell{6}{4} = {c},
  cell{7}{2} = {c},
  cell{7}{3} = {c},
  cell{7}{4} = {c},
  cell{8}{2} = {c},
  cell{8}{3} = {c},
  cell{8}{4} = {c},
  cell{9}{2} = {c},
  cell{9}{3} = {c},
  cell{9}{4} = {c},
  cell{10}{2} = {c},
  cell{10}{3} = {c},
  cell{10}{4} = {c},
  cell{11}{2} = {c},
  cell{11}{3} = {c},
  cell{11}{4} = {c},
  cell{12}{2} = {c},
  cell{12}{3} = {c},
  cell{12}{4} = {c},
  cell{13}{2} = {c},
  cell{13}{3} = {c},
  cell{13}{4} = {c},
  cell{14}{2} = {c},
  cell{14}{3} = {c},
  cell{14}{4} = {c},
  cell{15}{2} = {c},
  cell{15}{3} = {c},
  cell{15}{4} = {c},
  cell{16}{2} = {c},
  cell{16}{3} = {c},
  cell{16}{4} = {c},
  cell{17}{1} = {c=4}{},
  hlines,
  vlines,
}
Страна                                                 & OSI    & HCI    & TII    \\
Дания                                                  & 0,9797 & 0,9559 & 0,9795 \\
Финляндия                                              & 0,9833 & 0,964  & 0,9127 \\
Республика Корея                                       & 0,9826 & 0,9087 & 0,9674 \\
Новая Зеландия                                         & 0,9579 & 0,9823 & 0,8896 \\
Швеция                                                 & 0,9002 & 0,9649 & 0,958  \\
Исландия                                               & 0,8867 & 0,9657 & 0,9705 \\
Австралия                                              & 0,938  & 1      & 0,8836 \\
Эстония                                                & 1      & 0,9231 & 0,8949 \\
Нидерланды                                             & 0,9026 & 0,9506 & 0,962  \\
США                                                    & 0,9304 & 0,9276 & 0,8874 \\
Великобритании и Северная Ирландия                     & 0,8859 & 0,9369 & 0,9186 \\
Сингапур                                               & 0,962  & 0,9021 & 0,8758 \\
ОАЭ                                                    & 0,9014 & 0,8711 & 0,9306 \\
Япония                                                 & 0,9094 & 0,8765 & 0,9147 \\
Мальта                                                 & 0,8849 & 0,8734 & 0,9245 \\
\textit{Примечание. Составлео на основе источника [1]} &        &        &        
\end{tblr}
\end{table}

\begin{multicols}{2}
{\bfseries Результаты и обсуждение.} Каждая страна должна определять
уровень и степень выполнения задач цифрового правительства, исходя из
конкретных условий национального развития, потенциала, стратегии и
программ, а не из произвольного предположения о своем будущем положении
в рейтинге. Для анализа этой взаимосвязи используются индикаторы, такие
как Online Service Index (OSI), Human Capital Index (HCI) и
Telecommunication Infrastructure Index (TII), которые дают
количественную оценку зрелости цифровизации:

- EGDI - это инструмент сравнительного анализа для развития ЭП,
используемый в качестве косвенного показателя эффективности. Так,
например, своим прогрессом в области ЭП представлены следующие страны,
относящиеся к высшему (VH) рейтинговому классу группы с очень высоким
EGDI, где:

1. OSI - Индекс онлайн-услуг отражает уровень развития электронных услуг,
предоставляемых государством. Высокое значение OSI говорит о
доступности государственных цифровых сервисов, являясь ключевым
элементом цифровизации госуправления;

2. TII - Индекс телекоммуникационной инфраструктуры измеряет уровень
развития телекоммуникационной инфраструктуры (интернет, мобильные сети
и т. д.), которая является основой для успешной цифровизации;

3. НСI - Индекс человеческого капитала указывает на уровень человеческого
капитала в стране, включая грамотность и доступность технологий для
граждан (таблица 1) {[}1{]}.

Цифровизация требует квалифицированных специалистов и готовности
населения использовать цифровые инструменты (рисунок 1).
\end{multicols}


{\bfseries Рис. 1 - Применение госуслуг к цифровизации}

\emph{Примечание. Составлено автором}

\begin{multicols}{2}
Все 15 стран, относящихся к рейтинговому классу и указанные на рисунке 1
имеют национальную стратегию развития, включающую цели ЦУР. Эти страны
имеют национальную политику или стратегию, направленную на обеспечение
цифровой инклюзивности. Взаимосвязь между представленными данными и
процессом цифровизации проявляется через оценку уровня развития цифровых
технологий и их внедрения в различных странах.

Цифровизация требует сбалансированного развития всех трех аспектов.
Например, даже при высоком уровне OSI (электронные услуги) их
доступность будет ограничена при низких значениях TII (инфраструктура).
Аналогично, низкий HCI может ограничивать использование технологий из-за
нехватки знаний и навыков.

Для математического анализа данных по показателям OSI (Online Service
Index), HCI (Human Capital Index) и TII (Telecommunications
Infrastructure Index) приведём средние значения, стандартные отклонения,
корреляции и выявим взаимосвязи между показателями для предоставленных
стран.

\emph{1. Расчет среднего значения и стандартного отклонения}

Средние значения (Mean) и стандартные отклонения (SD) по каждому
показателю рассчитываются по формулам:

- среднее значение:

\begin{equation}
    X = \frac{\sum Xi}{n},
\end{equation}

- стандартное отклонение:

\begin{equation}
    \sigma = \frac{\sqrt{\sum (Xi - X)^2}}{n},
\end{equation}

Результаты расчетов представлены в таблице 2.
\end{multicols}

\begin{table}[H]
\caption*{Таблица 2 - Результаты расчетов}
\centering
\begin{tblr}{
  cells = {c},
  cell{5}{1} = {c=3}{},
  hlines,
  vlines,
}
Показатель                                                               & Среднее значение (Mean) & Стандартное отклонение (SD) \\
OSI                                                                      & 0,9317                  & 0,0414                      \\
HCI                                                                      & 0,9336                  & 0,0402                      \\
TII                                                                      & 0,9215                  & 0,0364                      \\
\textit{Примечание. Составлено автором на основе произведенных расчетов} &                         &                             
\end{tblr}
\end{table}

\begin{multicols}{2}
\emph{2. Корреляционный анализ}

Для анализа корреляции между компонентами используем формулу корреляции
Пирсона:

\begin{equation}
    r_{XY} = \frac{\sum (Xi - X)(Yi - Y)}{\sqrt{\sum (Xi - X)^2 \sum (Yi - Y)^2}},
\end{equation}

где:

хi, уi - значения сравниваемых индикаторов

xˉ,yˉ- средние значения индикаторов.

Результаты корреляционного анализа представлены в таблице 3.
\end{multicols}

\begin{table}[H]
\caption*{Таблица 3 - Результаты корреляционного анализа}
\centering
\begin{tblr}{
  cells = {c},
  cell{5}{1} = {c=2}{},
  hlines,
  vlines,
}
Пары показателей                                                         & Коэффициент корреляции (r) \\
OSI и HCI                                                                & 0,751                      \\
OSI и TII                                                                & 0,682                      \\
HCI и TII                                                                & 0,807                      \\
\textit{Примечание. Составлено автором на основе произведенных расчетов} &                            
\end{tblr}
\end{table}

\begin{multicols}{2}
Анализ взаимосвязей показал следующие результаты:

1) OSI и HCI (r = 0,751) - умеренно сильная положительная корреляция
указывает на то, что высокий уровень предоставления онлайн-услуг (OSI)
часто сопровождается высоким уровнем человеческого капитала (HCI);

2) OSI и TII (r = 0,682) - умеренная положительная корреляция между
уровнем онлайн-услуг и телекоммуникационной инфраструктурой;

3) HCI и TII (r = 0,807) - сильная положительная корреляция показывает,
что развитая телекоммуникационная инфраструктура способствует высокому
уровню человеческого капитала;

4) Наивысшие значения OSI достигаются у Эстонии (1), Финляндии (0,9833)
и Республики Корея (0,9826), демонстрируя высокую цифровизацию
онлайн-услуг;

5) HCI наиболее высок у Австралии (1), Новой Зеландии (0,9823) и
Финляндии (0,964), свидетельствуя о развитии человеческого капитала;

6) TII выше всего у Дании (0,9795), Исландии (0,9705) и Республики Корея
(0,9674), отражая сильную инфраструктурную поддержку.

На основании анализа данных можно предположить, что высокий уровень
телекоммуникационной инфраструктуры (TII) и человеческого капитала (HCI)
оказывает значительное влияние на качество предоставляемых онлайн-услуг
(OSI), что актуально для оценки уровня цифровизации в странах.

Данные OSI, HCI и TII служат важными инструментами для анализа прогресса
цифровизации в каждой стране, синергия которых отражает общую готовность
страны к цифровизации, влияя на доступность и качество госуслуг. Важно
учитывать эти показатели в комплексной оценке цифровой трансформации
госуправления и внедрения передовых технологий.

Первоначальные шаги в этом направлении были предприняты в США и Китае,
затем последовали Великобритания, страны Европейского союза, страны
Азиатско-Тихоокеанского региона, страны Содружества Независимых
государств и другие. Рассмотрим индекс развития электронного
правительства среди стран СНГ (таблица 4) {[}1{]}.
\end{multicols}

\begin{table}[H]
\caption*{Таблица 4 - Индекс развития электронного правительства (EGDI) среди стран СНГ в 2022 году}
\centering
\begin{tblr}{
  colspec = {X[2] X[1] X[1] X[1] X[1] X[1] X[1] X[1]},
  row{even} = {c},
  row{1} = {c},
  row{3} = {c},
  row{5} = {c},
  row{7} = {c},
  row{9} = {c},
  row{11} = {c},
  cell{13}{1} = {c=8}{},
  hlines,
  vlines,
}
Страна                                                          & Группа EGDI        & Место & EGDI 2022 & OSI 2022 & TII 2022 & HCI 2022 & Уровень дохода      \\
Азербайджан                                                     & Высокий EGDI       & 83    & 0,6937    & 0,6119   & 0,6761   & 0,7932   & доход выше среднего \\
Армения                                                         & Высокий EGDI       & 64    & 0,7364    & 0,7221   & 0,6925   & 0,7945   & доход выше среднего \\
Беларусь                                                        & Очень высокий EGDI & 58    & 0,758     & 0,5302   & 0,8426   & 0,9011   & доход выше среднего \\
Казахстан                                                       & Очень высокий EGDI & 28    & 0,8628    & 0,9344   & 0,752    & 0,9021   & доход выше среднего \\
Кыргызстан                                                      & Высокий EGDI       & 81    & 0,6977    & 0,6176   & 0,6637   & 0,8119   & доход ниже среднего \\
Молдова                                                         & Высокий EGDI       & 72    & 0,7251    & 0,738    & 0,576    & 0,8613   & доход выше среднего \\
Россия                                                          & Очень высокий EGD  & 42    & 0,8162    & 0,7368   & 0,8053   & 0,9065   & доход выше среднего \\
Туркменистан                                                    & Средний EGDI       & 137   & 0,4808    & 0,298    & 0,3551   & 0,7892   & доход выше среднего \\
Таджикистан                                                     & Высокий EGDI       & 129   & 0,5039    & 0,3968   & 0,377    & 0,738    & доход ниже среднего \\
Узбекистан                                                      & Высокий EGDI       & 69    & 0,7265    & 0,744    & 0,6575   & 0,7778   & доход ниже среднего \\
Среднее значение                                                &                    &       & 0,70061   & 0,632    & 0,639    & 0,827    &                     \\
\textit{Примечание. Составлено автором на основе источника [1]} &                    &       &           &          &          &          &                     
\end{tblr}
\end{table}

\begin{multicols}{2}

\begin{equation}
    \text{Среднее } EGDI = \frac{\sum EGDI{\text{ по странам}}}{\text{количество стран}}
\end{equation}

\begin{equation}
    \text{Среднее } OSI = \frac{\sum OSI{\text{ по странам}}}{\text{количество стран}}
\end{equation}

\begin{equation}
    \text{Среднее } TII = \frac{\sum TII{\text{ по странам}}}{\text{количество стран}}
\end{equation}

\begin{equation}
    \text{Среднее } HCI = \frac{\sum HCI{\text{ по странам}}}{\text{количество стран}}
\end{equation}

В соответствии с произведенными расчетами получили средние значения
компонентов:

- среднее OSI = 0,63578;

- среднее TII = 0,63293;

- среднее HCI = 0,82656;

- среднее EGDI = 0,70061

Казахстан (0,8628) значительно превышает среднее значение, подтверждая
его высокую степень цифровизации, способствуя улучшению качества
госуправления и его адаптации к современным вызовам.

{\bfseries Выводы.} Сравнение с мировым опытом, представленным в
литературе, цифровая трансформация указывает на необходимость
дальнейшего перенятия успешных практик, таких как интеграция мобильных
приложений, улучшение кибербезопасности и адаптация системы ЭП к новым
технологиям, где важную роль играет повышение уровня цифровой
грамотности населения, требуя комплексного подхода и господдержки. Такой
комплексный подход необходим для:

- разработки стратегии по сокращению цифрового разрыва между городом и
селом;

- расширения программ обучения цифровой грамотности, особенно среди
госслужащих;

- углубления интеграции международного опыта в области цифровизации, в
том числе в рамках сотрудничества с ЕС, Эстонией и Южной Кореей.

Такие результаты подчеркивают стратегическую важность цифровизации для
модернизации госуправления в Казахстане, создания прозрачной и
эффективной системы госуслуг и укрепления позиций страны на
международной арене.
\end{multicols}

\begin{center}
{\bfseries Литература}
\end{center}

\begin{references}
1. Доклад. Цифровая экономика: глобальные тренды и практика российского
бизнеса {[}Электронный ресурс{]} // Imi.hse.ru: Национальный
исследовательский университет Высшая школа экономики. URL:
\href{https://imi.hse.ru/pr2017_1}{https://imi.hse.ru} . Дата обращения: 25.10.24.

2. Frey C.B., Osborn M.A. The Future of Employment: How Susceptible Are
Jobs to Computerisation? // Technological Forecasting and Social
Change.-2017.-Vol.114.- P. 254-280
DOI \\10.1016/j.techfore.2016.08.019

3. Fung B.S.C., Halaburda H. Central Bank Digital Currencies: A Framework
for Assessing Why and How // Bank of Canada Staff Discussion
Paper.2016.-№.2016-22.
\href{url:https://www.bankofcanada.ca/wp-content/uploads/2016/11/sdp2016-22.pdf}{URL:https://www.bankofcanada.ca/wp-content/uploads/2016/11/sdp2016-22.pdf}. Date of circulation: 25.10.24

4.Davis S.J., Haltiwanger J. Labor Market Fluidity and Economic
Performance // NBER Working Paper Series. --Cambridge, MA: National
Bureau of Economic Research, 2014. - №. 20479. URL:\\
\href{https://www.nber.org/system/files/working_papers/w20479/w20479.pdf}{https://www.nber.org}.
Date of circulation: 25.10.24

5.Полянин А.В., Тенденции и прогнозы экономического роста для российской
экономики: региональный аспект. // Среднерусский вестник общественных
наук.-2017.- Т.12.(3).- С. 53-63. DOI \\10.22394/2071-2367-2017-12-3-53-63

6.Презентация концепции проекта государственной программы «Цифровой
Казахстан 2020» {[}Электронный ресурс{]}. URL:
\href{https://www.itk.kz/doc/images/Digital\_Kazakhstan.pdf}{https://www.itk.kz} Дата обращения:
25.10.24.

7.Грибанов Ю. И. Цифровая трансформация социально-экономических систем
на основе развития института сервисной интеграции: автореф. дис.
\ldots{} д-ра экон. наук: 08.00.05 / Грибанов Юрий Иванович.
--Санкт-Петербург, 2019.

8.Бурденко Е.В. Модели электронного правительства // Вопросы
инновационной экономики. -2023. -Т.13.(1). - С.59-76. DOI
10.18334/vinec.13.1.117234.

9.Сабденов, Р., Абдрахманова, Г., Жолдыбалина, А., Урпекова, А. Эволюция
электронного правительства в Казахстане: от формирования до цифровой
трансформации (2004-2022) // Казахстан-Спектр. -2024. - №111(3). DOI
10.52536/2415-8216.2024-3.07

10.Казиева А.Н. Цифровая трансформация как процесс изменения системы
государственного управления в Казахстане. // Государственный аудит. -
2022. -Т. 56.(3). -С. 47-57. DOI 10.55871/2072-9847-2022-56-3-47-57

11.Ғалы Ә.Ғ., Бажаева Н.А. Цифровизация государственных услуг в
контексте клиентоориентированного подхода: проблемы и перспективы
развития. // Государственный аудит. 2024. -Т. 64.(3). - С.155-162. DOI
\href{https://doi.org/10.55871/2072-9847-2024-64-3-155-162}{10.55871/2072-9847-2024-64-3-155-162}

12.Мусина Г.С. Цифровизация и региональная интеграция: сравнительный
анализ опыта ЕС и ЕАЭС. // Вестник Евразийского национального
университета имени Л.Н. Гумилева. -- 2023. - №2(143). -С.166-178. DOI
\href{https://doi.org/10.32523/2616-6887/2023-143-2-166-178}{10.32523/2616-6887/2023-143-2-166-178}

13.Рахметулина Ж.Б., Карипова А.Т. Перспективы сотрудничества между
Казахстаном и Китаем в процессе развития транспортного коридора Евразии
// Экономические отношения. -2019.-Т. 9(3). -C. 1615-1628. DOI
10.18334/eo.9.3.40816.

14.Исследование ООН: Электронное правительство 2022. Будущее цифрового
правительства/ Департамент по экономическим и социальным вопросам ООН. -
Нью-Йорк: ООН, 2022. -279 с. ISBN 978-92-1-12321-34.
\end{references}

\begin{center}
{\bfseries References}
\end{center}

\begin{references}
1.Doklad. Cifrovaja jekonomika: global' nye trendy i
praktika rossijskogo biznesa {[}Jelektronnyj resurs{]} // Imi.hse.ru:
Nacional' nyj issledovatel' skij
universitet Vysshaja shkola jekonomiki. URL:\\
\href{https://imi.hse.ru/pr2017\_1}{https://imi.hse.ru}. Data obrashhenija: 25.10.24. {[}in
Russian{]}

2.Frey C.B., Osborn M.A. The Future of Employment: How Susceptible Are
Jobs to Computerisation? // Technological Forecasting and Social
Change.-2017.-Vol.114.- P. 254-280 DOI \\10.1016/j.techfore.2016.08.019

3.Fung B.S.C., Halaburda H. Central Bank Digital Currencies: A Framework
for Assessing Why and How // Bank of Canada Staff Discussion
Paper.2016.-№.2016-22.
\href{https://www.bankofcanada.ca/wp-content/uploads/2016/11/sdp2016-22.pdf}{URL:https://www.bankofcanada.ca/wp-content/uploads/2016/11/sdp2016-22.pdf}.
Date of circulation: 25.10.24

4.Davis S.J., Haltiwanger J. Labor Market Fluidity and Economic
Performance // NBER Working Paper Series. --Cambridge, MA: National
Bureau of Economic Research, 2014. - №. 20479. URL:\\
\href{https://www.nber.org/system/files/working\_papers/w20479/w20479.pdf}{https://www.nber.org}.
Date of circulation: 25.10.24

5.Poljanin A.V., Tendencii i prognozy jekonomicheskogo rosta dlja
rossijskoj jekonomiki: regional' nyj aspekt. //
Srednerusskij vestnik obshhestvennyh nauk.-2017.- T.12.(3).- S. 53-63.
DOI 10.22394/2071-2367-2017-12-3-53-63{[}in Russian{]}

6.Prezentacija koncepcii proekta gosudarstvennoj programmy «Cifrovoj
Kazahstan 2020» {[}Jelektronnyj resurs{]}. URL:
https://www.itk.kz/doc/images/Digital\_Kazakhstan.pdf Data obrashhenija:
25.10.24. \\{[}in Russian{]}

7.Gribanov Ju. I. Cifrovaja transformacija
social' no-jekonomicheskih sistem na osnove razvitija
instituta servisnoj integracii: avtoref. dis. \ldots{} d-ra jekon. nauk:
08.00.05 / Gribanov Jurij Ivanovich. Sankt Peterburg, 2019. {[}in
Russian{]}

8.Burdenko E.V. Modeli jelektronnogo pravitel' stva //
Voprosy innovacionnoj jekonomiki. -2023. \\-T.13.(1). - S.59-76. DOI
10.18334/vinec.13.1.117234. {[}in Russian{]}

9.Sabdenov, R., Abdrahmanova, G., Zholdybalina, A., Urpekova, A.
Jevoljucija jelektronnogo pravitel' stva v Kazahstane: ot
formirovanija do cifrovoj transformacii (2004-2022) // Kazahstan-Spektr.
-2024. - №111(3). DOI 10.52536/2415-8216.2024-3.07 {[}in Russian{]}

10.Kazieva A.N. Cifrovaja transformacija kak process izmenenija sistemy
gosudarstvennogo upravlenija v Kazahstane. // Gosudarstvennyj audit. -
2022. -T. 56.(3). -S. 47-57. DOI 10.55871/2072-9847-2022-56-3-47-57.
{[}in Russian{]}

11.Ғaly Ә.Ғ., Bazhaeva N.A. Cifrovizacija gosudarstvennyh uslug v
kontekste klientoorientirovannogo podhoda: problemy i perspektivy
razvitija. // Gosudarstvennyj audit. 2024. -T. 64.(3). - S.155-162. DOI
10.55871/2072-9847-2024-64-3-155-162. {[}in Russian{]}

12.Musina G.S. Cifrovizacija i regional' naja
integracija: sravnitel' nyj analiz opyta ES i EAJeS. //
Vestnik Evrazijskogo nacional' nogo universiteta imeni
L.N. Gumileva. -2023. - №2(143). -S.166-178. DOI\\
10.32523/2616-6887/2023-143-2-166-178. {[}in Russian{]}

13.Rahmetulina Zh.B., Karipova A.T. Perspektivy sotrudnichestva mezhdu
Kazahstanom i Kitaem v \\processe razvitija transportnogo koridora Evrazii
// Jekonomicheskie otnoshenija. -2019.-T. 9(3). -C. 1615-1628. DOI
10.18334/eo.9.3.40816. {[}in Russian{]}

14.Issledovanie OON: Jelektronnoe pravitel' stvo 2022.
Budushhee cifrovogo pravitel' stva/ Departament po
jekonomicheskim i social' nym voprosam OON. -
N' ju-Jork: OON, 2022. -279 s. ISBN 978-92-1-12321-34.
{[}in Russian{]}
\end{references}

\begin{authorinfo}
\emph{{\bfseries Сведения об авторах}}

Рыстамбаева А.Н. - Карагандинский университет Казпотребсоюза, Караганда,
Казакстан, e-mail:
\href{mailto:aiga_das@mail.ru}{\nolinkurl{aiga\_das@mail.ru}};

Сихимбаев М.Р. - Карагандинский университет Казпотребсоюза, Караганда,
Казакстан, e-mail:
\href{mailto:aiga_das@mail.ru}{\nolinkurl{aiga\_das@mail.ru}}

\emph{{\bfseries Information about the authors}}

Rystambaev.N. - Karaganda University of Kazpotrebsoyuz, Karaganda,
Kazakhstan, e-mail: aiga\_das@mail.ru; Sikhimbayev M.R. - Karaganda
University of Kazpotrebsoyuz, Karaganda, Kazakhstan, e-mail:
aiga\_das@mail.ru
\end{authorinfo}
