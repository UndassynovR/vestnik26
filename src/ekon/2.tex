\id{МРНТИ 06.00.01}{}

\begin{articleheader}
\sectionwithauthors{M. Zhumazhanova, R.Zhappasova, A.Oryntaeva, M. Bolsynbek}{FORMATION OF THE BUSINESS ENVIRONMENT IN KAZAKHSTAN: THE ROLE OF
GOVERNMENT SUPPORT FOR ENTREPRENEURSHIP}

{\bfseries M. Zhumazhanova\authorid,
R.Zhappasova\authorid,
A.Oryntaeva\authorid,
M. Bolsynbek\authorid}
\end{articleheader}

\begin{affiliation}
\emph{Kazakh University of Technology and Business named after K.
Kulazhanov, Astana, Kazakhstan}

\raggedright \textsuperscript{\envelope }Corresponding author: \href{mailto:maral2804@mail.ru}{\nolinkurl{maral2804@mail.ru}}
\end{affiliation}

The development of entrepreneurship in Kazakhstan has a noticeable
impact on the country's economy, bringing significant changes to its
structure and dynamics. Entrepreneurial activity contributes to GDP
growth by creating new jobs and reducing unemployment levels. The
emergence of new businesses and the expansion of existing companies lead
to increased production volumes and an expanded economic base, which is
especially important for diversifying an economy traditionally dependent
on oil and gas.

One of the key aspects is improving the business climate. Government
programs such as the "Business Roadmap-2025" are aimed at simplifying
company registration procedures, enhancing the tax system, and
supporting startups. These measures create favorable conditions for
entrepreneurs and attract foreign investors, thereby increasing
investments in the economy.

The development of entrepreneurship also enhances market
competitiveness, leading to improvements in the quality of goods and
services, innovations, and lower prices. This, in turn, positively
affects consumers and stimulates further business development.
Additionally, entrepreneurial activity contributes to improving social
infrastructure and integrating local communities, creating new
opportunities for social initiatives and raising living standards.

However, entrepreneurship in Kazakhstan faces certain challenges such as
bureaucracy, lack of funding, and issues with law enforcement.
Addressing these problems is crucial for the sustainable and long-term
development of entrepreneurial activities in the country. Overall, the
development of entrepreneurship in Kazakhstan has a multifaceted impact
on the economy, contributing to its growth, diversification, and
improvement in the quality of life for its population.

{\bfseries Keywords:} entrepreneurship, business, PPP, lending; support;
economic growth.

\begin{articleheader}
{\bfseries ҚАЗАҚСТАНДАБИЗНЕС-ОРТАНЫҚАЛЫПТАСТЫРУ: КӘСІПКЕРЛІКТІ МЕМЛЕКЕТТІК
ҚОЛДАУДЫҢ РӨЛІ}

{\bfseries М.Т. Жұмажанова\textsuperscript{\envelope }, Р.Е.Жаппасова, Орынтаева
А.Е., Болсынбек М.К.}
\end{articleheader}

\begin{affiliation}
\emph{Қ.Құлажанов атындағыҚазақ технология және бизнес университеті,
Астана, Қазақстан,}

\emph{e-mail: maral2804@mail.ru}
\end{affiliation}

Қазақстанда кәсіпкерліктің дамуыоның құрылымымен динамикасына елеулі
өзгерістер енгізе отырып, ел экономикасына елеулі әсеретеді. Кәсіпкерлік
белсенділік жаңа жұмыс орындарын құру және жұмыссыздық деңгейін
төмендету арқылы жалпы ішкі өнімнің (ЖІӨ) өсуіне ықпал етеді.

Жаңа бизнестің пайда болуы және қолданыстағы компаниялардың кеңеюі
өндіріс көлемінің ұлғаюына және экономикалық базаның кеңеюіне әкеледі,
бұл дәстүрлі түрде мұнай мен газға тәуелді экономиканы әртараптандыру
үшін өте маңызды.

Негізгі аспектілердің бірі-бизнес климатын жақсарту. "Бизнестің
жолкартасы-2025" сияқты мемлекеттік бағдарламалар компанияларды тіркеу
рәсімдерін жеңілдетуге, салық жүйесін жетілдіруге және стартаптарды
қолдауға бағытталған.

Бұл шаралар кәсіпкерлер үшін қолайлы жағдай туғызады және экономикаға
инвестициялардың ұлғаюына ықпал ете отырып, шетелдік инвесторларды
тартады.

Кәсіпкерлікті дамыту сонымен қатар нарықта бәсекеге қабілеттілікті
арттыруға ықпал етеді, бұл тауарлармен қызметтердің сапасын жақсартуға,
инновацияларға және бағаның төмендеуіне әкеледі.

Бұл өз кезегінде тұтынушыларға оң әсер етеді және бизнестің одан әрі
дамуын ынталандырады. Сонымен қатар, кәсіпкерлік белсенділік әлеуметтік
бастамалармен өмір сүру деңгейін жақсарту үшін жаңа мүмкіндіктер жасай
отырып, әлеуметтік инфрақұрылымды жақсартуға және жергілікті
қауымдастықтардың интеграциясына ықпал етеді.

Алайда, Қазақстандағы кәсіпкерлік бюрократия, қаржыландырудың
жетіспеушілігі және құқық қолдану проблемалары сияқты белгілі бір
қиындықтарға тап болады.

Бұл проблемаларды шешу елдегі кәсіпкерлік қызметтің тұрақты және ұзақ
мерзімді дамуы үшін маңызды шарт болып табылады. Жалпы, Қазақстанда
кәсіпкерлікті дамыту экономикаға оның өсуіне, әртараптандырылуына және
халықтың өмір сүру сапасын жақсартуға ықпал ете отырып, көп қырлы әсер
етеді.

{\bfseries Түйін сөздер:} Кәсіпкерлік; бизнес; МЖС; несиелеу; қолдау;
экономикалық өсу.

\begin{articleheader}
{\bfseries Формирование бизнес-среды в Казахстане: роль государственной
поддержки предпринимательства}

{\bfseries М.Т. Жумажанова, Р.Е. Жаппасова, А.Е. Орынтаева, М.К.Болсынбек}
\end{articleheader}

\begin{affiliation}
\emph{Казахский университет технологии и бизнеса им.К.Кулажанова,
Астана, Казахстан}

\emph{e-mail: maral2804@mail.ru}
\end{affiliation}

Развитие предпринимательства в Казахстане оказывает заметное влияние на
экономику страны, внося значительные изменения в её структуру и
динамику. Предпринимательская активность способствует росту валового
внутреннего продукта (ВВП), создавая новые рабочие места и снижая
уровень безработицы. Появление новых бизнесов и расширение существующих
компаний приводит к увеличению объемов производства и расширению
экономической базы, что особенно важно для диверсификации экономики,
традиционно зависимой от нефти и газа.

Одним из ключевых аспектов является улучшение бизнес-климата.
Государственные программы, такие как «Дорожная карта бизнеса-2025»,
направлены на упрощение процедур регистрации компаний, совершенствование
налоговой системы и поддержку стартапов. Эти меры создают благоприятные
условия для предпринимателей и привлекают иностранных инвесторов,
способствуя увеличению инвестиций в экономику.

Развитие предпринимательства также способствует повышению
конкурентоспособности на рынке, что ведет к улучшению качества товаров и
услуг, инновациям и снижению цен. Это, в свою очередь, положительно
сказывается на потребителях и стимулирует дальнейшее развитие бизнеса.
Кроме того, предпринимательская активность способствует улучшению
социальной инфраструктуры и интеграции местных сообществ, создавая новые
возможности для социальных инициатив и повышения уровня жизни.

Однако предпринимательство в Казахстане сталкивается с определенными
вызовами, такими как бюрократия, нехватка финансирования и проблемы с
правоприменением. Решение этих проблем является важным условием для
устойчивого и долгосрочного развития предпринимательской деятельности в
стране. В целом, развитие предпринимательства в Казахстане оказывает
многогранное влияние на экономику, способствуя её росту, диверсификации
и улучшению качества жизни населения.

{\bfseries Ключевые слова:} предпринимательство; бизнес; ГЧП; кредитование;
поддержка; экономический рост.

\begin{multicols}{2}
{\bfseries Introduction.} The development of entrepreneurship in Kazakhstan
contributes to economic growth, economic development, innovation,
improved infrastructure, stimulation of competition, attraction of
investment and global integration, which together strengthen the
economic and social sustainability of the country.

{\bfseries Materials and methods.} The methods such as: content analysis,
quantitative analysis, SWOT analysis, PEST analysis, system analysis
were used in writing the article.

The purpose of the study is to identify and analyse the role of the
state in supporting entrepreneurship.

The hypothesis of the study is that effective government programmes
aimed at supporting business contribute to the improvement of the
business climate in the country, which in turn can affect the
development and increase the number of enterprises, increase
competitiveness.

The novelty of this topic lies in a comprehensive analysis of the impact
of government support and policy in the country on the development of
entrepreneurship. The study reviewed existing forms of state support,
including tax incentives, educational programmes and entrepreneurship
credit programmes.

Entrepreneurship in Kazakhstan is an important component of the
country' s economy. In recent years, the government of
Kazakhstan has actively supported the development of entrepreneurship by
introducing various measures and initiatives to stimulate business and
attract investment.

We consider a few key aspects of the entrepreneurial environment in
Kazakhstan:

\begin{enumerate}
\def\labelenumi{\arabic{enumi}.}
\item
  Legal framework and regulation: Kazakhstan has legislation that
  regulates various aspects of business activities, including business
  registration, taxation, intellectual property protection and other
  aspects.
\end{enumerate}

The legislative framework is constantly being improved to improve the
investment climate.

\begin{enumerate}
\def\labelenumi{\arabic{enumi}.}
\setcounter{enumi}{1}
\item
  Support for SMEs: The government actively supports the development of
  SMEs through various financing, consulting, training and other
  programmes. This includes access to financing through banks and
  financial institutions, as well as state concessional lending
  programmes.
\item
  Infrastructure and market access: Kazakhstan has a well-developed
  infrastructure, which facilitates doing business. The country is also
  a member of various international organisations and economic unions,
  which provides access to various markets and partners.
\item
  Technological innovation and digitalisation: In recent years,
  Kazakhstan has been actively developing the digital economy and
  supporting technological innovation, which has helped create new
  opportunities for entrepreneurs in IT, e-commerce and other high-tech
  industries.
\item
  International co-operation and investment: Kazakhstan actively
  attracts foreign investment, which contributes to the development of
  the economy and the creation of new jobs. The country also
  participates in various international projects and initiatives, which
  contributes to the expansion of business opportunities.
\end{enumerate}

The development of entrepreneurial activity plays a key role not only in
the economic growth of a country, but also in social and innovative
development. Entrepreneurship contributes to job creation, stimulates
innovation and technological progress, promotes regional development and
reduces social inequalities. It also contributes to globalisation,
international cooperation and sustainable development.

Thus, the development of entrepreneurial activity remains relevant and
important for the country, contributing to sustainable economic growth,
innovation and improving the quality of life of the population.

The history of entrepreneurship development is associated with risk. The
French economist Richard Contillon is considered to be the founder of
this direction. In his opinion: `an entrepreneur must anticipate future
opportunities and realise his existing opportunities in order to
generate income.Taking into account the ratio of supply and demand, the
entrepreneur buys goods at a lower price and sells them at a higher
price {[}1{]}.

All prerequisites for the development of entrepreneurial activity are
being intensively created in the Republic of Kazakhstan, in particular,
a great deal of work has been done on privatisation of property, thanks
to which a solid economic basis is being created for the development of
entrepreneurship, hence society as a whole. The country thrives thanks
to entrepreneurs, and entrepreneurs - thanks to the support of the
state.

According to the Entrepreneurial Code of the Republic of Kazakhstan,
entrepreneurship is an independent, initiative activity aimed at
generating net income through the use of property, production, sale of
goods, performance of work and provision of services {[}2{]}.

In his message to the people of Kazakhstan, the Head of State K. Tokayev
noted that effective small and medium-sized business is a solid
foundation for the development of the city and village, which plays an
important role in the socio-economic and political life of the country.
The President pays special attention to the support of small and
medium-sized businesses, the development of which has been changed since
2020 {[}3{]}.

Along with this, the Address of the Head of State of 2 September 2019
provides for income tax exemption for a period of three years only for
taxpayers who apply special tax regimes and are recognised as micro or
small businesses. Income tax is corporate income tax, individual income
tax, unified land tax for peasants and farmers, as well as social tax
for taxpayers working under a simplified declaration. All other taxes
and payments will be paid when obligations arise {[}4{]}.

According to official statistics, in the first nine months of 2020, the
share of small and medium enterprises (SMEs) in the Gross Domestic
Product (GDP) was 29.5 per cent, up 0.6 per cent compared to the same
period in 2019 (28.9 per cent in January-September 2019). In addition,
the volume of products produced by SMEs totalled K12,662.7 billion in
January-June 2020, an increase of 10.8 per cent compared to the same
period in 2019 (K10,638 billion). The number of registered SMEs reached
1.6 million units on 1 December 2020, up 3.1\% from the same date in
2019. The number of operating entities increased by 7.8 per cent to 1.3
million units {[}5{]}.

To solve the above problems, Kazakhstan is actively implementing a state
programme of business support and development called `Business Roadmap
2025'. This programme is aimed at implementing the goals of the messages
of the President of the Republic of Kazakhstan to the people: `Strategy
``Kazakhstan-2050'': a new political course of the established state'
dated 14 December 2012 and `Kazakhstan Way-2050: Common Goal, Common
Interests, Common Future' dated 17 January 2014.

Entrepreneurship in modern Kazakhstan, despite the existing
difficulties, has already established itself as an important element of
the economy. It is protected by legislation and will continue todevelop.

According to the chart below, we can see the annual growth in the number
of registered, operating medium and large enterprises in Kazakhstan over
the last 6 years by an average of 5\%. From 433,774 enterprises in 2018,
the number of legal entities of the state increased to 510,797
enterprises in 2023. The number of registered enterprises is higher by
77,023 units or 85\%. The data can be seen in figure 1.
\end{multicols}

{\bfseries Fig. 1 - The dynamics of growth of registered medium and large
enterprises in the}

{\bfseries Republic of Kazakhstan for 2018-2023.}

\emph{Note: Compiled by the author based on data from
https://stat.gov.kz/ {[}6{]}}

\begin{multicols}{2}
According to government statistics, at the end of 2023, there are
510,797 large legal entities registered in Kazakhstan, of which 410,744
are active enterprises. The number of individual entrepreneurs
1,652,564, of which 1,550,617 IE are active and registered 2,097,519
small and medium-sized enterprises, 1,904,656 are active enterprises.

If we analyse the dynamics of growth in the number of registered and
operating entrepreneurs by types of economic activity, it can be noted
that domestic entrepreneurs willingly choose the sphere of wholesale and
retail trade, repair of cars and motorbikes 27.9\%, construction 13.4\%,
provision of other types of services 9.9\%.The number of registered
entrepreneurs is shown in Table 1.
\end{multicols}


\begin{table}[H]
\caption*{Table 1 - Legal entities registered by type of activity and activity as of 31 December 2023}
\centering
\begin{tabular}{|lc|}
\hline
\multicolumn{1}{|p{0.7\textwidth}|}{Type of activity of legal entities}                                                       & \multicolumn{1}{l|}{Numberofenterprises} \\ \hline
\multicolumn{1}{|p{0.7\textwidth}|}{TotalbyKazakhstan}                                                                        & 510 797                                  \\ \hline
\multicolumn{1}{|p{0.7\textwidth}|}{Wholesale and retail trade; repair of cars and motorbikes}                                & 142 632                                  \\ \hline
\multicolumn{1}{|p{0.7\textwidth}|}{Construction}                                                                             & 68 470                                   \\ \hline
\multicolumn{1}{|p{0.7\textwidth}|}{Provision of other services}                                                              & 50 659                                   \\ \hline
\multicolumn{1}{|p{0.7\textwidth}|}{Professional, scientific and technical activities}                                        & 34121                                    \\ \hline
\multicolumn{1}{|p{0.7\textwidth}|}{Education}                                                                                & 28 496                                   \\ \hline
\multicolumn{1}{|p{0.7\textwidth}|}{Manufacturing industry}                                                                   & 25789                                    \\ \hline
\multicolumn{1}{|p{0.7\textwidth}|}{Administrative and support services activities}                                           & 24 456                                   \\ \hline
\multicolumn{1}{|p{0.7\textwidth}|}{Transactions with immovable property}                                                     & 23649                                    \\ \hline
\multicolumn{1}{|p{0.7\textwidth}|}{Transport and storage}                                                                    & 20643                                    \\ \hline
\multicolumn{1}{|p{0.7\textwidth}|}{Agriculture, forestryandfishery}                                                          & 20392                                    \\ \hline
\multicolumn{1}{|p{0.7\textwidth}|}{Information and communication}                                                            & 16554                                    \\ \hline
\multicolumn{1}{|p{0.7\textwidth}|}{Health care and social services}                                                          & 10223                                    \\ \hline
\multicolumn{1}{|p{0.7\textwidth}|}{Provision of accommodation and catering services}                                         & 9781                                     \\ \hline
\multicolumn{1}{|p{0.7\textwidth}|}{Public administration and defence; compulsory social security}                            & 9582                                     \\ \hline
\multicolumn{1}{|p{0.7\textwidth}|}{Financial and insurance activities}                                                       & 7991                                     \\ \hline
\multicolumn{1}{|p{0.7\textwidth}|}{Arts, entertainment and recreation}                                                       & 7599                                     \\ \hline
\multicolumn{1}{|p{0.7\textwidth}|}{Miningandquarrydevelopment}                                                               & 5118                                     \\ \hline
\multicolumn{1}{|p{0.7\textwidth}|}{Water supply; waste collection, treatment and disposal, pollution elimination activities} & 2763                                     \\ \hline
\multicolumn{1}{|p{0.7\textwidth}|}{Supply of electricity, gas, steam, hot water and conditioned air}                         & 1878                                     \\ \hline
\multicolumn{2}{|p{0.7\textwidth}|}{\textit{Note: Compiled by the author based on data from https://stat.gov.kz/ {[}6{]}}}                                               \\ \hline
\end{tabular}
\end{table}

\begin{multicols}{2}
Despite the recent pandemic period, the change in the economic situation
due to military actions in neighbouring countries, domestic
entrepreneurship continues to grow and develop. The volume of
entrepreneurship increases from year to year.

Analysis of official statistical data for 2023 in the field of
entrepreneurship shows that the largest number of registered legal
entities is concentrated in the cities of Almaty (142,271 or 27.9 per
cent) and Astana (95,697 or 18.7 per cent), as well as in the Karaganda
region (28,530 or 5.6 per cent). The smallest number of registered legal
entities is observed in Ұlytau (2,936 or 0.6 per cent), Abay (8,113 or
1.6 per cent) and Zhetisu (8,002 or 1.6 per cent) oblasts.The data can
be seen in table 2
\end{multicols}

\begin{table}[H]
\caption*{Table 2 - Registered and operating SMEs by regions of the Republic of Kazakhstan}
\centering
\begin{tabular}{|lcc|}
\hline
\multicolumn{1}{|l|}{Regions}                 & \multicolumn{1}{l|}{Theregistered} & \multicolumn{1}{l|}{Current} \\ \hline
\multicolumn{1}{|l|}{1}                       & \multicolumn{1}{l|}{2}             & \multicolumn{1}{l|}{3}       \\ \hline
\multicolumn{1}{|l|}{Republic of Kazakhstan}  & \multicolumn{1}{c|}{2 097 519}     & 1 904 656                    \\ \hline
\multicolumn{1}{|l|}{Abay}                    & \multicolumn{1}{c|}{56 918}        & 52 144                       \\ \hline
\multicolumn{1}{|l|}{Akmola region}           & \multicolumn{1}{c|}{61 185}        & 56 473                       \\ \hline
\multicolumn{1}{|l|}{Aktobe region}           & \multicolumn{1}{c|}{93 028}        & 85 694                       \\ \hline
\multicolumn{1}{|l|}{Almaty region}           & \multicolumn{1}{c|}{131 785}       & 122 867                      \\ \hline
\multicolumn{1}{|l|}{Atyrau region}           & \multicolumn{1}{c|}{71 309}        & 65 396                       \\ \hline
\multicolumn{1}{|l|}{West-Kazakhstan region}  & \multicolumn{1}{c|}{60 644}        & 55 488                       \\ \hline
\multicolumn{1}{|l|}{Zhambyl region}          & \multicolumn{1}{c|}{111 205}       & 98 266                       \\ \hline
\multicolumn{1}{|l|}{Jetisu}                  & \multicolumn{1}{c|}{61 353}        & 55 844                       \\ \hline
\multicolumn{1}{|l|}{Karaganda region}        & \multicolumn{1}{c|}{106 880}       & 96 316                       \\ \hline
\multicolumn{1}{|l|}{Kostanay region}         & \multicolumn{1}{c|}{66 626}        & 62 870                       \\ \hline
\multicolumn{1}{|l|}{Kyzylorda region}        & \multicolumn{1}{c|}{73 686}        & 69 503                       \\ \hline
\multicolumn{1}{|l|}{Mangistau region}        & \multicolumn{1}{c|}{84 918}        & 78 999                       \\ \hline
\multicolumn{1}{|l|}{Pavlodar region}         & \multicolumn{1}{c|}{59 591}        & 53 117                       \\ \hline
\multicolumn{1}{|l|}{North Kazakhstan region} & \multicolumn{1}{c|}{36 923}        & 33 895                       \\ \hline
\multicolumn{1}{|l|}{Turkestan region}        & \multicolumn{1}{c|}{199 393}       & 196 031                      \\ \hline
\multicolumn{1}{|l|}{Ulytau}                  & \multicolumn{1}{c|}{19 432}        & 18 224                       \\ \hline
\multicolumn{1}{|l|}{East Kazakhstan region}  & \multicolumn{1}{c|}{69 814}        & 63 144                       \\ \hline
\multicolumn{1}{|l|}{Astana}                  & \multicolumn{1}{c|}{239 311}       & 212 402                      \\ \hline
\multicolumn{1}{|l|}{Almaty}                  & \multicolumn{1}{c|}{364 139}       & 309 947                      \\ \hline
\multicolumn{1}{|l|}{Shymkent}                & \multicolumn{1}{c|}{129 379}       & 118 036                      \\ \hline
\multicolumn{3}{|l|}{\textit{Note: Compiled by the author based on data from https://stat.gov.kz/ {[}6{]}}}       \\ \hline
\end{tabular}
\end{table}

\begin{multicols}{2}
In recent years, Kazakhstan has been successfully implementing the
`National Entrepreneurship Development Project for 2021-2025'. The main
commitment and direction of the project is the financing of this
industry. Financial support for entrepreneurship under this project
includes subsidising, guaranteeing, preferential micro-credit, provision
of missing infrastructure and government grants.

Subsidisation here means reduction of interest rates on loans issued by
second-tier banks for business development. Collateral - partial
guarantee as collateral for bank loans.

This project with all related legislative and normative acts has been
developed by the Ministry of National Economy with the integration of
all previously developed state programmes. Among the ministries
responsible for implementation, along with the developer, are the
Ministries of Trade and Integration, Labour and Social Protection,
Culture and Sports, Industry and Infrastructure Development, Interior,
Finance, Energy and Foreign Affairs.

The purpose of the national project is to ensure qualitative changes in
the structure of entrepreneurship, development of small business to
increase employment, reliance on medium-sized businesses as a driving
force for the diversification of economic sectors, comprehensive
development of competition with the creation of equal conditions for
businesses.

The `National project for the development of entrepreneurship for
2021-2025' approved by Government Decision No. 728 of 12 October 2021
provides for 8,455,329,919 thousand tenge. Including from the republican
budget - 1 030 884427 thousand, from local budgets - 124 695 492
thousand, extrabudgetary - 7 299 750 000 thousand tenge. Information
concerning the giant resolution by years can be seen in the table below.
The data can be seen in table 2.
\end{multicols}

\begin{table}[H]
\caption*{Table 3 - Financing of the `National Project on Entrepreneurship Development for 2021-2025' thousand tg.}
\centering
\begin{tabular}{|lllll|}
\hline
\multicolumn{1}{|l|}{2021} & \multicolumn{1}{l|}{2022} & \multicolumn{1}{l|}{2023} & \multicolumn{1}{l|}{2024} & 2025 \\ \hline
\multicolumn{1}{|l|}{1}    & \multicolumn{1}{l|}{2}    & \multicolumn{1}{l|}{3}    & \multicolumn{1}{l|}{4}    & 5    \\ \hline
\multicolumn{1}{|c|}{1272207923,3} &
  \multicolumn{1}{c|}{1370367038,7} &
  \multicolumn{1}{c|}{1620056904,0} &
  \multicolumn{1}{c|}{1963862120,0} &
  \multicolumn{1}{c|}{2228835933,0} \\ \hline
\multicolumn{5}{|l|}{\textit{Compiled by the author on the basis of data: https://damu.kz/programmi {[}7{]}}}         \\ \hline
\end{tabular}
\end{table}

\begin{multicols}{2}
As of the end of 2023, 11,913 projects have been subsidised under the
`National Entrepreneurship Development Project for 2021-2025' for a
total loan amount of KZT 435.36 billion, with a subsidy amount of KZT
86.6 billion. The state pledged 11,813 projects for a total loan amount
of KZT 207.5 billion, with a guarantee amount of KZT 111.1 billion

For young entrepreneurs from the national budget allocated a state grant
of 307.17 million tenge for 91 projects. To attract the missing
infrastructure in 2022 spent 22.1 billion tenge, which was enough to
finance 48 projects, in 2023 25 billion tenge and in 2024 it is planned
to allocate 30 billion tenge.Figure 2 shows the ways and amount of
business financing.
\end{multicols}

{\bfseries Fig. 2 - Methods and volume of entrepreneurship financing under
the `National Entrepreneurship}

{\bfseries Development Project for 2021-2025'}

\emph{Note: Compiled by the author based on https://kapital.kz {[}8{]}}

\begin{multicols}{2}
Economic efficiency of the project for five years aims to bring the
share of small and medium-sized enterprises in the gross domestic
product up to 35\%. The growth of tourism in GDP should reach 8.4
trillion tenge. It is planned to reduce the share of the state in the
economy to 14\%. According to the results of the national document, the
state plans to create 995.3 thousand jobs for the population. Of them
permanent - 335.1 thousand, temporary - 660.2 thousand jobs.

The expected social effect of the project is as follows: 1.7 million
people will be employed on a permanent basis, 3.5 million people will be
covered by active measures to promote employment, and the share of the
rural population with incomes below the subsistence minimum is expected
to decrease by 6.5 per cent.

The project also integrates the state programme for business support and
development `Business Roadmap 2025' and the programme `Economy of Simple
Things'.

According to the list of goods for priority projects subject to lending,
favourable lending and loan guarantees are provided to entrepreneurs.
For 2023, 125 projects were subsidised for a total loan amount of 194.58
billion tenge.

During the project implementation period, SMEs created more than 106
thousand new jobs, produced products and services worth 36 trillion
tenge, and paid taxes to the budget totalling 2.3 trillion tenge.

{\bfseries Results of the discussion.}The following objectives are
envisaged in the state regulation of entrepreneurial activity:

- development of entrepreneurship in the interests of the state and
society;

- protecting the legitimate interests and rights of entrepreneurs;

- ensuring the equality of all business entities before the law.

As for non-financial support for entrepreneurs under the National
Entrepreneurship Development Project for 2021-2025, it includes a set of
projects to provide training and advisory support to potential and
aspiring entrepreneurs on how to run a business. Entrepreneur Service
Centres are functioning in regional and district centres,
single-industry towns and small towns. They will provide assistance
under the Business Roadmap 2025 programme, the Simple Things Economy
programme, and the Enbek programme on a free-of-charge basis, offline
and online. In addition, entrepreneurs are provided with service support
in running their existing business in the form of accounting and tax
accounting, statistical reporting, internet marketing, legal services,
information technology coverage, public procurement, and assistance to
domestic producers in selling their products.

Also within the framework of the `National Project on Entrepreneurship
Development for 2021-2025' it is envisaged to provide microcredits at
6\% interest rates. For mono and small towns and rural communities,
these loans are provided without sectoral restrictions, for investment
purposes in an amount not exceeding 20 million tenge, and for
replenishment of working capital on an unsecured basis at 5\% up to 5
million tenge.

In addition, the implementation of lending mechanisms for priority
projects, such as subsidising part of the interest rate and private loan
guarantees, is ongoing.

Youth is the most important link in the entrepreneurial environment.
Youth today, youth entrepreneurship in 10 years will determine the face
of our country. Economic growth and the pace of development of the
country of tomorrow depend on the development of youth business today.

Currently, youth entrepreneurship has become one of the key areas of
small business development in Kazakhstan. The creation of favourable
conditions to encourage young people to engage in entrepreneurial
activities is considered in various programmes at the national and
regional levels.

However, many regions of Kazakhstan lack platforms that would allow
young people to gain relevant knowledge and skills, share information,
develop ideas and receive expert advice. As a result, innovative ideas
are either not realised or remain within universities and research
laboratories without reaching the market.

According to statistics, approximately 300 million young people between
the ages of 18 and 30 worldwide are either unemployed or not working at
all. About 20\% of them have the ability to start their own business,
but only 5\% of them can realise this potential for various reasons.

The singling out of youth entrepreneurship as a separate category is due
to its unique characteristics, strengths and weaknesses compared to
other types of entrepreneurship:

Strengths of youth entrepreneurship:

- High innovative activity and creativity.

- Flexibility, speed of action and ability to develop new markets.

- The ability to systematically update knowledge and skills in line with
market changes.

- Potential to withstand high labour and nervous loads, especially at
the initial stage.

- risk appetite.

Weaknesses of youth entrepreneurship:

- Lack of social experience.

- lack of business reputation.

- weak practical skills in applying economic principles.

- problems with the formation of start-up capital.

- Lack of personal connections in business and power structures.

- vulnerability to bureaucratic obstacles and criminal influence.

These characteristics highlight key challenges in supporting youth
entrepreneurship, including developing strengths and overcoming
weaknesses.

To date, several problems affecting the willingness of young people to
create their own business projects remain unresolved:

- social mood. Entrepreneurial spirit, willingness to take risks and
create something new, is not yet so pronounced among young people who
prefer careers in big business or public service.

- Public perception of entrepreneurs. Entrepreneurial activity is
perceived as a way of overcoming difficulties rather than as an
endeavour to achieve success.

- Educational content. Traditional educational institutions provide
basic economic knowledge, but do not develop the incentives and
competences necessary for successful business.

- start-up conditions. Small business support measures for young people
have not yet yielded noticeable results, and administrative, legal and
financial barriers are still significant.

- Young people' s lack of awareness of state
organisations supporting small businesses.

Young people are the most active part of society, able to react quickly
to changes and take advantage of them. Therefore, it can be argued that
young people have a significant potential for entrepreneurship. Support
of youth entrepreneurship requires special attention from the state
authorities. Properly organised and targeted support will help to
develop small business in the region, leading to economic growth and
increasing its attractiveness.

However, young people face great difficulties on the way to creating
this entrepreneurship. These challenges are related to both the
financial difficulties in setting up a business and the lack of
education of young people. They include high rates of interest and taxes
on credit, lack of start-up capital in the hands of young people,
limited or no economic and productive linkages due to the short-term
nature of their activities and lack of experience. These and other
problems require mandatory solutions through government support for
youth entrepreneurship. Due to the lack of awareness of young people
about the existing methods of support for youth entrepreneurship, there
is a need for various campaigns, competitions, creation of information
bases in educational institutions. All these measures will help to stop
young people' s perception of entrepreneurship and
business as an inaccessible sphere of activity {[}9{]}.

The growth of production output and increase in revenues from sales in
small businesses contribute to the increase in tax revenues to the
budget. In particular, there is a positive trend in the corporate income
tax (CIT) expenditures of small businesses.

- Over the last five years, small business expenditures on KPN have
increased more than 5 times, reaching 6,381.6 billion tenge in 2023.

- The share of small business expenditures on KPN in the total volume of
all enterprises increased from 26.2 per cent to 53.4 per cent.

Despite the positive analysis, there are still problems that hinder the
development of the industry:

Today, a large number of small businesses operate in low-productivity
sectors, with trade remaining the dominant activity, with about 35 per
cent of all small businesses in the country.

Trade is an important sector of the economy; at the same time, it is
necessary to ensure growth in the number of small and medium-sized
enterprises, especially in the manufacturing sector.

The contribution of medium-sized companies to GDP will remain at the
2019 level, which is about 6 per cent. At the same time, about 30\% of
medium-sized companies show losses in the first half of 2023.

Instruments to facilitate business growth and expansion for effective
implementation should be strengthened. Further work is needed to improve
the business climate in the country.

Domestic Entrepreneurship Challenges:

Lack of engineering industrial infrastructure, difficulties in business
planning due to price controls and export bans.

The number of active SMEs over the last 5 years is 46.5\%. And the
number of medium-sized enterprises is only 11.6\%. SMEs account for only
0.2\% of medium-sized industries, and 99.8\% of small enterprises are
fertilised.

Over the last five years, the number of operating SMEs has increased by
46.5\%. The main contribution to this growth was a one and a half times
increase in the number of individual entrepreneurs. At the same time,
the number of medium-sized businesses has increased by only 11.6 per
cent. Currently, only 0.2 per cent of operating SMEs belong to
medium-sized businesses, while 99.8 per cent are small enterprises.

It is necessary to revise the system and mechanisms of state support for
entrepreneurship. Despite the long-term operation of a number of
programmes for the development of domestic business, their effectiveness
is subject to criticism.

Firstly, there is limited coverage of the commercial establishment with
recommended support measures, such as cash injections, from the budget.

The purpose of the national programme is an important issue, but
compliance with these objectives is another entrepreneurial problem. An
example is the Economics of Simple Things programme.

An example is the Economy of Simple Things programme, which was created
to saturate the domestic market with goods and services from domestic
producers. However, 35 per cent of the activities (53 out of 150)
eligible for concessional financing did not meet the stated objective.
This trend reduces the effectiveness of state support measures and is
observed in the framework of the National Project for the Development of
Entrepreneurship. Grant support for business is particularly
problematic.

Firstly, there are problems with monitoring the targeted use of grant
support funds. Today, monitoring is carried out only selectively and
covers a very narrow range of projects. For example, in 2022 only 27 out
of 264 projects approved in Almaty were checked, i.e. only one in ten. A
similar situation occurs in other regions. All grant financing projects
should be covered by control over the targeted use of funds and
achievement of final results.

Secondly, grant support is most often aimed at small projects with
funding of 5 million tenge, which leads to the dispersion of budgetary
funds. No account is taken of promising directions and orientation
towards final results.

Thirdly, there are legal gaps, such as the lack of a clear definition of
`novelty of a business idea', which leads to different interpretations
of this concept by both entrepreneurs and members of competition
commissions. There is also no requirement to confirm the `novelty',
which is a key condition for receiving a grant.

The criteria for determining the priority sectors of the economy for
which state support measures are provided also remain unclear. There are
plans to include in the list of priority sectors such areas as food
retailing (e.g., `convenience stores'), property rental, building
maintenance, and passenger car leasing.

Government support should help to increase domestic production, develop
the manufacturing industry and target sectors specialising in the
production of finished goods.

The business community continues to rely on government assistance, as
bank lending conditions remain unacceptable. Despite the growth in
lending to SMEs, the share of such loans in the total loan portfolio
decreased from 33\% in 2019 to 26\%.(Figure 4)
\end{multicols}

{\bfseries Fig.4 - Dynamics of business lending in the RK}

\emph{Note: Compiled by the author on the basis of data from the CIS
Internet portal /
\href{https://e-cis.info/news\%20\%5b10}{https://e-cis.info/news
{[}10}{]}}

\begin{multicols}{2}
According to the National Bank, last year 52 per cent of the total
volume of loans granted by banks to businesses went to large
entrepreneurs, 34 per cent to small businesses and only 14 per cent to
medium-sized enterprises. Industrial and trading companies continue to
lend more actively than others. The largest share of loans (51 per cent)
went to the trade sector, with large entrepreneurs receiving more than
half (54 per cent) of these funds. Industry accounts for 42.1 per cent
of the portfolio. The third place with a big lag is occupied by the
transport sector with a share of 6.9\%.

Trade also dominates the receipt of government support through the Damu
fund. More than one-third of the subsidised (37\%) and guaranteed (39\%)
projects are trade-related. This points to the need for a fundamental
review of approaches and mechanisms of state support.

There are serious problems with tax policy in relation to small and
medium-sized enterprises. The business community expresses
dissatisfaction, especially with the proposal to increase the value
added tax (VAT) rate from 12 per cent to 16 per cent. This issue
requires detailed discussion with domestic entrepreneurs. It is also
necessary to address the problems of VAT refunds and the high tax and
social burden on employers, which reaches 39 per cent.

Fiscal measures should be aimed at stimulating the growth of small
businesses and their consolidation.

In rail transport, business faces serious problems, including
deteriorated infrastructure and corruption risks in the provision of
freight wagons. An example of business dissatisfaction is the lack of
equality in freight tariffs, where the national carrier emphasises
transit traffic as its tariffs are 2.5 times higher.

According to Atameken, the government has imposed export restrictions on
15 types of agricultural products over the past two years. These
measures, which are aimed at stabilising prices in the domestic market,
lead to negative consequences for business, such as the disruption of
contracts, spoilage of products, stoppage of production and loss of
foreign markets.

In addition, there have been more than 50 changes to industry subsidy
rules over the past five years. Constant changes disorient farmers and
investors, preventing them from building long-term development plans.

According to the results of the SWOT analysis of the work, we have
achieved the following:
\end{multicols}

\begin{table}[H]
    \centering
    \renewcommand{\arraystretch}{1.3}
    \begin{tabular}{|p{0.48\textwidth}|p{0.48\textwidth}|}
        \hline
        \textbf{Strengths} & \textbf{Weaknesses} \\ 
        \hline
        \begin{minipage}[t]{0.46\textwidth} 
            \RaggedRight
            1. Government support: Availability of programmes and initiatives aimed at supporting small and medium-sized businesses. \\
            2. Strategic location: Kazakhstan is at the crossroads of important trade routes, which favours logistics and trade. \\
            3. Resources: Rich natural resources for energy and mining businesses.
        \end{minipage}
        &
        \begin{minipage}[t]{0.46\textwidth} 
            \RaggedRight
            1. Corruption: Can impede access to government support and licensing. \\
            2. Lack of skilled labour: High migration and low specialist training in some sectors. \\
            3. Uneven regional development: Varying business environments in urban and rural areas. \\
            4. Bureaucracy: Complex, slow administrative procedures discourage entrepreneurs.
        \end{minipage} \\ 
        \hline
        \textbf{Opportunities} & \textbf{Threats} \\ 
        \hline
        \begin{minipage}[t]{0.46\textwidth} 
            \RaggedRight
            1. Increased digitalisation: Adoption of digital technologies creates new start-up opportunities. \\
            2. Investment in innovation: Support for IT and environmental projects. \\
            3. International cooperation: Access to foreign investment and technology via global programmes. \\
            4. Agro-industrial growth: Rising interest in agriculture and eco-products fosters new businesses.
        \end{minipage}
        &
        \begin{minipage}[t]{0.46\textwidth} 
            \RaggedRight
            1. Economic instability: External factors like oil price fluctuations and crises. \\
            2. Competition: Neighbouring countries developing competitive business environments. \\
            3. Climate change: Environmental issues affecting resources and agribusiness.
        \end{minipage} \\ 
        \hline
    \end{tabular}
\end{table}

This type of analysis can help to understand the current state and
prospects of business in Kazakhstan, taking into account government
support. SWOT analysis serves as a basis for long-term planning and
prioritisation of business development.

In order to identify important social and technological trends, a PEST
analysis was conducted, which can help to develop new products or
services.

According to the results of PEST-analysis we have achieved the
following:

\begin{table}[h]
    \centering
    \renewcommand{\arraystretch}{1.3}
    \begin{tabular}{|p{0.48\textwidth}|p{0.48\textwidth}|}
        \hline
        \textbf{Political Factors} & \textbf{Economic Factors} \\ 
        \hline
        \begin{minipage}[t]{0.46\textwidth} 
            \RaggedRight
            1. State support for SMEs: Availability of programmes aimed at business development. \\
            2. Regulation and regulatory framework: Legislative changes aimed at improving the business climate and simplifying administrative procedures. \\
            3. Political stability: The level of stability in the country, which affects the confidence of investors and entrepreneurs. \\
            4. International agreements: Participation in various international economic and trade agreements, which opens new markets for business.
        \end{minipage}
        &
        \begin{minipage}[t]{0.46\textwidth} 
            \RaggedRight
            1. GDP growth: Sustained economic growth in recent years, creating opportunities for new investment. \\
            2. Inflation and exchange rate: Exchange rate fluctuations and inflation can affect the cost of imports and export opportunities. \\
            3. Unemployment rate: Reduced unemployment contributes to consumer demand and welfare. \\
            4. Investment climate: An assessment of Kazakhstan's attractiveness to foreign investment and the level of government support.
        \end{minipage} \\ 
        \hline
        \textbf{Social Factors} & \textbf{Technological Factors} \\ 
        \hline
        \begin{minipage}[t]{0.46\textwidth} 
            \RaggedRight
            1. Entrepreneurial culture: The prevailing culture and attitudes towards entrepreneurship that may favour or hinder its development. \\
            2. Education and skills: The level of education and skills of the labour force that affects the ability of entrepreneurs to compete. \\
            3. Demographic change: Changes in the population, including the growth of young people, can create new consumer trends. \\
            4. Demand for innovation: Increased interest in innovative products and services among the population.
        \end{minipage}
        &
        \begin{minipage}[t]{0.46\textwidth} 
            \RaggedRight
            1. Digitalisation of business: Increased adoption of digital technologies in various sectors of the economy. \\
            2. Innovative startups: Increase in the number of technology startups supported by the government and private investors. \\
            3. Infrastructure development: Investments in the development of digital and transport infrastructure, which improves access to technology.
        \end{minipage} \\ 
        \hline
    \end{tabular}
\end{table}

\begin{multicols}{2}
Kazakhstan demonstrates a more stable business environment compared to
neighbouring countries such as Uzbekistan, Kyrgyzstan, Tajikistan and
Russia, although it faces bureaucracy and corruption. Uzbekistan is
implementing reforms to improve the business environment, but still has
problems with respect for property rights. Kyrgyzstan is known for high
levels of corruption but offers more liberal approaches, while
Tajikistan faces strong state influence and limited access to finance.
Russia offers diverse financial opportunities but has a difficult
business environment and high risks. Overall, Kazakhstan has potential
for growth, but it needs to improve access to finance and technology
adoption to improve its international competitiveness.

{\bfseries Conclusions.} In Kazakhstan, entrepreneurship plays an
increasingly important role in the economy. It is designed to address
key challenges such as:

- Production of consumer goods and services using local sources of raw
materials without significant capital investment.

- Creating conditions for the employment of labour force released from
large enterprises.

- Acceleration of scientific and technological progress in the country.

- Creation of a positive alternative to criminal business.

As already mentioned, the state is implementing many projects to improve
and revive domestic entrepreneurship. The future of Kazakhstan depends
on entrepreneurial activity and the ability to develop an effective
concept. Therefore, a professional approach to entrepreneurial activity
and fulfilment of entrepreneurial functions is required.
\end{multicols}

\begin{center}
{\bfseries References}
\end{center}

\begin{references}
1.Lapusta, M.G. Predprinimatel' stvo: Uchebnik / M.G.
Lapusta - M:INFRA-M, 2009. - 608 s.{[}in Russian{]} ISBN
978-5-16-003252-8 {[}in Russian{]}

2.Predprinimatel' skij kodeks Respubliki Kazahstanot 29
oktjabrja 2015 goda № 375 -- V ZRK (s izmenenijami i dopolnenijami po
sostojaniju na 16.01.2021 g.). -Data obrashhenmija: 12.06.2024. {[}in
Russian{]}

3.Poslanie Glavygosudarstva Kasym-Zhomarta Tokaeva narodu Kazahstana:
«Kazahstan v novojreal' nosti: vremjadejstvij» ot 1
sentjabrja 2020 g. -Data obrashhenmija: 12.06.2024. {[}in Russian{]}

4.Poslanie Glavy gosudarstva Kasym-Zhomarta Tokaeva narodu Kazahstana:
«Konstruktivnyj obshhestvennyj dialog -- osnova
stabil' nosti i procvetanija Kazahstana» ot 2 sentjabrja
2019 goda.-Data obrashhenmija: 12.06.2024. {[}in Russian{]}

5.Oficial' nyj informacionnyj resurs
Prem' erMinistra RK / Moratorij na proverki i
osvobozhdenie ot nalogov - kak budet razvivat' sja biznes
v RK v blizhajshie 3 goda. URL: \url{https://primeminister.kz/} - Data
obrashhenmija: 21.06.2024. {[}in Russian{]}

6.Oficial' nye dannye agentstva po strategicheskomu
planirovaniju i reformam RK / Osnovnye pokazateli kolichestvasub\#ektov
v Respublike Kazahstan. URL: \url{https://stat.gov.kz/}.-Data
obrashhenmija: 21.06.2024. {[}in Russian{]}

7. Oficial' nye dannye AO «Fond razvitija
predprinimatel' stva «Damu» /Nacional' nyj
proekt po razvitiju

predprinimatel' stva na 2021-2025
gody.URL:https://damu.kz/programmi.- Data obrashhenmija: 21.06.2024.
{[}in Russian{]}

8.Kapital -- centrdelovojinformacii / V Kazahstaneopredelili 10
zadachporazvitijupredprinimatel' stva. URL:
\url{https://kapital.kz} - Data obrashhenmija: 21.06.2024. {[}in
Russian{]}

9.Oficial' nyedannyeNacional' nojpalatypredprinimatelej
RK «Atameken» / Problemy,
meshajushhieuspeshnomurazvitijumolodezhnogopredprinimatel' stva.
URL: \url{https://atameken.kz/}. Data obrashhenmija: 21.06.2024.{[}in
Russian{]}

10.Internet portal SNG / Razvitiemalogoisrednegobiznesa v Kazahstane:
problemireshenija. URL: \url{https://e-cis.info/news}. - Data
obrashhenmija: 21.06.2024.{[}in Russian{]}
\end{references}

\begin{authorinfo}
\emph{{\bfseries Information about the authors}}

Zhumazhanova M. -Kazakh University of Technology and Business named
after K. Kulzhanov, Senior Lecturer, Master's Degree, Astana,
Kazakhstan, e-mail: maral2804@mail.ru;

Zhappasova R. -Kazakh University of Technology and Business named after
K. Kulzhanov, Ph.D in Economics, ass.Professor, Astana, Kazakhstan,
e-mail:zhappasova\_77@mail.ru;

Oryntayeva A.Y. - Kazakh University of Technology and Business named
after K. Kulzhanov, lecturer Master's Degree\\
Astana, Kazakhstan, e-mail:
\href{mailto:oryntayeva.a13@gmail.com}{\nolinkurl{oryntayeva.a13@gmail.com}}

Bolsynbek M. - Kazakh University of Technology and Business named after
K. Kulzhanov, lecturer Master's Degree\\
Astana, Kazakhstan, e-mail:
\href{mailto:mbolsynbek@bk.ru}{\nolinkurl{mbolsynbek@bk.ru}}

\emph{{\bfseries Сведения об авторах}}

Жумажанова М.Т. \emph{-} Казахский университет технологии и бизнеса
им.К.Кулажанова, старший преподаватель, магистр, Астана, Казахстан,
e-mail: maral2804@mail.ru;

Жаппасова Р.Е. \emph{-} Казахский университет технологии и бизнеса
им.К.Кулажанова, к.э.н., асс.профессор.

Астана, Казахстан, e-mail: zhappasova\_77@mail.ru

Орынтаева А.Е\emph{. -}Казахский университет технологии и бизнеса
им.К.Кулажанова, преподаватель, магистр,

Астана, Казахстан, e-mail:
\href{mailto:oryntayeva.a13@gmail.com}{\nolinkurl{oryntayeva.a13@gmail.com}}

Болсынбек М.К. \emph{-}Казахский университет технологии и бизнеса
им.К.Кулажанова, преподаватель, магистр,

Астана, Казахстан, e-mail:
\href{mailto:mbolsynbek@bk.ru}{\nolinkurl{mbolsynbek@bk.ru}}

https://orcid.org/0009-0001-0233-198
\end{authorinfo}
