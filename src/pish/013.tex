\id{МРНТИ 65.33.35}{https://doi.org/10.58805/kazutb.v.1.26-735}

\begin{articleheader}
\sectionwithauthors{Б.Ж. Мулдабекова, М.Б. Султанкул, К.А. Куртибай, Ф.Б. Қаххоров, Г.Х. Исматуллаева, А.Т. Жумабекова}{ВЛИЯНИЕ АРАХИСОВОЙ МУКИ НА ОРГАНОЛЕПТИЧЕСКИЕ И ФИЗИКО-ХИМИЧЕСКИЕ ПОКАЗАТЕЛИ КРЕКЕРОВ}

{\bfseries  
\textsuperscript{1}Б.Ж. Мулдабекова\authorid,
\textsuperscript{1}М.Б. Султанкул\textsuperscript{\envelope } \authorid,
\textsuperscript{2}К.А. Куртибай\authorid,
\textsuperscript{3}Ф.Б. Қаххоров\authorid, 
\textsuperscript{1}Г.Х. Исматуллаева\authorid,
\textsuperscript{1}А.Т. Жумабекова\authorid}
\end{articleheader}

\begin{affiliation}
\emph{\textsuperscript{1}Алматинский Технологический Университет, Алматы, Казахстан,}

\emph{\textsuperscript{2}Евразийский национальный университет имени Л.Н. Гумилева, Астана, Казахстан,}

\emph{\textsuperscript{3}Джизакский политехнический институт, г. Джизак, Республика Узбекистан,}

\raggedright \textsuperscript{\envelope }{\em Корреспондент-автор: \href{mailto:m.sultankul@bk.ru}{\nolinkurl{m.sultankul@bk.ru}}}
\end{affiliation}

Статья посвящена исследованию влияния арахисовой муки на
органолептические и физико - химические показатели крекеров. В ходе работы
проведены эксперименты по добавлению арахисовой муки в различные
рецептуры теста для крекеров, с целью улучшения их вкусовых качеств,
текстуры и питательной ценности.

Оценка органолептических характеристик включала анализ запаха, вкуса,
текстуры и внешнего вида полученных изделий, что позволило объективно
оценить их качество. Физико - химические исследования охватывали такие
параметры, как влажность, щелочность и намокаемость, что позволило
понять, как арахисовая мука влияет на структуру теста и готовых изделий.
Результаты показали, что добавление арахисовой муки значительно улучшает
вкусовые и текстурные характеристики крекеров, при этом их питательная
ценность повышается за счет увеличения содержания растительного белка и
жиров. Данная статья подчеркивает потенциал арахисовой муки как
полезного ингредиента для производства функциональных продуктов питания,
а также открывает перспективы для дальнейших исследований в области
инновационных технологий в пищевой промышленности, что в конечном итоге
приведет к созданию более полезных и вкусных продуктов для потребителей.
Добавление арахисовой муки также может способствовать созданию продуктов
для специальных диетических потребностей.

{\bfseries Ключевые слова:} арахисовая мука, крекеры, органолептические
показатели, физико - химический анализ, питательная ценность.

\begin{articleheader}
{\bfseries ЖЕРЖАҢҒАҚ ҰНЫНЫҢ КРЕКЕРДІҢ ОРГАНОЛЕПТИКАЛЫҚ ЖӘНЕ ФИЗИКАЛЫҚ-ХИМИЯЛЫҚ КӨРСЕТКІШТЕРІНЕ ӘСЕРІ}

{\bfseries
\textsuperscript{1}Б.Ж. Мулдабекова,
\textsuperscript{1}М.Б. Султанкул\textsuperscript{\envelope },
\textsuperscript{2}К.А. Куртибай,
\textsuperscript{3}Ф.Б. Қаххоров,
\textsuperscript{1}Г.Х. Исматуллаева,
\textsuperscript{1}А.Т. Жумабекова}
\end{articleheader}

\begin{affiliation}
\emph{\textsuperscript{1}Алматы Технологиялық Университеті, Алматы, Қазақстан,}

\emph{\textsuperscript{2}Л.Н. Гумилев атындағы Еуразия ұлттық университеті, Астана, Қазақстан,}

\emph{\textsuperscript{3}Джизак политехникалық институты, Джизак қаласы, Өзбекстан,}

\emph{e- mail: \href{mailto:m.sultankul@bk.ru}{\nolinkurl{m.sultankul@bk.ru}}}
\end{affiliation}

Мақалада жержаңғақ ұнының крекерлердің органолептикалық және
физико-химиялық көрсеткіштеріне әсерін зерттеу қарастырылған. Жұмыс
барысында крекерлердің дәмдік сапасын, құрылымын және тағамдық
құндылығын жақсарту мақсатында, түрлі рецептерге жержаңғақ ұнын қосу
бойынша эксперименттер жүргізілді.

Органолептикалық көрсеткіштерді бағалау дайын өнімдердің иісі, дәмі,
құрылымы және сыртқы көрінісін талдауды қамтыды, бұл олардың сапасын
объективті түрде бағалауға мүмкіндік берді. Физико-химиялық зерттеулер
ылғалдылық, сілтілік және суланғыштық сияқты параметрлерді қамтыды, бұл
жержаңғақ ұнының қамыр мен дайын өнімдердің құрылымына қалай әсер
ететінін түсінуге мүмкіндік берді. Нәтижелер жержаңғақ ұнын қосу
крекерлердің дәмдік және текстуралық сипаттамаларын едәуір
жақсартатынын, сонымен қатар олардың тағамдық құндылығы өсімдік ақуызы
мен майларының құрамының жоғарылауына байланысты артатынын көрсетті. Бұл
мақала жержаңғақ ұнының функционалдық азық-түлік өнімдерін өндіруге
арналған пайдалы ингредиент ретіндегі әлеуетін ерекше атап өтеді,
сондай-ақ азық-түлік өнеркәсібінде инновациялық технологиялар
саласындағы одан әрі зерттеулердің перспективаларын ашады, бұл түптеп
келгенде тұтынушылар үшін пайдалы әрі дәмді өнімдердің құрылуына
әкеледі. Жержаңғақ ұнын қосу арнайы диеталық қажеттіліктері бар
өнімдерді жасауға да ықпал етуі мүмкін.

{\bfseries Түйін сөздер:} жержаңғақ ұны, крекерлер, органолептикалық
көрсеткіштер, физико-химиялық анализ, тағамдық құндылық

\begin{articleheader}
{\bfseries THE EFFECT OF PEANUT FLOUR ON THE ORGANOLEPTIC AND PHYSICO-CHEMICAL PARAMETERS OF CRACKERS}

{\bfseries  
\textsuperscript{1}B.Zh. Muldabekova,  
\textsuperscript{1}M.B. Sultankul\textsuperscript{\envelope },  
\textsuperscript{2}K.A. Kurtibay,  
\textsuperscript{3}F. Qaxxorov,  
\textsuperscript{1}G.Kh. Ismatullaeva,  
\textsuperscript{1}A.T. Zhumabekova}
\end{articleheader}

\begin{affiliation}
\emph{\textsuperscript{1}Almaty Technological University, Almaty, Kazakhstan,}

\emph{\textsuperscript{2}L.N.Gumilyov Eurasian National University, Astana, Kazakhstan,}

\emph{\textsuperscript{3}Jizzakh Polytechnic Institute, Jizzakh, Uzbekistan}
\end{affiliation}

The article is devoted to the study of the effects of peanut flour on
the organoleptic and physicochemical properties of crackers. Experiments
were conducted to add peanut flour to various cracker dough recipes in
order to improve their flavor, texture, and nutritional value.

The evaluation of organoleptic characteristics included the analysis of
the aroma, taste, texture, and appearance of the resulting products,
which allowed for an objective assessment of their quality.
Physico - chemical studies covered parameters such as moisture, alkalinity,
and wetting, which helped to understand how peanut flour affects the
structure of the dough and the finished products. The results showed
that the addition of peanut flour significantly improves the flavor and
texture characteristics of crackers, while their nutritional value is
enhanced due to the increased content of vegetable proteins and fats.
This article highlights the potential of peanut flour as a beneficial
ingredient for the production of functional food products, as well as
opens up prospects for further research in the field of innovative
technologies in the food industry, ultimately leading to the creation of
healthier and tastier products for consumers. The addition of peanut
flour may also contribute to the creation of products for special
dietary needs.

{\bfseries Key words:} peanut flour, crackers, organoleptic parameters,
physico-chemical analysis, nutritional value.

\begin{multicols}{2}
{\bfseries Введение.} Рост потребительского интереса к функциональным и
питательным продуктам питания стимулирует поиск альтернативных
растительных белковых компонентов, способных улучшить состав
традиционных хлебобулочных изделий. Одним из таких перспективных
ингредиентов является \emph{Arachis hypogaea}, широко известный как
арахис или земляной орех. Арахисовая мука, получаемая из обезжиренного
арахисового жмыха или измельченного жареного арахиса, рассматривается
как ценный пищевой компонент благодаря высокому содержанию белка,
наличию незаменимых аминокислот и биологически активных соединений.
Кроме того, ее функциональные свойства позволяют использовать ее в
качестве ингредиента для обогащения безглютеновой и высокобелковой
продукции {[}1{]}. Внедрение арахисовой муки в рецептуру крекеров
открывает новые перспективы для повышения их пищевой ценности, улучшения
органолептических характеристик и модификации текстурных параметров.

Арахисовая мука характеризуется высокой концентрацией белка,
составляющей около 50\% в обезжиренной форме, а также наличием
незаменимых аминокислот, полиненасыщенных жирных кислот, антиоксидантов
и микроэлементов {[}2{]}. Ее включение в состав хлебобулочных изделий
способствует улучшению пищевой ценности, повышению биодоступности белка
и микронутриентов, а также усилению антиоксидантного потенциала продукта
{[}3{]}. Кроме того, высокая липидная фракция арахиса способствует
формированию желаемой текстуры, улучшает реологические свойства теста и
повышает органолептические характеристики конечного продукта {[}4{]}. С
функциональной точки зрения арахисовая мука обладает выраженной
водопоглощающей способностью, что влияет на свойства теста и конечную
структуру выпеченных изделий. В частности, она способствует повышенной
гидратации теста, увеличению его вязкости и когезионности, что
определяет более плотную и однородную текстуру крекеров по сравнению с
традиционными вариантами на основе пшеничной муки {[}5{]}.

Физико-химические свойства крекеров, обогащенных арахисовой мукой,
изменяются в зависимости от уровня ее включения в рецептуру.
Исследования показывают, что увеличение концентрации арахисовой муки
приводит к изменению реологических свойств теста и готового продукта:
повышается водопоглощающая способность, что способствует формированию
более плотной и когезионной текстуры {[}5,6{]}; усиливаются реакции
неэнзимного потемнения (реакция Майяра), что приводит к формированию
более интенсивного золотисто-коричневого оттенка, повышая
привлекательность продукта {[}6{]}; изменяется химический состав
крекеров, в частности, увеличивается содержание белка, пищевых волокон и
минеральных веществ, таких как магний, цинк и железо {[}7{]}. Однако
чрезмерное включение арахисовой муки может привести к увеличению
твердости изделий, снижению их хрупкости и изменению текстурных
характеристик, что требует оптимизации рецептуры с точки зрения
соотношения компонентов {[}8{]}.

Сенсорные исследования подтверждают, что арахисовая мука положительно
влияет на аромат, вкус и текстуру крекеров, придавая им характерные
ореховые и поджаренные ноты, которые воспринимаются потребителями как
привлекательные {[}9{]}. Однако увеличение концентрации арахисовой муки
свыше 30\% может привести к появлению горьковатого послевкусия и
излишней твердости, что требует коррекции рецептуры {[}10{]}. Согласно
сенсорному анализу, крекеры, обогащенные арахисовой мукой, обладают
высокой степенью приемлемости среди потребителей, особенно при
использовании сбалансированного уровня включения (менее 30\% от общей
массы муки) {[}11{]}. Таким образом, оптимизация рецептуры с учетом
органолептических параметров играет ключевую роль в разработке
конкурентоспособных продуктов.

В связи с растущим спросом на высокобелковые растительные продукты,
арахисовая мука находит широкое применение в разработке инновационных
хлебобулочных изделий. В частности, она используется в безглютеновых
продуктах, функциональных закусках и белковых смесях для
специализированного питания {[}12{]}. Кроме того, активное изучение
сочетания арахисовой муки с альтернативными источниками растительного
белка, такими как маниоковая, нутовая или сорговая мука, позволяет
оптимизировать питательный состав крекеров и добиться гармоничного
сочетания текстурных и вкусовых характеристик {[}13{]}. Рыночные
исследования указывают на возрастающий интерес к разработке
гипоаллергенных форм арахисовой муки, а также к возможностям модификации
ее функциональных свойств с целью расширения областей применения в
хлебопекарной отрасли {[}14{]}.

Интеграция арахисовой муки в рецептуру крекеров обладает значительным
потенциалом для повышения их пищевой ценности, улучшения
органолептических характеристик и изменения текстурных параметров.
Однако достижение оптимального качества продукции требует детальной
проработки рецептурных решений с учетом влияния арахисовой муки на
физико-химические и сенсорные свойства изделий. Перспективными
направлениями дальнейших исследований являются изучение технологических
методов обработки арахисовой муки с целью улучшения ее функциональных
характеристик, разработка стратегий белкового обогащения для повышения
биодоступности аминокислот, внедрение технологий снижения аллергенности,
что позволит расширить область применения арахисовой муки в
функциональных продуктах питания. Таким образом, дальнейшее изучение
свойств арахисовой муки и механизмов ее взаимодействия с компонентами
теста будет способствовать созданию высококачественных и востребованных
продуктов на рынке функционального питания.

Целью данного исследования является изучение влияния добавления
арахисовой муки на органолептические характеристики и физико-химические
параметры крекеров, а также комплексная оценка её воздействия на
качество.

\emph{{\bfseries Задачи исследования:}}

1. Провести анализ химического состава арахисовой муки и сравнить его с
составом пшеничной муки.

2. Изучить органолептические показатели арахисовой муки.

3. Разработать рецептуру крекера путем экспериментального замещения
части пшеничной муки арахисовой в различных пропорциях.

4. Определить оптимальные методические соотношения ингредиентов для
достижения наилучших органолептических характеристик и текстуры готового
продукта.

5. Оценить органолептические свойства изготовленных крекеров (вкус,
аромат, текстура, внешний вид).

6. Измерить физико-химические характеристики крекеров, полученных с
использованием арахисовой муки.

7. Сделать выводы о влиянии арахисовой муки на качество крекеров и
обоснованности ее применения в рецептуре.

{\bfseries Материалы и методы.} \emph{Функциональные и технологические
свойства используемого сырья}

Арахисовая мука представляет собой ценный функциональный ингредиент
благодаря своему богатому химическому составу и многочисленным полезным
свойствам. Высокое содержание белка и аминокислот обуславливает её
широкое применение в производстве пищевых продуктов, направленных на
повышение их пищевой ценности. Кроме того, значительное количество
пищевых волокон в составе арахисовой муки способствует улучшению
структуры теста, увеличению вязкости и повышению влагоудерживающей
способности, что особенно важно при изготовлении хлебобулочных и
кондитерских изделий. Помимо этого, арахисовая мука служит натуральным
ароматизатором, придавая продуктам характерный ореховый вкус и аромат.
Анализ химического состава позволяет выявить преимущества арахисовой
муки по сравнению с традиционной пшеничной мукой. В таблице 2
представлены основные показатели их химического состава.
\end{multicols}

\begin{table}[H]
\caption*{Таблица 1 - Химический состав муки (мг/100 г)}
\centering
\begin{tblr}{
  cells = {c},
  cell{1}{1} = {r=2}{},
  cell{1}{2} = {r=2}{},
  cell{1}{3} = {r=2}{},
  cell{1}{4} = {r=2}{},
  cell{1}{5} = {r=2}{},
  cell{1}{6} = {r=2}{},
  cell{1}{7} = {c=4}{},
  vlines,
  hline{1,3-5} = {-}{},
  hline{2} = {7-10}{},
}
Мука       & Вода & Белки & Жиры & Углеводы & Зола & Минеральные вещества &      &      &      \\
           &      &       &      &          &      & Ca                   & Mn   & Fe   & Zn   \\
Пшеничная  & 14,0 & 11,20 & 1,35 & 54,37    & 0,45 & 17,88                & 0,57 & 1,22 & 0,73 \\
Арахисовая & 8,0  & 32,60 & 4,83 & 26,70    & 4,8  & 14,2                 & 1,52 & 5,45 & 1,54 
\end{tblr}
\end{table}

\begin{multicols}{2}
Данная таблица 1 демонстрируют существенные различия в составе
арахисовой муки по сравнению с пшеничной. В частности, содержание белка
в арахисовой муке (32,60 мг/100 г) значительно превышает его количество
в пшеничной муке (11,20 мг/100 г), что делает её ценным ингредиентом для
обогащения продуктов белком. Кроме того, арахисовая мука содержит
значительно больше золы (4,8 мг/100 г) по сравнению с пшеничной мукой
(0,45 мг/100 г), что свидетельствует о более высокой концентрации
минеральных веществ. В частности, арахисовая мука содержит больше
марганца, железа и цинка, что усиливает её пищевую ценность.

Органолептические свойства играют важную роль при выборе муки для
различных видов продуктов. В таблице 2 представлены основные различия
между арахисовой и пшеничной мукой по цвету, запаху, вкусу и наличию
минеральных примесей.
\end{multicols}

\begin{table}[H]
\caption*{Таблица 2 - Органолептические характеристики сырья}
\centering
\begin{tblr}{
  colspec = {X[1] X[1] X[1]},
  cells = {c},
  cell{1}{2} = {c=2}{},
  vlines,
  hlines,
}
Наименование показателя                                                        & Показатели муки                                                                &                                                                      \\
                                                                               & Мука пшеничная                                                                 & Мука арахисовая                                                      \\
Цвет                                                                           & Белый                                                                          & Светло-кремовый                                                      \\
Запах                                                                          & Свойственной пшеничной муке, без посторонних запахов, не затхлый, не плесневый & Аромат насыщенный, ореховый, без посторонних или прогорклых оттенков \\
Вкус                                                                           & Свойственной пшеничной муке, без посторонних привкусов, не кислый, не горький  & Приятный, характерный для арахиса, с выраженной ореховой сладостью   \\
Наличие минеральной примеси (при разжевывании муки не должно ощущаться хруста) & Отсутствуют                                                                    & Отсутствуют                                                          
\end{tblr}
\end{table}

\begin{multicols}{2}
Результаты органолептического анализа выявили значительные различия
между двумя видами муки. Пшеничная мука характеризуется белым цветом и
нейтральным вкусом без выраженных ароматических свойств, тогда как
арахисовая мука имеет светло-кремовый оттенок и насыщенный ореховый
аромат. Вкус арахисовой муки более выраженный, с характерной ореховой
сладостью. Оба вида муки продемонстрировали высокую степень чистоты,
поскольку минеральные примеси при разжёвывании не выявлены. Выраженные
вкусо-ароматические свойства арахисовой муки делают её ценным
ингредиентом для обогащения вкусового профиля хлебобулочных и
кондитерских изделий.

Для оценки технологического потенциала арахисовой муки проведено
сравнение её физико-химических характеристик с пшеничной мукой первого
сорта (таблица 3).
\end{multicols}

\begin{table}[H]
\caption*{Таблица 3 - Физико-химические показатели сырья}
\centering
\begin{tblr}{
  cells = {c},
  cell{1}{1} = {r=2}{},
  cell{1}{2} = {c=2}{},
  vlines,
  hlines,
}
Определяемый показатель    & Результаты испытаний муки &            \\
                           & пшеничной                 & арахисовой \\
Массовая доля влаги, \%    & 14,0                      & 8,0        \\
Массовая доля жира, \%     & 1,7                       & 26,1       \\
Массовая доля белка, \%    & 15,3                      & 45,1       \\
Массовая доля углевода, \% & 23,8                      & 10,1       
\end{tblr}
\end{table}

Результаты физико-химического анализа показывают, что арахисовая мука
содержит значительно больше белка (45,1\%) и жира (26,1\%) по сравнению
с пшеничной мукой (15,3\% и 1,7\% соответственно). Эти свойства делают
её перспективным ингредиентом для повышения белково-липидного состава
пищевых продуктов. Кроме того, арахисовая мука характеризуется более
низким содержанием углеводов (10,1\% против 23,8\% в пшеничной муке),
что делает её подходящей для продуктов с пониженным содержанием
углеводов. Низкий уровень влажности (8,0\%) по сравнению с пшеничной
мукой (14,0\%) может оказывать влияние на гидратацию теста и его
текстурные характеристики.

Таким образом, арахисовая мука обладает значительным потенциалом для
обогащения пищевых продуктов белками, жирами и минеральными веществами,
а также улучшения их органолептических и технологических характеристик.
Её использование в пищевой промышленности может способствовать созданию
более питательных и функциональных продуктов, отвечающих современным
требованиям здорового питания.

\emph{Сырьевая база и рецептура}

В ходе исследования изучались технологические параметры производства
крекеров с частичной заменой пшеничной муки первого сорта на арахисовую
муку. В качестве сырьевых компонентов использовались: пшеничная мука 1
сорта (ГОСТ 26574-85), арахисовая мука (ТУ 9146-042-70834238-14),
маргарин, соль, сахар, дрожжи сухие, сода пищевая и вода.

Для экспериментальной части были подготовлены контрольный образец, а
также опытные варианты с заменой 10, 20 и 30\% пшеничной муки на
арахисовую муку согласно рецептуре. Формулы рецептур представлены в
Таблице 4.

\begin{table}[H]
\caption*{Таблица 4 - Рецептурный состав крекеров (г на 1000 г смеси)}
\centering
\begin{tblr}{
  colspec = {X[1] X[1] X[1] X[1] X[1]},
  cells = {c},
  hlines,
  vlines,
}
Компонент              & Контрольный образец & Арахисовая мука 10\% & Арахисовая мука 20\% & Арахисовая мука 30\% \\
Пшеничная мука 1 сорта & 500                 & 450+50               & 400+100              & 350+150              \\
Арахисовая мука        & -                   & 50                   & 100                  & 150                  \\
Маргарин               & 100                 & 100                  & 100                  & 100                  \\
Соль                   & 7                   & 7                    & 7                    & 7                    \\
Сахар                  & 10                  & 10                   & 10                   & 10                   \\
Дрожжи сухие           & 5                   & 5                    & 5                    & 5                    \\
Сода пищевая           & 1                   & 1                    & 1                    & 1                    \\
Вода                   & 200                 & 200                  & 200                  & 200                  
\end{tblr}
\end{table}

\begin{multicols}{2}
\emph{Технологический процесс производства крекеров}

Изготовление крекеров проводилось опарным способом, состоящим из
нескольких технологических этапов, регламентированных стандартами
хлебопекарного производства.

На первом этапе осуществляли приготовление опары, смешивая пшеничную
муку, воду и сухие дрожжи с добавлением части сахара и соли. Ферментация
опары проходила при температуре 28--30 °C в течение 120 минут, что
способствовало активному размножению дрожжевых микроорганизмов и
накоплению углекислого газа (CO₂), необходимого для структурообразования
теста. В результате происходило повышение кислотности среды, что
обеспечивало улучшение физико-химических и реологических свойств
тестовой массы.

После завершения брожения в опару вводили маргарин, остатки сахара и
соли, соду и арахисовую муку в соответствии с рецептурой. Замес теста
осуществляли до получения однородной пластичной консистенции с
оптимальными структурно-механическими характеристиками. Далее тесто
подвергали разделке, раскатке до толщины 4--5 мм и перфорации, что
предотвращало вздутие заготовок при термообработке.

Термическая обработка проводилась в конвекционной печи Tecnoeka MKF 664
BM (Италия) при температуре 160--170 °C в течение 12--15 минут. Данный
температурный режим способствовал предотвращению термического разложения
липидных фракций, содержащихся в арахисовой муке, и обеспечивал
оптимальные условия для протекания реакций Майяра, определяющих цветовую
характеристику и аромат готовых изделий.

После завершения выпекания крекеры охлаждали при температуре 20--22 °C и
упаковывали в герметичные полиэтиленовые пакеты для предотвращения
влагообмена с окружающей средой и окислительных процессов при хранении.

\emph{Методы анализа качества крекеров}

При выполнении работы использовались органолептические и
физико-химические методы исследования. В качестве исследуемых материалов
рассматривали контрольные образцы крекера и опытные образцы крекера с
заменой пшеничной хлебопекарной муки 1 сорта на аналогичное количество
арахисовой муки. Оценку показателей качества осуществляли в соответствии
с нормативными документами: ГОСТ 5897-90 -- методы определения
органолептических показателей качества, ГОСТ 5898-87 -- методы
определения кислотности и щелочности, ГОСТ 5900-73 -- методы определения
влаги и сухих веществ, ГОСТ 10114-80 -- метод определения намокаемости.

Органолептическую оценку выполняли с привлечением дегустационной
комиссии по 5-балльной шкале, учитывая вкус, аромат, цвет, текстуру и
общий внешний вид образцов. Для повышения достоверности полученных
данных исследования проводили в трехкратной повторности. Анализируемые
образцы подвергали одинаковым условиям обработки и хранения, что
позволило минимизировать случайные погрешности и обеспечить
воспроизводимость результатов.

{\bfseries Результаты и обсуждение.} В рамках исследования, направленного
на изучение влияния альтернативного сырья на качество мучных
кондитерских изделий, была проведена модификация рецептуры крекера с
использованием арахисовой муки в качестве обогащающего компонента. В
ходе экспериментов были разработаны три варианта рецептур (Таблица 1),
предусматривающие частичную замену пшеничной муки на арахисовую в
количествах 10\%, 20\% и 30\% от общей массы муки.

Выбранный диапазон замещения обусловлен необходимостью комплексной
оценки влияния различных уровней добавления арахисовой муки на
органолептические, физико-химические и структурно-механические
характеристики готового продукта. Такой подход позволяет установить
оптимальное соотношение компонентов, обеспечивающее улучшение пищевой
ценности изделий без существенного ухудшения их технологических свойств.

Согласно утверждённой рецептуре, представленной в таблице 1, проведены
серии экспериментальных выпекании с различными уровнями замены пшеничной
муки на арахисовую. Технологический процесс включал ферментацию опары
(28--30 °C, 120 мин), после чего в неё добавляли маргарин, сахар, соль,
соду и арахисовую муку. Тесто замешивали до однородной пластичной
консистенции, раскатывали (4--5 мм), перфорировали и формовали
заготовки.

Выпекание проводилось при 160--170 °C (12--15 мин), что обеспечивало
оптимальное формирование цвета, аромата и структуры изделий. Готовые
крекеры охлаждали (20--22 °C) и герметично упаковывали, предотвращая
потерю влаги и окислительные процессы, что способствовало сохранению их
качества. Далее проводили описание и оценку структурных и
органолептических свойств готовых изделий (Таблицы 5, 6)
\end{multicols}

\begin{longtblr}[
  label = none,
  entry = none,
  caption = {\bfseries Таблица 5 - Описание структурных и органолептических характеристик готовых изделий},
]{
  colspec = {X[1] X[1] X[1] X[1] X[1]},
  cells = {c},
  hlines,
  vlines,
}
Показатель                           & Контрольный вариант                                                               & Арахисовая мука 10 \%                                                          & Арахисовая мука 20 \%                                                       & Арахисовая мука 30 \%                                                            \\
Внешний вид и поверхность            & Равномерно пропечённая поверхность, светло-золотистый цвет, без трещин и вздутий. & Светло-золотистый цвет с лёгким коричневым оттенком, однородная структура.     & Более тёмный золотистый цвет с коричневым оттенком, однородная поверхность. & Тёмно-золотистый, выраженный коричневый оттенок, небольшие тёмные включения.     \\
Вид на изломе и внутренняя структура & Плотная и однородная структура, мелкие поры, умеренная ломкость.                  & Мелкопористая равномерная структура, незначительные включения арахисовой муки. & Менее плотная структура, включения арахисовой муки визуально заметны.       & Рыхлая структура, отчётливо выраженные включения арахисовой муки.                \\
Консистенция и текстура              & Оптимальная хрусткость с умеренной пластичностью, структура стабильная.           & Слегка повышенная рассыпчатость, сохраняет пластичность.                       & Повышенная хрупкость, снижение эластичности, рыхлая текстура.               & Повышенная рассыпчатость, значительно снижена эластичность.                      \\
Вкус                                 & Нейтральный, сбалансированный вкус с умеренной солоноватостью.                    & Лёгкий ореховый привкус, умеренная солоноватость, гармоничный профиль.         & Выраженный ореховый вкус, сбалансированный солоновато-ореховый профиль.     & Ярко выраженный ореховый вкус, с насыщенным послевкусием.                        \\
Аромат                               & Слабовыраженный, характерный для выпечных изделий.                                & Слабовыраженный ореховый аромат, дополняющий основной вкус.                    & Умеренно выраженный ореховый аромат, усиливающийся при разжёвывании.        & Интенсивный ореховый аромат, преобладает над другими вкусовыми характеристиками. \\
Оценка хрупкости и рассыпчатости     & Умеренная хрупкость, сохраняет форму при механическом воздействии.                & Ломкость умеренная, структура стабильная.                                      & Ломкость несколько выше, заметная рассыпчатость.                            & Максимальная хрупкость, структура нестабильная, легко крошится.                  
\end{longtblr}

\begin{table}[H]
\caption*{Таблица 6 - Оценка структурных и органолептических характеристик готовых изделий}
\centering
\begin{tblr}{
  colspec = {X[2] X[1] X[1] X[1] X[1]},
  cells = {c},
  hlines,
  vlines,
}
Показатель                           & Контрольный вариант (0\%) & Арахисовая мука 10 \% & Арахисовая мука 20 \% & Арахисовая мука 30 \% \\
Внешний вид и поверхность            & 4.5                       & 4.6                   & 4.3                   & 3.8                   \\
Вид на изломе и внутренняя структура & 4.5                       & 4.6                   & 4.2                   & 3.7                   \\
Консистенция и текстура              & 4.7                       & 4.5                   & 4.0                   & 3.5                   \\
Вкус                                 & 4.0                       & 4.5                   & 4.6                   & 4.2                   \\
Аромат                               & 3.8                       & 4.2                   & 4.5                   & 4.6                   \\
Оценка хрупкости и рассыпчатости     & 4.8                       & 4.5                   & 4.0                   & 3.5                   
\end{tblr}
\end{table}

\begin{multicols}{2}
Данные таблицы представляют результаты описания и оценки структурных и
органолептических характеристик готовых изделий с различным уровнем
замены пшеничной муки на арахисовую. Оценка готовых издалий проводилась
дегустационной комиссией из 10 экспертов согласно требованиям по
5-балльной шкале, где 5 -- полное соответствие стандарту качества, а 1
-- неудовлетворительный результат. Готовые изделия были оценены и
описаны дегустационной комиссией в соответствии с требованиями ГОСТ
5897-90.

Таким образом, наилучшие органолептические характеристики наблюдаются
при частичной замене пшеничной муки на арахисовую в пределах
{\bfseries 10--20\%}. Данный диапазон обеспечивает улучшение вкуса и
аромата без значительных изменений текстуры и механической стойкости.
При 30\% замене отмечаются существенные изменения в структуре,
выраженная рассыпчатость и доминирование орехового вкуса, что может
ограничивать применение такой рецептуры в промышленном производстве.
Рекомендуемый уровень замены пшеничной муки на арахисовую для
оптимального баланса качественных характеристик -- 10--20\%.

Оценка качества готовой продукции является важным этапом в анализе
влияния частичной замены пшеничной муки на арахисовую в рецептуре мучных
изделий. Для объективного контроля были использованы стандартизированные
методы, регламентированные нормативными документами, обеспечивающими
точность и воспроизводимость результатов.

\emph{Определение массовой доли влаги в готовых изделиях}

Определение массовой доли влаги является одним из ключевых параметров,
определяющих качество и стабильность мучных изделий при хранении.
Согласно данным, представленным на рисунке 1, наблюдается закономерное
снижение содержания влаги с увеличением концентрации арахисовой муки в
рецептуре.
\end{multicols}

{\bfseries Рис. 1 - Массовая доля влаги в готовых изделиях, \%}

В контрольном образце данный показатель составил 7,0\%, в то время как
при замещении 10\%, 20\% и 30\% пшеничной муки арахисовой массовая доля
влаги последовательно уменьшалась до 6,7\%, 6,4\% и 6,1\%
соответственно. Этот эффект объясняется гидрофобными свойствами
арахисовой муки, которая снижает способность теста удерживать влагу за
счёт изменения структуры клейковинного каркаса и влагосвязывающей
способности белков.

Снижение содержания влаги в готовых изделиях оказывает значительное
влияние на их микробиологическую стабильность и срок хранения.
Уменьшение доступной влаги снижает активность воды, тем самым подавляя
рост микроорганизмов и замедляя окислительные процессы, что является
важным фактором сохранения качества продукта. Данный результат указывает
на потенциальную возможность продления срока хранения мучных изделий при
использовании арахисовой муки в количестве 10--30\%.

\emph{Определение щелочности готовых изделий}

Щелочность является важным показателем, отражающим изменения
кислотно-щелочного баланса теста и готового изделия, что в свою очередь
влияет на физико-химические свойства продукта. Согласно данным,
представленным на рисунке 2, наблюдается линейная зависимость между
уровнем замещения пшеничной муки арахисовой и увеличением щелочности.

{\bfseries Рис. 2 - Щелочность готовых изделий, град}

В контрольном образце уровень щелочности составил 0,86 градуса, при
добавлении 10\% арахисовой муки этот показатель увеличился до 0,91
градуса, а при 20\% и 30\% -- до 1,05 и 1,16 градуса соответственно.

Рост щелочности может быть обусловлен наличием в арахисовой муке
соединений, обладающих буферными свойствами, таких как белки и фосфатные
соли, которые способны изменять кислотно-щелочной баланс теста. Это
изменение может повлиять на органолептические характеристики изделий,
включая вкус и аромат, а также способствовать улучшению текстурных
свойств, увеличивая пластичность теста.

Полученные результаты позволяют сделать вывод о том, что введение
арахисовой муки может быть целесообразным для целенаправленной
модификации химических и органолептических характеристик мучных изделий.

\emph{Определение намокаемости готовых изделий}

Показатель намокаемости отражает способность готовых изделий впитывать
жидкость, что может играть важную роль в оценке текстуры, пористости и
механической стабильности продукта при взаимодействии с жидкими средами.

Как показано на рисунке 3, намокаемость контрольного варианта составила
110,3\%, в то время как при введении 10\% арахисовой муки данный
показатель увеличился до 130,7\%, что на 18,5\% больше. При замещении
20\% и 30\% намокаемость увеличивалась до 135,5\% и 140,2\%
соответственно, что на 29,9\% выше, чем в контрольном образце.

{\bfseries Рис. 3 - Намокаемость готовых изделий, \%}

\begin{multicols}{2}
Рост намокаемости связан с гидрофильными свойствами компонентов
арахисовой муки, таких как белки и полисахариды, которые обладают
способностью удерживать влагу и увеличивать водопоглощающую способность
теста. Увеличение этого показателя может улучшать взаимодействие изделий
с жидкими средами, что особенно важно для продуктов, предназначенных для
употребления с напитками (например, чаем или кофе).

Однако чрезмерное увеличение намокаемости может привести к потере
хрустящих свойств, что негативно скажется на текстурных характеристиках.
Таким образом, при разработке рецептур необходимо учитывать оптимальный
баланс между водопоглощением и структурной стабильностью изделий, не
допуская чрезмерного увеличения влагопоглощения.

На основании полученных экспериментальных данных можно сделать вывод о
том, что оптимальным уровнем замещения пшеничной муки на арахисовую
является 10--20\%, так как именно в этом диапазоне наблюдается улучшение
функционально-технологических и органолептических свойств изделий при
минимальном ухудшении текстурных характеристик.

Таким образом, наиболее оптимальной является замена пшеничной муки на
арахисовую в количестве 10--20\%. Этот диапазон позволяет улучшить
пищевую ценность изделий, увеличить срок их хранения и усилить вкусовые
характеристики без значительного ухудшения текстуры. Замена 30\%
пшеничной муки на арахисовую не рекомендуется из-за чрезмерной ломкости,
потери хрустящей структуры и выраженного влияния на вкус.

Результаты исследования показывают, что использование арахисовой муки в
количестве 10--20\% может быть эффективным решением для разработки
функциональных мучных изделий, обладающих повышенной пищевой ценностью,
стабильными органолептическими характеристиками и улучшенной
сохранностью.

{\bfseries Выводы.} В результате проведённых исследований установлено, что
частичная замена пшеничной муки на арахисовую в рецептуре крекеров
оказывает значительное влияние на физико-химические, органолептические и
структурно-механические характеристики готовых изделий. Оптимальным
уровнем замены пшеничной муки на арахисовую является 10--20\%, поскольку
именно в этом диапазоне наблюдается улучшение вкусовых и ароматических
характеристик при сохранении стабильной текстуры и механических свойств.
Введение 10--20\% арахисовой муки способствует уменьшению массовой доли
влаги, что повышает микробиологическую устойчивость и срок хранения
продукции. При этом увеличение щелочности не оказывает негативного
влияния на органолептические свойства, а умеренное повышение
намокаемости улучшает взаимодействие изделия с жидкими средами без
потери хрустящей структуры.

Однако при увеличении замены пшеничной муки до 30\% наблюдаются
негативные изменения, включая чрезмерную хрупкость и сухость, что
ухудшает механическую стабильность изделий. Щелочность возрастает до
1,16 градуса, что может привести к выраженному изменению вкуса, а рост
намокаемости на 29,9\% способствует избыточному поглощению влаги, что
отрицательно сказывается на текстуре.

Таким образом, использование 10--20\% арахисовой муки позволяет повысить
пищевую ценность изделий, увеличить их устойчивость к микробиологической
порче, сбалансировать вкусовые характеристики и сохранить текстурную
стабильность. Замена 30\% пшеничной муки на арахисовую не рекомендуется
для промышленного производства из-за выраженного изменения структуры и
ухудшения механических свойств продукта. Полученные результаты
подтверждают перспективность использования арахисовой муки для
обогащения мучных изделий и целенаправленной модификации их химических и
органолептических свойств. Однако для успешного промышленного внедрения
данной технологии необходимы дополнительные исследования, направленные
на оптимизацию технологических параметров и обеспечение максимального
качества продукции.
\end{multicols}

\begin{center}
{\bfseries References}
\end{center}

\begin{references}
1.Daud Suleman S. B., Shah F. U. H., Ikram A., Shahid M. Z., Tufail T.,
Khan A. A., Mohamed M. H. Nutritional and functional properties of
cookies enriched with defatted peanut cake flour // Cogent Food \&
Agriculture. -2023.- Vol.9(1) DOI
\href{https://doi.org/10.1080/23311932.2023.2238408}{10.1080/23311932.2023.2238408}.

2.Sobukola O., Ajayi F., Kayode O., Faloye O. Effect of processing
conditions on some quality attributes of fried cassava-defatted peanut
crackers // Croatian Journal of Food Science and Technology.-
2021.- Vol.13(1).- P. С. 111-121. DOI
\href{https://doi.org/10.17508/CJFST.2021.13.1.14}{10.17508/CJFST.2021.13.1.14}.

3.Shongwe S. G., Kidane S. W., Shelembe J. S., Nkambule T. P. Dough
rheology and physicochemical and sensory properties of wheat--peanut
composite flour bread // Legume Science-. 2022.- Vol.4(3): Е138.
DOI \href{https://doi.org/10.1002/leg3.138}{10.1002/leg3.138}.

4.Katyal M., Singh N., Singh H. Effects of incorporation of groundnut
oil and hydrogenated fat on pasting and dough rheological properties of
flours from wheat varieties // Journal of Food Science and
Technology\emph{.} -2019. - Vol.56. - P.1056--1065. DOI
\href{https://doi.org/10.1007/s13197-019-03633-9}{10.1007/s13197-019-03633-9}.

5.Arukwe D. C., Ezeocha V. C., Obiasogu S. P. Production and quality
evaluation of snacks from blends of groundnut cake and pigeon pea flour
// Journal of Agriculture and Food Sciences.- 2023.- Vol. 21(1).
- P. 90-113. DOI
\href{https://doi.org/10.4314/jafs.v21i1.7}{10.4314/jafs.v21i1.7}.

6.Granato D., Ellendersen L. D. S. N. Almond and peanut flours
supplemented with iron as potential ingredients to develop gluten-free
cookies // Food Science and Technology. -2009. --Vol. 29.- P. 395-400.
DOI
\href{https://doi.org/10.1590/S0101-20612009000200026}{10.1590/S0101-20612009000200026}.

7.Kahlon, T. S., Avena-Bustillos, R. J., Kahlon, A. K., \& Brichta, J.
L. Consumer sensory evaluation and quality of Sorghum-Peanut Meal-Okra
snacks//~Heliyon.-2021/-Vol.~7(5).
\href{https://doi.org/10.1016/j.heliyon.2021.e06874}{DOI\\
10.1016/j.heliyon.2021.e06874}.

8.Dada M. A., Bello F. A., Omobulejo F. O., Olukunle F. E. Nutritional
quality and physicochemical properties of biscuit from composite flour
of wheat, African yam bean and tigernut // Heliyon. -2023.-
Vol.9(11): e22477 DOI
\href{https://doi.org/10.1016/j.heliyon.2023.e22477}{10.1016/j.heliyon.2023.e22477}.

9.Wilkin J. The effects of storage and processing on the properties of
Arachis hypogeae (peanut): {[}Thesis{]} // Cardiff Metropolitan
University. 2013.- 201 p. DOI
\href{https://doi.org/10.25401/cardiffmet.20455623.v1}{10.25401/cardiffmet.20455623.v1}

10.Howard B. M., McWatters K. H., Saalia F., Hashim I. Formulation and
evaluation of snack crackers made with peanut flour // Cereal Foods
World. -2009. -- Vol. 54(4).- P. 166--171. DOI
\href{https://doi.org/10.1094/CFW-54-4-0166}{10.1094/CFW-54-4-0166}.

11.Sengev I. A., Damsa M. A., Bunde-Tsegba M. C. Evaluation of
proximate, physical and sensory \\properties of snacks produced from
wheat, banana puree and roasted peanut grits // Nigerian Food Journal.-
2024.- Vol.42(2) DOI:
\href{https://openurl.ebsco.com/EPDB\%3Agcd\%3A9\%3A2629225/detailv2?sid=ebsco\%3Aplink\%3Ascholar&id=ebsco\%3Agcd\%3A182511032&crl=c&link_origin=none}{10.4314/nifoj.v42i2.3}.
-Available at:
\href{https://openurl.ebsco.com/EPDB\%3Agcd\%3A9\%3A2629225/detailv2?sid=ebsco\%3Aplink\%3Ascholar&id=ebsco\%3Agcd\%3A182511032&crl=c&link_origin=none}{EBSCOhost}
(accessed March 15, 2025).

12.Mustapha B. O., Aderibigbe O. R., Idowu O. O., Otunla C. A.
Evaluation of the nutritional and chemical composition of crackers
developed from blends of wheat, unripe plantain, and mushroom flour //
Food and Humanity. \emph{-}2024.-Vol.3:100379. DOI
\href{https://doi.org/10.1016/j.foohum.2024.100379}{10.1016/j.foohum.2024.100379}.

13.Makahity H., Tuhumury H. C. D., Palijama S. The effects of margarine
substitution with peanut paste on the characteristics of sago cookies //
Journal of Applied Agricultural Science and Technology.
\emph{-}2024.-Vol.8(2).- P. 159-174. DOI:
\href{https://doi.org/10.55043/jaast.v8i2.276}{10.55043/jaast.v8i2.276}.

14.Suleman D., Bashir S., Hassan Shah F. U., Ikram A., Zia Shahid M.,
Tufail T., Hassan Mohamed M. Nutritional and functional properties of
cookies enriched with defatted peanut cake flour // Cogent Food \&
Agriculture. -2023.-Vol. 9(1): 2238408. DOI
\href{https://doi.org/10.1080/23311932.2023.2238408}{10.1080/23311932.2023.2238408}.
\end{references}

\begin{authorinfo}
\emph{{\bfseries Сведения об авторах}}

Мулдабекова Б.Ж. - к.т.н., профессор, Алматинский Технологический
Университет, Алматы, Казахстан, e-mail:\\
\href{mailto:bayan_1004@mail.ru}{\nolinkurl{bayan\_1004@mail.ru}};

Султанкул М.Б.- магистрант, Алматинский Технологический Университет,
Алматы, Казахстан, e-mail:
\href{mailto:m.sultankul@bk.ru}{\nolinkurl{m.sultankul@bk.ru}};

Куртибай К.А. - магистрант, Евразийский национальный университет имени
Л.Н. Гумилева, Астана, Казахстан, e-mail:
\href{mailto:kurtibayqb@gmail.com}{\nolinkurl{kurtibayqb@gmail.com}};

Қаххоров Ф.Б. {\bfseries -} ассистент, Джизакский политехнический институт,
Джизак, Узбекистан, е-mail:\\
\href{mailto:issaqovshokir93@gmail.com}{\nolinkurl{issaqovshokir93@gmail.com}};

Исматуллаева Г.Х.- магистрант, Алматинский Технологический Университет,
Алматы, Казахстан, e-mail:\\
\href{mailto:ismatullayeva02@inbox.ru}{\nolinkurl{ismatullayeva02@inbox.ru}};

Жумабекова А.Т. -магистрант, Алматинский Технологический Университет,
Алматы, Казахстан, e-mail:\\
\href{mailto:zhumabekova.aikhanym@mail.ru}{\nolinkurl{zhumabekova.aikhanym@mail.ru}}

\emph{{\bfseries Information about the authors}}

Muldabekova B.Zh. - Candidate of Technical Sciences, Professor, Almaty
Technological University, Almaty, Kazakhstan, e-mail:
bayan\_1004@mail.ru;

Sultankul M.B. - Master' s Student, Almaty Technological
University, Almaty, Kazakhstan, e-mail: m.sultankul@bk.ru;

Kurtibaу K.A. - Master' s Student, L.N. Gumilyov Eurasian
National University, Astana, Kazakhstan, e-mail: \\kurtibayqb@gmail.com;

Qaxxorov F. - Assistant, Jizzakh Polytechnic Institute, Jizzakh,
Uzbekistan, е-mail:
\href{mailto:issaqovshokir93@gmail.com}{\nolinkurl{issaqovshokir93@gmail.com}};

Ismatullaeva G.Kh.- Master' s Student, Almaty
Technological University, Almaty, Kazakhstan, e-mail:\\
ismatullayeva02@inbox.ru;

Zhumabekova A.T.- Master' s Student, Almaty Technological
University, Almaty, Kazakhstan, e-mail: \\zhumabekova.aikhanym@mail.ru;
\end{authorinfo}
