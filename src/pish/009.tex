%% DONE
\id{IRSTI 65.33.35}{}

\begin{articleheader}
\sectionwithauthors{A.M. Omaraliyeva, A.Zh. Khastaeva, A.A. Bekturganova, A.M. Rakhimzhanova, Zh.T. Botbayeva}{JUSTIFICATION AND SELECTION OF RAW MATERIALS FOR THE PRODUCTION OF FLOUR CONFECTIONERY PRODUCTS}

{\bfseries
A.M. Omaraliyeva\textsuperscript{\envelope } \authorid,
A.Zh. Khastaeva\authorid,
A.A. Bekturganova\authorid,
A.M. Rakhimzhanova\authorid,
Zh.T. Botbayeva\authorid}
\end{articleheader}

\begin{affiliation}
\emph{K.Kulazhanov named Kazakh University of Technology and Business, Astana, Kazakhstan,}

\raggedright \textsuperscript{\envelope }{\em Corresponding author: aigul-omar@mail.ru}
\end{affiliation}

The article discusses the rationale and selection of raw materials for
the production of flour confectionery in Kazakhstan. The main focus is
on the popularity of cookies and gingerbread among consumers, due to
their availability and low cost. However, the authors emphasize that
traditional flour confectionery products are overloaded with
carbohydrates and lack the content of micro- and macronutrients,
vitamins and dietary fiber that are important for the body.

The implementation of proposals to improve the nutritional value of
these products consists in the introduction of herbal additives into the
formulation, which makes it possible to increase the content of
essential nutrients and improve the overall composition of the product.
In particular, it is planned to use legumes and oilseeds as the main raw
materials, as well as powders of melons and fruit crops as fillers. This
will significantly increase the nutritional properties of cookies, as
well as reduce their calorie content, making the products more useful
for consumers.

It is very important that Kazakhstan focuses on local raw materials and
cultivated crops, which allows not only to improve and increase the
range of baked goods, but also is one of the factors supporting local
producers from an economic point of view. The authors of the article
refer to the principles of food com binatorics, which emphasize the
importance of developing and implementing new recipes to improve the
characteristics of confectionery products.

It is very important that Kazakhstan focuses on local resources and
grown crops, which can not only improve the range of baked goods, but
also support local producers. These studies show the importance of
integrating scientific approaches into traditional production processes,
which in turn can contribute not only to improving the quality of food,
but also to strengthening food security in the country. To summarize, we
can say that the introduction of new technologies and the use of
alternative raw materials is an important step towards creating more
nutritious and environmentally friendly flour products that can meet the
needs of the modern consumer.

{\bfseries Keywords:} flour confectionery, grain and legume crops,
zucchini, melon, apples, apricots, raspberries, sea buckthorn, currants,
dry powder

\begin{articleheader}
{\bfseries ҰННАН ЖАСАЛҒАН КОНДИТЕРЛІК ӨНІМДЕРДІ ӨНДІРУГЕ АРНАЛҒАН ШИКІЗАТТЫ НЕГІЗДЕУ ЖӘНЕ ТАҢДАУ}

{\bfseries
А.М.Омаралиева\textsuperscript{\envelope },
А.Ж.Хастаева,
А.А.Бектурганова,
А.М.Рахимжанова,
Ж.Т. Ботбаева}
\end{articleheader}

\begin{affiliation}
\emph{Қ.Құлажанов атындағы Қазақ технология және бизнес университеті, Астана, Қазақстан,}

\emph{e-mail:aigul-omar@mail.ru}
\end{affiliation}

Мақалада Қазақстанда ұннан жасалған кондитерлік өнімдерді өндіруге
арналған шикізатты таңдау мәселесі қарастырылады. Негізгі назар
тұтынушылар арасында печенье мен пряниктің танымалдылығына аударылған,
бұл олардың қолжетімділігі мен төмен бағасымен түсіндіріледі. Алайда,
авторлар дәстүрлі ұннан жасалған кондитерлік өнімдердің көмірсуларға бай
екенін және адам ағзасына қажетті микро- және макроэлементтер,
дәрумендер мен тағамдық талшықтардың жеткіліксіздігін атап өтеді.

Аталған өнімдердің тағамдық құндылығын арттыру бойынша ұсыныстарды іске
асыру рецептураға өсімдік текті қоспаларды енгізу арқылы жүзеге
асырылады, бұл қажетті заттардың мөлшерін арттыруға және өнімнің жалпы
құрамын жақсартуға мүмкіндік береді. Атап айтқанда, негізгі шикізат
ретінде дәнді-бұршақты және майлы дақылдарды, ал толтырғыш ретінде бақша
дақылдары мен жеміс-жидек ұнтақтарын қолдану ұсынылады. Бұл печеньенің
қоректік қасиеттерін айтарлықтай жақсартуға, сондай-ақ оның калориялық
құрамын өзгертуге мүмкіндік береді, нәтижесінде өнімдер тұтынушылар үшін
пайдалырақ болады.

Негізгі екпін ұннан жасалған кондитерлік өнімдерді, әсіресе кеңінен
тұтынылатын өнімдерді байыту олардың сапасы мен тағамдық құндылығын
арттыру үшін маңызды фактор екеніне қойылған. Мақала авторлары тағамдық
комбинаторика қағидаларына сүйене отырып, кондитерлік өнімдердің
сипаттамаларын жақсарту үшін жаңа рецептураларды әзірлеу мен енгізудің
маңыздылығын атап көрсетеді.

Қазақстанда жергілікті ресурстар мен өсірілетін дақылдарға бағдарлану
өте өзекті, бұл тек нан-тоқаш өнімдерінің ассортиментін жақсартуға ғана
емес, сонымен қатар жергілікті өндірушілерді қолдауға да мүмкіндік
береді. Зерттеулер дәстүрлі өндіріс процестеріне ғылыми тәсілдерді
енгізудің маңыздылығын көрсетеді, бұл өз кезегінде тағам сапасын
арттырумен қатар елдегі азық-түлік қауіпсіздігін нығайтуға да ықпал ете
алады. Қорытындылай келе, жаңа технологияларды енгізу және баламалы
шикізатты пайдалану -- қазіргі заманғы тұтынушылардың қажеттіліктерін
қанағаттандыра алатын дәрумендерге бай және экологиялық таза ұн
өнімдерін жасау жолындағы маңызды қадам болып табылады.

{\bfseries Түйін сөздер:} ұннан жасалған кондитерлік өнімдер, дәнді және
дәнді-бұршақты дақылдар, асқабақ, қауын, алма, өрік, таңқурай, шырғанақ,
қарақат, құрғақ ұнтақ.

\hl{}

\begin{articleheader}
{\bfseries ОБОСНОВАНИЕ И ВЫБОР СЫРЬЯ ДЛЯ ПРОИЗВОДСТВА МУЧНЫХ КОНДИТЕРСКИХ ИЗДЕЛИЙ}

{\bfseries
А.М. Омаралиева\textsuperscript{\envelope },
А.Ж. Хастаева,
А.А. Бектурганова,
А.М. Рахимжанова,
Ж.Т. Ботбаева}
\end{articleheader}

\begin{affiliation}
\emph{Казахский университет технологии и бизнеса им. К. Кулажанова, Астана, Казахстан,}

\emph{e-mail: aigul-omar@mail.ru}
\end{affiliation}

В статье рассматривается обоснование и подбор сырья для производства
мучных кондитерских изделий в Казахстане. Основное внимание уделяется
популярности печенья и пряников среди потребителей, что обусловлено их
доступностью и невысокой стоимостью. Однако авторы подчеркивают, что
традиционные мучные кондитерские изделия перегружены углеводами и
недостаточны по содержанию важных для организма микро- и макроэлементов,
витаминов и пищевых волокон.

Реализация предложений по улучшению пищевой ценности данных изделий
заключается в введении в рецептуру добавок растительного происхождения,
что позволяет увеличить содержание необходимых полезных веществ и
улучшить общий состав продукта. В частности, предполагается использовать
зернобобовые и масличные культуры в качестве основного сырья, а также
порошки бахчевых и плодово-ягодных культур в качестве наполнителей. Это
позволит значительно повысить питательные свойства печенья, а также
снизить его калорийность, делая продукты более полезными для
потребителей.

Основной акцент делается на том, что обогащение мучных изделий, особенно
тех, которые потребляются широкими слоями населения, является ключевым
моментом для повышения их качества и пищевой ценности. Авторы статьи
ссылаются на принципы пищевой комбинаторики, которые подчеркивают
важность разработки и внедрения новых рецептур для улучшения
характеристик кондитерских изделий.

Весьма актуально, что в Казахстане ориентируются на местные сырьевые
ресурсы и выращиваемые культуры, что позволяет не только улучшить и
увеличить ассортимент выпечки, но и является одним из факторов поддержки
местных производителей с экономической точки зрения. Данные исследования
показывают важность интеграции научных подходов в традиционные процессы
производства, что в свою очередь может способствовать не только
повышению качества пищи, но и укреплению продовольственной безопасности
в стране. Подводя итог, можно сказать, что внедрение новых технологий и
использованием альтернативного сырья --- это важный шаг к созданию более
питательных и экологически чистых мучных продуктов, способных
удовлетворить потребности современного потребителя.

{\bfseries Ключевые слова:} мучные кондитерские изделия, зерновые и
зернобобовые культуры, кабачки, дыня, яблоки, абрикосы, малина,
облепиха, смородина, сухой порошок

\begin{multicols}{2}
{\bfseries Introduction.} In Kazakhstan, the most popular flour
confectionery product for most consumers is cookies and gingerbread.
This is due to the relatively low cost and, accordingly, availability to
a wide consumer audience. However, these products are overloaded with
carbohydrates, contain an insufficient amount of micro- and
macroelements, vitamins and dietary fiber.

To increase the nutritional value, additives containing essential
nutrients are introduced into the recipe of flour products. Plant-based
raw materials containing many important macro- and micronutrients are
best suited for this purpose. These can be various fruit and berry and
melon crops. In this regard, it is advisable to continue research in the
field of developing technology for the production of flour confectionery
products. Legumes and oilseeds will be used as the main raw materials
for the production of flour confectionery products, and powders of melon
and fruit and berry crops grown in Kazakhstan will be used as fillers.

The use of non-traditional raw materials allows changing the caloric
content of cookies, increases the content of dietary fiber, macro- and
microelements, vitamins and other minor substances.

According to the principles of food combinatorics {[}1-5{]}, flour
products of mass demand, those products that are consumed by a large
number of different segments of the population, are subject to
enrichment. Much research has been devoted to improving the quality of
confectionery products, increasing their nutritional and biological
value. Thus, in many countries, bakery and flour confectionery products
are modified by adding dietary fiber, vitamins, and minerals to the
recipes of these products, using various cereals, processed vegetable
products, and oilseeds - sesame, sunflower, etc. {[}2{]}.

The use of melon and fruit and berry powder in the processing of grain
products is an optimal solution for improving their nutritional
qualities with health benefits for consumers, and is increasingly being
used by processors in this area {[}6, 7{]}. Along with increasing the
nutritional value, adding melon and fruit and berry crops to baking
dough improves the physical, chemical, sensory and microbiological
properties of finished products {[}8, 9, 10{]}. Due to the richness of
nutrients, the consumption of functional products made from melon and
fruit and berry powders can prevent some diseases {[}6, 11, 12{]}. Based
on the above, it can be noted that the problem of forming an assortment
of functional flour confectionery products cannot be considered solved
at present.

An analysis of scientific literature on this topic shows that there are
many studies on the use of plant materials in the technology of flour
confectionery products. For example, in the work {[}13{]} it is proposed
to use biologically active substances and dietary fibers isolated from
fruit and vegetable powders in the technology of flour confectionery
products for the purpose of enriching the products.

S.E. Kharkov, V.V. Gonchar, I.V. Roslyakov developed technologies for
brewed gingerbread products using non-traditional raw materials, such as
flour from powdered melon seeds, which helps improve the organoleptic
and physicochemical properties of finished products {[}14{]}.

Scientists from the Kuban State Technological University have proven the
value of processed melon crops -- watermelons and pumpkins -- as
functional ingredients through their research. They have developed a
food additive based on watermelon pomace and its seeds, which has
membrane-protective, antitoxic and radioprotective properties. The use
of this additive in bakery products is facilitated by the fact that this
food additive is highly soluble in water. This ability appears as a
result of processing the raw materials in a rotary roller disintegrator
{[}15{]}.

Pumpkin processing products are widely used in the technology of bread,
bakery and flour confectionery. Pumpkin pomace, seeds, and fermented
pumpkin pulp are used as enriching ingredients {[}16-19{]}. The use of
pumpkin in the technology of bread products is also the subject of the
works of I.B. Isabaev. He developed a method for producing rusks with
pumpkin puree additives. The finished products had not only good
organoleptic properties, but also an optimal calcium and magnesium ratio
{[}19{]}.

Thus, the use of dry powders based on melons and fruit crops as
ingredients that increase the nutritional value of flour confectionery
products is due to the fact that they are harmless additives of natural
origin and, based on the results of the literature review, flour
products will be developed using non-traditional raw materials.

{\bfseries Materials and methods.} The objects of the study are:

- grain and leguminous crops such as rice, oats, buckwheat, barley,
corn, chickpeas, lentils;

- melons - pumpkin, squash, melon;

- fruit crops - apples, apricots;

- berries - raspberries, sea buckthorn, currants.

During the research, traditional methods of assessing the quality of
plant materials were used.

GOST 7975-2013 Fresh edible pumpkin. Specifications; GOST 7178-2015
Fresh melons. Specifications; GOST 31822-2012 Fresh Courgettes sold in
retail. Specifications; GOST 34314-2017 Fresh apples sold in retail.
Specifications; GOST 27572 - 2017 Fresh apples for industrial
processing. Specifications; GOST 32787-2014 Fresh apricots.
Specifications; GOST 33915-2016 Fresh raspberries and blackberries.
Specifications; GOST 6829-2015 Fresh black currants. Specifications;
GOST 33954-2016 Fresh red and white currants; GOST R 59661-2021 Fresh
sea buckthorn. Technical conditions; TR CU 021/2011 Technical
regulations of the Customs Union "On the safety of food products"; GOST
13586.3-2015 Grain. Acceptance rules and sampling methods.

{\bfseries Results and discussion.} According to the data of the conducted
literature review, at the present stage, raw materials of plant origin
play an important role in the creation of food products. From this
position, grain and leguminous crops are a promising crop in the
Republic of Kazakhstan.

From the main growing regions of grain (wheat, barley, oats, buckwheat,
corn, rice) and legumes (chickpeas, lentils) crops, the following
species and varieties were selected based on their technological
characteristics for the production of flour confectionery:

- barley varieties "Sabir" (Akmola region), "Arpa elite" (Almaty
region);

- buckwheat varieties "Saulyk" (Almaty region), "Batyr" (Akmola region);

- lentil varieties "Stepnaya" (Almaty region), "L-4 400" (Kostanay
region);

- chickpea variety "Ersultan" (Almaty region);

- corn variety "Dobrynya" (Almaty region);

- rice varieties ``Syr Syluy'', ``Aikerim'' and ``Marzhan'' (Kyzylorda
region);

- oats varieties ``Duman'' and ``Bitik'' (Akmola region); "Arman"
(Kostanay region).

The chemical composition of the selected varieties was analyzed. The
results of the analysis are summarized in Table 1.

The analysis of the presented table -- 1 shows that domestic varieties
of grain and leguminous crops have a high content of mass fraction of
protein. Studies of grain crop samples show that the maximum value of
mass fraction of protein corresponded to the grain of buckwheat of the
variety ``Saulyk'' (Almaty region) and amounted to 12.5\%,
\end{multicols}

\begin{table}[H]
\caption*{Table 1 - Results of the chemical analysis of the selected grain and leguminous crops}
\centering
\begin{tblr}{
  colspec = {X[0.2] X[3] X[1] X[1] X[0.5] X[1] X[1] X[1]},
  cells = {c},
  cell{2}{1} = {c=8}{},
  cell{5}{1} = {c=8}{},
  cell{7}{1} = {c=8}{},
  cell{9}{1} = {c=8}{},
  cell{12}{1} = {c=8}{},
  cell{15}{1} = {c=8}{},
  cell{19}{1} = {c=8}{},
  hlines,
  vlines,
}
№         & Name/variety                    & Humidity, \% & Protein, \% & Fat,\% & Fiber,\% & Starch content,\% & Ash content,\% \\
Lentils   &                                 &              &             &        &          &                   &                \\
1         & "Stepnaya" (Almaty region)      & 5,7          & 27,8        & 2,6    & 3,6      & 47,2              & 2,6            \\
2         & "L-4 400" (Kostanay region)     & 6            & 27,8        & 2,4    & 3,2      & 46,8              & 2,4            \\
Chickpeas &                                 &              &             &        &          &                   &                \\
3         & "Ersultan" (Almaty region)      & 5,3          & 24,6        & 5,8    & 5        & 38,5              & 0,9            \\
Corn      &                                 &              &             &        &          &                   &                \\
4         & "Dobrynya" (Almaty region)      & 6,4          & 7,7         & 1,3    & 1,7      & 64,4              & 1,36           \\
Barley    &                                 &              &             &        &          &                   &                \\
5         & "Arpa elite" (Almaty region)    & 6,8          & 10,2        & 3,5    & 9,4      & 51,9              & 3,2            \\
6         & "Sabir" (Akmola region)         & 6            & 10          & 2,3    & 8,9      & 51,8              & 2,2            \\
Buckwheat &                                 &              &             &        &          &                   &                \\
7         & "Saulyk" (Almaty region)        & 9,5          & 12,5        & 3,3    & 9,1      & 47,9              & 1,8            \\
8         & "Batyr" (Akmola region)         & 9            & 10,8        & 3      & 9        & 48,7              & 2              \\
Rice      &                                 &              &             &        &          &                   &                \\
9         & “Syr Syluy” (Kyzylorda region); & 9,7          & 7,96        & 1,2    & 2,8      & 68,3              & 0,41           \\
10        & “Aikerim” (Kyzylorda region)    & 10,5         & 6,62        & 1,79   & 2,5      & 70,1              & 0,45           \\
11        & “Marzhan” (Kyzylorda region)    & 11,5         & 7,4         & 1,93   & 2,7      & 69,4              & 0,48           \\
Oats      &                                 &              &             &        &          &                   &                \\
12        & “Duman” (Akmola region)         & 10,2         & 12,77       & 1,8    & 9,2      & 57,5              & 2,54           \\
13        & "Arman" (Kostanay region)       & 10           & 12,5        & 3,2    & 9,3      & 55,4              & 2,45           \\
14        & “Bitik” (Akmola region)         & 9,8          & 11,3        & 3,6    & 9,8      & 56,2              & 2,58           
\end{tblr}
\end{table}

\begin{multicols}{2}
The analysis showed a high starch content in the selected batches of
lentils - up to 47.2\% for the "Almatinskaya" variety. High starch
content was found in: corn of the "Dobrynya" variety (Almaty region)
(64.4\%), and barley of the "Arpa Elite" variety (up to 51.9\%), the
minimum value was noted in the sample of buckwheat grain of the "Saulyk"
variety (Almaty region) (47.9\%).

Table 1 shows that the carbohydrate content in the studied samples
varies in the range of 70-76\%. The highest carbohydrate content was
found in the grain of rice of the "Marzhan" variety. It is known that
the quantitative content of carbohydrates in rice grain depends not only
on genetic traits, but also on many external factors and growing
conditions (chemical composition of the soil, its acidity and humidity).
Dietary fiber, poorly absorbed by the human body, accelerates intestinal
peristalsis, normalizes lipid and carbohydrate metabolism in the body,
and promotes the elimination of heavy metals. The high nutritional value
of rice is provided by the protein composition of rice varieties.
Analysis of protein in the grain of selected rice varieties showed that
the "Syr syluy" variety has higher rates (7.96\%), and the "Aikerim"
variety has a mass fraction of protein of 6.62\%.

The analysis of Table 1 allows us to state that in the studied samples
of oat grains of the Duman variety, the mass fraction of carbohydrates
is 2.8\%; 1.54\% higher than in the Bitik and Arman varieties. As we
know, carbohydrates are the main source of energy for physical and
mental activity. In addition, carbohydrates are necessary for
uninterrupted cell division, muscle strengthening and normalization of
growth dynamics {[}20{]}. In terms of protein content, the Duman variety
also leads, in comparison with other studied varieties, the mass
fraction of protein is higher by 1.47\%; 0.27\%, respectively. And in
terms of fat content, the Bitik variety leads.

The results of the research made it possible to identify varieties of
grain and leguminous crops with the highest nutritional value, which can
serve as the main raw material in the development of technologies for
the production of flour confectionery products.
\end{multicols}

\begin{table}[H]
\caption*{Table 2 - Comparative analysis of the quality of melons and gourds with the requirements of regulatory documents}
\centering
\begin{tblr}{
  colspec = {X[0.2] X[3] X[1] X[1] X[1] X[1] X[1] X[1]},
  row{1} = {c},
  row{2} = {c},
  cell{1}{1} = {r=2}{},
  cell{1}{2} = {r=2}{},
  cell{1}{3} = {c=2}{},
  cell{1}{5} = {c=2}{},
  cell{1}{7} = {c=2}{},
  cell{3}{1} = {c},
  cell{3}{3} = {c},
  cell{3}{4} = {c},
  cell{3}{5} = {c},
  cell{3}{6} = {c},
  cell{3}{7} = {c},
  cell{3}{8} = {c},
  cell{4}{1} = {c},
  cell{4}{3} = {c},
  cell{4}{4} = {c},
  cell{4}{5} = {c},
  cell{4}{6} = {c},
  cell{4}{7} = {c},
  cell{4}{8} = {c},
  cell{5}{1} = {c},
  cell{5}{3} = {c},
  cell{5}{4} = {c},
  cell{5}{5} = {c},
  cell{5}{6} = {c},
  cell{5}{7} = {c},
  cell{5}{8} = {c},
  cell{6}{1} = {c},
  cell{6}{3} = {c},
  cell{6}{4} = {c},
  cell{6}{5} = {c},
  cell{6}{6} = {c},
  cell{6}{7} = {c},
  cell{6}{8} = {c},
  cell{7}{1} = {c},
  cell{7}{3} = {c},
  cell{7}{4} = {c},
  cell{7}{5} = {c},
  cell{7}{6} = {c},
  cell{7}{7} = {c},
  cell{7}{8} = {c},
  cell{8}{1} = {c},
  cell{8}{3} = {c},
  cell{8}{4} = {c},
  cell{8}{5} = {c},
  cell{8}{6} = {c},
  cell{8}{7} = {c},
  cell{8}{8} = {c},
  vlines,
  hlines,
}
№ & Name of raw material                                                                                         & Pumpkin                     &                    & Melon                       &                    & Courgettes                   &                    \\
  &                                                                                                              & according to GOST 7975-2013 & sample under study & according to GOST 7178-2015 & sample under study & according to GOST 31822-2012 & sample under study \\
1 & 2                                                                                                            & 3                           & 4                  & 5                           & 6                  & 7                            & 8                  \\
1 & Mass fraction of fruits (in case of calibration) that do not meet calibration requirements, \%, no more than & 10                          & 4                  & 10                          & 5                  & 5                            & 3                  \\
2 & Mass fraction of bruised and mechanically damaged fruits, \%. no more than                                   & not allowed                 & not found          & not allowed                 & not found          & not allowed                  & not found          \\
3 & Mass fraction of foreign impurities (twigs, stalks, leaves). \%, no more than                                & not allowed                 & not found          & not allowed                 & not found          & not allowed                  & not found          \\
4 & The presence of rotten, withered, moldy and dry fruits                                                       & not allowed                 & not found          & not allowed                 & not found          & not allowed                  & not found          \\
5 & Presence of agricultural pests and their waste products                                                      & not allowed                 & not found          & not allowed                 & not found          & not allowed                  & not found          
\end{tblr}
\end{table}

\begin{multicols}{2}
The ideological principle of the functional products being developed is
the absence of preservatives, dyes, flavors and other artificial food
additives. The main ingredients of the products are melons, fruits,
berries (powders) dried using a special innovative technology, as well
as various food components that allow varying the taste, aroma and
functional properties of the products.

Dried crushed melons, fruits and berries contain dietary fiber, pectin
and cellulose, which have prebiotic properties.

Next, a comparative analysis of the quality of the studied samples of
melons (pumpkin, melon, squash) and fruit and berry (apples, apricots,
raspberries, currants and sea buckthorn) crops was carried out with the
requirements of regulatory documents presented in Table 2.

The studies showed that all the analyzed samples were in a healthy
condition, the mass fraction of fruits was dented and mechanically
damaged, foreign impurities; rotten, withered, moldy and dry fruits were
not found in the studied samples. The conducted studies showed that the
analyzed samples of melons and gourds meet the requirements of
regulatory documents, have high technological properties, which can
subsequently ensure the maximum output of products during their
processing.
\end{multicols}

\begin{longtblr}[
  label = none,
  entry = none,
  caption = {\bfseries Table 3 - Comparative analysis of the quality of fruit and berry crops with the requirements of regulatory documents},
]{
  colspec = {X[0.2] X[3] X[1] X[1] X[1] X[1] X[1] X[1] X[1] X[1] X[1] X[1]},
  row{1} = {c},
  row{2} = {c},
  row{3} = {c},
  cell{1}{1} = {r=2}{},
  cell{1}{2} = {r=2}{},
  cell{1}{3} = {c=2}{},
  cell{1}{5} = {c=2}{},
  cell{1}{7} = {c=2}{},
  cell{1}{9} = {c=2}{},
  cell{1}{11} = {c=2}{},
  cell{4}{1} = {c},
  cell{4}{3} = {c},
  cell{4}{4} = {c},
  cell{4}{5} = {c},
  cell{4}{6} = {c},
  cell{4}{7} = {c},
  cell{4}{8} = {c},
  cell{4}{9} = {c},
  cell{4}{10} = {c},
  cell{4}{11} = {c},
  cell{4}{12} = {c},
  cell{5}{1} = {c},
  cell{5}{3} = {c},
  cell{5}{4} = {c},
  cell{5}{5} = {c},
  cell{5}{6} = {c},
  cell{5}{7} = {c},
  cell{5}{8} = {c},
  cell{5}{9} = {c},
  cell{5}{10} = {c},
  cell{5}{11} = {c},
  cell{5}{12} = {c},
  cell{6}{1} = {c},
  cell{6}{3} = {c},
  cell{6}{4} = {c},
  cell{6}{5} = {c},
  cell{6}{6} = {c},
  cell{6}{7} = {c},
  cell{6}{8} = {c},
  cell{6}{9} = {c},
  cell{6}{10} = {c},
  cell{6}{11} = {c},
  cell{6}{12} = {c},
  cell{7}{1} = {c},
  cell{7}{3} = {c},
  cell{7}{4} = {c},
  cell{7}{5} = {c},
  cell{7}{6} = {c},
  cell{7}{7} = {c},
  cell{7}{8} = {c},
  cell{7}{9} = {c},
  cell{7}{10} = {c},
  cell{7}{11} = {c},
  cell{7}{12} = {c},
  cell{8}{1} = {c},
  cell{8}{3} = {c},
  cell{8}{4} = {c},
  cell{8}{5} = {c},
  cell{8}{6} = {c},
  cell{8}{7} = {c},
  cell{8}{8} = {c},
  cell{8}{9} = {c},
  cell{8}{10} = {c},
  cell{8}{11} = {c},
  cell{8}{12} = {c},
  cell{9}{1} = {c},
  cell{9}{3} = {c},
  cell{9}{4} = {c},
  cell{9}{5} = {c},
  cell{9}{6} = {c},
  cell{9}{7} = {c},
  cell{9}{8} = {c},
  cell{9}{9} = {c},
  cell{9}{10} = {c},
  cell{9}{11} = {c},
  cell{9}{12} = {c},
  cell{10}{1} = {c},
  cell{10}{3} = {c},
  cell{10}{4} = {c},
  cell{10}{5} = {c},
  cell{10}{6} = {c},
  cell{10}{7} = {c},
  cell{10}{8} = {c},
  cell{10}{9} = {c},
  cell{10}{10} = {c},
  cell{10}{11} = {c},
  cell{10}{12} = {c},
  vlines,
  hlines,
}
№ & Name of raw material                                                                                                       & Apple                        &                    & Apricot                      &                    & Raspberry                    &                    & Currant                      &                    & Sea buckthorn                  &                    \\
  &                                                                                                                            & accord ing to GOST 27572-2017 & sample under study & accord ing to GOST 32787-2014 & sample under study & accord ing to GOST 33915-2016 & sample under study & accord ing to GOST 33954-2016 & sample under study & accord ing to GOST R 59661-2021 & sample under study \\
1 & 2                                                                                                                          & 3                            & 4                  & 5                            & 6                  & 7                            & 8                  & 9                            & 10                 & 11                             & 12                 \\
1 & Fruit size by largest transverse diameter, cm, not less than                                                               & 6                            & 7                  & 10                           & 12                 & -                            & -                  & -                            & -                  & -                              & -                  \\
2 & Mass fraction of berries that do not correspond to the commercial grade, but correspond to a lower grade, \%, no more than & -                            & -                  & -                            & -                  & 5                            & 3                  & 5                            & 2                  & -                              & -                  \\
3 & Mass fraction of fruits that have not reached removable maturity and color. \%, not more than                              & -                            & -                  & -                            & -                  & -                            & -                  & -                            & -                  & 2                              & 1                  \\
4 & Mass fraction of bruised and mechanically damaged fruits, \%. no more than                                                 & not allow ed                  & not found          & not allow ed                  & not found          & -                            & -                  & -                            & -                  & 5                              & 3                  \\
5 & Mass fraction of foreign impurities (twigs, stalks, leaves). \%, no more than                                              & not allow ed                  & not found          & not allow ed                  & not found          & 0,3                          & 0,1                & 0,3                          & 0,2                & 1                              & 0,5                \\
6 & The presence of rotten, withered, moldy and dry fruits                                                                     & not allow ed                  & not found          & not allow ed                  & not found          & not allow ed                  & not found          & not allow ed                  & not found          & not allow ed                    & not found          \\
7 & Presence of agricultural pests and their waste products                                                                    & not allow ed                  & not found          & not allow ed                  & not found          & not allow ed                  & not found          & not allow ed                  & not found          & not allow ed                    & not found          
\end{longtblr}

\begin{multicols}{2}
The studies showed (Table 3) that the samples were in a healthy
condition. The presence of agricultural pests and their waste products
were not detected in the studied samples. The mass fraction of berries
that did not correspond to the commercial grade was within the norm. The
studies showed that the analyzed samples of fruit and berry crops meet
the requirements of regulatory documents.

Analysis of tables 2 and 3 allows us to state that the studied samples
of melons and fruit and berry crops comply with the requirements of
regulatory documentation and can be used for research and production of
dry powders.

Thus, dry powders from melons and fruit and berry crops can become an
excellent basis for developing technologies for the production of flour
confectionery products due to their beneficial properties and ability to
improve the texture and nutritional value of the final product.

{\bfseries Conclusions.} Based on the conducted research, it can be
concluded that the development of functional flour confectionery
products using non-traditional raw materials, such as legumes, melon and
fruit and berry powders, is promising. Such products will have increased
nutritional value, improved organoleptic properties and can help prevent
various diseases due to the content of natural antioxidants and
biologically active substances. The obtained results open up prospects
for further research in the field of developing technologies for the
production of flour confectionery products using non-traditional raw
materials. This will expand the range of functional products enriched
with natural antioxidants and biologically active substances, which is
especially important in the context of the modern healthy food market.

\emph{{\bfseries Funding:} This study was funded by the Science Committee
of the Ministry of Science and Higher Education of the Republic of
Kazakhstan within the framework of the PCF for the scientific and
technical program IRN No. BR24993031 on the topic "Development of
technology for the preparation of healthy food products for daily
consumption, enriched with natural antioxidants and biologically active
substances"}
\end{multicols}

\begin{center}
{\bfseries References}
\end{center}

\begin{references}
1. Dzakhmisheva Z.A., Dzakhmisheva I.SH. Funktsional' nye
pishchevye produkty gerodieticheskogo naznacheniya //
Fundamental' nye issledovaniya. - 2014. - № 9-9. - S.
2048 - 2051. {[}in Russian{]}

2. Doronin A.F., Shenderov B.A. Funktsional' noe pitanie
/A.F. Doronin, B. A. Shenderov.- Izd-vo \\«GranT», 2002. - 294 s. ISBN
5-89135-219-2 {[}in Russian{]}

3. Nikberg I.I. Funktsional' nye produkty v strukture
sovremennogo pitaniya// Praktikuyushchemu \\ehndokrinologu /
Mezhdunarodnyj jendokrinologicheskij zhurnal. - 2011. -№ 6 (38). - S. 64
- 69. {[}in Russian{]}

4. Bulgakova N.N. Razrabotka i sovershenstvovanie tehnologii
hlebobulochnyh izdelij funkcional' nogo naznachenija:
dis. ... kand. tehn. nauk. - Voronezh, 2004. - 243 s. {[}in Russian{]}

5. Shenderov B.A. Sovremennoe sostoyanie i perspektivy razvitiya
kontseptsii «Funktsional' noe pitanie»// Pishchevaya
promyshlennost'. - 2003. - № 5. - S. 4-7. {[}in
Russian{]}

6. Salehi F. Recent applications of powdered fruits and vegetables as
novel ingredients in biscuits: A review// Nutrire. - 2020. -Vol 45(1).
DOI 10.1186/s41110-019-0103-8

7. Potter R., Stojceska V., Plunkett A. The use of fruit powders in
extruded snacks suitable for Children's diets// LWT-Food Sci. Technol. -
2013. -Vol 51(2). -P. 537-544.

DOI 10.1016/j.lwt.2012.11.015

8. Salehia F., Aghajanzadeh S. Effect of dried fruits and vegetables
powder on cakes quality: A review// Trends Food Sci. Technol. - 2020. -
Vol. 95. - P. 162-172. DOI 10.1016/j.tifs.2019.11.011

9. Elleuch M., Bedigian D., Roiseux O., Besbes S., Blecker C., Attia H.
Dietary fibre and fibre-rich by-products of food processing:
Characterisation, technological functionality and commercial
applications: A review//Food Chem.-2011.-Vol.124(2) - P.411-421.

DOI 10.1016/j.foodchem.2010.06.077

10. Puvanenthiran A., Stevovitch-Rykner C., McCann T.H., Day L.
Synergistic effect of milk solids and carrot cell wall particles on the
rheology and texture of yoghurt gels// Food Res. Int. - 2014. -Vol. 62.
- P. 701 - 708. DOI 10.1016/j.foodres.2014.04.023

11. Sudha M.L., Dharmesh S.M., Pynam H., Bhimangoude S.V., Eipson S.W.,
Somasundaram R. \\Antioxidant and cyto/DNA protective properties of apple
pomace enriched bakery products// J. Food Sci. Technol. - 2016. - Vol
53(4). - P. 1909 - 1918. DOI 10.1007/s13197-015-2151-2

12. Górnas P., Juhnevica-Radenkova K., Radenkovs V., Mišina I., Pugajeva
I., Soliven A., Seglin A.D. The impact of different baking conditions on
the stability of the extractable polyphenols in muffins enriched by
strawberry, sour cherry, raspberry or black currant pomace// LWT-Food
Sci. Technol. -2016. --Vol. 65. - P. 946 - 953. DOI
10.1016/j.lwt.2015.09.029

13. Khar' kov S.E., Gonchar'{} V.V.,
Roslyakov I.V. Novaya tekhnologiya zavarnykh pryanichnykh izdelii s
ispol' zovaniem netraditsionnogo
rastitel' nogo syr' ya // Izvestiya vuzov.
Pishchevaya tekhnologiya. - 2012. - № 5-6. -S.112 -- 113. {[}in
Russian{]}

14. Pat. 2357444 Rossiiskaya Federatsiya, MPK A23L1/30, (2006.01).
Biologicheski aktivnaya dobavka k pishche, obladayushchaya
radioprotektornymi membranoprotektornymi svoistvami / Martovshchuk V.I.,
Ul' yanova O.V. i dr.; zayavitel'{} i
patentoobladatel'. - GOU VPO «KuBGTU. - № 2007144906
zayavl. 03.12.07; opubl. 10.06.09, Byul. №16 - 4 s. {[}in Russian{]}

15. Tamazova, S.YU. Pishchevye dobavki na osnove
rastitel' nogo syr' ya, primenyaemye v
proizvodstve khlebobulochnykh i muchnykh konditerskikh izdelii/S.YU.
Tamazova, T.V. Pershakova, M.A. Kazimirova // Nauchnyi zhurnal KubGAU.-
2016.- №122 (08).- S. 112-113. DOI 10.21515/1990-4665-122-076 {[}in
Russian{]}

16. Isabaev I. B., Mazhidov K.KH. i dr Pyure iz passirovannoi tykvy v
proizvodstve sukharei/ Isabaev, K.H. Mazhidov i dr. //Khlebopechenie
Rossii. - 2000. - № 4. - S. 30-31. {[}in Russian{]}

17. Loktev D.B., Zonova L.N. Produkty funktsional' nogo
naznacheniya i ikh rol'{} v pitanii cheloveka //
Obshchestvennoe zdorov' e i organizatsiya
zdravookhraneniya, ehkologiya i gigiena cheloveka. Vyatskii meditsinskii
vestnik. - 2010. -№ 2. - S. 48 - 53. {[}in Russian{]}

18. Ashoka S, Shamshad Begum S, Vijayalaxmi KG. Byproduct utilization of
watermelon to develop watermelon rind flour based cookies// The Pharma
Innovation Journal. - 2021. -Vol. 10(2). -P. 196 - 199. DOI
10.22271/tpi.2021.v10.i2c.5658

19. Bakulina O.A. Razvitie pishchevykh tekhnologii:
ispol' zovanie rastitel' nykh
ehkstraktov// Pishchevaya promyshlennost'.-2007. -№ 5.
-S. 32-33. {[}in Russian{]}

20. Kak uglevody vliyayut na organizm cheloveka {[}Ehlektronnyi
resurs{]}.-2019.-URL:\\
\href{https://mygenetics.ru/blog/food/kak-uglevody-vliyayut-na-organizm-cheloveka}{https://mygenetics.ru}/Data
obrashhenija: 27.02.2025). {[}in Russian{]}
\end{references}

\begin{authorinfo}
\emph{{\bfseries Information about authors}}

Omaralieva A.М. - Candidate of Technical Sciences, ass.professor, Kazakh
University of Technology and Business named after K. Kulazhanov, Astana,
Kazakhstan, e-mail:
\href{mailto:aigul-omar@mail.ru}{\nolinkurl{aigul-omar@mail.ru}};

Khastayeva A.Zh.- PhD, ass.professor Kazakh University of Technology and
Business named after K. Kulazhanov, Astana, Kazakhstan, e-mail:
\href{mailto:gera_or@mail.ru}{\nolinkurl{gera\_or@mail.ru}};

Bekturganova A.A.-Candidate of Technical Sciences, ass.professor, Kazakh
University of Technology and Business named after K. Kulazhanov, Astana,
Kazakhstan, e-mail:
\href{mailto:1968al1@mail.ru}{\nolinkurl{1968al1@mail.ru}};

Rakhimzhanova A.M.- master, senior lecturer, Kazakh University of
Technology and Business named after K. Kulazhanov, Astana, Kazakhstan,
e-mail: \href{mailto:r.ayagoz@mail.ru}{\nolinkurl{r.ayagoz@mail.ru}};

Botbayeva Zh.T.- candidate of Biological Sciences, Kazakh University of
Technology and Business named after K. Kulazhonov, Astana,
Kazakhstan,e-mail:.zhanar.b.t@mail.ru. h

\emph{{\bfseries Сведения об авторах}}

Омаралиева А.М.-кандидат технических наук, доцент, Казахский университет
технологии и бизнеса им. К. Кулажанова, Астана, Казахстан, e-mail:
\href{mailto:aigul-omar@mail.ru}{\nolinkurl{aigul-omar@mail.ru}};

Хастаева А.Ж.- PhD, асс.профессор Казахский университет технологии и
бизнеса им. К. Кулажанова, Астана, Казахстан, e-mail:
\href{mailto:gera_or@mail.ru}{\nolinkurl{gera\_or@mail.ru}};

Бектурганова А.А.- кандидат технических наук, доцент Казахский
университет технологии и бизнеса им. К. Кулажанова, Астана, Казахстан,
e-mail: \href{mailto:1968al1@mail.ru}{\nolinkurl{1968al1@mail.ru}};

Рахимжанова А.М.- магистр, старший преподаватель Казахский университет
технологии и бизнеса им. К. Кулажанова, Астана, Казахстан, e-mail:
\href{mailto:r.ayagoz@mail.ru}{\nolinkurl{r.ayagoz@mail.ru}};

Ботбаева Ж.Т.-биология ғылымдарының кандидаты, Қ.Құлажанов атындағы
Қазақ технология және бизнес университеті, Астана, Қазақстан,
\href{mailto:e-mail.zhanar.b.t@mail.ru}{\nolinkurl{e-mail.zhanar.b.t@mail.ru}}
\end{authorinfo}
