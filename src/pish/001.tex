%% DONE
\newpage
\let\cleardoublepage\clearpage
\part{Производственные и обрабатывающие отрасли}
\chapter{Пищевая технология}
\ID{ҒТАМР 65.09.03}{}

\begin{articleheader}
\sectionwithauthors{У.Ч. Чоманов, Г.Е. Жумалиева, А.Б. Байзақова}{БАҚША ЖӘНЕ КӨКӨНІС ДАҚЫЛДАРЫНЫҢ ҚҰРАМЫНДАҒЫ ЛИКОПИННІҢ ҚҰРАМЫН АНЫҚТАУ ЖӘНЕ ЗЕРТТЕУ}

{\bfseries У.Ч. Чоманов\authorid,
Г.Е. Жумалиева\textsuperscript{\envelope } \authorid,
А.Б. Байзақова\authorid}
\end{articleheader}

\begin{affiliation}
\emph{«Қазақ қайта өңдеу және тағам өнеркәсіптері ғылыми-зерттеу институты» ЖШС Алматы, Казақстан}

\raggedright \textsuperscript{\envelope }{\em Corresponding-author: \href{mailto:guljan_7171@mail.ru}{\nolinkurl{guljan\_7171@mail.ru}}}
\end{affiliation}

Мақалада онкологиялық аурулардың алдын алу үшін жоғары сапа
көрсеткіштері бар тағамдық қоспалар мен функционалдық тағамдық өнімдерді
өндіру технологиясын жасау мақсатында бақша дақылдары мен көкөністердегі
(қарбыз, қызанақ) ликопиннің құрамын анықтау және қарбыз бен қызанақтан
алынған шырындардың сапалық көрсеткіштерін анықтау бойынша жүргізілген
зерттеулер нәтижелері ұсынылған.

Зерттеу нәтижесі бойынша ликопиннің ең көп мөлшері қызанақ (23,5\%) және
қарбыз (20\%) үлгілерінен табылды. Қарбыз бен қызанақтың отандық
сорттарынан алынатын шырындардың сапалық көрсеткіштері анықталды. Қарбыз
шырынында Brix бойынша еритін заттардың мөлшері 8,1-ден 8,4-ке дейін
екені, ал титрленетін қышқылдық бойынша 0,38-ден 0,46 градусқа дейін
ауытқитындығы анықталды. Томат шырынында Brix бойынша еритін заттардың
мөлшері 6,8-ден 7,6-ға дейін, титрленетін қышқылдық ауытқушылығы
0,46-дан 0,51 градусқа дейінгі көрсеткіштер анықталды. Физико-химиялық
көрсеткіштер бойынша жүргізілген зерттеулердің нәтижесінде барлық
үлгілер титрленетін қышқылдық бойынша рұқсат етілген шектерде.

Зерттеу барысында шырындарда (қарбыз, қызанақ) анықталған
физикалық-химиялық көрсеткіштер оны ликопин және функционалдық белсенді
тағамдық қоспаларды алу үшін қолданудың өзектілігін растайды, бұл тағам
өнімдерінің жекелеген түрлерін өндіру үшін шикізат базасын кеңейтуге
мүмкіндік береді.

{\bfseries Түйін сөздер:} ликопин, қарбыз, қызанақ, каротиноидтар, бақша
мен көкөніс дақылдары, каротиноидтар.

\begin{articleheader}
{\bfseries ОПРЕДЕЛЕНИЕ И ИССЛЕДОВАНИЕ СОДЕРЖАНИЯ КОЛИЧЕСТВА ЛИКОПИНА В БАХЧЕВЫХ И ОВОЩНЫХ КУЛЬТУРАХ}

{\bfseries
У.Ч. Чоманов,
Г.Е. Жумалиева\textsuperscript{\envelope },
А.Б. Байзакова}
\end{articleheader}

\begin{affiliation}
\emph{ТОО «Казахский научно-исследовательский институт перерабатывающей и пищевой промышленности», Алматы, Казахстан,}

\emph{e-mail: \href{mailto:guljan_7171@mail.ru}{\nolinkurl{guljan\_7171@mail.ru}}}
\end{affiliation}

В статье представлены результаты исследований по определению и
содержанию ликопина в бахчевых и овощных культурах (арбуз, томаты) и по
определению показателей качества соков, полученных из арбуза и томатов с
целью разработки технологии производства БАДов и пищевых продуктов
функционального назначения с высокими качественными показателями для
профилактики онкологических заболеваний.

В ходе проведенных исследований было установлено, что наибольшее
содержание ликопина обнаружено в образцах томатов (23,5\%) и арбуза
(20\%). Определены качественные показатели соков из отечественных сортов
арбуза и томата. Арбузные соки по содержанию растворимых веществ по Brix
находятся от 8,1 до 8,4. По титруемой кислотности от 0,38 до 0,46 град.
Томатные соки по содержанию растворимых веществ по Brix находятся от 6,8
до 7,6. По титруемой кислотности тоже 0,46 до 0.51 град. В результате
проведенных исследований по физико-химическим показателям все образцы
находятся по титруемой кислотности в пределах допустимой нормы.

Физико-химические показатели, определенные в соках (арбузном, томатном),
подтверждают актуальность его использования для получения ликопина и
БАДа функционального назначения, который позволит расширить сырьевую
базу для производства некоторых видов пищевых продуктов.

{\bfseries Ключевые слова:} ликопин, арбуз, томат, каротиноиды, бахчевые и
овощные культуры, каротиноиды.

\begin{articleheader}
{\bfseries DETERMINATION AND STUDY OF THE CONTENT OF LYCOPENE IN MELONS AND VEGETABLE CROPS}

{\bfseries
U. Chomanov,
G. Zhumalieva\textsuperscript{\envelope },
A. Baizakova}
\end{articleheader}

\begin{affiliation}
\emph{«Kazakh research institute of processing and food industry» LTD, Almaty, Kazakhstan,}

\emph{e-mail: \href{mailto:guljan_7171@mail.ru}{\nolinkurl{guljan\_7171@mail.ru}}}
\end{affiliation}

The article presents the results of studies on the determination and
content of lycopene in melons and vegetables (watermelon, tomatoes) and
on the determination of the quality indicators of juices obtained from
watermelon and tomatoes in order to develop a technology for the
production of dietary supplements and functional food products with high
quality indicators for the prevention of oncological diseases.

In the course of the studies, it was found that the highest content of
lycopene was found in samples of tomatoes (23.5\%) and watermelon
(20\%). The quality indicators of juices from domestic varieties of
watermelon and tomato were determined. Watermelon juices by the content
of soluble substances according to Brix are from 8.1 to 8.4. By
titratable acidity, also from 0.38 to 0.46 degrees. Tomato juices by the
content of soluble substances according to Brix are from 6.8 to 7.6. By
titratable acidity, also 0.46 to 0.51 degrees. As a result of the
conducted studies on physicochemical indicators, all samples are within
the permissible norm for titratable acidity.

Physicochemical indicators determined in juices (watermelon, tomato)
confirm the relevance of its use for obtaining lycopene and functional
dietary supplements, which will expand the raw material base for the
production of certain types of food products.

{\bfseries Keywords:} lycopene, watermelon, tomato, carotenoids, melons and
vegetables, carotenoids.

\begin{multicols}{2}
{\bfseries Кіріспе.} Толыққанды тамақтану адам өмірінің негізі болып
табылады және адамның жүрек-қан тамырлары (атеросклероз, миокард
инфарктісі, инсульт, гипертония және т.б.), онкологиялық, асқазан-ішек,
метаболикалық (семіздік, остеохондроз т.б.) ауруларының көпшілігінің
алдын алудың маңызды факторы болып саналады. Қазіргі уақытта еліміздің
әртүрлі аймақтарында халықтың аса биологиялық құнды азық-түлік өнімдерін
тұтынуының айтарлықтай төмендеуі, витаминдер мен бірқатар минералды
заттардың жеткіліксіз тұтынуы байқалады. Бұл мәселені шешудің бір жолы
антиоксиданттық белсенділігі бар және адам денсаулығын сақтауда,
иммундық жүйені нығайтуда және созылмалы аурулардың алдын алуда маңызды
физиологиялық рөл атқаратын каротиноидтарды тамақ рационына қосу болуы
мүмкін.

Биологиялық қасиеттеріне байланысты каротиноидтар адам өмірінде маңызды
физиологиялық рөл атқарады және олардың қолдану аясының кеңеюі
каротиноидтар мен оларды қолданатын тағам өнімдерінің өндірісін
арттыруды қажет етеді. Осыған сәйкес, каротиноидтарды алудың
қолданыстағы технологияларын жетілдіру және шикізат базасын кеңейту,
оның ішінде құрамында каротині бар шикізатты қайта өңдеу зауытында
қайталама ресурстарды пайдалану арқылы кеңейту саласындағы зерттеулер
өзекті болып табылады.

Бақша дақылдары мен көкөніс дақылдарында каротиноидтар (ликопин) әдемі
түс беріп қана қоймайды, сонымен қатар ағзадағы асқын тотығумен
күресетін, бос радикалдарды, канцерогендерді дезактивациялауға
көмектесетін және қатерлі ісік пен жүрек-қан тамырлары ауруларының
дамуын тежейтін жоғары антиоксиданттық қасиеттерге ие {[}1{]}.

Ликопин - қызанақ, гуава және қарбыз сияқты кейбір өсімдіктердің
жемістерінің түсін анықтайтын каротиноидты пигмент. Ликопин
антиоксиданттық белсенділікке ие, атап айтқанда, қатерлі ісіктерінің
пайда болуы мен дамуына жол бермейтін ДНҚ-ны қорғауда негізгі рөл
атқарады. Ол атеросклероздың, катарактаның алдын алу шарасы ретінде
әрекет етеді және кейбір қабыну процестерін емдейді {[}2{]}.

Ликопин -- майда еритін каротиноид, ол қызанақта, қарбыз целлюлозасында
(3,55-4,86 мг/100 г), қызыл грейпфрутта, шырғанақ пен итмұрында
кездеседі. Ликопин Е-160 тағамдық қоспасы ретінде тіркелген және
Ресейде, Беларусьте, АҚШ-та, Австралияда, Жаңа Зеландияда және Еуропалық
Одақта пайдалануға рұқсат етілген. Ликопиннің ұсынылатын мөлшері
тәулігіне 5-10 мг құрайды {[}3{]}. Ликопиннің адам ағзасындағы негізгі
қызметі антиоксидант болып табылады. Тотығу стрессін азайту
атеросклероздың дамуын бәсеңдетеді, сонымен қатар онкогенезді
болдырмайтын ДНҚ қорғанысын қамтамасыз етеді. Ликопин және құрамында
ликопин бар тағамдарды тұтыну адамдарда тотығу стресс маркерлерінің
айтарлықтай төмендеуіне әкеледі. Ликопин адам қанындағы ең күшті
каротиноид және антиоксидант болып табылады {[}4{]}.

Ғалымдар өз зерттеулерінің нәтижесінде қартаю үдерісін баяулату,
иммундық жүйенің қызметін арттыру, жүрек-қан тамырлары ауруларының
қаупін төмендету, жасушааралық метаболизмнің тұрақтылығын қамтамасыз ету
және онкологиялық патологиялардың алдын алу механизмдерін анықтауға
ұмтылады. Осы мақсатта каротиноидтар арасындағы ең күшті
антиоксиданттардың бірі болып табылаты ликопиннің биологиялық әсері
кеңінен зерттелуде. Ең күшті антиоксиданттардың бірі ретінде оның
синглетті оттегін бейтараптандыру қабілеті β-каротиннен екі есе,
α-токоферолдан он есе және глутатионнан жүз жиырма бес есе тиімдірек
{[}5{]}. 1903 жылы \emph{Lycopersicum esculentum} (қызанақ) тұқымынан
бөлініп алынған ликопин өзі бөлініп алынған жемістің атымен аталған
{[}6{]}.

Адамдарда қатерлі ісіктерді емдеуде ликопинді зерттеу оның простата
безінің жергілікті қатерлі ісігін емдеу тиімділігін арттыратынын
көрсетеді {[}7{]}. Сондай-ақ, қуық асты безінің қатерлі ісігіне шалдығу
қаупі жоғары ерлер арасында простата обырына байланысты өлім деңгейін
төмендетуге ықпал етеді және гормоналды сезімталдық күйіне тәуелсіз
түрде қайталанатын простата обырының дамуын тежейді {[}8{]}. Бұдан
бөлек, ликопин өмір сүру сапасын жақсартуға, сүйек ауырсынуын
жеңілдетуге және төменгі зәр шығару жолдарының симптомдарын бақылауды
қамтамасыз етуге қабілетті екені анықталған {[}9{]}.

Сондай-ақ, ликопин көздің көптеген ауруларының дамуында маңызды рөл
атқаратын тотығу стрессінен қорғанысты қамтамасыз етеді. Зерттеулер
көрсеткендей, ликопин егде жастағы адамдарда көру қабілетінің жоғалуына
алып келуі мүмкін макулярлы дегенерацияның дамуына әсер ететін химиялық
және қабыну процестерін тежейді.

Ликопин - жемістер мен көкөністерге сары, қызғылт сары және қызыл түстер
беретін және суда ерімейтін каротиноидтар тұқымдасының қосылысы. Біздің
ағзамыз ликопинді өздігінен синтездей алмайды, сондықтан қызыл түсті
көкөністерді, жемістерді жеу керек немесе байытқыш ретінде тағам
рационын тағамдық қоспалармен байытуға болады. Ағзада ликопин күшті
антиоксидант ретінде әрекет етіп, жасушаларды бос радикалдар тудыратын
тотығу әсерінен қорғайды. Тотығу үдерісі қабыну реакцияларын қоздырып,
қант диабеті, Альцгеймер ауруы, қатерлі ісік және жүрек-қан тамыр
жүйесімен байланысты патологиялардың дамуына ықпал етеді.
Антиоксиданттық қасиеттерінің арқасында ликопин зиян холестерин деңгейін
төмендету арқылы инфаркт пен инсульт қаупін азайтуға көмектеседі,
сонымен қатар, бұл қосылыс теріні ультракүлгін сәулелердің зиянды
әсерінен қорғай отырып, меланоманың даму қаупін төмендетуге ықпал етеді.
Ликопиннің сүйек тінінің жасушаларына да оң әсері бар, ол олардың
өмірлік циклін ұзартуға және сүйек тінінің жалпы күйін жақсартуға
көмектеседі.

Ликопин термиялық өңдеу арқылы жойылмайды, сондықтан ол негізінен
қызанақтан жасалған тағамдар мен өнімдерде кездеседі. Сонымен қатар,
жоғары температура каротиноидтың құрылымын өзгертеді, осылайша ол денеге
жақсырақ және толық сіңе бастайды. Томат пастасында жаңа піскен
қызанақтарға қарағанда 30 есе көп ликопин бар. Бірқатар зерттеулер
каротиноид ликопині бар тағамдарды үнемі тұтыну жүрек-қан тамырлары
аурулары мен қуық асты безінің қатерлі ісігінің қаупін төмендететінін
көрсетті {[}10{]}.

Жергілікті көкөністер мен бақша дақылдары ликопин алу үшін
пайдаланылғандықтан, бұл өндіріс шығындарының төмен болуына әкеледі. Бір
қызығы, құрамында ликопин бар өнімдер шикі күйінде тұтынылғанда, ол
ағзаға аз мөлшерде сіңеді. Оларды термиялық өңдеу кезінде ликопин
концентрациясы артады, бірақ бұл жағдайда да жеткіліксіз болуы мүмкін.

Осыған байланысты, ликопинді тамақ өндірісінде қолдану қазіргі өмір
салты жағдайында маңызды және қажетті элементке айналуда. Оның
антиоксиданттық, қабынуға қарсы және қорғаныш қасиеттері тағам
өнімдерінің биологиялық құндылығын арттырып, түрлі аурулардың алдын алу
мүмкіндігін кеңейтеді.

{\bfseries Материалдар мен әдістер.} Зерттеу объектілері - қарбыз және
қызанақ шырындары, каротиноидты ликопин. Зерттеу үшін қарбыз мен
қызанақтың әртүрлі отандық сорттарынан сынамалар алынды. Қарбыздан
1-үлгі -- «Асар» қарбызы, 2-үлгі «Достық10», 3-үлгі -- «Кримсон Свит»
қарбызы; қызанақтан 4-үлгі - «Солнечный» қызанағы; 5-үлгі --«Колхозный
34» қызанағы; 6-үлгі -- «Ахтанақ» қызанағы.

Зерттеу үшін жалпы қабылданған стандартты әдістер қолданылды:
органолептикалық, физика-химиялық, биохимиялық, санитарлық-гигиеналық.

Зерттеу үшін келесі әдістер қолданылды:

- МЕМСТ ISO 750-2013 Өңделген жеміс-көкөніс өнімдері. Титрленетін
қышқылдықты анықтау.

- МЕМСТ ISO 2173-2013 еритін қатты заттарды рефрактометриялық әдіспен
анықтау.

- МЕМСТ 29031-91 Құрғақ заттарды анықтау әдісі.

Құрғақ заттардың мөлшері SNEL-104 рефрактометрі арқылы анықталды.
Шикізат пен өнімдердің ылғалдылығы MX-50 ылғал өлшегішінің көмегімен
анықталды. Өлшеу нұсқауларға сәйкес жүргізіледі: 5 грамм сынама алынады,
шыныаяққа біркелкі қабатта жайылды және 130°С температурада кептіріледі.

- МЕМСТ 51433-99 Жеміс-көкөніс шырындары. Рефрактометр көмегімен еритін
қатты заттардың құрамын анықтау әдісі.

- ликопинді анықтау әдісі. Гександы сусыздандыру әдісі.

Ликопин мөлшері гександы сусыздандыру әдісі арқылы анықталды, онда
шырындардың бензинде ерітілуі нәтижесінде ликопин гексан фракциясына
ауысады. Өсімдік пигментін бөліп алу үшін әр сынамадан 2,5 мл алынып,
пробиркаларға құйылды, содан кейін әр пробиркаға тең мөлшерде бензин
қосылды. Қоспа мұқият шайқалып, анық екі фазалы жүйе түзілгенше
қалдырылды. Жоғарғы фаза (гексан) ашық, мөлдір, сары-қызғылт сары түсті
болды, ал төменгі фаза (сулы) ақшыл қызыл түсті және мөлдір емес болып
көрінді. Жоғарғы мөлдір қабат тамшуыр көмегімен мұқият алынып, фарфор
шыныаяққа құйылды, содан кейін су моншасында 0,25--0,5 мл қалғанға дейін
буландырылды. Ликопиннің мөлшері жоғарғы фазаның биіктігі бойынша
анықталды.

- МЕМСТ 5898 -- 87 Қышқылдық пен сілтілілікті анықтау.

- МЕМСТ Р 51478-99 (ISO 2917-74) сутегі ионының рН концентрациясын
анықтау.

- МЕМСТ Р ISO 22935-2-2011 органолептикалық бағалаудың ұсынылған
әдістері.

- МЕМСТ 51433-99 «Жеміс-көкөніс шырындары. Рефрактометр көмегімен еритін
қатты заттардың құрамын анықтау әдісі.

- белсенді қышқылдықты Testo 206-pH1 рН өлшегішінің көмегімен анықтау.

{\bfseries Нәтижелер мен талқылау}.Зерттеу үшін бақша дақылдары (қарбыз)
және көкөніс дақылдары (қызанақ) алынды. Жұмыс үшін қарбыз мен
қызанақтың келесі сорттары пайдаланылды:

1. «Асар» қарбыз сорты. Орташа салмағы: 7-12 кг, қабығы қалың, қара
жолақты ерте пісетін сорт. Целлюлозасы ашық қызыл, ірі түйіршікті,
шырынды және қарбызға тән хош иіске ие. Бұл сорт негізгі ауруларға
төзімділігімен және ұзақ сақталу қабілетімен ерекшеленеді.

2. «Достық-10» қарбызы. Орташа салмағы 5-10 кг аралығында болатын,
салыстырмалы түрде орташа кеш пісетін қарбыз сорты. Қабығы жұқа, қара
жолақты. Целлюлозасы ашық қызыл, ұсақ түйіршікті, шырынды және қарбыз
хош иісі айқын білінеді. Сорт негізгі ауруларға төзімді және жиналған
өнімді 3 айға дейін сақтауға қабілетті. Өсіп-өну кезеңі 90-95 күнді
құрайды. Бұл сорт оңтүстік аймақтарда ашық алаңдарда және пленкалы
жылыжайларда өсіруге бейімделген.

3. «Кримсон Свит» сорты -- ерте пісетін, ашық және қорғалған топырақ
жағдайларына арналған қарбыз сорты. Өсу мерзімі 65-80 күн. Өсімдіктің
ұзындығы 3-5 м аралығында, ірі жемісті, ұзын сопақша пішінді, ұзындығы
50-60 см, салмағы 12-15 кг дейін жетеді. Қабығы ашық жасыл түсті,
қара-жасыл мәрмәр өрнектері бар. Целлюлозасы бай қызғылт-қызыл, тығыз,
шырынды және тәтті. Бір өсімдіктен 25-30 кг өнім алынады.

4. «Солнечный» сорты -- ерте пісетін қызанақ сорты. Жемістері қызыл
түсті, сопақша, сәл жұмыртқа тәрізді, салмағы 85-105 г аралығында.
Құрылымы тығыз, 3-5 тұқым камерасы бар, жапырақшадан оңай бөлінеді.
Егіннің біркелкі пісуі өнімді жинау мен тасымалдау барысында
артықшылық береді. Көшет отырғызғаннан кейін 53-56 күнде өнім береді.
Сорт негізгі қызанақ ауруларына төзімді.

5. «Колхозный 34» сорты -- орта маусымдық қызанақ сорты. Жемістері жалпақ
дөңгелек пішінді, қызыл фонда сары жолақтары бар. Пісу барысында күн
сәулесінің әсерінен түсі қанық қызыл реңкке ие болады. Целлюлозасы көп
камералы, шырынды және тәтті. Жемістерінің орташа салмағы 200-400 г.
Бұл сорт салаттарға, шырындарға, соустар мен макарон өнімдерін
дайындауға қолайлы.

6. «Ахтанақ» сорты -- орта маусымдық қызанақ сорты, өну кезеңі 115-127
күнді құрайды. Жемістері дөңгелек, тегіс, піскен кезде қызыл түске ие,
салмағы 60-100 г аралығында. Шырынды, дәмдік қасиеттері жоғары,
жарылуға бейім емес. Сорт жаңа піскен күйінде тұтынуға, консервілеуге
және қайта өңдеуге жарамды.

Таңдап алынған бақша және көкөніс дақылдарынан (қарбыз, қызанақ)
шырындар

дайындалды.

Қарбыз шырыны шырынды целлюлозаны престеу әдісі арқылы бөлу
технологиясымен дайындалды. Өңдеу процесінде ірі целлюлоза бөлшектері
жойылады, бірақ өнім құрамында белгілі бір мөлшерде диеталық талшықтар
сақталады. Шикізат ретінде піскен, ашық түсті және зақымданбаған
қарбыздар таңдалып алынды.

Піскен қарбыз ағынды су астында 20-25°С температурада жуылды, қабығы
аршылып, қалыңдығы 1 мм-ден аспайтын ақ целлюлоза қабаты қалдырылды.
Қарбыз целлюлозасы үлкен кесектерге кесіліп, тұқымдары алынып тасталды.
Целлюлоза шырынның ауырлық күшімен ағуына мүмкіндік беру үшін 1 минут
бойы көлбеу перфорацияланған бетте ұсталды, содан кейін 80 л/сағатқа
дейінгі престеу жылдамдығымен сығылды. Алынған шырын 85-90°С
температурада 8-10 минут бойы пастерленді және алдын ала жуылып,
стерильденген ыдыстарға құйылды. Одан кейін орау процесі жүргізіліп,
өнім 18-20°С температураға дейін салқындатылды. Дайын шырын 2-4°С
температурада сақталды.

Томат шырынын алу үшін піскен, қабығы аршылған және сұрыпталған
қызанақтар 20-25°С температурадағы ағынды сумен жуылды. Өңдеу барысында
олар 60-70°С-қа дейін қыздырылып, ұсақталды, содан кейін пресс астында
сығылды. Алынған шырын тұқымдардан, ірі қосындылардан (жемістердің жасыл
бөліктері мен қабығынан) тазарту мақсатында електен (дәке арқылы)
өткізілді және стерильді ыдыстарға құйылды. Микроорганизмдерді жою
қажеттілігі қызанақ шырынының қышқылдығының төмен болуымен байланысты,
себебі оның рН деңгейі 5,5-6,5 аралығында, бұл бірқатар
микроорганизмдердің, соның ішінде спора түзетіндердің дамуына қолайлы
жағдай туғызады. Осыған байланысты қызанақ шырыны 110°С температурада
20-25 минут бойы зарарсыздандырылды.

Зерттеулер нәтижесінде қарбыз мен қызанақтың отандық сорттарының
үлгілеріндегі каротиноидты ликопиннің мөлшері анықталды. Қарбыз бен
қызанақтың құрамындағы ликопиннің болуы зерттеліп, талдау жүргізілді.
Алдын ала таңдап алынған қарбыз бен қызанақтың әртүрлі отандық
сорттарының жаңа үлгілері зерттеліп, сынақтан өткізілді.

Қарбыз және қызанақ шырындарының органолептикалық және физика-химиялық
қасиеттері зерттелді.

Қарбыз және қызанақ шырынының әртүрлі үлгілерінің органолептикалық және
физика-химиялық көрсеткіштері 1-кестеде келтірілген.

Үлгілер: қарбыз: 1-үлгі, 2-үлгі, 3-үлгі. қызанақ: 4-үлгі, 5-үлгі,
6-үлгі.
\end{multicols}

\begin{table}[H]
\caption*{1. кесте. Қарбыз бен қызанақтың органолептикалық және физика-химиялық көрсеткіштері}
\centering
\begin{tblr}{
  colspec = {X[1] X[3] X[1] X[1] X[1]},
  cells = {c},
  cell{1}{1} = {r=2}{},
  cell{1}{2} = {c=4}{},
  cell{3}{2} = {r=3}{},
  cell{6}{2} = {r=3}{},
  vlines,
  hline{1,3,6,9} = {-}{},
  hline{2} = {2-5}{},
  hline{4-5,7-8} = {1,3-5}{},
}
Үлгілер & Шырын көрсеткіштері                                                                                             &                                                &                           &                                              \\
        & сыртқы түрі, консистенциясы, дәмі, иісі және түсі                                                               & {титрленетін\\қышқылдық, \textsuperscript{о}Т} & {белсенді\\қышқылдық, pH} & {еритін қатты\\заттардың\\мөлшері,\\(°Brix)} \\
1- үлгі & айқын қарбыз дәмі, тәтті, біркелкі, ашық қызғылт, мөлдір емес, жұмсақсыз және тұнбалы                           & 0,41                                           & 4,4                       & 8,1                                          \\
2- үлгі &                                                                                                                 & 0,38                                           & 4,3                       & 8,4                                          \\
3- үлгі &                                                                                                                 & 0,45                                           & 4,4                       & 8,2                                          \\
4- үлгі & айқын тән қызанақ хош иісі, бүкіл массаға таралған майда ұнтақталған жұмсағы бар біртекті сұйықтық, қызыл түсті & 0,46                                           & 5,8                       & 7,6                                          \\
5- үлгі &                                                                                                                 & 0,49                                           & 5,5                       & 7,0                                          \\
6- үлгі &                                                                                                                 & 0,51                                           & 6,0                       & 6,8                                          
\end{tblr}
\end{table}

\begin{multicols}{2}
1-кестеде ұсынылған мәліметтерге сәйкес, алынған барлық шырындардың сапа
көрсеткіштері белгіленген рұқсат етілген шектерге сәйкес келетіні
анықталды. Барлық қарбыз шырындары ашық қызғылт түсті болып, қарбызға
тән тәтті дәмге ие болды. Brix шкаласы бойынша еритін заттардың мөлшері
8,1-ден 8,4\%-ға дейін, ал титрленетін қышқылдық 0,38-ден 0,45°-қа дейін
өзгерді.

Томат шырындары қызыл түсті, құрамында массаға біркелкі бөлінген майда
ұнтақталған целлюлоза болды. Brix бойынша еритін заттардың мөлшері
6,8-ден 7,6\%-ға дейін, ал титрленетін қышқылдық 0,46-дан 0,51°
аралығында ауытқыды. Алынған нәтижелерге сәйкес, «Асар» қарбызы мен
«Солнечный» қызанақ сорттарынан алынған шырындар органолептикалық және
физика-химиялық қасиеттері бойынша ең жоғары бағаға ие болды.

Зертханалық жағдайда ликопин қарбыз бен қызанақ шырынынан гексан
әдісімен бөлініп алынды.

Зерттеу нәтижелері барлық зерттелген қарбыз және қызанақ үлгілерінде
ликопиннің бар екенін, алайда оның мөлшері мен түс ерекшеліктері әртүрлі
болатынын көрсетті. 2-кестеде гександы сусыздандыру әдісі арқылы
анықталған ликопин көрсеткіштері ұсынылған.

Ликопинді оқшаулау бойынша жүргізілген талдаулар нәтижесі ликопиннің ең
көп мөлшері қарбыз үлгісінің 1-үлгісінде (20\%) және 4-үлгідегі
қызанақта (23,5\%) бар екені екенін көрсетті. Осылайша, алынған
эксперименттік деректер ликопиннің барлық зерттелген қарбыздар мен
қызанақтарда кездесетінін растайды. Ликопиннің ең жоғары мөлшері «Асар»
қарбыз сорты мен «Солнечный» қызанақ сортында анықталды.

Жүргізілген эксперименттік зерттеулер мен талдаулардың нәтижелеріне
сүйене отырып, қызанақ құрамындағы ликопин мөлшері қарбызбен
салыстырғанда жоғары екендігі туралы қорытынды жасауға болады.
\end{multicols}

\begin{table}[H]
\caption*{2 - кесте. Ликопинді гександы сусыздандыру әдісімен анықтау}
\centering
\begin{tblr}{
  colspec = {X[2] X[1] X[1] X[1] X[1] X[1] X[1]},
  cells = {c},
  cell{1}{1} = {r=2}{},
  cell{1}{2} = {c=3}{},
  cell{1}{5} = {c=3}{},
  vlines,
  hline{1,3-7} = {-}{},
  hline{2} = {2-7}{},
}
Көрсеткіштер                     & Қарбыздар              &                        &                        & Қызанақтар             &                   &                   \\
                                 & 1- үлгі                & 2- үлгі                & 3- үлгі                & 4- үлгі                & 5- үлгі           & 6- үлгі           \\
Түсі                             & ашық сары қызғылт сары & ашық сары қызғылт сары & ашық сары қызғылт сары & ашық сары-қызғылт сары & сары-қызғылт сары & ашық қызғылт сары \\
Гексан мөлшері, мл               & 1,5                    & 1,9                    & 2,2                    & 1,7                    & 2,0               & 1,8               \\
Буландырудан кейінгі мөлшері, мл & 0,3                    & 0,3                    & 0,3                    & 0,4                    & 0,4               & 0,4               \\
Ликопин мөлшері, \%              & 20,0                   & 15,8                   & 13,6                   & 23,5                   & 20,0              & 22,2              
\end{tblr}
\end{table}

\begin{multicols}{2}
{\bfseries Қорытынды.} Зертханалық жағдайда қарбыз және қызанақтың отандық
сорттарының органолептикалық және физика-химиялық көрсеткіштері
анықталды.

Отандық қарбыздың «Асар», «Достық10», «Кримсон Свит» және қызанақтардың
«Солнечный», «Колхозный 34», «Ахтанақ» сорттарының органолептикалық және
физика-химиялық көрсеткіштері зертханалық жағдайда анықталды. Барлық
қарбыз шырындары ашық қызғылт түсті болып, қарбызға тән тәтті дәмге ие
болды. Brix шкаласы бойынша еритін заттардың мөлшері 8,1-ден 8,4\%-ға
дейін, ал титрленетін қышқылдық 0,38-ден 0,45°-қа дейін өзгерді.

Томат шырындары қызыл түсті, құрамында массаға біркелкі бөлінген майда
ұнтақталған целлюлоза болды. Brix бойынша еритін заттардың мөлшері
6,8-ден 7,6\%-ға дейін, ал титрленетін қышқылдық 0,46-дан 0,51°
аралығында ауытқыды. .

Физико-химиялық көрсеткіштер бойынша жүргізілген зерттеулердің
нәтижесінде барлық үлгілер титрленетін қышқылдық бойынша рұқсат етілген
шектерде.

Қарбыз бен қызанақтың отандық сорттарының үлгілеріндегі каротиноидты
ликопиннің мөлшері анықталды. Отандық қарбыз мен қызанақтың әртүрлі
сорттарынан ликопинді бөліп алу бойынша жүргізілген зерттеулердің
нәтижесінде барлық үлгілерде ликопин бар, бірақ әртүрлі мөлшерде екені
анықталды. Ликопинді анықтау кезінде ең көп мөлшері қарбызда 1-үлгіде
(20\%), ал қызанақтың 4-үлгісінде (23,5\%) болатыны анықталды.

Жүргізілген зерттеулердің нәтижелерін ескере отырып, «Асар» қарбыз сорты
мен «Солнечный» қызанақ қолайлы баға алды.

\emph{{\bfseries Қаржыландыру}. Материалдар BR22886613 «Инновациялық
технологияларды дамыту» ғылыми-техникалық бағдарламасы аясында
«Онкологиялық аурулардың алдын алу үшін құны төмен және жоғары сапа
көрсеткіштерімен функционалды тағамдық қоспаларды өндіру технологиясын
жасау» жобасы аясында дайындалған. Ауыл шаруашылығы министрлігінің 101
«Ғылыми зерттеулер мен іс-шараларды бағдарламалық-нысаналы қаржыландыру»
кіші бағдарламасы бойынша 267 «Білім мен ғылыми зерттеулердің
қолжетімділігін арттыру» бюджеттік бағдарламасы бойынша ауыл шаруашылығы
өсімдік шаруашылығы өнімдері мен шикізатын өңдеу және сақтау
технологиялары»; Қазақстан Республикасы 2024-2026 жж.}
\end{multicols}

\begin{center}
{\bfseries References}
\end{center}

\begin{references}
1. Collins E.J., Bowyer C., Tsouza A., Chopra M. Tomatoes: An extensive
review of the associated health impacts of tomatoes and factors that can
affect their cultivation// Biology (Basel). - 2022.-Vol.11(2): 239. DOI
10.3390/biology11020239.

2. Kristal A.R., Till C., Platz E.A., Song X., King I.B., Neuhouser M.L.,
Ambrosone C.B., Thompson I.M. Serum lycopene concentration and prostate
cancer risk: Results from the Prostate Cancer Prevention Trial //Cancer
Epidemiol. Biomark. Prev. -2011.-Vol.20(4).- P.638--646. DOI
10.1158/1055-9965.EPI-10-1221

3. Imran M.,~Ghorat F.,~Ul-haq I.,~Ur-rehman H.,~Aslam F.,~Heydari
M.,~Shariati M. A.,~Okuskhanova E.,~Yessimbekov Z.,~Thiruvengadam
M.,~Hashempur M. H., and~Rebezov M.~Lycopene as a natural \\antioxidant
used to prevent human health disorders//~Antioxidants.- 2020.-Vol.9
(8).- P.706 -- 727.~\href{https://doi.org/10.3390/antiox9080706}{DOI
10.3390/antiox9080706}.~

4. M. Viuda-Martos,~E. Sanchez-Zapata,~E. Sayas-Barberá,~E. Sendra,~J. A.
Pérez-Álvarez,~J. Fernández-López.~Tomato and Tomato Byproducts. Human
Health Benefits of Lycopene and Its Application to Meat Products: A
Review //Critical Reviews in Food Science and
Nutrition\emph{.-}~2014.-Vol.54\emph{~}(8).-P.1032-1049.~\href{https://doi.org/10.1080/10408398.2011.623799}{DOI
10.1080/10408398.2011.623799}

5. Joshi B.,~Kar S. K.,~Yadav P. K.,~Yadav S.,~Shrestha L., and~Bera T.
K.,~Therapeutic and medicinal uses of lycopene: a systematic
review//International Journal of Research in Medical Sciences.-
2020.-Vol.8
(3):1195\href{https://doi.org/10.18203/2320-6012.ijrms20200804}{DOI
10.18203/2320-6012.ijrms20200804}.

6. Bramley P. M.,~Is lycopene beneficial to human
health?//~Phytochemistry. -2000.-Vol.~54(3)-P.~233 --
236.~\href{https://doi.org/10.1016/S0031-9422(00)00103-5}{DOI
10.1016/S0031-9422(00)00103-5}

7. Morgia G., Voce S., Palmieri F., Gentile M., Iapicca G., Giannantoni
A., Blefari F., Carini M., Vespasiani G., Santelli G., et al.
Association between selenium and lycopene supplementation and incidence
of prostate cancer: Results from the post-hoc analysis of the procomb
trial// Phytomedicine.- 2017.-Vol.-34.-P. 1-5. DOI
10.1016/j.phymed.2017.06.008.~

8. Clark P.E., Hall M.C., Borden L.S.J., Miller A.A., Hu J.J., Lee W.R.,
Stindt D., D'Agostino R.J., Lovato J., Harmon M., et al. Phase I-II
prospective dose-escalating trial of lycopene in patients with
biochemical relapse of prostate cancer after definitive local therapy
//Urology.- 2006.-Vol.67(6).- P.1257--1261. DOI
10.1016/j.urology.2005.12.035.

9. Usman Mir Khan, Mustafa Sevindik, Ali Zarrabi, Mohammad Nami, Betul
Ozdemir, Dilara Nur Kaplan, Zeliha Selamoglu, Muzaffar Hasan, Manoj
Kumar, Mohammed M. Alshehri, Javad Sharifi-Rad,~Lycopene: Food Sources,
Biological Activities, and Human Health Benefits//Oxidative Medicine and
Cellular \\Longevity.-~2021.-Vol.2021(1): 2713511.- 10
p.~2021.~\href{https://doi.org/10.1155/2021/2713511}{DOI
10.1155/2021/2713511}

10. J. Costa-Rodrigues, O. Pinho, and P. R. R. Monteiro, ``Can lycopene
be considered an effective \\protection against cardio-vascular disease?//
Food Chemistry.- Vol. 245.-P.1148--1153,2018.
\href{https://doi.org/10.1155/2021/2713511}{DOI\\
10.1016/j.foodchem.2017.11.055}
\end{references}

\begin{authorinfo}
\emph{{\bfseries Авторлар туралы мәліметтер}}

Чоманов У.Ч. - т.ғ.д., «Қазақ өңдеу және тамақ өнеркәсібі ҒЗИ» ЖШС,
Алматы, Қазақстан, e-mail:
\href{mailto:сhomanov@mail.ru}{\nolinkurl{сhomanov@mail.ru}};

Жұмалиева Г.Е. -- техника ғылымдарының кандидаты, «Қазақ өңдеу және
тамақ өнеркәсібі ғылыми-зерттеу институты» ЖШС, Алматы, Қазақстан,
e-mail:
\href{mailto:guljan_7171@mail.ru}{\nolinkurl{guljan\_7171@mail.ru}};

Байзақова А.Б. - жаратылыстану ғылымдарының магистрі, ғылыми қызметкер
Қазақ өңдеу және тамақ өнеркәсібі ғылыми-зерттеу институты, Алматы,
Қазақстан, e-mail:
\href{mailto:ainel0026@gmail.com}{\nolinkurl{ainel0026@gmail.com}}

\emph{{\bfseries Information about the authors}}

Chomanov U.Ch. - doctor of technical sciences, LLP "Kazakh Research
Institute of Processing and Food Industry" Almaty, Kazakhstan, e-mail:
\href{mailto:сhomanov@mail.ru}{\nolinkurl{сhomanov@mail.ru}};

Zhumalieva G.E. - candidate of technical sciences, LLP "Kazakh Research
Institute of Processing and Food Industry" Almaty, Kazakhstan, e-mail:
\href{mailto:guljan_7171@mail.ru}{\nolinkurl{guljan\_7171@mail.ru}};

Baizakova A.B. - master of natural sciences, research fellow, Kazakh
Research Institute of Processing and Food Industry, Almaty, Kazakhstan,
e-mail:
\href{mailto:ainel0026@gmail.com}{\nolinkurl{ainel0026@gmail.com}}
\end{authorinfo}
