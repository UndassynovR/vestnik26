\id{ҒТАМР 06.35.31}{}

\begin{articleheader}
\sectionwithauthors{Д.Н. Беделова, С.Б. Мақыш}{МЕМЛЕКЕТТІК АКТИВТЕРДІ БАСҚАРУ ТИІМДІЛІГІНІҢ АУДИТІН ҰЙЫМДАСТЫРУ ЖӘНЕ ЖҮРГІЗУ ӘДІСТЕМЕСІ}

{\bfseries
\textsuperscript{1}Д.Н. Беделова\textsuperscript{\envelope } \authorid,
\textsuperscript{2}С.Б. Мақыш\authorid}
\end{articleheader}

\begin{affiliation}
{\bfseries \textsuperscript{1}}\emph{Л.Н.Гумилев атындағы Еуразия Ұлттық Университеті, Астана, Қазақстан,}

{\bfseries \textsuperscript{2}} Esil University\emph{, Астана, Қазақстан}

{\bfseries \textsuperscript{\envelope }}Корреспондент-автор:everest-astana@mail.ru
\end{affiliation}

Біздің республика үшін салыстырмалы түрде жаңа құрал тиімділік аудиті
дамыған елдерде кеңінен тараған мемлекеттік аудиттің ерекше түрі болып
табылады. Оның мақсаты қаржылық есеп берудің белгіленген нормаларға
сәйкестігін тексерумен шектелетін дәстүрлі қаржылық аудиттен
ерекшеленетін мемлекеттік ресурстарды басқарудың тиімділігіне,
тиімділігіне және тиімділігіне терең талдау жасау болып табылады.

Бірқатар шет елдерде тиімділік аудитін қолдану үнемі кеңейіп келеді және
кейбір жағдайларда жоғары мемлекеттік бақылау органдарының барлық
қызметінің 60-70\% құрайды.Бұл үрдіс биліктің ашықтығы мен
жауапкершілігіне жоғары талаптар қоятын жетілген азаматтық қоғамның
пайда болуына байланысты. Мұндай қоғамдардағы азаматтар мемлекеттік
активтерді басқаруға белсенді қызығушылық танытады және оларды
пайдалануды оңтайландыруды талап етеді, бұл тиімділік аудитін кеңірек
деңгейде енгізуді ынталандырады.

Мемлекеттің жауапкершілігінің жоғары деңгейі тек күшті тәртіптік
құрылымды қамтамасыз етіп қана қоймай, үкіметті мемлекеттік ресурстарды
басқаруда әлеуметтік және қаржылық тиімділікті арттыру әдістерін шұғыл
іздеуге мәжбүр етті. Атап өтілгендей, мемлекеттік қаражатты ұтымсыз,
ысырап ету немесе тиімсіз пайдалану жағдайларын анықтауда тиімділік
аудиті шешуші рөл атқарады. Осындай тексерулерді жүргізу арқылы
мемлекеттік бақылау органдары бюджеттік ресурстарды басқаруды
айтарлықтай жақсартуға және мемлекеттік сектордағы жалпы өнімділікті
арттыруға бағытталған ұсыныстар әзірлейді. Бұл шаралар мемлекеттік
шығындарды оңтайландыруға және ресурстарды неғұрлым жауапкершілікпен
және тиімді пайдалануды қамтамасыз етуге көмектеседі, бұл өз кезегінде
азаматтардың сенімін нығайтады және қоғамдық әл-ауқаттың өсуіне ықпал
етеді.

{\bfseries Түйін сөздер:} аудит, басқару, жоспарлау, жүйелі тәсіл, кешенді
талдау, мемлекеттік активтер, стандарттар, тиімділік аудиті.

\begin{articleheader}
{\bfseries МЕТОДОЛОГИЯ ОРГАНИЗАЦИИ И ПРОВЕДЕНИЯ АУДИТА ЭФФЕКТИВНОСТИ УПРАВЛЕНИЯ ГОСУДАРСТВЕННЫМИ АКТИВАМИ}

{\bfseries
\textsuperscript{1}Д.Н. Беделова\textsuperscript{\envelope },
\textsuperscript{2}С.Б. Мақыш}
\end{articleheader}

\begin{affiliation}
\textsuperscript{1}\emph{Евразийский Национальный Университет им. Л.Н.Гумилев, Астана, Казахстан,}

\textsuperscript{2}\emph{EsilUniversity, Астана, Казахстан,}

e-mail:everest-astana@mail.ru
\end{affiliation}

Аудит эффективности -- относительно новый для нашей республики
инструмент -- представляет собой особый вид государственного аудита,
широко распространенный в развитых странах. Его целью является
проведение углубленного анализа эффективности, результативности и
результативности управления государственными ресурсами, что отличается
от традиционных финансовых проверок, которые ограничиваются проверкой
соответствия финансовой отчетности установленным нормам.

В ряде зарубежных стран использование аудитов эффективности постоянно
расширяется и в ряде случаев составляет 60-70\% всей деятельности высших
органов государственного контроля.Эта тенденция обусловлена появлением
зрелого гражданского общества, предъявляющего высокие требования к
прозрачности и подотчетности правительства. Граждане в таких обществах
проявляют активный интерес к управлению общественными активами и требуют
оптимизации их использования, что способствует внедрению проверок
эффективности на более широком уровне.

Высокий уровень ответственности государства не только обеспечил сильную
дисциплинарную структуру, но и заставил правительство срочно искать пути
повышения социальной и финансовой эффективности управления
государственными ресурсами. Как отмечалось, аудит эффективности играет
решающую роль в выявлении случаев нерационального, расточительного или
неэффективного использования государственных средств. Проводя такие
проверки, органы государственного контроля вырабатывают рекомендации,
направленные на существенное улучшение управления бюджетными ресурсами и
повышение общей производительности в государственном секторе. Данные
меры помогают оптимизировать государственные расходы и обеспечить более
ответственное и эффективное использование ресурсов. В свою очередь, это
укрепляет доверие граждан и способствует росту общественного
благосостояния.

{\bfseries Ключевые слова:} аудит, менеджмент, планирование, системный
подход, комплексный анализ, государственные активы, стандарты, аудит
эффективности.

\begin{articleheader}
{\bfseries METHODOLOGY OF ORGANIZING AND CONDUCTING AN AUDIT OF THE EFFICIENCY OF STATE ASSETS MANAGEMENT}

{\bfseries
\textsuperscript{1}D.N. Bedelova\textsuperscript{\envelope },
\textsuperscript{2}S.B. Makysh}
\end{articleheader}

\begin{affiliation}
\textsuperscript{1}\emph{L.N.Gumilyov Eurasian National University , Astana, Kazakhstan,}

\textsuperscript{2}\emph{Esil University, Astana, Kazakhstan,}

\emph{e-mail:} everest-astana@mail.ru
\end{affiliation}

Efficiency audit, a relatively new tool for our republic, is a special
type of government audit, widespread in developed countries. Its purpose
is to provide an in-depth analysis of the effectiveness, efficiency and
effectiveness of the management of public resources, which differs from
traditional financial audits, which are limited to checking the
compliance of financial statements with established norms.

In a number of foreign countries, the use of performance audits is
constantly expanding and in some cases accounts for 60-70\% of all
activities of supreme state control bodies.This trend is driven by the
emergence of a mature civil society with high demands for government
transparency and accountability. Citizens in such societies take an
active interest in the management of public assets and demand
optimization of their use, which encourages the introduction of
performance audits at a broader level.

The high level of state responsibility not only provided a strong
disciplinary structure, but also forced the government to urgently look
for ways to improve the social and financial efficiency of the
management of public resources. As noted, performance auditing plays a
critical role in identifying cases of wasteful, wasteful or ineffective
use of public funds. By conducting such audits, state control bodies
develop \\recommendations aimed at significantly improving the management
of budgetary resources and increasing overall productivity in the public
sector. These measures help optimize government spending and ensure more
responsible and efficient use of resources. In turn, this strengthens
citizen confidence and contributes to the growth of public welfare.

{\bfseries Keywords:} audit, management, planning, systems approach,
comprehensive analysis, state assets, \\standards, performance audit.

\begin{multicols}{2}
{\bfseries Кіріспе.} Тиімділік аудиті - бюджет қаражатының пайдаланылуын
кешенді талдауға және атқарушы билік органдары жүзеге асыратын
бағдарламалардың тиімділігін бағалауға бағытталған тәуелсіз бағалау
{[}1{]}.

Оның мақсаты - осы бағдарламалардың тиімділігін арттыру бойынша мақсатты
ұсыныстар әзірлеу. Бұл аудит процеске қатысушылардың барлығын қамтиды --
бюджет қаражатын басқаратын менеджерлер мен әкімшілерден бастап түпкі
пайдаланушыларға дейін, бюджет ресурстарын басқаруды жан-жақты
жетілдіруді және оларды оңтайлы пайдалануды қамтамасыз етеді {[}2{]}.

Заңнамалық тұрғыдан тиімділік аудитін мемлекеттік басқару тәжірибесіне
енгізу бойынша маңызды қадамдар жасалуда. Атап айтқанда, Бюджет
кодексіне және «Мемлекеттік аудит және қаржылық бақылау туралы»
Қазақстан Республикасының Заңына енгізілген соңғы өзгерістер мемлекеттік
бағдарламаларды бағалау процесінде тиімділік аудитінің рөлін күшейтті.

Маңызды өзгерістер өңірлік тексеру комиссияларын Жоғарғы есеп
палатасының аумақтық департаменттеріне айналдыруды көздейтін аудит
жүйесін реформалаумен де байланысты. Бұл шара аудиторлық процедураларды
стандарттауға және жүргізілетін аудиттердің тәуелсіздігін арттыруға
бағытталған.

Тиімділік аудитін дамытуда 2030 жылға дейінгі мемлекеттік қаржыны
басқару тұжырымдамасы маңызды рөл атқарады. Ол ұзақ мерзімді жоспарлауды
және мемлекеттік ресурстарды пайдалану тиімділігін бағалауды қамтамасыз
ететін тәуекелге бағытталған тәсілді және стратегиялық аудитті енгізу
арқылы аудит әдіснамасын жетілдіруді көздейді. Тиімділік аудитін дамыту
перспективалары басқару шешімдерін неғұрлым сапалы және объективті
бағалауды қамтамасыз ететін деректерді жинау мен талдаудың бірыңғай
ақпараттық жүйесін құрумен де байланысты.

Қазақстан Республикасындағы мемлекеттік активтерді басқарудың тиімділік
аудиті дамудың тұрақты тенденциясын көрсетеді. Осы құралды іске асыруды
қолдайтын заңнамалық бастамалар, сондай-ақ қол жеткізілген практикалық
нәтижелер оның мемлекеттік басқару тиімділігін арттыру және мемлекеттік
ресурстарды ұтымды пайдалану үшін маңыздылығын растайды.

Тиімділік аудиті мемлекеттік органдар мен бюджет қаражатын алушылардың
өз функциялары мен міндеттерін орындау кезінде осы қаражатты
пайдаланудың тиімділігі мен ұтымдылығын кешенді бағалауға бағытталған
іс-шаралар кешенін қамтиды {[}3{]}.

Тиімділік аудиті ұлттық ресурстарды және мемлекеттік қаржыларды
басқаруды жақсарту жолдарын анықтайды, олардың сапасын жақсарту үшін
қажетті шараларды қабылдауға көмектеседі.

Қазақстанның ерекшеліктерін ескере отырып, бұл мәселенің маңыздылығы
еліміздің алдында экономиканы әртараптандыру, инфрақұрылымды жаңғырту,
табиғи ресурстарды басқару және әлеуметтік саясаттың тұрақтылығын
қамтамасыз ету сияқты бірқатар міндеттер тұрғанымен түсіндіріледі. Бұл
міндеттерді мемлекеттік активтерді тиімді басқарусыз шешу мүмкін емес,
бұл олардың пайдаланылуын тексеруді ерекше өзекті етеді.

Зерттеу нәтижелері мемлекеттік сектордың қаржылық жағдайын жақсартуға
тікелей әсер ететін мемлекеттік активтерді басқарудың қолданыстағы
процестерін бағалау және оңтайландыру тетіктерін жетілдіру бойынша
ұсынымдарды әзірлеу үшін пайдаланылуы мүмкін.

Мақаланың мақсаты -- мемлекеттік ресурстарды басқару сапасын арттыру
және оларды ұтымды пайдалануды қамтамасыз ету үшін мемлекеттік
активтерді басқару тиімділігіне аудитті ұйымдастыру және жүргізу
әдістемесін әзірлеу және негіздеу.

\emph{Зерттеудің гипотезасы:} кешенді тәсілді және қазіргі заманғы
талдау әдістерін қолдану негізінде мемлекеттік активтерді басқару
тиімділігінің аудитін ұйымдастыру мен жүргізудің әдіснамалық негізделген
жүйесін әзірлеу және енгізу мемлекеттік басқару органдарының ашықтығын,
тиімділігін және ұтымдылығын айтарлықтай арттырады. мемлекет активтерін
пайдалану. Бұл өз кезегінде активтерді басқаруды оңтайландыруға, тиімсіз
пайдалану тәуекелдерін азайтуға және олардың елдің экономикалық дамуына
қосатын үлесін арттыруға әкеледі.

{\bfseries Материалдар мен әдістер.} Аудиторлық қызметтің тәуелсіз саласы
ретінде тиімділік аудитінің бастауы 1970 жылдан басталады, ол кезде
«атқарушы аудит» термині алғаш рет айтылып, Жоғары қаржылық бақылау
органдарының халықаралық конгресінде (INTOSAI) ресми түрде танылды.

Бұл термин 1977 жылғы Лима декларациясында да көрініс тапты. Қаражаттың
дұрыс пайдаланылуын тексеруге және қаржылық есептерді жасауға
бағытталған стандартты қаржылық аудиттен басқа мемлекеттік қаражаттардың
тиімділігі мен рентабельділігін бағалауға бағытталған бақылаудың жеке
түрі бар екенін атап өткен жөн.

Бақылаудың бұл нысаны басқарудың жекелеген элементтерін, оның ішінде
басқару қызметінің барлық саласын, оның ішінде тұтастай басқару
жүйесінің құрылымы мен ұйымын зерттеу шеңберінен әлдеқайда асып түседі
{[}4{]}.

Алибекова Б.А., Зейнелгабдин А.Б., Карибаев А.А.-К., Макыш С.Б.,
Мухаметкарим А.М., Ногербекова С.Н., Нурхалиева Д.М., Тажикенова С.К.,
Торебекова Б.О., К.Э. Тиімділік аудиті шешуші рөл атқарады деп саналады,
өйткені ол бағалау мүмкіндігін ғана емес, сонымен қатар бюджет қаражатын
пайдалану тиімділігінің дәрежесін егжей-тегжейлі талдауды қамтамасыз
етеді {[}5{]}.

1-кестеде шетелдік ғылыми сала өкілдерінің тиімділік аудитінің
анықтамасы берілген.
\end{multicols}

\begin{table}[H]
\caption*{1 - кесте. Шетелдік ғылыми сала өкілдерінің пікірі бойынша тиімділік аудитінің анықтамасы}
\centering
\begin{tblr}{
  colspec = {X[1] X[6]},
  cell{1}{1} = {c},
  cell{2}{1} = {c},
  cell{3}{1} = {c},
  cell{4}{1} = {c},
  cell{5}{1} = {c=2}{},
  hlines,
  vlines,
}
А.Н.Саунин                                          & мемлекет қаражатын пайдалану нәтижесінде алынған экономикалық нәтижелерді талдауға бағытталған қаржылық бақылау механизмі [6].                                                                                                                                                                                             \\
Е.Н. Синиева                                        & бюджет қаражатын пайдалану тиімділігін неғұрлым егжей-тегжейлі және жан-жақты бағалауға бағытталған мемлекеттік қаржылық бақылаудың құрамдас құралы болып табылады [7].                                                                                                                                                    \\
С.Н.Рябухин                                         & ақшаның қаншалықты тиімді және үнемді пайдаланылғанын жан-жақты бағалауға бағытталған ұйымның қызметін терең зерттеу болып табылады. Ол сондай-ақ басқару процестері мен ресурстарды пайдалануды жақсартудың ықтимал бағыттарын анықтай отырып, қол жеткізілетін мақсаттар мен міндеттердің барлық спектрін бағалайды [8]. \\
Ф.В.Голубев                                         & бұл ұйымның ресурстарын пайдаланудың үнемділігін, тиімділігі мен тиімділігін зерттеуге бағытталған кешенді аналитикалық бағалау [9].                                                                                                                                                                                       \\
Ескерту - [6-10] дереккөз негізінде автор әзірлеген &                                                                                                                                                                                                                                                                                                                            
\end{tblr}
\end{table}

\begin{multicols}{2}
Тиімділік аудиті жағдайында мемлекеттік ресурстар мемлекеттің
билігіндегі активтердің кең ауқымын қамтиды. Қазақстан Республикасының
заңнамасына сәйкес оларға мемлекеттің стратегиялық мақсаттары мен
міндеттеріне қол жеткізу үшін пайдаланылатын қаржы ресурстары, табиғи
ресурстар, өндірістік қуаттар, еңбек және ақпараттық ресурстар жатады.
Бұл анықтамада мемлекеттік ресурстарға қаржылық (бюджеттік қорлар,
мемлекеттік мүлік) де, қаржылық емес активтер де (табиғи ресурстар,
еңбек ресурстары, ақпарат) жататынын атап көрсетеді.

Мемлекет қаржысы өз кезегінде мемлекет қаржысын қалыптастыру, бөлу және
пайдалану жүйесін білдіреді. Оларға бюджет қаражаты, мемлекеттік
бюджеттен тыс қорлар және мемлекеттік органдардың қарамағындағы басқа да
қаржылық активтер жатады. Мемлекеттік қаржыны басқару мемлекеттік
бағдарламалар мен функцияларды іске асыру үшін қаржы ресурстарын тиімді
және мақсатты пайдалануды қамтамасыз етуге бағытталған.

Тиімділік аудитінде мемлекеттік ресурстар мен қаржыны пайдалануды
бағалау мемлекеттік органдар өз мақсаттарына жету үшін қаржылық және
қаржылық емес активтерді қаншалықты тиімді және тиімді басқаратынын
талдауды қамтиды.

Тиімділік аудиті мемлекеттік органдардың жауапкершілігін арттыруда және
азаматтардың мемлекеттік институттарға сенімін нығайтуда басты рөл
атқарады. Мемлекеттік органдардың қызметін дербес және объективті
бағалау арқылы аудит кемшіліктерді, ресурстарды тиімсіз пайдалануды және
бұзушылықтарды анықтауға көмектеседі, бұл оларды жою бойынша негізделген
шараларды қабылдауға мүмкіндік береді.

Аудит тетіктері мемлекеттік органдардың жауапкершілігін арттырады:

- аудит нәтижелерін жариялау азаматтардың мемлекеттік органдардың
қызметі туралы ақпаратқа қол жеткізуін қамтамасыз етеді, бұл ашықтыққа
ықпал етеді және сыбайлас жемқорлық тәуекелдерін азайтады;

- бұзушылықтарды немесе нашар тәжірибелерді анықтау және құжаттау
лауазымды тұлғаларды түзету шараларын қолдануға міндеттейді, бұл олардың
қоғам алдындағы жауапкершілігін арттырады;

- аудит қорытындысы бойынша берілген ұсынымдар көрсетілетін мемлекеттік
қызметтердің сапасына тікелей әсер ететін процестерді оңтайландыруға
және басқару тиімділігін арттыруға бағытталған.

Тиімді басқаруды және мемлекеттік ресурстарды пайдалану үшін
жауапкершілікті көрсетуге үкіметтің адалдығын көрсету арқылы тиімділік
аудиті арқылы есеп беруді жақсарту қоғамның сенімін арттырады.

Тиімділік аудиті халықтың сенімін арттыруға және халықтың әл-ауқатын
жақсартуға көмектеседі:

- сыбайлас жемқорлық тәуекелдерін азайту: тұрақты және мұқият тексерулер
сыбайлас жемқорлық жағдайларын анықтайды және алдын алады, ресурстарды
неғұрлым әділ және әділетті бөлуді қамтамасыз етеді;

- қаражатты бөлудің ашықтығын қамтамасыз ету: аудит халықты мемлекет
қаражатының қалай және қайда жұмсалатыны туралы ақпаратпен қамтамасыз
етеді, бұл азаматтардың мемлекеттік шығындардың орындылығы мен
негізделуіне деген сенімін арттырады;

- тиімсіз шығыстарды анықтау және оларды оңтайландыру бойынша ұсыныстар
беру үнемделген қаражатты әлеуметтік маңызы бар жобаларға жұмсауға,
халықтың өмір сүру сапасын арттыруға мүмкіндік береді.

Осылайша, тиімділік аудиті мемлекеттік ресурстардың пайдаланылуын
бақылауды қамтамасыз етіп қана қоймайды, сонымен қатар азаматтардың
мемлекеттік институттарға сенімін арттыру құралы ретінде қызмет етеді,
халықтың әл-ауқатын нығайтуға ықпал етеді.

Бұл материалдарды мақалаға қосу бізге негізгі терминдердің мағынасын
толық ашуға және мемлекеттік ресурстарды басқару және халықтың сенімін
нығайту контекстінде тиімділік аудитінің практикалық маңыздылығын
көрсетуге мүмкіндік береді.

Аудиттің бұл түрі әлеуметтік топтардың мемлекет қаражатын пайдаланумен
қанағаттануын жай ғана өлшеумен шектелмейді; ол сондай-ақ осы
қанағаттанудың негізгі себептерін анықтауға мүмкіндік береді. Тиімділік
аудиті ашықтықтың жоғарылауын ғана емес, сонымен қатар елдің мемлекеттік
қаржысы мен ресурстарын басқару тиімділігін кеңінен түсінуді қамтамасыз
етеді. Сонымен қатар, ол оңтайландыру мүмкіндіктерін анықтайды, сол
арқылы жалпы мемлекеттік ресурстарды басқару мен пайдалануды жақсартуға
көмектеседі.

Зерттеу әдістері: зерттеу теориялық талдау, жіктеу және индукция
әдістерін қолдану арқылы жүзеге асырылады.

{\bfseries Нәтижелер мен талқылау.}Тиімділік аудиті процесі тексерілетін
ұйымның немесе бағдарламаның өз міндеттерін тиімді және нәтижелі орындау
және қойылған мақсаттарға жету үшін қолда бар ресурстарды қалай
пайдаланғанын зерттейді. Тиімділік аудитінің негізгі мақсаты қызметтің
немесе бағдарламаның тиімділігін, үнемділігін және тиімділігін
қамтамасыз ету болып табылады. Мемлекеттік активтерді басқару аудиті
тиімділік аудитінің он негізгі бағытының ішінде ерекшеленеді.

Тиімділік аудиті барысында талдау бағдарламалық құжаттардың орындалуын
және алға қойылған мақсаттарға қол жеткізуді ғана емес, сонымен қатар
көрсетілетін мемлекеттік қызметтердің сапасын бағалауды, сондай-ақ
әртүрлі адами, қаржылық, табиғи және басқа да ресурстарды басқаруды
қамтиды.

Тиімділік аудиті мемлекеттік аудит субъектісінің нақты аспектілерін
немесе қызметін зерттеуге бағытталған болуы мүмкін немесе әртүрлі
әдістемелерді, соның ішінде көлденең және тік тәсілдерді қолдану арқылы
жүзеге асырылуы мүмкін. Сыртқы мемлекеттік аудит және қаржылық бақылау
бойынша Қазақстан Республикасының Есеп комитетінде қолданылатын рәсімдік
стандарт тиімділік аудитін жүргізуді ғана емес, сонымен қатар олардың
тиімділігін ескере отырып, мемлекеттік аудит субъектілерінің қызметін
бағалауды және зерделеуді қамтиды.

Қазіргі уақытта тиімділік аудиті аудиторлық іс-шаралар ішінде басымдыққа
ие (2-кесте).
\end{multicols}

\begin{table}[H]
\caption*{2 - кесте. 2021-2023 жылдар аралығындағы аудит типтері бойынша ЖАП аудиторлық іс-шараларының құрамы мен құрылымы}
\centering
\begin{tblr}{
  row{1} = {c},
  row{2} = {c},
  cell{1}{1} = {r=2}{},
  cell{1}{2} = {c=2}{},
  cell{1}{4} = {c=2}{},
  cell{1}{6} = {c=2}{},
  cell{1}{8} = {c=2}{},
  cell{3}{2} = {c},
  cell{3}{3} = {c},
  cell{3}{4} = {c},
  cell{3}{5} = {c},
  cell{3}{6} = {c},
  cell{3}{7} = {c},
  cell{3}{8} = {c},
  cell{3}{9} = {c},
  cell{4}{2} = {c},
  cell{4}{3} = {c},
  cell{4}{4} = {c},
  cell{4}{5} = {c},
  cell{4}{6} = {c},
  cell{4}{7} = {c},
  cell{4}{8} = {c},
  cell{4}{9} = {c},
  cell{5}{2} = {c},
  cell{5}{3} = {c},
  cell{5}{4} = {c},
  cell{5}{5} = {c},
  cell{5}{6} = {c},
  cell{5}{7} = {c},
  cell{5}{8} = {c},
  cell{5}{9} = {c},
  cell{6}{2} = {c},
  cell{6}{3} = {c},
  cell{6}{4} = {c},
  cell{6}{5} = {c},
  cell{6}{6} = {c},
  cell{6}{7} = {c},
  cell{6}{8} = {c},
  cell{6}{9} = {c},
  cell{7}{2} = {c},
  cell{7}{3} = {c},
  cell{7}{4} = {c},
  cell{7}{5} = {c},
  cell{7}{6} = {c},
  cell{7}{7} = {c},
  cell{7}{8} = {c},
  cell{7}{9} = {c},
  cell{8}{2} = {c},
  cell{8}{3} = {c},
  cell{8}{4} = {c},
  cell{8}{5} = {c},
  cell{8}{6} = {c},
  cell{8}{7} = {c},
  cell{8}{8} = {c},
  cell{8}{9} = {c},
  cell{9}{1} = {c=9}{},
  vlines,
  hline{1,3-10} = {-}{},
  hline{2} = {2-9}{},
}
Аудиторлық іс-шаралар                           & 2021 жыл &      & 2022 жыл &      & 2023 жыл &      & 2024 жыл &      \\
                                                & саны     & \%   & саны     & \%   & саны     & \%   & саны     & \%   \\
Тиімділік аудиті                                & 9        & 47,4 & 12       & 32,4 & 13       & 36,1 & 38       & 61,2 \\
Тиімділік пен сәйкестік аудиті                  & 8        & 42,1 & 13       & 35,1 & 16       & 44,4 & 14       & 22.6 \\
Сәйкестік аудиті                                & 1        & 5,3  & 7        & 18,9 & 5        & 13,9 & 1        & 1.6  \\
Қаржылық есептілік аудиті                       & 1        & 5,2  & 1        & 2,7  & 1        & 2,8  & 1        & 1.6  \\
Өзге де                                         & 2        &      & 4        & 11   & 1        & 2,8  & 8        & 13   \\
Барлығы                                         & 21       & 100  & 37       & 100  & 36       & 100  & 62       & 100  \\
Ескерту – қайнар көздің негізінде құрастырылған &          &      &          &      &          &      &          &      
\end{tblr}
\end{table}

\begin{multicols}{2}
2-кестеде келтірілген деректермен расталғандай, талданатын кезеңдегі
аудиторлық және сараптамалық-талдау іс-шараларының саны айтарлықтай оң
серпіні көрсетеді, бұл мемлекеттік аудит саласының кеңеюін және оның
маңыздылығының артқанын көрсетеді. Бұл іс-шаралар аудиттің
мамандандырылған түрлерін және басқа да құпия салаларын қоса алғанда,
аудиторлық қызметтің кең ауқымын қамтиды, бұл Қазақстандағы аудит
процестерінің барған сайын күрделі сипатын көрсетеді.

Мемлекеттік активтерді басқарудың тиімділік аудитін ұйымдастыру нақты
құрылымды және әрекеттердің ретін талап етеді. Бұл жұмыс тобын құруды,
аудиттің негізгі критерийлері мен мақсаттарын анықтауды, қажетті
процедураларды белгілеуді қамтиды.

Бірінші қадам -- қандай мемлекеттік активтер аудитке жататынын және
қандай мақсаттарға жету керектігін анықтау. Қаржылық көрсеткіштерді,
стратегиялық мақсаттарға қол жеткізуді және заңнаманың сақталуын қамтуы
мүмкін оларды пайдалану тиімділігін бағалау критерийлерін әзірлеу
маңызды.

Содан кейін жұмыстың барлық кезеңдерін қамтуы тиіс аудит жоспарын
дайындау керек: мәліметтерді жинау мен ақпаратты талдаудан бастап
қорытындылар мен ұсыныстарға дейін. Бұл кезеңде қажетті құзыреттері бар
аудиторлар командасын жасақтау, қажетті құжаттамаға қолжетімділікті
қамтамасыз ету және жауапты органдармен өзара іс-қимыл орнату қажет.

Одан әрі тексерулер активтерді басқарудың барлық деңгейлерінде
жүргізіледі. Бұл есептілікті талдауды, ішкі бақылауды, қолданыстағы
басқару процестерінің тиімділігін зерттеуді және процестің негізгі
қатысушыларымен сұхбатты қамтуы мүмкін.

Қажетті ақпарат жиналып, талданғаннан кейін активтерді басқарудың
ағымдағы жағдайы туралы қорытындылары бар қорытынды есеп жасалады. Ол
анықталған проблемаларды да, оларды шешу бойынша ұсыныстарды да,
тиімділікті арттыру үшін басқару стратегиясын жетілдіру бойынша
ұсыныстарды да көрсетуі керек.

Қорытынды кезең - аудит нәтижелерін шешім қабылдауға жауапты органдарға
ұсыну. Мемлекеттік активтерді басқаруды оңтайландыру және олардың
мемлекет пен азаматтар үшін құндылығын арттыру бойынша ұсыныстар мен
іс-шаралар жоспарларына басты назар аудару керек.

Мемлекеттiк активтердi басқарудағы мемлекеттiк аудиттiң әдiстемелiк
негiзi аудиторлық процесте аудиторлардың алдына қойылған мақсаттар мен
мiндеттердi анықтау болып табылады. Мұндай аудиттің негізгі
мақсаттарының бірі дағдарыстық жағдайлардан шығуға бағытталған
дағдарысқа қарсы бюджет қаражатын пайдаланудың тиімділігі мен
тиімділігін бағалау болып табылады. Осы мақсатқа жету үшін аудиторларға
бюджеттік рәсімдер мен бюджет қаражатын пайдалануды реттейтін заңнаманың
сақталуына кешенді аудит жүргізу, сондай-ақ белгіленген дағдарыстық
міндеттерді ескере отырып, оларды пайдалану тиімділігін бағалау қажет.

Жалпы алғанда, мемлекеттік активтерді басқарудың тиімділік аудиті бюджет
қаражатының жұмсалу тиімділігін бақылау мен бағалаудың негізгі құралы
болып табылады. Мемлекеттік аудиттің әдіснамалық негіздерін дұрыс
қолдану арқылы мемлекеттік активтерді басқарудағы проблемалар мен
кемшіліктерді анықтауға, басқару шешімдерін қабылдау үшін тиісті
ақпаратпен қамтамасыз етуге, сондай-ақ бюджет қаражатын жұмсау
тиімділігін және халықтың сенімін арттыруға көмектесуге болады.

Қазіргі уақытта мемлекеттік активтерді басқару тиімділігіне аудит
жүргізу үшін арнайы әдістемелік қамтамасыз ету құралдары жоқ. Дегенмен,
Қазақстан Республикасында тиімділік аудитінде қолданылатын процедуралық
стандарттар сенімді негізге ие. Жоғары қаржылық бақылау органдарының
халықаралық стандарттары (ISSAI) және Ішкі аудиторлар институтының (IIA)
ішкі аудиттің кәсіби тәжірибесіне арналған стандарттары стандарттар мен
олардың әртүрлі ұлттық контексттерде қолданылуы туралы түсінік беретін
ортақ стандартты қамтамасыз етеді.

Стандарттарды енгізу ішкі аудит қызметі жұмыс істейтін ортаға сәйкес
және қолданыстағы заңнаманы ескере отырып реттеледі.ISSAI стандарттары
мемлекеттік активтерді тексеруді қоса алғанда, барлық мемлекеттік аудит
мамандарына қолдануға арналған әмбебап нұсқауларды береді. Бұл олардың
объективтілігі мен сенімділігін қамтамасыз ете отырып, аудит жүргізудің
бірыңғай стандартты тәсілін қамтамасыз етеді. Осылайша, дағдарысқа қарсы
шаралар аясындағы мемлекеттік аудит бюджет қаражатын тиімді пайдалануға
және әлеуметтік-экономикалық мақсаттарға қол жеткізуге бағытталған
құрамдас құралға айналады.

Дағдарыс кезінде, экономикалық тұрақсыздық халықтың қаржы институттары
мен мемлекеттік органдарға деген сенімін сынайтын кезде аудиторлардың
тәуелсіздігі басты рөл атқарады. Ол мемлекеттік аудит кезіндегі
объективтілік пен бейтараптықтың таптырмас кепілі болып табылады, бұл
аудит нәтижелеріне сенімділік үшін негіз жасауға мүмкіндік береді.
Аудиторлардың тәуелсіздігі заңнамада да, кәсіби стандарттарда да
бекітілген, ол кез келген мүдделі тұлғалардың, мейлі ол мемлекеттік
органдар немесе кәсіпорындардың басқару органдары болсын, ықпал етуден
босатылады. Бұл аудитордың тәуелсіздігі аудиттердің адалдығы мен
объективтілігіне халықтың сенімінің негізі болып табылады, бұл өз
кезегінде қаржылық ортаның тұрақтылығын және басқару жүйесіне үлкен
сенімнің артуына ықпал етеді.

Мәліметтердің дәлдігі мен бейтараптығын қамтамасыз ету үшін мемлекеттік
аудиторлар жоғары біліктілікке ие және заңмен және халықаралық ISSAI
стандарттарында белгіленген процедуралар мен стандарттарды қатаң
сақтайды. Олар қаржылық құжаттарды талдайды, қосымша ақпарат жинайды
және оның қолданыстағы заңнамаға сәйкестігін тексереді, сол арқылы аудит
процесінің сенімділігі мен объективтілігін қамтамасыз етеді.

Тиімділік аудиті принциптерін қолдану жөніндегі нұсқаулық бағдарлама мен
жоба аудиті, ресурстарды бағалау және ұйымдық өнімділікті бағалау сияқты
әртүрлі жағдайларда осы принциптерді пайдалану бойынша практикалық кеңес
береді. Бұл бөлімде сонымен қатар тиімділік аудиті есептерінің үлгілері
және мемлекеттік аудиторлардың осы принциптерді түсінуі мен тәжірибеде
қолдануы үшін басқа пайдалы нұсқаулар бар. Қазақстанда мемлекеттік
аудиттің ұлттық стандарттарын қалыптастыру кезінде халықаралық тәжірибе
мен стандарттар, соның ішінде ISSAI ескеріледі. Қазақстанның мемлекеттік
аудитінің ұлттық стандарты мемлекеттік аудитті жүргізуге және оның
қорытындыларына қойылатын негізгі талаптарды анықтайды, сондай-ақ осы
процеске барлық қатысушылардың рөлдері мен міндеттерін
бөледі.Қазақстандағы мемлекеттік аудиттің ұлттық стандарттары ISSAI
халықаралық стандарттарымен, соның ішінде тәуелсіздік, объективтілік,
құзыреттілік және құпиялылық сияқты принциптермен үйлестірілген. Бұл
қағидаттар Қазақстандағы мемлекеттік аудит тәжірибесінің әлемдік
стандарттарға сәйкестігін және елдің мемлекеттік аудит процесінің сапасы
мен сенімділігін қамтамасыз етеді {[}11{]}. Қазақстанда мемлекеттік
аудиттің ұлттық стандарттарын әзірлеу кезінде ұлттық нормаларды
халықаралық стандарттармен үйлестіру және мемлекеттік аудиттің сапасын
арттыру мақсатында халықаралық стандарттар мен ISSAI қағидаттары
ескеріледі.

Мемлекеттік активтерді басқарудағы мемлекеттік аудиттің әдіснамалық
негізі үш деңгейден тұрады: жалпы, заңнамалық және әдістемелік.
Мемлекеттік аудиттің құрылымын анықтауда Қазақстан Республикасының
Конституциясы басты рөл атқарады. Қазақстан Республикасының Бюджет
кодексі бюджеттік процесс шеңберінде ішкі аудит пен бақылаудың
қажеттілігін бекітеді {[}12-15{]}.

Мемлекеттік активтерді басқарудағы мемлекеттік аудиттің әдістемелік
негіздері 1-суретте көрсетілген.
\end{multicols}

{\bfseries 1 - сурет. Мемлекеттік активтерді басқарудағы мемлекеттік аудиттің әдістемелік негіздері}

\emph{Ескерту - {[}12-15{]} дереккөздерден құрастырылған}

\begin{multicols}{2}
Қазақстанның мемлекеттік активтерін басқарудағы мемлекеттік аудиттің
әдіснамалық негізі үш деңгейде құрылымдалған:

- жалпы деңгей - мемлекеттік аудитті ұлттық деңгейде де, халықаралық
деңгейде де реттейтін негізгі принциптер мен нормаларды қамтиды;

- заңнамалық деңгей мемлекеттік аудит пен бақылау жүргізуді реттейтін
мамандандырылған құқықтық актілермен көрсетіледі. Атап айтқанда,
Қазақстан Республикасының Бюджет кодексінде бюджеттік процесс ішкі
аудитті де, бақылауды да міндетті түрде ұйымдастыру мен жүргізуді талап
ететіні анық көрсетілген. Заңдар мемлекеттік органдардың міндеттерін
анықтайды және олардың активтерді басқару саласындағы есептілігін
реттейді;

- әдістемелік деңгей аудиторлық тапсырмаларды іс жүзінде орындауға
бағытталған әдістемелік ұсыныстар мен процедураларды әзірлеуді көздейді.
Оған ұлттық аудит стандарттары, сондай-ақ ISSAI сияқты ел жағдайларына
бейімделген халықаралық стандарттар кіреді.

Барлық мемлекеттік аудит органдарының қызметін реттеу «Мемлекеттік аудит
және қаржылық бақылау туралы» Қазақстан Республикасының 2015 жылғы 12
қарашадағы № 392 Заңымен көзделген. Барлық мемлекеттік аудит
органдарының қызметін реттеу «Мемлекеттік аудит және қаржылық бақылау
туралы» Қазақстан Республикасының 2015 жылғы 12 қарашадағы № 392 Заңымен
көзделген.

Мемлекеттік активтерді басқарудағы әдістемелік мемлекеттік аудиттің
негіздері:

- «Сыртқы мемлекеттік аудит және қаржылық бақылау қағидаларын бекіту
туралы» Республикалық бюджеттің атқарылуын бақылау жөніндегі есеп
комитетінің 2020 жылғы 30 шілдедегі № 6-НҚ нормативтік қаулысы;

- «Ішкі мемлекеттік аудитті және қаржылық бақылауды жүргізу қағидаттарын
бекіту туралы» Қазақстан Республикасы Қаржы министрінің 2018 жылғы 19
наурыздағы No 392 бұйрығы.

Мемлекеттік активтерді басқаруда мемлекет қаражатын пайдалану
тиімділігін бағалау күрделі де маңызды міндеттердің бірі деп есептейміз.
Бұл күрделі процедура, өйткені аудиттен өтетін органдарға жүктелген
функциялар мен міндеттерді орындауға байланысты тұтастай алғанда қоғамға
немесе оның жекелеген сегменттеріне әсер етудің түпкілікті нәтижелерін
анықтау және өлшеу қажет.

Тиімділік аудиті әдістемесін қолдану келесі қадамдарды қамтиды:

- аудитті жоспарлау кезеңінде аудиторлар аудиттің мақсаттары мен
міндеттерін анықтауға кіріседі, нәтижелерді бағалау критерийлерін
белгілейді, тексерілетін салаларды анықтайды және аудитті өткізу бойынша
іс-шаралар жоспарын жасайды;

- бұл кезеңде аудиторлар алдын ала белгіленген критерийлерге сәйкес
өнімділікті егжей-тегжейлі бағалауды жүргізеді. Мұны істеу үшін олар
құжаттаманы қарау, ұйымдағы адамдармен сөйлесу және процестер мен
әрекеттерді бақылау сияқты әртүрлі әдістерге жүгінеді. Бұл кезең
аудиторлар орындалған жұмыстың алдын ала белгіленген стандарттарға
немесе тиімділік критерийлеріне қаншалықты сәйкес келетінін анықтауға
ұмтылатын маңызды кезең болып табылады;

-деректерді талдау сатысында аудиторлар ақпаратты неғұрлым
егжей-тегжейлі зерттеуді жүзеге асырады, критерийлермен
сәйкессіздіктерді ғана емес, сонымен қатар олардың тамырларын да
анықтайды. Олар не болғанын ғана емес, оның неліктен болғанын және оның
салдары қандай болуы мүмкін екенін түсінуге тырысады. Нәтижелерді талдау
кезінде олар ұйымның жұмысына ең көп әсер ететін негізгі салаларға назар
аударады. Қорытынды есепте аудиторлар фактілер мен сандарды ғана емес,
сонымен қатар басшылыққа операцияның тиімділігі мен тиімділігін арттыру
үшін негізделген шешімдер қабылдауға көмектесетін мағыналы талдауды
ұсынады;

- бұл кезеңде аудиторлар ұйым басшылығымен белсенді диалогқа түседі,
анықталған проблемалардың мәнін зерттейді және оларды шешу бойынша нақты
ұсыныстарды ұсынады. Тиімділік аудитінің әдістемесі ұйымның ағымдағы
жағдайы мен қызметін бағалауға ғана емес, сонымен қатар оның
табыстылығын анықтайтын негізгі факторларды анықтауға мүмкіндік береді.
Осы аспектілерді басшылықпен бірге қарастыра отырып, аудиторлар
бизнес-процестерді үздіксіз жақсартуға және стратегиялық мақсаттарға қол
жеткізуге бағытталған бағытталған іс-қимыл жоспарын жасайды {[}15{]}.

Бұл тәсіл ұйымның тұрақты дамуына және оның нарықтағы бәсекеге
қабілеттілігіне ықпал етеді. Мемлекеттік активтерді басқарудың тиімділік
аудитін жүргізу барысында этикалық нормалар мен қағидаттарды сақтау
маңызды аспект болып табылады. Аудитор активтерді пайдалану тиімділігін
бағалауда тәуелсіз, объективті және бейтарап болуы керек. Сонымен қатар,
мәліметтерді бұрмалауды болдырмау және ашықтықты қамтамасыз ету үшін
құпия ақпаратты қорғауға және қаржылық есептілік стандарттарын сақтауға
ерекше назар аудару қажет. Аудиторлық тәжірибеде этикалық нормаларды
сақтау мысалдарын мемлекеттік жер ресурстарын пайдалану саласындағы
аудит мысалында келтіруге болады. Бір жағдайда заңбұзушылықтар
активтерді мақсатсыз пайдалануға әкеліп соқтырған жер операцияларының
ашықтығының жеткіліксіздігімен байланысты болды. Бұл ретте аудиторлық
топ барлық деректерді тәуелсіз тексеруді қамтамасыз етті және жер
активтерін есепке алу жүйесіне өзгерістер енгізуге және олардың
пайдаланылуына бақылауды жақсартуға мүмкіндік беретін нәтижеге қол
жеткізді (3 - кесте).
\end{multicols}

\begin{table}[H]
\caption*{3 - кесте. Мемлекеттік активтерді басқару тиімділігін тексеру әдістемесін жетілдірудің көп өлшемді көрсеткіштері}
\centering
\begin{tblr}{
  colspec = {X[1] X[6]},
  row{1} = {c},
  cell{7}{1} = {c=2}{},
  hlines,
  vlines,
}
Көп өлшемді көрсеткіштер                      & Нақты көрсеткіштердің мысалдары                                                                                                                                                                                 \\
{Қаржылық\\көрсеткіштер}                      & - Мемлекеттік активтерді пайдаланудан түсетін кірістердің жалпы бюджеттегі үлесі.- Мемлекеттік жобаларға инвестиция көлемі.                                                                                     \\
Экономикалық көрсеткіштер                     & - Мемлекеттік жобаларға инвестицияларға байланысты ЖІӨ өсу қарқыны.- Инфляция деңгейі және оның мемлекеттік тауарлармен қызметтер бағасына әсері.                                                               \\
{Әлеуметтік\\көрсеткіштер}                    & - Халыққа білім беру және медициналық қызметтердің қолжетімділігі мен сапасы- Мемлекеттік активтерді пайдаланатын аймақтардағы кедейлік пен жұмыссыздық деңгейі.                                                \\
Экологиялық көрсеткіштер                      & -мемлекеттік активтермен байланысты қызмет нәтижесінде қоршаған ортаға ластаушы заттардың шығарындыларының деңгейі- мемлекеттік өндірістік процестермен объектілердің энергия тиімділігі.                       \\
Техникалық көрсеткіштер                       & - жолдарды, көпірлерді, электржелілерін және су жүйелерін қосаалғанда, мемлекеттік инфрақұрылымның жай-күйімен қызмет көрсетуі-мемлекеттік активтерді басқаруда заманауи технологияларды қолданудың тиімділігі. \\
Ескерту -дереккөздерден құрастырылған [12-15] &                                                                                                                                                                                                                 
\end{tblr}
\end{table}

\begin{multicols}{2}
Эмпирикалық талдау әртүрлі мемлекеттік мекемелерде мемлекеттік
активтерді пайдаланудың нақты жағдайларын қарастыруды, сондай-ақ осы
активтерді бақылау мен мониторингілеудің қолданыстағы шетелдік және
қазақстандық тәжірибесін тексеруді қамтиды.

Дамушы елдерде мемлекеттік активтерді басқарудың тиімділік аудиті
ресурстардың шектеулілігіне, инфрақұрылымның дамымағандығына және нақты
экономикалық жағдайларға байланысты бірқатар ерекше қиындықтарға тап
болады. Бұл ерекшеліктер аудитті ұйымдастыру мен өткізуге айтарлықтай
әсер етеді, ерекше көзқарасты және халықаралық стандарттарды жергілікті
жағдайларға бейімдеуді талап етеді.

Дамушы елдерде аудитті ұйымдастыруға әсер ететін негізгі факторлардың
бірі қаржылық және адам ресурстарының шектеулілігі болып табылады.
Шектеулі бюджеттер жағдайында мемлекеттік органдар мен аудиторлық
құрылымдар білікті мамандардың, деректерді өңдеу мен талдауға арналған
техникалық құралдардың тапшылығына, сондай-ақ процестердің жеткіліксіз
автоматтандырылуына жиі ұшырайды {[}16{]}.

Үндістан, Кения және Қазақстан сияқты дамушы елдерде ресурстардың
шектеулері мемлекеттік активтерді басқару тиімділігінің аудитін
ұйымдастыруға және жүзеге асыруға айтарлықтай әсер етеді. Мысалы,
Үндістанда көптеген мемлекеттік органдар білікті аудит және қаржылық
бақылау мамандарының тапшылығына тап болады. Бұл федералдық және
аймақтық үкімет деңгейлерінде жан-жақты аудит жүргізуді қиындатады,
мұнда штат активтерінің көпшілігі дұрыс құжатталмаған немесе есепке
алынбаған. Сонымен қатар, мемлекеттік аудитте кадрлардың жоғары
тұрақсыздығы және білім беру бағдарламаларының жоқтығы білікті
мамандарды дайындау міндетін қиындатады.

Кенияда жағдай аудиторлық процестерді автоматтандырудың жоқтығынан
қиындады. Мемлекеттік органдарда көбінесе активтерді есепке алу мен
мониторингілеудің заманауи ақпараттық жүйелері жетіспейді, бұл аудиттің
кешігуіне және есеп беру қателерінің тәуекелдерінің артуына әкеледі.
Жер, мүлік және қаржылық активтер туралы маңызды деректер көбінесе қағаз
түрінде сақталады, бұл аудитті қиындатады және көп уақытты алады.
Сондай-ақ, деректерді талдауға арналған мамандандырылған бағдарламалық
қамтамасыз ету немесе активтерді пайдалану тиімділігін бағалау процесін
жылдамдататын технологиялар сияқты техникалық құралдардың жетіспеушілігі
бар.

Кенияда 2021 жылы денсаулық сақтау саласындағы мемлекеттік қаржылар мен
активтердің тиімділігіне бірқатар аудиттер жүргізілді. Тексерулер
көрсеткендей, көптеген медициналық жабдықтарды жеткізу және бюджет
қаражатының жұмсалуы бойынша құжаттама мен есеп беру жеткіліксіз.
Дегенмен, электрондық есеп беру платформаларын және жеткізу тізбегін
бақылауға арналған блокчейн технологияларын енгізу мемлекеттік
активтерді пайдаланудың ашықтығы мен тиімділігін арттыруға көмектесті.
Бұл мысал аудит процестеріне заманауи технологияларды енгізудің
маңыздылығын растайды, бұл да зерттеудің теориялық бөлігінің
ұсыныстарына сәйкес келеді. Мемлекеттік активтерді басқару тиімділігінің
аудитін жақсарту бойынша халықаралық тәжірибе, сәтті мысалдар мен
ұсыныстар 4 кестеде көрсетілген.
\end{multicols}

\begin{longtblr}[
  label = none,
  entry = none,
  caption = {\bfseries 4 - кесте. Мемлекеттік активтерді басқару тиімділігінің аудитін жақсарту бойынша халықаралық тәжірибе, сәтті мысалдар мен ұсыныстар},
]{
  colspec = {X[1] X[1] X[1] X[3] X[2]},
  cell{2}{3} = {c},
  cell{3}{3} = {c},
  cell{4}{3} = {c},
  cell{5}{3} = {c},
  cell{6}{3} = {c},
  cell{7}{1} = {c=5}{},
  hlines,
  vlines,
}
Ел/Аймақ                                   & {Тиімділік\\аудитінің\\органы}              & {Аудит\\құрылы-мындағы\\тиімділік\\аудитінің үлесі (\%)} & Сәтті мысалдар мен жетістіктер                                                                                                                   & Қазақстан үшін халықаралық тәжірибеге негізделген ұсыныстар                                          \\
Еуропа одағы                               & {Еуропалық\\аудиторлар\\соты}               & 65                                                      & ЕО қаражатын пайдалануды бағалау инфрақұрылымдық жобаларды қаржыландыруды оңтайландыруға және шығыстардың ашықтығын арттыруға мүмкіндік берді.   & ЕО стандарттарын бейімдеу, тәуекелге негізделген тәсілді енгізу.                                     \\
АҚШ                                        & {Government Account-ability\\Office (GAO)}   & 70                                                      & Федералды бағдарламалардың тиімділігін тексеру қайталанатын бағдарламаларды жою арқылы бюджет шығыстарын 50 миллиард долларға қысқартуға әкелді. & Аудит сапасын бағалау жөніндегі тәуелсіз орган құру, стратегиялық аудитті жүзеге асыру.              \\
Ұлы-британия                                & {Ұлттық аудит басқармасы\\(ҰАО)}            & 60                                                      & Мемлекеттік IT бағдарламаларының аудиті £4 млрд үнемдеді және мемлекеттік сектордың өнімділігін арттырды.                                        & Тиімділік аудитінің ұлттық стандарттарын әзірлеу, аудиторлардың біліктілігін арттыру.                \\
Канада                                     & {Office of the\\Auditor General\\(OAG)}     & 68                                                      & Жерге орналастыруды оңтайландыру ренталық кірісті 20 пайызға арттырып, мемлекеттік мүлікті пайдалануды жақсартты.                                & Аудиторлық процестерді және мемлекеттік активтерді бағалауды автоматтандыру үшін заманауи АТ енгізу. \\
Австралия                                  & {Australian\\National\\Audit Office (ANAO)} & 63                                                      & Мемлекеттік келісім-шарттарды басқару аудиті көрсетілетін қызметтердің құны мен сапасын жақсартуға әкелді.                                       & Экономиканың әртүрлі салаларындағы тиімділік аудиті бойынша әдістемелік ұсыныстар әзірлеу.           \\
Ескерту -дереккөздерден құрастырылған [17] &                                             &                                                         &                                                                                                                                                  &                                                                                                      
\end{longtblr}

\begin{multicols}{2}
Дамыған елдерде тиімділік аудиті мемлекеттік бақылау органдарының
қызметінің маңызды бөлігін алады. Мысалы, Еуропалық Одақ пен Америка
Құрама Штаттарында тиімділік аудитінің үлесі аудиттің жалпы көлемінің
60-70\% жетеді. Бұл мемлекеттік ресурстарды пайдаланудың тиімділігі мен
тиімділігін бағалауға баса назар аударумен байланысты.

Еуропалық Одақ елдерінде тиімділік аудиті Еуропалық аудиторлар соты
аясында жүзеге асырылады, ол ЕО қаражатының қаншалықты тиімді және
тиімді пайдаланылғанын бағалайды. Америка Құрама Штаттарында осыған
ұқсас функцияларды олардың тиімділігін арттыру үшін федералдық
бағдарламалар мен мемлекеттік органдарды тексеретін Үкіметтің есеп беру
басқармасы (GAO) атқарады.

Тиімділік аудитін сәтті қолдану мысалдары оның мемлекеттік активтерді
басқаруды жақсарту үшін маңыздылығын көрсетеді. Ұлыбританияда Ұлттық
аудит кеңсесі мемлекеттік органдардағы АТ жүйелерін жаңарту
бағдарламасының тиімділігіне аудит жүргізді, нәтижесінде бюджетті
айтарлықтай үнемдеуге және еңбек өнімділігін арттыруға мүмкіндік берді.
Канадада мемлекеттік жер ресурстарын басқару қызметінің аудиті жерді
пайдалануды жақсартуға және жалға беруден түсетін кірісті арттыруға
әкелді. Халықаралық тәжірибе тиімділік аудиті мемлекеттік ресурстарды
пайдалануды оңтайландырудың негізгі құралы болып табылатынын көрсетеді.
Ұсынылған халықаралық стандарттарды бейімдеу, аудит сапасын бағалау
жөніндегі тәуелсіз органдарды құру, заманауи технологияларды пайдалану
және аудиторлардың кәсіби құзыреттерін дамыту сияқты шараларды іске
асыру Қазақстан Республикасында мемлекеттік активтерді басқару
тиімділігін айтарлықтай арттыруға мүмкіндік береді.

Қазақстанда мемлекеттік құрылымды реформалау және активтерді басқару
жүйесін жетілдіру бойынша күш-жігерге қарамастан, ресурстық мәселелер де
сақталуда. Мысалы, шағын және орта мемлекеттік органдарда қаржылық
бақылау және аудит саласында қажетті білімі бар мамандар жетіспеуі
мүмкін. Бухгалтерлік есеп пен мониторинг процестерін автоматтандырудың
техникалық мүмкіндіктері шектеулі болып қала береді, бұл аудиторлық
органдарды тексерулерді қолмен жүргізуге мәжбүр етеді, бұл процесті
айтарлықтай баяулатады және қателер ықтималдығын арттырады {[}17{]}.

Үндістан үлкен мемлекеттік секторы бар және мемлекеттік активтерінің
алуан түрлілігі бар ел ретінде аудит әдістерін қолдануды көрсету үшін
жақсы мысал келтіреді. 2020 жылы Үндістан үкіметі Ұлттық инфрақұрылымдық
стратегия бағдарламасы аясында инфрақұрылымдық жобаларда мемлекеттік
активтерді пайдалану аудитін жүргізді. Аудит барысында ірі
инфрақұрылымдық жобалардың үлкен саны, олардың жоғары бюджеттеріне
қарамастан, дұрыс жоспарланбағандықтан және шығыстар мен күтілетін
нәтижелердің сәйкес келмеуі салдарынан тиімсіз пайдаланылғаны анықталды.
Аудит барысында жер ресурстарын мақсатсыз пайдалану фактілері анықталды,
бұл шығындардың асып кетуіне әкеп соқтырды, сондай-ақ келісімшарттың
орындалуына тиісті бақылау жүргізілмейді. Бұл мысал зерттеудегі
теориялық ұсыныстарға сәйкес келетін тиімдірек бақылау және мониторинг
жүйелерін әзірлеу қажеттілігін көрсетеді {[}18{]}.

Қазақстанда соңғы мысалдардың бірі мемлекеттік жер ресурстарын пайдалану
тиімділігінің аудитіне қатысты. Тексеру барысында көптеген мемлекет
меншігіндегі жер телімдері орталықтандырылған есепке алу мен бақылау
жүйесінің жоқтығынан тиімсіз пайдаланылып жатқаны анықталды. Бұл
сыбайлас жемқорлық схемалары мен активтерді қайта бөлудегі
заңсыздықтарға мүмкіндік туғызды. Аудит автоматтандырылған есепке алу
жүйесін және жер операцияларын бақылаудың қатаң тетіктерін енгізу
активтерді басқару тиімділігін айтарлықтай арттырып, сыбайлас жемқорлық
тәуекелдерін азайтатынын көрсетті. Осылайша, Қазақстандағы тиімділік
аудиті бойынша шетелдік тәжірибесінің нақты мысалдарына негізделген
эмпирикалық талдау, сондай-ақ мемлекеттік активтерді басқару тиімділігін
арттыруға бағытталған әзірленген ұсынымдар зерттеуге практикалық мән
береді және жұмысты жан-жақты және практикалық тұрғыдан бағдарлауды
қажет етеді. Эмпирикалық талдау негізінде бірнеше негізгі мәселелер
анықталды:

1.Есеп берудің төмен ашықтығы - көптеген мемлекеттік органдарда
активтердің жай-күйі мен пайдаланылуы туралы ақпарат берудің анық және
қолжетімді тетіктері жоқ;

2. Ресурстарды тиімсіз басқару-мемлекеттік активтер санының көп болуына
қарамастан, оларды пайдалану белгіленген мақсаттар мен даму
стратегияларына сәйкес келе бермейді, бұл тиімсіздік пен қажетсіз
шығындарға әкеледі;

3. Мемлекеттік органдар арасындағы үйлестірудің нашарлығы - активтерді
басқарумен айналысатын әртүрлі субъектілер жиі үйлесімсіз әрекет етеді,
бұл кешенді шешімдер қабылдауды қиындатады және активтерді басқарудың
жалпы жағдайына әсер етеді.

Анықталған проблемаларды ескере отырып, мемлекеттік активтерді басқару
тиімділігін арттыру бойынша келесі ұсыныстар ұсынылды:

- ішкі бақылау жүйесін жетілдіру - активтерді пайдалану тиімділігін
бағалау үшін неғұрлым қатаң ішкі аудит пен мониторинг процедураларын
енгізу қажет. Бұл активтердің қозғалысын бақылау үшін мамандандырылған
бөлімшелерді құруды, сондай-ақ тұрақты тәуелсіз аудиттерді қамтиды;

- мониторингтің анық тетіктерін енгізу - ашықтық пен активтерді
басқаруды жақсарту үшін орталықтандырылған мониторинг жүйесін құру
қажет. Мұндай жүйе әрбір мемлекеттік активтің ағымдағы жай-күйін
бақылауға мүмкіндік беретін және жедел шешім қабылдау мүмкіндігін
қамтамасыз ететін деректер қорын қамтуы керек;

- активтерді пайдалану тиімділігін бағалау жүйесін әзірлеу және енгізу
-- басқаруды жетілдірудегі маңызды қадам тиімсіз немесе нашар
басқарылатын ресурстарды уақтылы анықтауға мүмкіндік беретін әрбір
активтің тиімділігін бағалау әдістемесін әзірлеу болып табылады. оларды
оңтайландыру бойынша шаралар қабылдау.

4. Персоналды оқыту және олардың біліктілігін арттыру -- активтерді
басқару сапасын арттыру мақсатында мемлекеттік қызметшілерді, әсіресе
қаржылық есеп, аудит және стратегиялық менеджмент салаларында оқыту
бағдарламаларын ұйымдастыру қажет.

5. Мемлекеттік және жеке құрылымдар арасындағы ынтымақтастық деңгейін
арттыру -мемлекеттік институттар мен жеке сектор арасындағы
серіктестікті дамыту ресурстарды тиімдірек пайдалануға, активтерді
басқаруды оңтайландыру үшін тәжірибе мен технологияларды тартуға
мүмкіндік береді.

Біздің ойымызша, мемлекеттік активтерді басқарудың тиімділігін тексеру
әдістемесін жетілдіру тек қаржылық аспектілерді ғана емес, сонымен бірге
осы активтерді басқарудың әлеуметтік, экономикалық және ұйымдастырушылық
аспектілерін қамтуы мүмкін көп деңгейлі және күрделі көрсеткіштерді
әзірлеуді талап етеді.Мемлекеттік активтерді басқару тиімділігін талдау
кезінде көп өлшемді көрсеткіштерді пайдалану ұйымның жұмысы мен
жетістіктерінің әртүрлі аспектілерін жан-жақты және дәл бағалауға
мүмкіндік береді. Жеке көрсеткіштерді оқшаулаудың орнына көп өлшемді
көрсеткіштер бір уақытта бірнеше негізгі факторларды ескереді, бұл ұйым
қызметінің толық және объективті көрінісін жасауға көмектеседі.

{\bfseries Қорытынды.} Мемлекеттік активтерді басқару тиімділігін талдау
кезінде қызметтің әртүрлі аспектілерін қамтитын көп өлшемді көрсеткіштер
қолданылады:

- қаржылық тұрақтылықты және бюджетті басқарудың тиімділігін бағалау
үшін кірістерді, шығыстарды және пайданы қамтитын қаржылық көрсеткіштер
қолданылады;

- ЖІӨ, инфляция және жұмыссыздық сияқты экономикалық көрсеткіштер
активтерді басқарудың экономикалық дамуға әсерін көрсетеді;

- өмір сүру деңгейі және әлеуметтік қызметтердің қолжетімділігі сияқты
әлеуметтік көрсеткіштер ұйым қызметінің әлеуметтік әсерін бағалауға
мүмкіндік береді;

- экологиялық көрсеткіштер қоршаған ортаға әсерді, оның ішінде табиғи
ресурстарды пайдалануды және экологиялық қауіпсіздік деңгейін бағалайды;

инфрақұрылымның жағдайы және техникалық ресурстардың қолжетімділігі
сияқты техникалық көрсеткіштер техникалық дайындықты және активтерді
басқаруды қолдауды көрсетеді.

Мемлекеттік активтерді басқарудың тиімділік аудитін жүргізу барысында
этикалық нормалар мен қағидаттарды сақтау маңызды аспект болып табылады.
Аудитор активтерді пайдалану тиімділігін бағалауда тәуелсіз, объективті
және бейтарап болуы керек. Сонымен қатар, мәліметтерді бұрмалауды
болдырмау және ашықтықты қамтамасыз ету үшін құпия ақпаратты қорғауға
және қаржылық есептілік стандарттарын сақтауға ерекше назар аудару
қажет.

Аудиторлық тәжірибеде этикалық нормаларды сақтау мысалдарын мемлекеттік
жер ресурстарын пайдалану саласындағы аудит мысалында келтіруге болады.
Бір жағдайда заңбұзушылықтар активтерді мақсатсыз пайдалануға әкеліп
соқтырған жер операцияларының ашықтығының жеткіліксіздігімен байланысты
болды. Бұл ретте аудиторлық топ барлық деректерді тәуелсіз тексеруді
қамтамасыз етті және жер активтерін есепке алу жүйесіне өзгерістер
енгізуге және олардың пайдаланылуына бақылауды жақсартуға мүмкіндік
беретін нәтижеге қол жеткізді.

Қорытындылай келе, мемлекеттік активтерді басқару саласындағы
мемлекеттік аудитті әдістемелік қамтамасыз етуді талдау үш негізгі
деңгейді анықтағанын атап өткен жөн: жалпы, заңнамалық және әдістемелік.
Бұл бөлініс ішкі мемлекеттік аудиттің әдіснамалық негізін құрылымдауға
және нормативтік құқықтық актілердің қолданылу аясын нақты түсінуге
мүмкіндік беретінімен маңызды.Мемлекеттік активтерді пайдалану мен
басқаруды бағалайтын тиімділік аудитінің қорытындылары тек деректерге
ғана емес, сонымен қатар сараптамалық қорытындыларға негізделгендіктен
субъективті түрде қабылдануы мүмкін. Демек, аудиттің негізгі құндылығы
тек тексеру нәтижелері ғана емес, сонымен қатар анықталған кемшіліктерді
түзету бойынша ұсыныстар болып табылады. Осылайша, мемлекеттік
активтерді басқаруды жақсарту бойынша практикалық ұсыныстар жасаудың
маңыздылығын атап көрсете отырып, аудит процесінде конструктивті тәсіл
басты рөл атқарады.
\end{multicols}

\begin{center}
{\bfseries Әдебиеттер}
\end{center}

\begin{references}
1. Davis, Jim. What Is Asset Management and Where Do You Start? American
Water Works Association Journal.-2007.-Vol. 99(10). -P.26-34. DOI
\href{http://dx.doi.org/10.1002/j.1551-8833.2007.tb08042.x}{10.1002/j.1551-8833.2007.tb08042.x}

2. Государственная программа Российской Федерации "Управление федеральным
имуществом" {[}Электронный ресурс{]} // Официальный интернет-портал
правовой информации. URL:
https://www.consultant.ru/document/cons\_doc\_LAW\_155198/.Дата
обращения: 10.03.2025.

3. Сембиева Л. М. Введение в финансы: учебное пособие / Л.М. Сембиева. --
Алматы : ТехноЭрудит, 2019. - Т. 1.-244 с. ISBN 978-601-342-204-6.

4. Закон Республики Казахстан "О государственном аудите и финансовом
контроле" от 12 ноября 2015 года {[}Электронный ресурс{]}. URL:
\url{https://online.zakon.kz/document/?doc_id=37724730}. Дата
обращения: 10.03.2025.

5. Алибекова Б.А. Государственный аудит (продвинутый): учебное пособие. -
Нур-Султан: ЕНУ им. Л.Н. Гумилева, 2021 -- 286 с. ISBN
978-601-337-481-9

6. Саунин А.И. Аудит эффективности использования государственных средств:
вопросы теории и практики. - Московский государственный университет
имени М.В. Ломоносова, 2015.-336 с. ISBN 978-5-19-011000-5

7. Бейсенова Л.З. Аудит эффективности: учебное пособие. Нур-Султан: ЕНУ
им. Л.Н. Гумилева. - 2020. - 292 с. ISBN 978-601-337-281-5.

8. Туребекова Б.О. Статистика: учебно-методический комплекс для студентов
специальности 050508 «Учет и аудит». -Астана: ЕНУ им. Л.Н. Гумилева,
2008. - 116 с. ISBN 9965-31-200-1.

9. Дюсембаев К. Ш., Сатенов Б. И. Анализ финансового положения
предприятия : учебное пособие. - Алматы : Экономика, 1998. -- 184 с.
ISBN 9965-448-06-Х.

10. Официальный сайт организации «Intosai» {[}Электронный ресурс{]}. URL:
\url{https://www.intosai.org} (дата обращения: 10.03.2025).

11. Конституция Республики Казахстан от 30 августа 1995 года
{[}Электронный ресурс{]}. URL:
\url{http://adilet.zan.kz/rus/docs/K950001000_\#z42}. Дата обращения:
10.03.2025.

12. Бюджетный Кодекс Республики Казахстан от 4 декабря 2008 года № 95-IV
{[}Электронный ресурс{]} .URL:
\url{http://adilet.zan.kz/rus/docs/K080000095}. Дата обращения:
10.03.2025.

13. «Қазақстан Республикасы Жоғарғы есеп палатасының кейбір мәселелері
туралы» Қазақстан Республикасы Президентінің 2022 жылғы 26 қарашадағы
№5 Жарлығы {[}Электрондық ресурс{]}.-http://online.zakon.kz/ Жүгінген
күні: 10.03.2025.

14. Мемлекеттік аудит және қаржылық бақылау туралы Қазақстан
Республикасының 2015 жылғы 12 қарашадағы Заңы {[}Электрондық
ресурс{]}.-http://online.zakon.kz/ Жүгінген күні: 10.03.2025

15. «Сыртқы мемлекеттік аудит және қаржылық бақылау жүргізу қағидаларын
бекіту туралы» Республикалық бюджеттің атқарылуын бақылау жөніндегі
есеп комитетінің 2020 жылғы 30 шілдедегі № 6-НҚ нормативтік
қаулысы.-{[}Электрондық ресурс{]}.-\url{http://online.zakon.kz/}
Жүгінген күні: 10.03.2025.

16. Сембаев Д.К. Совершенствование системы показателей аудита
эффективности деятельности государственных органов Республики
Казахстан : дис. \ldots{} д-ра PhD: 6D052100 - Государственный аудит /
Д.К. Сембаев; науч. конс. Б.А. Алибекова; заруб. науч. конс. Е.В.
Никифорова; Евразийский национальный университет им. Л.Н. Гумилева. --
Нур-Султан, 2022. - 158 с.

17. Spatayeva S.B., Karabayev E.B. , Nikiforova E.V. An external state
audit of the effectiveness of the use of budgetary funds aimed at
ensuring food security of the Republic of Kazakhstan // Economic
Series of the Bulletin of the L.N. Gumilyov ENU.-2021.- № 4.- P.
214-221. DOI 10.32523/2789-4320-2021-4-214-221

18. Rakhimov R., Niemenmaa V., Kusherbaeva D., Rakhmetova А. State audit
in the field of assessing the effectiveness of the use of natural
resources: world experience // State audit. -2023. -№3 (60). -P.134-148.
DOI 10.55871/2072-9847-2023-60-3-134-148
\end{references}

\begin{center}
{\bfseries References}
\end{center}

\begin{references}
1. Davis, Jim. What Is Asset Management and Where Do You Start? American
Water Works Association Journal.-2007.-Vol. 99(10). -P.26-34. DOI
10.1002/j.1551-8833.2007.tb08042.x

2. Gosudarstvennaja programma Rossijskoj Federacii "Upravlenie
federal' nym imushhestvom" {[}Jelektronnyj resurs{]} //
Oficial' nyj internet-portal pravovoj informacii. URL:
https://www.consultant.ru/document/cons\_doc\_LAW\_155198/ (data
obrashhenija: 10.03.2025) {[}in Russian{]}

3. Sembieva L. M. Vvedenie v finansy: uchebnoe posobie / L.M. Sembieva.
-- Almaty : TehnoJerudit, 2019. -- T. 1. -- 244 s. ISBN
978-601-342-204-6. {[}in Russian{]}

4. Zakon Respubliki Kazahstan "O gosudarstvennom audite i finansovom
kontrole" ot 12 nojabrja 2015 goda {[}Jelektronnyj resurs{]}. URL:
https://online.zakon.kz/document/?doc\_id=37724730 (data obrashhenija:
10.03.2025). {[}in Russian{]}

5. Alibekova B.A. Gosudarstvennyj audit (prodvinutyj): uchebnoe posobie.
- Nur-Sultan: ENU im. L.N. Gumileva, 2021 -- 286 s. ISBN
978-601-337-481-9 {[}in Russian{]}

6. Saunin A.I. Audit jeffektivnosti ispol' zovanija
gosudarstvennyh sredstv: voprosy teorii i praktiki. - Moskovskij
gosudarstvennyj universitet imeni M.V. Lomonosova, 2015.-336 s. ISBN
978-5-19-011000-5 {[}in Russian{]}

7. Bejsenova L.Z. Audit jeffektivnosti: uchebnoe posobie. Nur-Sultan: ENU
im. L.N. Gumileva. -- 2020. -- 292 s. ISBN 978-601-337-281-5. {[}in
Russian{]}

8. Turebekova B.O. Statistika: uchebno-metodicheskij kompleks dlja
studentov special' nosti 050508 «Uchet i audit».
-Astana: ENU im. L.N. Gumileva, 2008. - 116 s. ISBN 9965-31-200-1.
{[}in Russian{]}

9. Djusembaev K. Sh., Satenov B. I. Analiz finansovogo polozhenija
predprijatija : uchebnoe posobie. -- Almaty : Jekonomika, 1998. -- 184
s. ISBN 9965-448-06-H. {[}in Russian{]}

10. Oficial' nyj sajt organizacii «Intosai» {[}Jelektronnyj
resurs{]}. URL: https://www.intosai.org (data obrashhenija:
10.03.2025). {[}in Russian{]}

11. Konstitucija Respubliki Kazahstan ot 30 avgusta 1995 goda
{[}Jelektronnyj resurs{]}. URL:
http://adilet.zan.kz/rus/docs/K950001000\_\#z42 (data obrashhenija:
10.03.2025). {[}in Russian{]}

12. Bjudzhetnyj Kodeks Respubliki Kazahstan ot 4 dekabrja 2008 goda №
95-IV {[}Jelektronnyj resurs{]} .URL:
http://adilet.zan.kz/rus/docs/K080000095\_ (data obrashhenija:
10.03.2025). {[}in Russian{]}

13. "Qazaqstan Respýblıkasy Prezıdentiniń 2022 jylǵy 26 qarashadaǵy №5
Jarlyǵy {[}Elektrondyq resýrstar{]}.-http://online.zakon.kz/ (jyǵylǵan
qyny: 10.03.2025) {[}in Kazakh{]}

14. Memlekettik aýdıt jańa qarjy baqılaý turaly Qazaqstan Respýblıkasynyń
2015 jylǵy 12 qarashadaǵy Zańy {[}Elektrondyq
resýrstar{]}.-http://online.zakon.kz / (jyǵylǵan qyny: 10.03.2025)
{[}in Kazakh{]}

15. "Syrttaǵy memlekettik aýdıt jáne qarjy mınıstrligi baqılaý jyrǵyzý
qaǵanattarynyń bekit turaly" Respýblıkalyq búdjettiń atqarylýyn
baqylaý komıtetiniń 2020 jylǵy 30 shildedegi № 6-NQ normatıvtik
quqyqtyq aktileriniń jobasy.- {[}Elektrondyq
resýrstar{]}-http://online.zakon.kz / (jyǵylǵan qyny: 10.03.2025)
{[}in Kazakh{]}

16. Sembaev D.K. Sovershenstvovanie sistemy pokazatelej audita
jeffektivnosti dejatel' nosti gosudarstvennyh organov
Respubliki Kazahstan : dis. \ldots{} d-ra PhD: 6D052100 --
Gosudarstvennyj audit / D.K. Sembaev; nauch. kons. B.A. Alibekova;
zarub. nauch. kons. E.V. Nikiforova; Evrazijskij
nacional' nyj universitet im. L.N. Gumileva. --
Nur-Sultan, 2022. -- 158 s. {[}in Russian{]}

17. Spatayeva S.B., Karabayev E.B. , Nikiforova E.V. An external state
audit of the effectiveness of the use of budgetary funds aimed at
ensuring food security of the Republic of Kazakhstan // Economic
Series of the Bulletin of the L.N. Gumilyov ENU.-2021.- № 4.- P.
214-221. DOI 10.32523/2789-4320-2021-4-214-221

18. Rakhimov R., Niemenmaa V., Kusherbaeva D., Rakhmetova A. State audit
in the field of assessing the effectiveness of the use of natural
resources: world experience // State audit. -2023. -№3 (60).
-P.134-148. DOI 10.55871/2072-9847-2023-60-3-134-148
\end{references}

\begin{authorinfo}
\emph{{\bfseries Авторлар туралы мәліметтер}}

Беделова Д.Н. - докторант,Л.Н.Гумилев атындағы Еуразия Ұлттық
Университеті Астана, Қазақстан, e-mail: everest-astana@mail.ru;

Мақыш С.Б.- экономика ғылымдарының докторы, профессор Esil University,
Астана,Қазақстан, e-mail:
\href{mailto:mtbb1986@gmail.com}{\nolinkurl{mtbb1986@gmail.com}}

\emph{{\bfseries Information about the authors}}

BedelovaD. N. - doctoral student, L.N.Gumilyov Eurasian National
University,Astana, Kazakhstan, e-mail: everest-astana@mail.ru;

Makysh S.B. - doctor of economic sciences, professor, Esil University,
Astana, Kazakhstan, e-mail:
\href{mailto:mtbb1986@gmail.com}{\nolinkurl{mtbb1986@gmail.com}}
\end{authorinfo}
