\id{МРНТИ }{}\href{https://grnti.ru/?p1=82&p2=13&p3=01}{82.13.01}

\href{https://zhetysu.edu.kz/\%D1\%81\%D0\%B5\%D1\%80\%D0\%B8\%D0\%B5\%D0\%B2-\%D0\%B1\%D0\%BE\%D0\%BB\%D0\%B0\%D1\%82-\%D0\%B0\%D0\%B1\%D0\%B4\%D1\%83\%D0\%BB\%D0\%B4\%D0\%B0\%D2\%B1\%D0\%BB\%D1\%8B/}{{\bfseries МЕМЛЕКЕТТІК
ҚЫЗМЕТТЕГІ ҰЙЫМДАСТЫРУШЫЛЫҚ МӘДЕНИЕТІ: (ЖЕТІСУ ОБЛЫСЫ, ҚАРАТАЛ АУДАНЫ
ӘКІМШІЛІГІ МЫСАЛЫНДА)}}


\emph{І.Жансүгіов атындағы Жетісу университеті, Талдықорған қ.,
Қазақстан}

{\bfseries \textsuperscript{\envelope }}Корреспондент-автор: gulima8888888@mail.ru

Мемлекеттік қызметшілердің ұйымдастырушылық мәдениеті мемлекетік қызмет
атқаруда қайшылықты басқару тәжірибесінде жиі кездеседі. Мемлекетік
қызметшілерді басқару, ұйымдастыру саласында басшының жеке дара
көшбасшылық қасиеттеріне және біліміне байланысты. Мемлекетттік
қызметшілерді басқаруда ұйымдастырушылық қабілеті мен кәсіби қызметінің
тиімділігі игеруіне байланысты анықталады. Мемлекеттік қызметтегі
ұйымдастырушылық мәдениетті қалыптастыру мекеме басшысына тікелей
байланысты екені сөзсіз, осы тұрғыдан қарағанда ұйымдастырушылық
мәдениетті мемлекеттік қызметте жұмыс жасайтын қызметкерлердің
құндылықтары, нанымдары, дәстүрлері мен мінез-құлық нормалары жиынтығы
құрайды. Мақалада мемлекеттік қызметтегі ұйымдастырушылық мәдениеті жеке
мәдениет элементтері қарастырылады. Сол себепті мемлекеттік қызметке
қабылдау кезінде білім мен тәжірибеге ғана емес, құндылықтар мен
мінез-құлық нормаларына мән беру керектігі қарастырылған. Аталған
жағдайды есептеу кезінде қызметкердің түрлі оқиғалар мен жағдайларға
көзқарасы, еңбектің қолайлығына, отбасы мен жұмыс арасында уақытты
бөлуде ұйымдастырушылық мәдениеті маңызды. Мемлекеттік қызметтегі
ұйымдастырушылық мәдениетді дамытуда реформалау тәжірибесінде кезінде
дамыған елдердегі мемлекеттік қызметшілердің мәдинетіне ғылыми
зерттеулер жүргізген. Мемлекеттік қызметтің және оның ұйымдастырушылық
қызметінің принциптері мен құндылықтары мен мәдениет маңызы
қарастырылды. Бұл зерттеу жұмысында мемлекеттік қызметтегі
ұйымдастырушылық мәдениетіне құқықтық талдаулар{\bfseries ,} Жетісу
облысындағы Қаратал ауданының мысалында зерттеулер жүргізілген.

{\bfseries Түйін сөздер:} Мемлекеттік қызмет, ұйымдастырушылық мәдениеті,
корпоративтік мәдениет, мінез-құлық нормасы, мемлекеттік орган.

\href{https://zhetysu.edu.kz/\%D1\%81\%D0\%B5\%D1\%80\%D0\%B8\%D0\%B5\%D0\%B2-\%D0\%B1\%D0\%BE\%D0\%BB\%D0\%B0\%D1\%82-\%D0\%B0\%D0\%B1\%D0\%B4\%D1\%83\%D0\%BB\%D0\%B4\%D0\%B0\%D2\%B1\%D0\%BB\%D1\%8B/}{{\bfseries ОРГАНИЗАЦИОННАЯ
КУЛЬТУРА НА ГОСУДАРСТВЕННОЙ СЛУЖБЕ: (НА ПРИМЕРЕ АКИМАТА КАРАТАЛЬСКОГО
РАЙОНА ОБЛАСТИ ЖЕТІСУ)}}

{\bfseries Б.А.Сериев, Г.А.Жакупова\textsuperscript{\envelope }}

\emph{І.Жансүгіов атындағы Жетісу университеті, Талдықорған қ.,
Қазақстан}

\emph{e-mail: gulima8888888@mail.ru}

Организационная культура государственных служащих часто встречается в
практике противоречивого управления на государственной службе. В области
управления, организации государственных служащих зависит от
индивидуальных лидерских качеств и знаний руководителя. Организаторские
способности и эффективность профессиональной деятельности в управлении
государственными служащими определяются в зависимости от их усвоения.
Несомненно, формирование организационной культуры на государственной
службе напрямую зависит от руководителя учреждения, с этой точки зрения
организационную культуру составляет совокупность ценностей, убеждений,
традиций и норм поведения работников, работающих на государственной
службе. В статье рассматриваются элементы индивидуальной культуры
организационной культуры на государственной службе. Поэтому при приеме
на государственную службу необходимо уделять внимание ценностям и нормам
поведения, а не только знаниям и опыту. При расчете указанной ситуации
важно отношение работника к различным событиям и ситуациям, его
организационная культура в распределении времени между семьей и работой,
комфортность труда. В ходе практики реформирования в развитии
организационной культуры на государственной службе проводились научные
исследования культуры государственных служащих в развитых странах.
Рассмотрены принципы и ценности государственной службы и ее
организационной деятельности и культуры. В данной исследовательской
работе проведен правовой анализ организационной культуры на
государственной службе, на примере Каратальского района области Жетісу.

{\bfseries Ключевые слова.} Государственная служба, организационная
культура, корпоративная культура, норма поведения, государственный
орган.

\href{https://zhetysu.edu.kz/\%D1\%81\%D0\%B5\%D1\%80\%D0\%B8\%D0\%B5\%D0\%B2-\%D0\%B1\%D0\%BE\%D0\%BB\%D0\%B0\%D1\%82-\%D0\%B0\%D0\%B1\%D0\%B4\%D1\%83\%D0\%BB\%D0\%B4\%D0\%B0\%D2\%B1\%D0\%BB\%D1\%8B/}{{\bfseries ORGANIZATIONAL
CULTURE IN PUBLIC SERVICE: (USING THE EXAMPLE OF THE AKIMAT OF THE
KARATAL DISTRICT OF THE ZHETYSU REGION)}}

{\bfseries Б.А.Сериев, Г.А.Жакупова\textsuperscript{\envelope }}

\emph{І.Жансүгіов атындағы Жетісу университеті, Талдықорған қ.,
Қазақстан}

\emph{e-mail: gulima8888888@mail.ru}

The organizational culture of civil servants is often found in the
practice of contradictory management in the civil service. In the field
of management, the organization of civil servants depends on the
individual leadership qualities and knowledge of the head.
Organizational skills and the effectiveness of professional activities
in the management of civil servants are determined depending on their
assimilation. Undoubtedly, the formation of organizational culture in
the civil service directly depends on the head of the institution, from
this point of view, organizational culture is a set of values, beliefs,
traditions and norms of behavior of employees working in the civil
service. In the article, organizational culture in the civil service
consists of separate cultural elements. Therefore, when applying for
public service, it is necessary to pay attention to values and norms of
behavior, rather than knowledge and experience. When calculating this
situation, the employee' s attitude to various events and
situations, his organizational culture in the allocation of time between
family and work, and the comfort of work are important. In the course of
the reform practice in the development of organizational culture in the
civil service, scientific research was conducted on the culture of civil
servants in developed countries. The principles and values of the civil
service and its organizational activities and the importance of culture
are considered. In this research paper, a legal analysis of
organizational culture in the civil service is carried out, using the
example of the Karatal district of the Zhetysu region.

{\bfseries Keywords.} Civil service, organizational culture, corporate
culture, standard of conduct, government agency.

{\bfseries Кіріспе.} XX ғасырдың 70-жылдарында ұйымдастыру мәдениетін
әлеуметтік және гуманитарлық ғалымдармен зерттеу аясында жүргізіле
бастады. Әр түрлі меншік нысанына қарамастан қызметкердің жұмыс сапасын
күшейту, еңбегінің тиімділігін көтеру және өндірістің өнімділігін
арттыру мақсатында басқарушылық буын қалыптастырылады. Осы мақсатты іске
асыру үшін мекеме қызметкерінің мінез-құлқын және сапалы қызмет
жиындығын қажетті нәтижелі жетістікке бағыттау болып табылады.
Ұйымдастырушылық мәдениетте аталған жиынтықты көздейтін бағыттарға
байланысты әлеметтік-гуманитарлық ғылымдар тұрғысынан қарастыруы тиіс.

Ұйымдастырушылық мәдениетін зерделеу барысында шетелдік және
Қазақстандық ғалымдар ұйымдастырушылық мәдениет ұғымының түрлі
мағынадағы анықтамаларын берді. Сондай-ақ ұйымдастырушылық мәдениет
жөнінде ғылыми пәндердің әрқайсысы оқытылатын саланың ерекшелігін
негізге ала отырып, өзіндік мағынасын нақтылады. Қазақстан
Республикасының «Мемлекеттік қызмет туралы» заңында мемлекеттік
қызметкер, мемлекеттік қызмет ұғымдарына анықтама берілген. Қазақстан
Республикасының «Мемлекеттік қызмет туралы» Заңының 1-бабының
1-тармағының 12-тарамағына сәйкес «Мемлекеттiк қызметшi -- мемлекеттiк
органда Қазақстан Республикасының заңнамасында белгiленген тәртiппен
республикалық немесе жергiлiктi бюджеттерден не Қазақстан Республикасы
Ұлттық Банкiнiң және (немесе) Қазақстан Республикасының заңсыз
иемденілген активтерді мемлекетке қайтару туралы заңнамасында
айқындалған арнаулы мемлекеттік қордың қаражатынан ақы төленетiн
мемлекеттiк лауазымды атқаратын және мемлекеттiң мiндеттерi мен
функцияларын iске асыру мақсатында лауазымдық өкiлеттiктерді жүзеге
асыратын Қазақстан Республикасының азаматы» {[}1{]}. Әкімшілік құқықтық
қатынастарының субъектісі ретінде мемлекеттік қызметкерлер мемлекеттік
органдардағы лауазымдық өкілеттіктерін жүзеге асыру кезінде мемлекеттік
биліктің мүддесі үшін фунционалдық міндеттері мен өзге де қызметтерін
атқаруға бағытталған іс-әрекеті.

Қазақстан Республикасында мемлекеттік қызметкерлердің қызметін
ұйымдастыру және олардың еңбек қатынастарын Қазақстан Республикасының
Конституциясы, Қазақстан Республикасының Еңбек кодексі, Қазақстан
Республикасының «Мемлекеттік қызмет туралы» Заңы және тағы басқада
нормативтік құқықтық актілер реттестіреді. Мемлекеттік қызметкерлер
еңбек еркіндігін таңдауға және әділ еңбек жағдайына, соның ішінде барлық
азаматтар сияқты қауіпсіз және қолайлы еңбек жағдайында еңбек етуіне
құқығы бар {[}2{]} делінген.

Мемлекет басшысы Қасым-Жомарт Тоқаевтың «Әділетті Қазақстан: заң мен
тәртіп, экономикалық өсім, қоғамдық оптимизм» атты Қазақстан халқына
арналған жолдауында «Халық үніне құлақ асатын мемлекет» тұжырымдамасын
ұсынғалы бес жылдан астам уақыт өтті. Осы уақыт ішінде қоғам мен билік
құрылымдары арасындағы қарым-қатынас мәдениетін өзгерте алдық. Пікір
білдірудің бірқатар тиімді тетігі және түрлі диалог алаңдары пайда
болды. Мемлекеттік қызметшілердің азаматтармен тікелей сөйлесуі қалыпты
жағдайға айналды. Бұл тұжырымдама мемлекеттік қызметшілер мінез-құлқының
жаңа моделін қалыптастырды. Олар жауапкершілікпен, проактивті, ашық және
тиімді жұмыс істей бастады. Дегенмен мемлекет пен қоғам арасындағы
диалогтың сапасын одан әрі арттыра түсу үшін бәріміз адал әрі әділ
болуымыз, тек заң аясында әрекет етуіміз, берген уәдеміз бен ісіміз үшін
жауап бере білуіміз керек» {[}3{]}. Сөз етіп отырған «Халық үніне құлақ
асатын мемлекет» тұжырымдасы мемлекеттік органдарының жұмысының
тиімділігін арттыру мақсатында арнайы дайындығы бар кадрларды тарту
мәселесіне назар аударылды. Аталған концепцияға сәйкес мемлекеттік билік
өкілдері функционалдық міндеттемелерді орындауда, қоғаммен салиқалы
қатынас орнатуда ұйымдастырушылық мәдениетті қалыптастыруы тиіс дегенді
білдіреді.

Мемлекеттік басқарудың «адами орталықты» моделін қалыптастыру шеңберінде
адами ресурстардың сапасын арттыру және мемлекеттік аппаратты
кәсібилендіру ұйымда корпоративтік мәдениетті күшейтуді талап етеді.
Мемлекеттік басқарудың жаңа моделі «халық үніне құлақ асатын», тиімді,
есеп беретін, кәсіби және прагматикалық мемлекет қағидаттарына
негізделетін болады. Қағидаттар мемлекеттік басқару жүйесінің негізгі
қырларын, маңызды сипаттамаларын көрсетеді және мемлекеттік саясатты
қалыптастыру мен іске асыруда, сондай-ақ мемлекеттік шешімдерді
қабылдауда бағдар ретінде қызмет етуге арналған {[}4{]}. Мемлекеттік
билік пен жасалып отырған реформалар «Ең алдымен адамдар» басты
қағидатына бағынуға тиіс екендігі басты назарда болады. Қазіргі әлемде
күшті және жетістікке жеткен ұйымның негізі болып құндылықтарға толы
корпоративтік мәдениет құрайды. Қызметкерлердің адалдығына ықпал ететін
ұйымдастырушылық мәдениет ретінде бағаланады, яғни соңғы нәтиже ұйымның
жетістігіне әсер етеді деген ойлар келесідей.

Д.Элдриджу және А.Кромбидің айтуы бойынша «Ұйымдастырушылық мәдениет -
жалпы мақсаттарға қол жеткізу үшін ұйым ішіндегі бірлескен топтар немесе
жекелеген тұлғалар, құндылықтар, нормалар, мінез-құлық үлгісі және т.б.
құрайды {[}5{]}. Мемлекеттік қызметтегі ұйымдастырушылық мәдениетін
әрбір қызметкер ескеруі қажет. Бірінші кезекте, ресми емес көшбасшылар,
жоғары буын, сондықтан қызметкерлерді ынталандыру, қолдау маңызды болып
табылады. Мемлекеттік органдардың атқаратын қызметтерінің тиімділігін,
жұмысының сапалығын арттыруда және өнімділігін көбейтуде мемлекеттік
қызметтегі ұйымдастырушылық мәдениет негізгі рөлді атқарады. Қоғамға
қызмет ету және басқарудағы жоғары нәтижелерге жету тұрғысында
мінез-құлық ережелері мен нормалар сияқты өзіне қатысты құндылық
жиынтығын қамтиды.

{\bfseries Материалдар мен әдістер.} Ғылыми еңбек теориялық деңгейде
дайындалған, онда мемлекеттік қызметкердің ұйымдастырушылық мәдениет
мәселелері теориялық, әдіснамалық және практикалық тұрғыда жүйелі
қарастырылған.

Мемлекеттік қызметкерге мемлекеттік функцияларды атқару қоғам мен
мемлекет тарапынан білдірілген ерекше сенімі болып табылатыныны сөзсіз,
сондықтан ғылыми еңбекте мемлекеттік қызметшілердің адамгершілігіне,
әдебіне мен моральдық бейнесіне жоғары талаптар қойылатыны зерттелді.
Мемлекеттік қызметкерлердің ұйымдастырушылық мәдениетін қарастыруды
экономика және құқықтық тұрғыдан зерттеу маңызды, сонымен қатар зерттеу
барысында салыстырмалы-талдау әдістері қолданылды. Мемлекеттік
қызметкерлердің еңбек қатынастарын реттеуде еңбек-құқықтық нормалары мен
өзге де заңнамалық актілер қолданылды. Осылайша, мемлекеттік
қызметкерлердің ұйымдастырушылық мәдениетін қалыптастыруда іскерлік
қатынастарды зерттеуде психологиялық, философиялық, филологиялық,
әлеуметтану, мәдениеттану, экономикалық және құқықтық тұрғыдан кешенді
талдау жүргізілді.

{\bfseries Нәтижелер мен талқылау.} Заманға сай ұйымдасқан жүйені дамыту
үшін интеграциялық психологиялық үдерісте - мәдени парадигма үлкен рөл
атқарады. Біз өз зерттеуімізді Жетісу облысы Қаратал ауданындағы
мемлекеттік қызметтегі ұйымдастырушылық мәдениетін зерттеуге арнадық.

Қаратал ауданындағы ұйымдастырушылық мәдениетінің жекелеген элементтерін
дамыту бағытында 20 сұрақтан тұратын сауалнама жүргізілді, олар төрт
негізгі бағытқа топтастырылды. Атап айтқанда, зерттелген мәселелер:
қызметкерлердің жалпы жас ерекшеліктері және мемлекеттік органдағы жұмыс
өтілі; ұжымдағы моральдық-психологиялық ахуал; еңбек тәртібін сақтау;
жұмыс және еңбек жағдайларын ыңғайлы ету; ынталандыру жүйесі мен
мансаптық өсу мүмкіндіктері; еңбек өнімділігін арттыруға бағытталған
мотивация механизмдері; қызметкерлерге әлеуметтік-экономикалық қолдау
көрсету; қызметтік міндеттерге жатпайтын қосымша жұмыстар немесе шамадан
тыс жүктеме мәселелері; адалдық және кәсіби этика қағидаттарын сақтау;
жас мамандарды қолдау шаралары.

«Әдеп және өзара қарым-қатынас» көрсеткіші бойынша мемлекеттік
қызметшілер «әріптестер арасындағы қолдау мен өзара іс-қимыл деңгейі»
өздерінің мемлекеттік органында «жоғары деңгейде» (79\%) екенін атап
өтті. Алайда, респонденттердің 20\%-дан астамы ұжымдық ортада әрдайым
өздерін жайлы сезінбейтінін білдірді. Бұл ұжымның бірлігін нығайту,
проблемалық жағдайларды реттеу, топтық жұмыстың тиімділігін арттыру,
іскерлік қатынастар мәдениетін қалыптастыру және белгіленген шектеулер
мен тыйымдардың сақталуын бақылауды қамтамасыз ету қажеттілігін
көрсетеді.

Мәселен, мемлекеттік әкімшілік қызметке алғаш кірген және мемлекеттік
әкімшілік лауазымға алғаш тағайындалған мемлекеттік қызметшілерді қайта
даярлау курстары аясында «әдеп және парасаттылық», «Сыбайлас жемқорлыққа
қарсы мәдениет және парасаттылық» сияқты пәндер енгізілді. Сонымен
қатар, «Мемлекеттік қызметшілердің этикасы мен имиджі», «Сыбайлас
жемқорлыққа қарсы комплаенс» сияқты тақырыптар бойынша вебинарлар мен
жаттықтырушы-тренингтер ерекше сұранысқа ие болды.

Мәдениеттің өзгеруі және бірқатар дамыған елдердің мемлекеттік
қызметшілерінің санасының өзгеруі тәжірибесінде көшбасшылықты оқыту және
дамыту арқылы айтарлықтай нәтижелерге қол жеткізіледі. Оқыту қозғаушы
күшке (көшбасшылар, өзгеріс агенттері, басшылар) және басқа
қызметкерлерге арналған оқыту бағдарламаларының кешенін білдіреді.

Ол үшін негізгі қызметкерлер көшбасшылық, жанжал - менеджмент,
стресс-менеджмент бойынша құзыреттерді күшейтетін "өзгерістер
агенттерінің білім орталығын" құру ұсынылады. Оқытуды мемлекеттің
бағдарламалық міндеттері мен көзқарастарымен байланыстыру керек. Егер
мемлекеттік қызметшілердің барлық санаттары үшін оқытуда мемлекеттік
қызметшілердің мінез-құлқын түзететін Мемлекеттік қызмет құндылықтарын
ілгерілету негіз болатын болса, онда қозғаушы күш үшін оқыту
басқарушылық мәдениетті, көшбасшылықты, стратегиялық ойлауды, командалық
білім беруді дамытуға негізделуге тиіс.

Осыған байланысты мыналар ұсынылады:

- мемлекеттік қызметшілерді дамыту және мансаптық ілгерілету шеңберінде
тыңдаушылардың барлық деңгейлері үшін" мемлекеттік қызметтің
құндылықтары " (Оңтүстік Корея тәжірибесінде);

- "А" корпусының мемлекеттік қызметшілері үшін мынадай бағыттар бойынша
оқытудың арнаулы қысқа мерзімді бағдарламаларын құруға: мемлекеттік
қызметтегі жанжалдарды басқару бойынша дәріс.

Осылайша, ұсыныстарды іске асыру үшін көшбасшылықты дамытуда, күшті
ақпараттық жұмысты қоса алғанда, ішкі және сыртқы орта үшін
коммуникациялар құруда, көшбасшылық бағдарламалары бойынша персоналды
оқытуда, Сингапур мен Корея Республикасының тәжірибесі бойынша
Мемлекеттік қызметтің құндылықтарында, персоналмен тұрақты идеологиялық
жұмыста көрініс табатын ресурстармен өзгерістердің стратегиялық маңызды
процесін үздіксіз қоректендіру қажет.

«Еңбек нормалылығы» бөлімі «Жұмыс жүктемесі», «Еңбек жағдайларына
қанағаттанушылық», «Мамандардың жетіспеушілігі» сияқты мәселелерді
қамтиды. Жүргізілген талдау нәтижелері бойынша деректер келесідей
бөлінді: жүктемеге қанағаттанушылық деңгейі 79,6\%-ды құрады, ал еңбек
жағдайларына 87\% респондент қанағаттанатындарын айтты. Сонымен қатар,
13\% жұмыс орындарының жеткілікті деңгейде жабдықталмағанын, барлық
қажетті ақпараттық жүйелерге үзіліссіз қол жеткізу үшін техникалық
қамтамасыз етудің жеткіліксіздігін атап өтті.

Респонденттердің басым көпшілігі (80\%-дан астамы) жұмыс көлемін
орындауға қажетті мамандардың жүйелі түрде жетіспейтінін көрсетті, бұл
өз кезегінде әрбір қызметкерге түсетін жүктеменің едәуір артуына
әкеледі.

\emph{{\bfseries 1. Сыйақы жүйесі}}

1.1. Жалақы деңгейіне қаншалықты қанағаттанасыз?

\begin{itemize}
\item
  Өте қанағаттанамын 6\%
\item
  Қанағаттанамын 26,5\%
\item
  Бейтарап 18,1\%
\item
  Қанағаттанбаймын 36,1\%
\item
  Мүлде қанағаттанбаймын 14,5\%
\end{itemize}

{\bfseries Диаграмма 1 - Жалақы деңгейі}

1.2. Сыйақы мен бонустар жүйесі қаншалықты ашық және әділ?

\begin{itemize}
\item
  Толық ашық және әділ 25,3\%
\item
  Көбіне ашық 49,4\%
\item
  Көбіне ашық емес 18,1\%
\item
  Мүлде ашық емес 8,4\%
\end{itemize}

{\bfseries Диаграмма 2 - Сыйақылар}

1.3. Жұмыстағы күш-жігеріңіз тиісті деңгейде бағаланады деп санайсыз ба?

\begin{itemize}
\item
  Иә, толық 30,1\%
\item
  Көбіне иә 42,2\%
\item
  Көбіне жоқ 21,7\%
\item
  Мүлде жоқ 6\% {[}6{]}.
\end{itemize}

{\bfseries Диаграмма 3 - Жұмыстағы күш-жігер деңгейі}

«Сыйақы тағайындау» бойынша сауалнамаға қатысқан респонденттердің
50\%-дан астамы өз жалақыларына қанағаттанбағанын, бейтараптылықты 18\%
ұстаным білдірді, ал қызметкерлердің 32,5\%-ы оң баға бергенін атап
өткен жөн. Атап айтқанда, сауалнамаға қатысушылардың 74\%-дан астамы
сыйақы беру және ынталандыру механизмдеріне көңілі толатындағын және
әділ әрі ашық деп санайды және тұтастай алғанда ол басқа жеке немесе
басқа ұйымдардағы деңгейге сәйкес келеді. Бірақ сонымен бірге, 18,1\%
бұл пікірмен келіспейді, ал 8,4\% бұл пікірмен абсолютті келіспеушілік
бар. Мемлекеттік қызметшілерді ынталандыру тетіктері заңнамалық деңгейде
реттелетінін атап өткен жөн. Дегенмен, бүгінгі күні мемлекеттік
қызметшіні көтермелеу басшылықтың қалауы бойынша ғана шешіледі, бұл
сауалнама нәтижелері бойынша әрқашан объективті бола бермейді.

Жалпы алғанда, Жетісу облысы Қаратал ауданында мемлекеттік қызметшілері
ұйымдастырушылық мәдениетті оң бағалады. Аудандағы
моральдық-психологиялық климатты жоғары дейгейде екендігін атап өтті.
Басшылық қол астындағы қызметкерлерге құрмет пен сыпайылықпен
қарым-қатынасын қалыптастырылғандығын, іскерлік байланыс орнатқандығын
мақтанышпен жеткізді. Дегенменде жұмысты жақсарту үшін келесідей
ұсыныстар айтылды: мемлекеттік қызметке алғаш кірген мамандар және
мемлекеттік әкімшілік лауазымға алғаш тағайындалған мемлекеттік
қызметшілерді қайта даярлау курстары аясында «Мемлекеттік қызметшілердің
этикасы мен имиджі», «Әдеп және парасаттылық» «Сыбайлас жемқорлыққа
қарсы мәдениет және парасаттылық» сияқты тақырыптар бойынша вебинарлар
мен жаттықтырушы-тренингтер ерекше сұранысқа ие болды. Екіншіден,
аудандық әкімшіліктегі мамандардың жетіспеушілігі, жұмыс жүктемесінің
ауырлығы, кейбір қызметкерлерге артық жүктеме беру жағдаяттары орын
алуда, бұл қызметкердің шамадан тыс жұмыстар орындау барысында шаршауына
және қызмет сапасының төмендеуіне әкеп соғады. Үшіншіден, жұмыс
уақытының аяқталуына қарамастан, үйге уақытылы қайта алмауы. Мекеме
басшылығының жұмыс орнында жиі ұстап қалады. Олар функционалдық
лауазымдық міндеттеріне сай келмейтін жұмыстар тапсыратындығын ескертті.
Мекеме басшысы күні бойы өзі жұмыс орнында отырмағанына қарамастан,
қызметкерді басшылықтан бұрын жұмыс орнынан шығып кетуіне жол бермеуге
тырысатындығын байқатты. Сауалнамаға қатысушылар тапсырмалар орындау
мерзімінің қысқа мерзімге берілуі, жұмысты қайта жасауға себеп болуы деп
түсіндіреді.

Сонымен қатар, жұмыс орындарының жеткілікті деңгейде құрал-саймандармен,
компьютерлік-техникамен жабдықталмағанын, барлық қажетті ақпараттық
жүйелерге үзіліссіз қол жеткізу үшін техникалық қамтамасыз етудің
жеткіліксіздігін атап өтті.

Жұмыстан босатылатын қызметшілермен жүргізілген сауалнамаға сәйкес
мемлекеттік қызметтен кетудің негізгі себептері еңбектің қалыптан тыс
болуы, жалақының төмендігі, жұмыстың мақсаттар мен үміттерге сәйкес
келмеуі, сондай-ақ мансаптық өсу перспективасының болмауы болып
табылады. Бұл ретте өз еркімен жұмыстан шығарылғандардың 72\%-ы қызметте
3 жылдан аз жұмыс істеді {[}6{]}. А.Ж. Аяғанова мен Б.А. Тұрғазы
«Мемлекеттік қызметкердің кәсіби және психологиялық мәдениеті» атты
еңбегінде «бүгінгі таңда мемлекеттік қызмет жүйесінің мазмұны мен рөліне
деген көзқарастың өзгеруі осы сала мамандарының кәсіби және
психологиялық мәдениетіне талапты күшейтіп отыр. Билік пен халық
арасындағы өзара ынтымақтастыққа қол жеткізуге, әлеуметттік-экономикалық
жаңғыруды жүзеге асыруда басқарушылық тетіктерді оңтайландырып қоймай,
сонымен қатар мемлекеттік қызметшілердің қызмет тиімділігін арттыру
жолдарын анықтау маңызды мәселелердің бірі болып табылады {[}7{]}.
Авторлар ойынан шығатын қорытынды мемлекеттік қызметкерлердің кәсіби
және психологиялық мәдениетін жетілдіру мәселесіне көңіл аударылып,
билік пен халық арасында ашық диалог орнату арқылы серіктестікті орнату
керектігі, сонымен қатар билік механизмдерін жаңғырту арқылы мемлекеттік
қызметшілердің функционалдық міндеттеріне жауапкершілікпен қарауды және
жұмыс тиімділігін көтеру нысаналы бағыттарын анықтау тетіктерін
іздестіру өзекті проблема екендігі анықталды.

Мемлекеттік органдар және олардың басшылары мемлекеттік қызметкердің
еңбек нормаларын бұзуға өздері итермелейтіндігі, жасанды түрде
жасайтындығы байқалды. Аталған сұрақтарды шешу үшін облыс және
республикалық маңызы бар қалалардың әкімдігі аппаратының басшылары үшін
жұмыс режимін сақтауды, оның ішінде қолайлы жұмыс кестесі, үстеме еңбегі
үшін ақы төлеу, қосымша еңбек демалыстарын беруді көздейтін негізгі
нысаналы индикаторлар белгілеуді ұсынған болатын {[}8, 21 б.{]}.
Авторлар зерттеу барысында мемлекеттік қызметтегі ұйымдастырушылық
мәдениетті қалыптастыру үшін қолайлы еңбек жағдайын жасауды ұсынған.
Сондай-ақ ҚР мемлекеттік органдарда корпоративтік мәдениеттің қалыптасуы
мен дамуына ерекше көңіл аударылады. Жүргізілген зерттеулерге сәйкес
Ж.Байжомартова мен М.Кадырова мемлекеттік қызметкерлердің арасындағы
жүргізілген сауалнаманың талдауы бойынша жұмысты жақсарту қажеттілігі
анықталды: ұжымдағы қолайлы еңбек жағдайының моральді-психологиялық
климат және қызметкерлерді ынталандыру жүйесіне байланысты практикалық
ұсыныстар енгізеді.

Мемлекеттік қызметшілердің ұйымдастырушылық мәдениетін қалыптастыру
бойынша әлеуметтанушылармен, философтармен, басқару жүйесіне қатысты
ғалымдармен зерттелді.

Ресейлік ғалым Т.С. Соломанидина келесіні ұсынған болатын
«Ұйымдастырушылық мәдениет -- ұйымның жетістікке жетуіне мүмкіндік
беретін, компания қызметкерлерінің мінез-құлқы және мәселелерді шешу
тәсілдері құрайды. Сондай-ақ материалдық және материалдық емес көтеру
арқылы қалыптасқан, анық және жасырын, процестер мен құрыбылыстарға
саналы және саналы емес философиялық, идеологиялық және құндылықтар
бірлігін анықтайтын ұйымның әлеуметтік-рухани өрісі {[}9, 43 б.{]}.

И.В. Грошевтың пікірінше «Ұйымдастырушылық мәдениет -- ұйымның басым
көпшілік мүшелелерін бөлінетін сыртқы ортаға бейімдеу құралы болып
табылатын ынталандыру және олардың қызметін қамтамасыз ету барысында
нормалар мен құндылықтар ретінде қарастырды» {[}10,105 б.{]}.
Қызметкерлердің мінез-құлқы уақыт өте келе белгілі бір заңдылықтарды,
стильдерді, нормаларды қалыптастырады. Қызметкердің жекелеген әрекеттері
жалпы нәтижелерге әсер етуі ықтимал, ал нақты жағдайларды талқылау
құндылықтарға негізделеді. Қызметкер жұмыс бабында жүріп, тәжірибие
жинақтау арқылы уақыт өте «нығайып», ұйымдастырушылық мәдениетті
қалыптастыру кезінде қызметкерлердің мінез-құлқын өзі анықтайды.
Хофштеде әдістемесі бойынша ұйымдастырушылық мәдениеттің бес негізгі
көрсеткіші есептелді. Даралық көрсеткішке сүйенсек, жекелеген адамдар
арасындағы байланыс болмашы қоғамды сипаттайды деген пайымдау
қалыптасқан. Әрбір қызметкер өзіне және отбасына қамқорлық жасайтыны
анық {[}11{]}.

Жалпы мемлекеттік қызметкердің кәсіби мәдениетін құндылықтар кешені мен
мінез-құлық нормалары кешені құрайды. Әсіресе бұл мемлекеттік басқару
жүйесінің көрсеткіші болып табылады. Осы орайда біз құндылықтарға
тоқталып өтуді жөн көрдік. Құндылық - бұл адам қол жеткізу үшын бар
ресурсын; күш-жігерін, уақытын, қаржысы және т.б. жұмсайтын жағдай.
Құндылықтарды мәнез-құлықтың мәнді зерделеушісі (алға тартушы) ретінде
тұжырымдауға болады. Зерттеулерде құндылықтардың көптеген сипаттамасы
қарастырылды {[}11 - 13{]}. Ғалымдар М.С. Яницкий пен Isa Bertram, Robin
Bouwman, ғылыми еңбектерінде мемлекеттік қызметкердің кәсіби мәдениетке
құндылықтар мен мінез-құлық нормалары кешенін қарастырды. Сондай-ақ
құндылықтардың белгілерін нақтылады. Мемлекетік қызметтегі
ұйымдастырушылық мәдениеті - белгілі бір мемлекеттік органның
мемлекеттік қызметшілері қабылдайтын және бірлескен қызметінің өнімі
болып табылатын жағдайларда жүзеге асырылады, қызметтік жүріс-тұрыс
нормалары мен құндылықтар жиынтығы құрайды {[}14{]}. Бұл жерде
мемлекеттік қызметтегі ұйымдастырушылық мәдениетін қалыптастыруда
белгілі бір мемлекеттік органның қабылдайтын және бірлескен қызметінің
өнімі болып табылатын жағдайларда жүзеге асырылатын қызметтік
жүріс-тұрыс нормалары мен құндылықтар жиынтығы құрайды деген ойды
білдіреді, бұл құптарлық жағдай. Ғалым П.Кенсок құндылықтарға негізделе
отырып, өмірді ұйымдастырудың тиімді құралын ұсынады. Аталмыш ғалым
еңбектерінде көрсетілген құрал жеке тұлғаның, сонымен қатар команданың
құндылықтары мен роліне талдау жасауға көмектеседі. Осы зерттеулер
көрсеткендей құндылықтарға бағдарланған басқару қарапайым тәжірибеге
және шешім қабылдауға бағытталған болуы тиіс. Ұйымдастыру мәдениеті --
менеджмент ғылымдарының сериясына енетін ең жаңа білім саласы. Ол
салыстырмалы түрде жаңа білім саласында ерекшеленді - компаниядағы жалпы
комбинацияларды, негіздерді, заңдар мен үлгілерді зерттейтін ұйымдық
мінез-құлық.

Әр түрлі елдердің мемлекеттік қызмет құндылықтарының жиынтығын зерттеу,
жалпы алғанда, бұл құндылықтар өте ұқсас екенін көрсетті. Көптеген
құндылықтардың негізгі біріктіруші хабары азаматтарға адал қызмет ету
болып табылады. Қазақстанның мемлекеттік қызметінің ұйымдастырушылық
мәдениетін дамыту үшін Сингапурдың тәжірибесі жақсы үлгі болып табылады.
Мемлекеттік қызметшілердің ұйымдық мәдениеті мен санасын өзгерту
құралдарының бірі 1995 жылы мамырда енгізілген "21 ғасырдағы Мемлекеттік
қызмет" бағдарламасы деп танылады. Бағдарлама азаматтарға шығармашылық
көзқарас пен тиімді қызмет көрсету мақсатында өзгерістерді болжауға,
қарсы алуға және қабылдауға бағытталған Сингапурдың Мемлекеттік қызмет
стандартының тасымалдаушысы болды. Ps21 бағдарламасының миссиясы бүкіл
мемлекеттік секторға өзгерістер енгізу және екі негізгі тапсырманы
орындау болып табылады:

- жоғары сапа, сыпайылық және жеделдік стандарттарымен азаматтарға
қызмет көрсетудің жоғары сапасына ұмтылуға тәрбиелеу;

- мемлекеттік қызметшілердің моральдық, адамгершілік рухы мен әл-ауқатын
ескеретін қазіргі заманғы басқару құралдары мен әдістерін пайдалану
арқылы нәтижелілік пен экономикалық тиімділікті арттыру үшін үздіксіз
өзгерістерді қолдау және ынталандыру жағдайларын жасауға жәрдемдесу.

Бағдарлама Сингапур мемлекеттік қызметінің ойлау, мінез-құлық,
ұйымдастырушылық мәдениетінің дамуына әсер етті. Сингапур мемлекеттік
қызметінің негізгі мақсаты-ұлт пен мемлекетке адал қызмет ету. 2003 жылы
қайта қаралған және қайта бекітілген Мемлекеттік қызметтің негізгі
құндылықтары-адалдық, қызмет және шеберлік. Бұл құндылықтар бағдар
ретінде қызмет етеді, мемлекеттік қызметшілердің мінез-құлық ережелері
мен жұмыс мазмұнын анықтайды, мақтаныш пен мемлекеттік қызметшілер
қатарына жату сезімін тудырады. Осы үш құндылыққа берілгендік Сингапур
мемлекеттік қызметінің басқару саласындағы халықаралық беделінің өсуіне
ықпал етті. Бұл құндылықтар ережелер мен нұсқаулықтарға енгізілген,
мемлекеттік қызметтегі жаңартылған мінез-құлық кодексіне енгізілген.

Ұлыбританияда мемлекеттік қызметтің негізгі құндылықтары мен барлық
мемлекеттік қызметкерлер ұстанатын мінез-құлық стандарттары азаматтық
қызмет кодексінде сипатталған. Мемлекеттік қызметтің негізгі
құндылықтары адалдық (integrity), объективтілік (объективтілік) және
бейтараптық болып табылады impartiality). Бұл құндылықтар мемлекеттік
қызметке қатысты барлық мүмкін болатын ең жоғары стандарттарға қол
жеткізуді қамтамасыз етеді. Бұл өз кезегінде мемлекеттік қызметке
үкіметті, парламентті, қоғамды және қызмет алушыларды құрметтеуге
көмектеседі. Канаданың тәжірибесі басқа елдерден біршама ерекшеленеді.
Канадада мемлекеттік қызметтегі құндылықтар мен этика бір мемлекеттік
қызметтегі құндылықтар мен этика кодексіне біріктірілген - Values and
Ethics Code For The Public Service. Мемлекеттік қызметтің құндылықтары
құндылықтардың келесі үш түрінің теңдестірілген шеңберін құрайды:

- демократиялық құндылықтар (Democratic Values) -- шенеуніктерге заң
шеңберінде азаматтардың мүдделеріне қызмет етуге жәрдемдесу;

- кәсіби құндылықтар (кәсіби құндылықтар) - Құзыретті, шеберлікпен,
тиімді, объективті және бейтарап қызмет ету; --

этикалық құндылықтар (этикалық құндылықтар) - әрқашан азаматтардың
сеніміне ие болу үшін әрекет ету;

- адами құндылықтар (People Values) -- азаматтармен және әріптестермен
жұмыста құрмет, әділдік, сыпайылық көрсету.

Мемлекеттік қызмет жүйесіндегі кәсіби мәдениеттің тағы бір құрамдас
бөлігі тұлғаның жеке қасиеттері болып табылады. Мемлекеттік қызметші
құзыреттілігінің бірыңғай шеңберіндегі жеке қасиеттерге жауапкершілік,
бастамашылдық және күйзеліске төзімшілдік жатады. Жауапкершілік -- бұл
өз әрекеті мен нәтижелеріне жауапкершілік таныту. Бастамашылдық
дегеніміз жаңа идеяларды әзірлеу және ұсыну, негізгі міндеттерінен бөлек
қосымша жқмыстарды орындау болып табылады. Ал күйзеліске төзімшілдік --
сынға сабырлық таныту және ол негізделген болса, кемшіліктерді жою
бойынша шараларды қабылдау деп тұжырымдауға болады {[}15, 44 б.{]}.

Ғалым М.Багатыров «Организационная культура: сущность и роль в системе
управления» атты диссертациялық жұмысында шетелдік және ресейлік
ғалымдарың еңбектерін талдай келе, ұйымдастырушылық мәдениеттің негізгі
үш позициясын:

- басқару объектісі ретінде;

- басқару әдістемесі не басқару құралы ретінде;

- ұйым қызметкерелерін басқару ортасы ретінде ұсынған болатын {[}16, 44
б.{]}.

Ресейлік ғалым М.Багатыровтың ұйымдастырушылық мәдениетке қатысты
айтылған үш позициясы да орынды деуге болады. Кеңестік ғалым Е.В.
Козниевская зерттеулерде өзіндік бағалау кәсіби өзіндік сананың жүйе
қалыптастырушы бөлшегі болып табылады. Бұл жағдайда кәсіби өзіндік
бағалау мемлекеттік қызметшілердің жетістікке жету мотивациясының
факторы ретінде алға шығады {[}17,122 б.{]}.

Мемлекеттік органдарда корпартивтік мәдениетті қалыптастыру үшін ресми
іскерлік қарым-қатынас орнату маңызды. Ресми іскерлік қарым-қатынас
мәдениеті мемлекеттік органдардың кәсіби қызметінің құрамдас бөлігі
ретінде қалыптасады.

Отандық ғалым К.О. Айтмухаметова атап өткендей «Мемлекеттік қызметте
негізінен ресми стиль қолданылады. Ресми іскерлік қатынас деп адамдардың
кәсібі, мамандығымен байланысты қызмет орындарында, мәдени-көпшілік
мекемелерде жүзеге асатын қатынастың түрін айтамыз. Мұндай орындардағы
мұндай қатынастың түрі ресми сипатқа ие. Мемлекеттік қызметшілердің,
оқытушы мен магистранттың арасындағы қарым-қатынас ресми іскерлік
қарым-қатынасқа жатады {[}18,84 б.{]}. Тілдік нормалардан ауытқу және
іскерлік коммуникация қағидаларын бұзу жұмыс тиімділігіне теріс әсер
етуі мүмкін» деген болатын. Сондықтан, мемлекеттік қызметкерлер алдында
жариялы баяндама жасау дағдысы мен сөйлеу мәдениеті дамытуға назар
аудару маңызды. Осылайша, мемлекеттік органдардағы жұмыс тиімділігін
арттыруда және халыққа сапалы қызмет көрсетуде - мемлекетік қызметтегі
ұйымдастырушылық мәдениеті маңызды құрал ретінде бағаланады.

Қоғамдық өмірдің әр саласында қызмет етіп жүрген адамдардың еңбегіне
лайық әр түрлі атақ-дәрежелері болады. Мәдениетті елдің дәстүрінде ондай
атақ, кісі фамилиясымен қоса айтады {[}19,105 б.{]}. Ақырында, кәсіби
білім мен қабілеттердің өтімділік дәрежесі менеджер тек осы ұйымда ғана
талап етілетін бірегей жұмыстарды және тиісті лауазым үшін стандартты
жалпы сипаттағы жұмыстарды орындайтын пропорцияны білдіреді. Кәсіби
маманның табыстылығы адамның өзін бағалауына байланысты болады. Сонымен
қатар өзіндік бағалау адамның күшін ұтқырландыруға, жасырын мүмкіндігін,
шығармашылық әлеуетін жүзеге асыруға жағдай туындататын маңызды фактор
болып табылады. Осы тұста айта кететін жайт төмен өзіндік бағалауды
шетелдік зерттеулерде ұят сезімінде болады деп тұжырымдайды {[}20{]}.

Бұл зерттеулерде ұят сезімі тұлғаның әлеуметтік Мен құрылымын
құнсыздандырады. Мәселен өзіндік бағалаудың төмен болуы әлеуметтік
тұрғыда нашар бағаланатын қызметте көрініс береді. Қызмет тиімділігінің
төмендеуі әлеуметтік жағдайға қауіп төндіретін психологиялық күйлердің
жоғары реакцияларына әкеледі. Атап айтқанда, әлеуметтік өзін өзі
бағалаудың төмендеуі және ұят сезімінің жоғарылауы {[}21{]}.

«Мемлекеттік қызметшілер өз қызметінде мемлекеттік саясатты ұстануға
және оны дәйекті түрде іс жүзіне асыруға, қоғамның мемлекеттік қызметке,
мемлекет пен оның институттарына деген сенімін сақтауға және нығайтуға
ұмтылуы тиіс» {[}22{]}- деп көрсетілген. Мемлекеттік қызметшілердің
қызметтік әдебі мынадай қағидаттарға негізделеді:

- адал ниеттілік -- қоғам игілігі үшін мемлекетке кәсіби және жауапты
қызмет ету;

- адалдық -- өз міндеттеріне шынайы көзқарас;

-әділдік -- жеке және заңды тұлғалардың, қоғамдық топтар мен ұйымдардың
ықпалына қарамастан заңды шешім қабылдау және кез келген мән-жайлар
бойынша біржақтылық пен субъективтілік себебінен адамдарды кемсітуге жол
бермеу;

- ашықтық -- жұртшылықпен жұмыс істеуге дайын екенін көрсету және өз
іс-қимылының ашықтығын қамтамасыз ету;

- сыпайылық -- азаматтар мен әріптестерге дұрыс және құрметпен қарау;

- клиентке бағдарлану -- мемлекеттік қызметтерді тұтынушы ретінде
халықтың сұраныстарына толықтай бағдарлай отырып, көрсетілетін
мемлекеттік қызметтердің сапасын арттыру және жолданымдарды қарау
кезiнде төрешiлдiк көрiнiстерiне және әуре-сарсаңға салуға жол бермеу
жөнінде шаралар қабылдау.

Жоғарыдағыларды тарқатсақ, мемлекеттік қызмет оны атқаратын азаматқа
қоғам мен мемлекет атынан артылатын «сенім жүгі», оны атқаруда азаматқа
қоғам тарапынан биік адамгершілік құндылықтарды ұстану, кәсіби әдеп
нормаларын сақтау талаптар қойылады. Елімізде шенеуніктерді жұмысқа
қабылдауда азаматтардың шынайы түрде ұстанатын әдеп стандарттары мен
адалдық талаптары әлі қалыптаспаған. ҚР мемлекеттік қызмет істері
агенттігі мен БҰҰ ДБ (ПРООН) бірігіп жасаған есебінен, мемлекеттік
қызметке кіруде өзінің туындауы мүмкін мүдделер қақтығысы туралы
жарияламау, қызметтік ілгерілеудегі мүдделер қақтығысының жиі ұшырасуы
салдарынан мемлекеттік құрылымдарға деген сенімнің төмендеп келе
жатқанын көруге болады {[}23{]}. Зерттеу нәтижесі көрсеткендей,
мемлекеттік қызметтің қоғам мен халық алдындағы міндетін атқару
барысында адамгершілік құндылықтарды ұстану, кәсіби әдеп нормаларын
сақтау жөніндегі талаптар қойылған. Францияның мемлекеттік қызметін
этикалық реттеу тәжірибесін талдай отырып, жыл сайынғы тәртіптік
санкциялар кестесін жүргізу тәжірибесі қызығушылық тудырады. Францияда
бұл кестеде нақты мемлекеттік органның қызметкерлері жасаған құқық
бұзушылықтар және оларға қолданылған санкциялар туралы мәліметтер бар.
Бұл ретте құқық бұзушылық жасаған қызметкерлердің тегі мен лауазымы
ашылмаған күйінде қалады.

{\bfseries Қорытынды.} Мемлекеттік қызметкерлердің жүріс-тұрысы мен
мінез-құлқы іскерлік қатынастарын реттеуде мемлекеттік қызметті
ұйымдастыру маңызды болып табылады.

Осының ішінде ең маңыздысы жоғарғы органдар мен лауазымды адамдардың
шешімдері және мемлекеттің жүктеген міндеттерін орындау барысында әділ,
парасатты, адал, қиянаттыққа бармайтын, көңілге сенім ұялататын, отанға
деген парызының беріктік қасиеттері атқару болып саналады. Біздің
ойымызша төмендегідей қорытындыларға келдік:

Біріншіден, мемлекеттік қызметтегі ұйымдастырушылық мәдениетті
қалыптастыруда ұйым қызметкерлері арасында функцияларды теңдей бөлмеу,
кейбір қызметкерлерге артық жүктеме беру жағдаяттары орын алуда, бұл
қызметкердің шамадан тыс жұмыстар орындау барысында шаршауына және
қызмет сапасының төмендеуіне әкеп соғады. Сондай-ақ құрылымдық
бөлімдерде мамандардың жетіспеушілігі орын алатын жағдайлар тағы бар.
Жұмыс жүктемесінің шамадан тыс болуы қызметкердің жұмыстан кетуіне себеп
болуы мүмкін. Сондықтан, мекеме басшысына қарамағындағы қызметкерлерге
жұмысты біркелкі беруі тиіс деген ойдамыз.

Екіншіден, мемлекеттік қызметкерлердің жиі жұмыстан ауысу мәселесі
жылдан-жылға қайталанып келеді. Мемлекеттік қызметтегі басшылықтың
командалық әдіспен жұмысқа келуі, іскер азаматтардың жұмыстан кетуіне
әкеліп соғуда. Қызметкерлерге лауазымдық міндеттерге сай келмейтін
жұмыстар тапсыру жұмыстан өз еркімен кетіруге негіз болуы мүмкін.
Облыстық маңызы бар қалалар, аудан, ауданға теңестірілген қалалардың
әкімдігінде ауыс-түйіске жол бермеу шараларын іздестіру маңызды, себебі
жергілікті әкімшіліктерде сол ауданның тұрғылықты халқы жұмыс істейтінін
назарда ұстауы қажет.

Үшіншіден, мемлекеттік қызметкердің қызметінің тиімділігін және жұмыс
сапасын арттыру үшін ынталандыру механизмдерін енгізу маңызды.
Ынталандыру материалдық және материалдық емес нысанда болуы мүмкін.
Мемлекеттік органдарда ынтыландыру кезінде лайықты емес қызметкерлер
иеленуі не марапатталуы орын алады деген теріс пікірлер бар. Бұл
мемлекеттік органдағы қызметкерлер арасында алауыздық пен сенімсіздік
туғызады. Ынталандыруды кей жерлерде ақылы және ақылы емес депте
қарастыралады.

Оңтүстік Кореяда қызметкерді ақшалай және ақшалай емес ынталандыруды
ұсынған. Ақшалай ынталандыру сыйақы, үстемақы және жұмысты сәтті
орындаған тұлғаға төлем. Ақшалай емес ынталандыру қызметкердің
нәтижелері жетістіктерді есепке алу жүйесінде ескеріледі, марапаттар
беріледі. Бұл бағалаулар жыл қорытындысы бойынша жалақы көтеруде
басшылыққа алынады.

Мемлекеттік қызметкерлерді ынтыландыру үшін сыйақы беруде ашықтық пен
жариялық қағидалары арқылы іске асыру қажет. Сондай-ақ, сыйақы беру
жалғыз басшымен шешілмеу керектігі қозғалды, яғни түрлі құрылымдық
бөлімшелер қызметкерлер арасынан құрылған комиссия жасырын дауыс беру
арқылы анықтағаны жөн деген ойдамыз. Үздіктерді және ақшалай сыйақыға
лайықты қызметкерлерді анықтау үшін жасырын дауыс беру жүйесін енгізу
пайымдалды. Осылайша, сыйақы беру кезінде мемлекеттік қызметкердің жұмыс
сапасын арттыру жағдайын және қызметінің тиімділігін бағалау
критерийлерін ескеру маңызды.

Төртіншіден, мемлекеттік қызметкердің ұйымдастырушылық мәдениетін
қалыптастыру үшін әдеп ережелерін сақтау ерекше маңызға ие. Басшы
тарапынан қарамағындағы мемлекеттік қызметкерге қатысты әдепке жатпайтын
(кекету, мұқату, дөрекілік, қорлау, балағатты сөздер, харассмент, қол
көтеру) іс-әрекеттер орын алуы мүмкін. Республикалық маңызы бар қалалар
мен облыстық әкімшіліктерде мемлекеттік қызметтегі ұйымдастырушылық
мәдениеттің қалыптасуы мен әдеп ережелерінің сақталуын әдеп жөніндегі
уәкіл іске асырады. Мемлекеттік органдарда әдеп жөніндегі уәкіл институт
енуімен байланысты мекеме ішінде іскерлік қарым-қатынас орнауына,
ұйымдастырушылық мәдениеттің қалыптасуына, қызметтегі мүдделер
қақтығысының алдын алу, қызметкерлердің жүріс-тұрысы мен мінез-құлқын
реттеуде, белгілі бір шектеулер мен тыйым салу процесін сақтауда бақылау
күшейтілді. Әдеп жөніндегі уәкілдік жүктелген қызметкер тапсырмаларды
орындамаған не тиісті дәрежеде орындамаған жағдайда жауаптылыққа тарту
қиынырақ, себебі негізгі жұмысы болып табылмайды. Осыған байланысты,
облыстық маңызы бар қалалар, аудан, ауданға теңестірілген қалалар әдеп
жөніндегі уәкілдік штаттық бірлігін енгізу ұсынылып отыр.

Бесіншіден, мемлекеттік қызметкердің ұйымдастырушылық мәдениетін
қалыптастыру үшін тағы бір маңызды қоғамдық институттың қажеттілігі бар.
Мемлекеттік органдардағы әдеп ережелері мен мемлекеттік қызметкердің
ұйымдастырушылық мәдениетін қалыптастыру үшін әдеп жөніндегі Қоғамдық
ұйымды құру қажеттілігі бар деп есептеймін, қоғамдық кеңес мүшелері
жасырын дауыс беру арқылы әдеп жөніндегі уәкіл мемлекеттік органдардағы
абыройлы және құрметке лайық мемлекеттік қызметкерлер ішінен сайлануы
тиіс.

Алтыншыден, Бекітілген заңнама бойынша мемлекеттік қызметкердің зейнетке
шығу жасы 63 құрап отыр. Егде жасқа жеткет қызметкер тез шаршайды,
жұмыстың тез әрі нәтижелі бітуін емес, жұмыс уақытының тез аяқталуын
қалайтыны ақиқат. Тиісінше жұмыс өнімділігі мен тиімділігінің азаюына,
сапаның төмендеуіне әкеп соғады. Шаршаған мемлекеттік қызметкердің
ешнәрсеге зауқы болмайды, нәтиже бермейді. Сондықтан, мемлекеттік
қызметкердің зейнетке шығу жасын 60-жасқа азайту қажеттілігі бар, мұны
өмір көрсетті. Бұл зейнетке шығу жас шамасы халықаралық тәжірибеге сай
келеді. Сондай-ақ, зейнетке шығу жағдайы әлемде қалыптасып отырған
экономикалық және демографиялық жағдайға сүйенген процесс.

Жетіншіден, зерттеу негізінде ұйымдық мәдениетті басқару процесін
ұйымдастырудың төрт негізгі тетігі ұсынылады: адамдармен жұмыс істеу,
қызметкерлермен идеологиялық жұмысты күшейту, қызметкерлерді оқыту,
"өзгерістер агенттерінің білім орталығын"құру. Қызметкерлердің
мотивациясы және олардың ұйымға деген адалдығы оларды жеңістер мен
сәтсіздіктердің жалпы тарихы біріктіретін топтардағы, топтардағы
іс-шаралар жағдайында айтарлықтай артатынын есте ұстаған жөн. Оқыту
мемлекеттік қызметшілердің санасын жаңғыртуға, Мемлекеттік қызмет
құндылықтарын ілгерілетуге, басқарушылық мәдениетті, көшбасшылықты,
стратегиялық ойлауды дамытуға, командалық білім беруге бағытталуы тиіс.

\emph{{\bfseries Қаржыландыру.} Жұмыс Қазақстан Республикасы Білім және
Ғылым министрлігінің ғылым комитеті қаржыландыратын Грант №BR 27195163
бағдарламасы}

{\bfseries Әдебиеттер}

1. Қазақстан Республикасының Мемлекеттік қызметі туралы: Қазақстан
Республикасының Заңы 2015 жылғы 23 қарашадағы № 416-V ҚРЗ. //
\url{https://adilet.zan.kz/kaz/docs/Z1500000416} Жүгінген күні:
01.12.2024

2.Сарсен Ж. Трудовые государственых
служащих//https://www.gov.kz/memleket/entities/

zem-shahtinsk/press/news/details/813166.Жүгінген күні: 01.12.2024

3.Мемлекет басшысы Қасым-Жомарт Тоқаевтың «Әділетті Қазақстан: заң мен
тәртіп, экономикалық өсім, қоғамдық оптимизм» атты Қазақстан халқына
Жолдауы 2024 жылғы 2
қыркүйек.//\url{https://www.akorda.kz/kz/memleket-basshysy-kasym-zhomart-tokaevtyn-adiletti-kazakstan-zan-men-tartip-ekonomikalyk-osim-kogamdyk-optimizm-atty-kazakstan-halkyna-zholdauy-285659}
Жүгінген күні: 01.12.2024

4. Қазақстан Республикасында Мемлекеттік басқаруды дамытудың 2030 жылға
дейінгі тұжырымдамасын бекіту туралы: Қазақстан Республикасының
Президенті 2021ж.26 ақпандағы № 522
Жарлығы//\url{https://adilet.zan.kz/kaz/docs/U150000015}. Жүгінген
күні:01.12.2024

5. Gschwantner, S.,~Hiebl, M.R.W.
\href{https://www.scopus.com/record/display.uri?eid=2-s2.0-84978811157&origin=reflist&sort=plf-f&src=s&sid=9294c12779ee47479a1a0997e8991897&sot=b&sdt=b&sl=99&s=TITLE\%28An+Examination+of+Civil+Servants\%e2\%80\%99+Assessment+of++the+New+Civil+Service+Reforms+in+Kazakhstan\%29}{Management
control systems and organizational ambidexterity}.//Journal of
Management Control.- 2016.- Vol.~27(4).-P.371-404.~

DOI 10.1007/s00187-016-0236-3

6. Қазақстан Республикасындағы мемлекеттік қызметтің жағдайы туралы
ұлттық баяндама
//https://www.gov.kz/memleket/entities/qyzmet/documents/details/454043?lang=kk

Жүгінген күні: 01.12.2024

7. Аяғанова А.Ж., Тұрғазы Б.А «Мемлекеттік қызметкердің кәсіби және
психологиялық мәдениеті» Психология және социология сериясы.// 2021.- №
1(76). -104- 112.б.

DOI 10.26577/JPsS.2021.v76.i1.010

8. Байжомартова Ж., Кадырова М. Развитие корпоративной культуры как
инструмент повышения эффективности деятельности государствееных служащих
// Мемлекеттік басқару және мемлкеттік қазмет.// 2022. - №4 (83). -
13-23 бб. DOI 10.52123/1994-2370-2022-914

9. Соломанидина Т.О. Организационная культура компании: Учеб. пособие.
-- 2-е изд., перераб. и доп. -- М.: ИНФРА-М, 2015. -624 с. ISBN
978-5-16-003946-6

10. Грошев И.В. Организационная культура в системе менеджмента
современного российского предприятия: дис. ... д-ра экон. наук. -
Тамбов, 2007. - 493 с.

11. Barodi M., Lalaou S. Civil servants' readiness for ai adoption: the
role of change management in morocco's public sector//` Problems and
Perspectives in Management. -2025.-Vol.23(1). - P.63-75
\href{http://dx.doi.org/10.21511/ppm.23(1).2025.05}{DOI
10.21511/ppm.23(1).2025.05}

12. Birken, S.A.,~Currie, G.
\href{https://www.scopus.com/record/display.uri?eid=2-s2.0-85104625356&origin=reflist&sort=plf-f&src=s&sid=9294c12779ee47479a1a0997e8991897&sot=b&sdt=b&sl=99&s=TITLE\%28An+Examination+of+Civil+Servants\%e2\%80\%99+Assessment+of++the+New+Civil+Service+Reforms+in+Kazakhstan\%29}{Correction
to: Using organization theory to position middle-level managers as
agents of evidence-based practice implementation //Implementation
Science.- 2021.-Vol.16(1). -- P.1-6. DOI 10.1186/s13012-021-01106-2)}

13. Bertram I., Bouwman R., \& Tummers L. Getting what you expect: How
civil servant stereotypes affect citizen satisfaction and perceived
performance// Public Administration.-2024.- -Vol.102(4).-P.1468-1491.
\href{https://doi.org/10.1111/padm.12986}{DOI 10.1111/padm.12986}

14. Жаркешова А., Джунусбекова Г. Применение метода Organizational
Culture Assessment Instrument в диагностике организационной культуры в
государственной организации // Экономика и статистика.- 2017.- № 3. - С.
99-109.

15. Жақыпова Ф.Н., Ахметов А.А., Давлетбаева Ж.Ж., Рысбекова Ж.К., Жаров
Е.К. Мемлекеттік қызметшінің әдебі және мінез-құлық нормалары. -Астана,
2018. - 84 б.

16. Богатырев М.Р. Организационная культура: сущность и роль в системе
управления: дис. ... канд. экон. наук. - M., 2005. - 178 с.

17. Козиевская Е.В. Профессиональная самооценка в развитии мотивации
достижения государственных служащих: дис. ... канд. психол. наук:
19.00.13. -- М., 1998. -- 155 c.

18. Айтмұқаметова Қ.Ө. Мемлекеттік қызметкерлердің ресми іскерлік
қарым-қатынас мәдениеті // Мемлекеттік басқару және мемлекеттік қызмет.
- 2013.-11-15 бб. //
URI:~\href{http://localhost:8080/xmlui/handle/123456789/1020}{http://localhost:8080/xmlui/handle/123456789/1020}

19. Issenova G., Bokayev G., Nauryzbek M., Kosherbayeva A., Amirova A.
\href{https://www.scopus.com/record/display.uri?eid=2-s2.0-85208677871&origin=resultslist&sort=plf-f&src=s&sot=b&sdt=b&s=TITLE\%28An+Examination+of+Civil+Servants\%E2\%80\%99+Assessment+of++the+New+Civil+Service+Reforms+in+Kazakhstan\%29&sessionSearchId=9294c12779ee47479a1a0997e8991897&relpos=0}{\hfill\break
An Examination of Civil Servants'{} Assessment of the New
Civil Service Reforms in Kazakhstan}// Central European Journal of
Public Policy. -2024.-Vol.18(2).-P. 2--16 р. DOI 10.2478/cejpp-2024-006

20. Ehrenreich B. Patterns for college Writing (12th ed.). -- Boston:
Bedford/St. Martin's, 2007. -р. 680. ISBN: 978-0-312-67684-1

21. Gruenewald T.L., Kemeny M.E., Aziz N., Fahey J.L. Acute threat to
the social self: Shame, social self-esteem, and cortisol activity //
Psychosomatic Medicine.- 2004.- Vol.66(6).- P. 915-924. DOI
10.1097/01.psy.0000143639.6169.3.ef

22. Қазақстан Республикасы Қазақстан Республикасы мемлекеттік
қызметшілерінің әдептілік нормаларын және мінез-құлық қағидаларын одан
әрі жетілдіру жөніндегі шаралар туралы Қазақстан Республикасы
Президентінің 2015 ж. 29 желтоқсандағы Жарлығы {[}электрондық ресурс{]}
-- Айналым режимі https://adilet.zan.kz/kaz/docs/U1500000153 Жүгінген
күні: 01.12.2024

23. БҰҰДБ мен ҚР Мемлекеттік қызмет істері агенттігі мемлекеттік
қызметшілердің әдеп деңгейі бойынша зерттеу нәтижелерін ұсынды /
Қазақстан. - 2021 ж. {[}Электрондық ресурс{]} - Айналыс режимі:
https://www.undp.org/kk/kazakhstan/press-releases/buudb-men-kr-memlekettik-kyzmet-isteri-agenttigi-
memlekettik-kyzmetshilerdin-dep-dengeyi-boyynsha-zertteu-ntizhelerin
(жүгінген күні: 27.03.2024) Жүгінген күні:01.12.2024

{\bfseries References}

1. Қazaқstan Respublikasynyң Memlekettіk қyzmetі turaly: Қazaқstan
Respublikasynyң Zaңy 2015 zhylғy 23 қarashadaғy № 416-V ҚRZ. //
https://adilet.zan.kz/kaz/docs/Z1500000416 Zhүgіngen kүnі:
01.12.2024.{[}in Kazakh{]}

2.Sarsen Zh. Trudovye gosudarstvenyh
sluzhashhih//https://www.gov.kz/memleket/entities/

zem-shahtinsk/press/news/details/813166.Zhүgіngen kүnі: 01.12.2024.
{[}in Russian{]}

3.Memleket basshysy Қasym-Zhomart Toқaevtyң «Әdіlettі Қazaқstan: zaң men
tәrtіp, jekonomikalyқ өsіm, қoғamdyқ optimizm» atty Қazaқstan halқyna
Zholdauy 2024 zhylғy 2
қyrkүjek.//https://www.akorda.kz/kz/memleket-basshysy-kasym-zhomart-tokaevtyn-adiletti-kazakstan-zan-men-tartip-ekonomikalyk-osim-kogamdyk-optimizm-atty-kazakstan-halkyna-zholdauy-285659
Zhүgіngen kүnі: 01.12.2024.{[}in Kazakh{]}

4. Қazaқstan Respublikasynda Memlekettіk basқarudy damytudyң 2030 zhylғa
dejіngі tұzhyrymdamasyn bekіtu turaly: Қazaқstan Respublikasynyң
Prezidentі 2021zh.26 aқpandaғy № 522
Zharlyғy//https://adilet.zan.kz/kaz/docs/U150000015. Zhүgіngen
kүnі:01.12.2024.{[}in Kazakh{]}

5. Gschwantner, S.,~Hiebl, M.R.W.
\href{https://www.scopus.com/record/display.uri?eid=2-s2.0-84978811157&origin=reflist&sort=plf-f&src=s&sid=9294c12779ee47479a1a0997e8991897&sot=b&sdt=b&sl=99&s=TITLE\%28An+Examination+of+Civil+Servants\%e2\%80\%99+Assessment+of++the+New+Civil+Service+Reforms+in+Kazakhstan\%29}{Management
control systems and organizational ambidexterity}.//Journal of
Management Control.- 2016.- Vol.~27(4).-P.371-404.~

DOI 10.1007/s00187-016-0236-3

6. Қazaқstan Respublikasyndaғy memlekettіk қyzmettің zhaғdajy turaly
ұlttyқ bajandama
//https://www.gov.kz/memleket/entities/qyzmet/documents/details/454043?lang=kk

Zhүgіngen kүnі: 01.12.2024.{[}in Kazakh{]}

7. Ajaғanova A.Zh., Tұrғazy B.A «Memlekettіk қyzmetkerdің kәsіbi zhәne
psihologijalyқ mәdenietі» Psihologija zhәne sociologija serijasy.//
2021.- № 1(76). -104- 112.b.

DOI 10.26577/JPsS.2021.v76.i1.010.{[}in Kazakh{]}

8. Bajzhomartova Zh., Kadyrova M. Razvitie korporativnoj
kul' tury kak instrument povyshenija jeffektivnosti
dejatel' nosti gosudarstveenyh sluzhashhih // Memlekettіk
basқaru zhәne memlkettіk қazmet.// 2022. - №4 (83). - 13-23 bb. DOI
10.52123/1994-2370-2022-914. {[}in Russian{]}

9. Solomanidina T.O. Organizacionnaja kul' tura kompanii:
Ucheb. posobie. -- 2-e izd., pererab. i dop. -- M.: INFRA-M, 2015. -624
s. ISBN 978-5-16-003946-6. {[}in Russian{]}

10. Groshev I.V. Organizacionnaja kul' tura v sisteme
menedzhmenta sovremennogo rossijskogo predprijatija: dis. ... d-ra
jekon. nauk. - Tambov, 2007. - 493 s. {[}in Russian{]}

11. Barodi M., Lalaou S. Civil servants' readiness for ai adoption: the
role of change management in morocco's public sector//` Problems and
Perspectives in Management. -2025.-Vol.23(1). - P.63-75
\href{http://dx.doi.org/10.21511/ppm.23(1).2025.05}{DOI
10.21511/ppm.23(1).2025.05}

12. Birken, S.A.,~Currie, G.
\href{https://www.scopus.com/record/display.uri?eid=2-s2.0-85104625356&origin=reflist&sort=plf-f&src=s&sid=9294c12779ee47479a1a0997e8991897&sot=b&sdt=b&sl=99&s=TITLE\%28An+Examination+of+Civil+Servants\%e2\%80\%99+Assessment+of++the+New+Civil+Service+Reforms+in+Kazakhstan\%29}{Correction
to: Using organization theory to position middle-level managers as
agents of evidence-based practice implementation //Implementation
Science.- 2021.-Vol.16(1). -- P.1-6. DOI 10.1186/s13012-021-01106-2)}

13. Bertram I., Bouwman R., \& Tummers L. Getting what you expect: How
civil servant stereotypes affect citizen satisfaction and perceived
performance// Public Administration.-2024.- -Vol.102(4).-P.1468-1491.
\href{https://doi.org/10.1111/padm.12986}{DOI 10.1111/padm.12986}

14. Zharkeshova A., Dzhunusbekova G. Primenenie metoda Organizational
Culture Assessment Instrument v diagnostike organizacionnoj
kul' tury v gosudarstvennoj organizacii // Jekonomika i
statistika.- 2017.- № 3. - S. 99-109.{[}in Russian{]}

15. Zhaқypova F.N., Ahmetov A.A., Davletbaeva Zh.Zh., Rysbekova Zh.K.,
Zharov E.K. Memlekettіk қyzmetshіnің әdebі zhәne mіnez-құlyқ normalary.
-Astana, 2018. - 84 b. {[}in Kazakh{]}

16. Bogatyrev M.R. Organizacionnaja kul' tura:
sushhnost'{} i rol'{} v sisteme
upravlenija: dis. ... kand. jekon. nauk. - M., 2005. - 178 s. {[}in
Russian{]}

17. Kozievskaja E.V. Professional' naja samoocenka v
razvitii motivacii dostizhenija gosudarstvennyh sluzhashhih: dis. ...
kand. psihol. nauk: 19.00.13. -- M., 1998. 155 c. {[}in Russian{]}

18. Ajtmұқametova Қ.Ө. Memlekettіk қyzmetkerlerdің resmi іskerlіk
қarym-қatynas mәdenietі // Memlekettіk basқaru zhәne memlekettіk қyzmet.
- 2013.-11-15 bb. // URI:
\url{http://localhost:8080/xmlui/handle/123456789/1020}. {[}in Kazakh{]}

19. Issenova G., Bokayev G., Nauryzbek M., Kosherbayeva A., Amirova A.
\href{https://www.scopus.com/record/display.uri?eid=2-s2.0-85208677871&origin=resultslist&sort=plf-f&src=s&sot=b&sdt=b&s=TITLE\%28An+Examination+of+Civil+Servants\%E2\%80\%99+Assessment+of++the+New+Civil+Service+Reforms+in+Kazakhstan\%29&sessionSearchId=9294c12779ee47479a1a0997e8991897&relpos=0}{\hfill\break
An Examination of Civil Servants'{} Assessment of the New
Civil Service Reforms in Kazakhstan}// Central European Journal of
Public Policy. -2024.-Vol.18(2).-P. 2--16 р. DOI 10.2478/cejpp-2024-006

20. Ehrenreich B. Patterns for college Writing (12th ed.). -- Boston:
Bedford/St. Martin's, 2007. -р. 680. ISBN: 978-0-312-67684-1

21. Gruenewald T.L., Kemeny M.E., Aziz N., Fahey J.L. Acute threat to
the social self: Shame, social self-esteem, and cortisol activity //
Psychosomatic Medicine.- 2004.- Vol.66(6).- P. 915-924. DOI
10.1097/01.psy.0000143639.6169.3.ef

22. Қazaқstan Respublikasy Қazaқstan Respublikasy memlekettіk
қyzmetshіlerіnің әdeptіlіk normalaryn zhәne mіnez-құlyқ қaғidalaryn odan
әrі zhetіldіru zhөnіndegі sharalar turaly Қazaқstan Respublikasy
Prezidentіnің 2015 zh. 29 zheltoқsandaғy Zharlyғy {[}jelektrondyқ
resurs{]} -- Ajnalym rezhimі https://adilet.zan.kz/kaz/docs/U1500000153
Zhүgіngen kүnі: 01.12.2024. {[}in Kazakh{]}

23. BҰҰDB men ҚR Memlekettіk қyzmet іsterі agenttіgі memlekettіk
қyzmetshіlerdің әdep deңgejі bojynsha zertteu nәtizhelerіn ұsyndy /
Қazaқstan. - 2021 zh. {[}Jelektrondyқ resurs{]} - Ajnalys rezhimі:
https://www.undp.org/kk/kazakhstan/press-releases/buudb-men-kr-memlekettik-kyzmet-isteri-agenttigi-
memlekettik-kyzmetshilerdin-dep-dengeyi-boyynsha-zertteu-ntizhelerin
(zhүgіngen kүnі: 27.03.2024) Zhүgіngen kүnі:01.12.2024. {[}in Kazakh{]}

\emph{{\bfseries Авторлар туралы мәліметтер}}

Сериев Б. А.- заң ғылымдарының кандидаты, профессор, І.Жансүгіров
атындағы Жетісу университетінің Басқарма мүшесі - жастар саясаты
жөніндегі проректоры, Талдықорған қ., Қазақстан, e-mail:
\href{mailto:seriev_bolat@mail.ru}{\nolinkurl{seriev\_bolat@mail.ru}};

Жакупова Г.А.- оқытушы-дәріскер І.Жансүгіров атындағы Жетісу
университеті, Талдықорған қ., Қазақстан, e-mail: gulima88888888@mail.ru

\emph{{\bfseries Information about the authors}}

Seriyev B. A. - candidate of law, professor, Member of the Management
Board - Vice-Rector for Youth Policy of Zhetysu University named after
I. Zhansugurov, Taldykorgan, Kazakhstan, e-mail:
\href{mailto:seriev_bolat@mail.ru}{\nolinkurl{seriev\_bolat@mail.ru}};

Zhakupova G. A.lecturer, Zhetysu University named after I.
Zhansugurov,Taldykorgan, Kazakhstan, e-mail: gulima88888888@mail.ru
