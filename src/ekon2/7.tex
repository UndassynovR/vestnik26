\id{ҒТАМР 06.73.15}{}

{\bfseries ҚАЗАҚСТАНДАҒЫ ШАҒЫН ЖӘНЕ ОРТА КӘСІПКЕРЛІКТІ}

{\bfseries ДАМЫТУДЫ ЖАҚСАРТУ}

\begin{figure}[H]
	\centering
	\includegraphics[width=0.8\textwidth]{media/ekon2/image6}
	\caption*{}
\end{figure}

\begin{figure}[H]
	\centering
	\includegraphics[width=0.8\textwidth]{media/ekon2/image6}
	\caption*{}
\end{figure}

\begin{figure}[H]
	\centering
	\includegraphics[width=0.8\textwidth]{media/ekon2/image6}
	\caption*{}
\end{figure}

\begin{figure}[H]
	\centering
	\includegraphics[width=0.8\textwidth]{media/ekon2/image6}
	\caption*{}
\end{figure}

\begin{figure}[H]
	\centering
	\includegraphics[width=0.8\textwidth]{media/ekon2/image6}
	\caption*{}
\end{figure}


{\bfseries \textsuperscript{1,3}} \emph{Л.Н.Гумилев атындағы Еуразия Ұлттық
Университеті, Астана, Қазақстан,}

\emph{{\bfseries \textsuperscript{4}}С.Сейфуллин атындағы Қазақ
агротехникалық зерттеу университеті,} Астана, Қазақстан,

\emph{{\bfseries \textsuperscript{2,5}} Қ.Құлажанов атындағы Қазақ
технология және бизнес университеті,} Астана, Қазақстан

{\bfseries \textsuperscript{\envelope }}Корреспондент-автор:
\href{mailto:mtbb1986@gmail.com}{\nolinkurl{mtbb1986@gmail.com}}

Қазіргі уақытта мемлекеттің, оның аймақтарының және ұйымдарының қоғамдық
және кәсіпкерлік қызмет саласындағы экономикалық мүдделерін қорғаудың
әдістері мен тетіктерін қалыптастырудың жаңа тұжырымдамалық тәсілдерін
іздестіру жүргізілуде, ол Қазақстанның әлеуметтік-экономикалық даму
стратегиясында көрсетілген. Бұл мәселені шешу үшін кәсіпкерліктің
экономикалық мүдделерін жүзеге асыруға бағытталған аймақтық саясатқа
жаңа көзқарасты қолдану қажет.

Шағын және орта бизнес қазіргі өркениетті экономикалық дамудың ең
перспективалы бағыттарының бірі болып табылады. Нақты сектор, сауда
кәсіпорындары мен компаниялары Қазақстан экономикасының дамуы мен
өсуіне, ұлттық табыстың, жалпы ішкі өнімнің, жалпы ішкі өнімнің,
жұмыспен қамтудың және т.б. өсуіне тікелей әсер етеді.

Кәсіпкерлік секторды дамыту қоғамның саяси, экономикалық және әлеуметтік
тұрақтылығын арттырудың стратегиялық қажеттілігі болып табылады. Ол
барлық деңгейдегі бюджеттер үшін салық базасын ұлғайтуға, жұмыссыздықты
азайтуға, нарықты инвестициялық ресурстармен толтыруға көмектеседі.

Кәсіпкерлік экономиканың дамуына үлкен үлес қосады, өйткені осы
қызметтің арқасында ұлттық табыстың едәуір бөлігі құрылады, жаңа жұмыс
орындары пайда болады, техника мен технологиялар игеріледі, жаңа
өндірістер мен қызмет көрсетулер қалыптасады, жаңа аймақтар дамиды.
Бүгінде көпшілік өндірістің, нарықтың, демек, жалпы қоғамның дамуын
қозғайтын кәсіпкерлік екенін түсіне бастады. Ел кәсіпкерлердің
арқасында, кәсіпкерлер мемлекеттік қолдаудың арқасында өркендеп келеді.
Сондықтан Қазақстанда кәсіпкерлікті қолдау жүйесін дамыту өте маңызды.

{\bfseries Түйін сөздер:}аймақ, екінші деңгейлі банктер, инфрақұрылым,
кәсіпкерлік, Қазақстан, несие, субъект, шағын және орта кәсіпкерлік.

{\bfseries СОВЕРШЕНСТВОВАНИЕ РАЗВИТИЯ МАЛОГО И СРЕДНЕГО БИЗНЕСА В
КАЗАХСТАНЕ}

{\bfseries \textsuperscript{1} Т.Б.Мукушев\textsuperscript{\envelope },
\textsuperscript{2} Б.А. Жуматаева, \textsuperscript{3}А.Б.Амерханова,
\textsuperscript{4}А.Е.Баярлин, \textsuperscript{5}К.Д.Кожабергенова}

\emph{\textsuperscript{1,3} Евразийский национальный университет им. Л.
Н. Гумилёва, Астана, Казахстан,}

\emph{\textsuperscript{4}Казахский агротехнический
научно-исследовательский университет им.С.Сейфуллина, Астана,
Казахстан,}

\emph{\textsuperscript{2,5} Казахский университет технологии и бизнеса
им. К.Кулажанова, Астана, Казахстан,}

\emph{e-mail:\href{mailto:mtbb1986@gmail.com}{\nolinkurl{mtbb1986@gmail.com}}}

В настоящее время ведется поиск новых концептуальных подходов к
формированию методов и механизмов защиты экономических интересов
государства, его регионов и организаций в сфере общественной и
предпринимательской деятельности, что находит отражение в стратегии
социально-экономического развития Казахстана. Для решения этой проблемы
необходимо использовать новый подход к региональной политике,
направленный на реализацию экономических интересов предпринимательства.

Малый и средний бизнес или бизнес являются одним из наиболее
перспективных направлений современного цивилизованного экономического
развития. Реальный сектор, торговые предприятия и компании оказывают
непосредственное влияние на развитие и рост экономики Казахстана, рост
национального дохода, валового внутреннего продукта, валового
внутреннего продукта, занятости и многое другое.

Развитие сектора предпринимательства является стратегической
необходимостью повышения политической, экономической и социальной
стабильности общества. Оно способствует увеличению налогооблагаемой базы
для бюджетов всех уровней, снижению уровня безработицы, насыщению рынка
инвестиционными ресурсами.

Предпринимательство вносит большой вклад в развитие экономики, так как
благодаря этой деятельности создается значительная часть национального
дохода, появляются новые рабочие места, развивается техника и
технологии, формируются новые отрасли производства и услуг, происходит
освоение новых регионов. Сегодня многие стали понимать, что именно
предпринимательство двигает развитие производства, рынка, а значит, и
общества в целом. Страна процветает благодаря предпринимателям, а
предприниматели - благодаря поддержке государства. Поэтому очень важно
развивать систему поддержки предпринимательства в Казахстане.

{\bfseries Ключевые слова:} регион, банки второго уровня, инфраструктура,
предпринимательство, Казахстан, кредит, субъект, малые и средние
предприятия.

{\bfseries IMPROVEMENT OF SMALL AND MEDIUM BUSINESS DEVELOPMENT}

{\bfseries IN KAZAKHSTAN}

{\bfseries \textsuperscript{1} T.B. Mukushev\textsuperscript{\envelope },
\textsuperscript{2} B.A. Zhumataeva, \textsuperscript{3}A.B.
Amerkhanova, \textsuperscript{4}A.E. Bayarlin,}

{\bfseries \textsuperscript{5}K.D. Kozhabergenova}

\emph{\textsuperscript{1,3} L.N. Gumilyov Eurasian National University,
Astana, Kazakhstan,}

\emph{\textsuperscript{4}S.Seifullin Kazakh Agrotechnical Research
University, Astana, Kazakhstan,}

\emph{\textsuperscript{2,5} K.Kulazhan Kazakh University of Technology
and Business, Astana, Kazakhstan,}

\emph{e-mail:\href{mailto:mtbb1986@gmail.com}{\nolinkurl{mtbb1986@gmail.com}}}

Currently, there is a search for new conceptual approaches to the
formation of methods and mechanisms for protecting the economic
interests of the state, its regions and organizations in the sphere of
public and entrepreneurial activity, which is reflected in the strategy
of socio-economic development of Kazakhstan. To solve this problem, it
is necessary to use a new approach to regional policy aimed at
implementing the economic interests of entrepreneurship.

Small and medium businesses or business are one of the most promising
areas of modern civilized economic development. The real sector, trading
enterprises and companies have a direct impact on the development and
growth of the economy of Kazakhstan, the growth of national income,
gross domestic product, gross domestic product, employment and much
more.

The development of the entrepreneurship sector is a strategic need to
improve the political, economic and social stability of society. It
contributes to an increase in the tax base for budgets at all levels, a
decrease in unemployment, and saturation of the market with investment
resources.

Entrepreneurship makes a great contribution to the development of the
economy, since thanks to this activity a significant part of the
national income is created, new jobs appear, equipment and technologies
are developed, new industries and services are formed, new regions are
being developed. Today, many have come to understand that it is
entrepreneurship that drives the development of production, the market,
and therefore society as a whole. The country prospers thanks to
entrepreneurs, and entrepreneurs - thanks to state support. Therefore,
it is very important to develop a system of support for entrepreneurship
in Kazakhstan.

{\bfseries Keywords:} region, second-tier banks, infrastructure,
entrepreneurship, Kazakhstan, credit, entity, small and medium
enterprises.

{\bfseries Кіріспе.} Қазақстандағы шағын кәсіпкерліктің дамуын тежейтін
негізгі мәселе қаржылық қолдау жүйесінің, әсіресе инвестициялық несиелер
берудегі тиімділігінің жеткіліксіздігі болып табылады. Екінші деңгейлі
банктер, Шағын кәсіпкерлікті дамыту қоры, Еуропалық қайта құру және даму
банкінің несиелік желілері, Орталық Азия мен Американың қолдау қорлары
сияқты бірнеше әлеуетті қаржыландыру көздерінің болуына қарамастан, бұл
ресурстарға қолжетімділік әлі де шектеулі. Оның басты себебі шағын
кәсіпкерліктің үдемелі дамуын тежейтін осы қаражатты бөлу және пайдалану
механизмінің жетілдірілмегендігі.Бұл тұрғыда бірінші кезектегі міндет --
бөлінген қаражаттың мақсатты түрде мүлтіксіз пайдаланылуын қамтамасыз
ету. Бұл несиелер мен қаржылық көмектің тұрақты өсу үшін ұзақ мерзімді
инвестицияларды қажет ететін жобаларға бағытталғанын қамтамасыз ету үшін
нақты және ашық бақылау орнату қажеттілігін білдіреді. Дегенмен, бұл
процесс қаражаттың пайдаланылуын бақылауды және барлық деңгейде бақылау
инфрақұрылымын құруды қоса алғанда, неғұрлым жүйелі көзқарасты талап
етеді.Екінші негізгі аспект -- несиелік ресурстар бойынша пайыздық
мөлшерлемелерді қайта қарау. Өндірістік немесе инновациялық жобалар
сияқты қызметтің кейбір түрлері өте ұзақ уақытты және сәйкесінше
несиелеудің қолайлы шарттарын талап ететіндіктен, мөлшерлемелерді
саланың ерекшеліктеріне және аймақтық ерекшеліктерге байланысты саралау
маңызды. Бұл әсіресе Қазақстанның шалғай және ауылдық аймақтарына
қатысты, мұнда географиялық кедергілер мен дамыған инфрақұрылымның
жоқтығынан қаржыландыруға қолжетімділік одан да шектеулі.Ұзақ мерзімді
несие ресурстарының жеткіліксіздігі екінші деңгейлі банктердің шектеулі
мүмкіндіктерінде де маңызды мәселе болып табылады. Бұл шағын кәсіпкерлік
субъектілеріне берілген несиелердің шамамен 80-90\%-ы қысқа мерзімді
және ең алдымен айналым қаражатына немесе шаруашылық операцияларына
арналғанына әкеледі. Мұндай несиелер өндірістік процестерге ұзақ
мерзімді инвестициялауды жеңілдете алмайды, бұл сайып келгенде шағын
бизнестің өсу әлеуетін шектейді және оларды стратегиялық дамуға емес,
бизнесті жедел қолдауға бағытталған қысқа мерзімді қаржылық шешімдерге
тәуелді етеді.Сонымен қатар, шағын бизнесті қолдаудың ең күшті және жиі
бағаланбайтын шараларының бірі салық жүйесі болып табылады. Қазіргі
жағдайда Қазақстан шағын бизнеске таза фискалдық көзқарасты ұстанады,
бұл салықтың шамадан тыс жоғары ставкаларынан көрінеді. Мамандардың
айтуынша, шағын кәсіпкерлік субъектілеріне түсетін салық ауыртпалығы
олардың табысының 50-70 пайызын құрайды, бұл кәсіпкерлерді салықтан
жалтарудың немесе несиені қайтаруды кейінге қалдырудың жолдарын іздеуге
мәжбүр етеді. Бұл жоғары салықтар кәсіпкерлік белсенділікті тежейтін,
бизнестің өсу және өндірісті кеңейтуге немесе тиімділікті арттыруға
инвестиция салу мүмкіндігін шектейтін тұйық шеңберді
тудырады.Мемлекеттің кәсіпкерлікті тікелей қаржылық қолдауға
мүмкіндіктері шектеулі жағдайда салық жүйесі икемді және сараланған
болуы керек. Бұл қызмет түріне байланысты салық ставкаларын қайта
қарауды қамтиды. Атап айтқанда, өңдеуші өнеркәсіп кәсіпорындары қосымша
құнға, инновацияларға және жұмыспен қамтуға нақты үлес қосатындықтан,
неғұрлым қолайлы салық режимін пайдалануы керек. Бұл ретте коммерциялық
және делдалдық кәсіпорындарға салық жүктемесі артуы мүмкін, себебі
олардың қызметі экономикада ұзақ мерзімді өсуге қажетті құрылымдық
өзгерістерді жасамайды. АҚШ сияқты экономикасы дамыған елдерде
өндірістік кәсіпорындардың пайдасына салынатын салықтар 20-30\%
аралығында болса, коммерциялық және делдалдық операцияларға салынатын
салықтар 90-95\% жетуі мүмкін. Салық ауыртпалығын осылайша қайта бөлу
экономиканың тұрақтылығына шынайы әсер ететін және құрылымдық
проблемаларды еңсеруді қамтамасыз ететін экономика секторларын
ынталандырады.Осылайша, Қазақстандағы шағын бизнесті стратегиялық қолдау
қаржылық және салықтық қолдау жүйесін жан-жақты қарастыруға негізделуі
керек. Бұған несиелеу тетіктерін оңтайландыру, ұзақ мерзімді
инвестициялау үшін қолайлы жағдай жасау, өндірістік секторларға салық
жүктемесін азайту және коммерциялық және делдалдық операциялар бойынша
фискалдық саясатты қайта қарау кіреді. Осындай жағдайларда ғана шағын
кәсіпкерлік тұрақты экономикалық өсудің нақты қозғалтқышына айналып, ел
экономикасының тереңірек құрылымдық мәселелерін шешуге ықпал ете алады
{[}1{]}.

{\bfseries Материалдар мен әдістер.} Қазақстандағы шағын және орта
бизнестің дамуын жақсартуға бағытталған зерттеу барысында шетелдік және
отандық экономистердің еңбектеріне негізделген әртүрлі әдістер мен
тәсілдер қолданылды. Теориялық негіз ретінде экономикалық дамудағы
кәсіпкерліктің инновациялық рөліне назар аударған Джозеф Шумпетердің
және әлемдік экономикадағы қазақстандық ШОБ-тың бәсекеге қабілеттілігін
талдауға қолданылатын бәсекелестік артықшылықтарды зерттеген Майкл
Портердің жұмыстары алынды. Бұл зерттеулер негізгі инновациялар мен
бәсекеге қабілетті стратегиялардың ШОБ табысына қалай ықпал ететіні
туралы түсінік берді.Қазақстандық ғалымдардың ішінде ел экономикасындағы
кәсіпкерліктің рөлі мен мемлекеттік қолдаудың тиімділігін талдаған
А.Смайылов пен Ж.Әбдиевтің еңбектеріне ерекше көңіл бөлінді. Олардың
еңбектері қазақстандық бизнес-ортаның ерекшеліктерін және кәсіпкерлер
үшін бар кедергілерді тереңірек қарастыруға мүмкіндік берді.Зерттеу
салыстырмалы талдау әдістеріне негізделді, бұл қазақстандық ШОБ қолдау
жүйесін табысты шетелдік мысалдармен салыстыруға мүмкіндік берді.
Кәсіпкерліктің дамуына әртүрлі факторлардың әсерін болжау үшін де
экономикалық-математикалық модельдеу қолданылды. Бұл салық саясаты мен
мемлекеттік қолдаудағы өзгерістер ШОБ секторына қалай әсер ететінін
бағалауға мүмкіндік берді.Сонымен қатар, бизнес өкілдерімен және
кәсіпкерлік саласындағы сарапшылармен сұхбатты қамтитын сапалы талдау
қолданылды. Бұл сұхбаттар қазақстандық кәсіпкерлер тап болатын негізгі
проблемаларды анықтауға, сондай-ақ олардың мемлекеттік қолдаудан және
нарық конъюнктурасынан күтетінін түсінуге көмектесті.Эмпирикалық база
ретінде Қазақстан Республикасы Статистика комитетінің деректері, Ұлттық
экономика министрлігінің есептері, сондай-ақ Дүниежүзілік банктің «Doing
Business» сияқты халықаралық рейтингтері мен есептері пайдаланылды. Бұл
деректер әлемдік трендтерді ескере отырып, Қазақстандағы шағын және орта
бизнестің қазіргі жағдайы мен даму болашағын бағалауға мүмкіндік берді.

Осылайша, теориялық тәсілдер мен эмпирикалық талдаудың үйлесуі негізгі
проблемаларды анықтауға және Қазақстандағы шағын және орта бизнесті
қолдау мен дамытуды жақсарту жолдарын ұсынуға мүмкіндік берді.

{\bfseries Нәтижелер мен талдау.}Қазақстан Республикасының бюджеттік
стратегиясына өсудің елеулі әлеуеті мен экономикалық белсенділігі жоғары
өңірлерді кешенді қолдауға бағытталған жаңа мақсатты қаржы құралын --
Өңірлерді дамыту бағдарламасын интеграциялау ұсынылады. Бұл құрал
белгілі бір аумақтардың дамуына кедергі келтіретін жасырын құрылымдық
кедергілер мен теңгерімсіздіктерді анықтауға және жоюға бағытталған және
нәтижесінде осы өңірлердегі жедел экономикалық өсудің катализаторына
айналады. Жүйелі және ойластырылған тәсіл арқылы бағдарлама өсу
нүктелерін қолдау арқылы экономикаға мультипликативтік әсерді қамтамасыз
ететін жинақтаушы әсерді жасайды. Мұндай нәтижеге даму әлеуеті жоғары
өңірлер ұлттық деңгейде өсуді ынталандыра отырып, айтарлықтай қосымша
құн тудырып, өңіраралық байланыстарды нығайта алатындығына байланысты
қол жеткізуге болады.

Бағдарламаның негізгі бағыттары өңірлік дамуды күшейтетін және олардың
стратегиялық әлеуетін ашатын бастамаларды қаржылық қолдау арқылы ұзақ
мерзімді әлеуметтік-экономикалық проблемаларды жоюға бағытталған.
Бағдарлама сондай-ақ аймақтар арасындағы құрылымдық теңгерімсіздікті
барынша азайтуға, олардың әлеуметтік-экономикалық даму деңгейлеріндегі
алшақтықты азайтуға арналған. Экономикалық белсенділік сыртқы
факторларға ең аз тәуелділікпен ішкі дамуға бағытталуы мүмкін
аймақтардың өзін-өзі қамтамасыз етуіне жағдай жасауға ерекше назар
аударылады.Бағдарламаның маңызды аспектісі шағын және моноқалалардағы
шағын және орта бизнесті қолдау арқылы экономиканы әртараптандыру болып
табылады. Көбінесе бір немесе бірнеше салаға тәуелді бұл қалалар
стратегиялық маңызды «зәкірлік» инвестициялық жобаларды жүзеге асыру
арқылы трансформация процесіне тартылады. Бұл жобалар ұзақ мерзімді
тұрақты дамуға, аталған аймақтардың экономикалық осалдығын төмендетуге
және олардың экономикалық қызмет аясын кеңейтуге жағдай жасайды. Бұл
тұрғыда шағын және моноқалаларды кешенді аумақтық дамытудың берік
негізін құра отырып, экономикалық тәуелсіздіктің жаңа деңгейіне шығуға
мүмкіндік беретін индустрияландыру бағдарламаларының құралдары ерекше
орын алады {[}2{]}.

Жергілікті атқарушы органдар индустриялық-инновациялық қызметті
мемлекеттік қолдауға жауапты уәкілетті органмен бірлесіп әрбір шағын
немесе моноқалалар үшін 1-3 негізгі «зәкірлік» инвестициялық жобаларды
таңдайды. Бұл жобалардың стратегиялық маңызы бар және олардың бірегей
сипаттамалары мен әлеуетін ескере отырып, өңірлерде тұрақты экономикалық
өсуді ынталандыруға бағытталған.

Шағын және моноқалалардың экономикалық әлеуетін дамытудың негізгі
бағыттарына мыналар жатады:

- қосалқы және қызмет көрсету өндірісін орналастыру, сондай-ақ шағын
және моноқалалардағы ұлттық холдингтердің тапсырыстарын орындау. Бұл әр
қаланың ерекшелігін ескере отырып жасалады. «Самұрық-Қазына» Ұлттық
әл-ауқат қоры» АҚ және «НГК ҚазАгро» АҚ сияқты холдингтер осы аумақтарда
инвестициялық жобаларды жүзеге асыру бойынша шаралар қабылдауға
міндеттенеді. Бұл тек қаржылық-экономикалық орынды ғана емес, сонымен
қатар шаруашылық белсенділікті арттыруға және экономиканы
әртараптандыруға көмектесетін холдингтердің өндірістік ерекшеліктері мен
стратегиялық бағытын ескереді.

- қала құрушы кәсіпорындардан тапсырыстар мен қосалқы өндірістерді
орналастыру. Шағын және моноқалалардың ерекшеліктерін ескере отырып, бұл
кәсіпорындар жергілікті атқарушы органдармен бірлесе отырып,
экономиканың мамандандырылған секторларын дамытуға бағытталған кем
дегенде бір инвестициялық жобаны іске асыру бойынша жұмыс жүргізуде.
Мақсат -- не бар өндіріс профильдерін жаңғырту, не стратегиялық
инвесторды тарту арқылы бұрынғы экономикалық мамандандыруды қалпына
келтіру. Бұл тәсіл жергілікті экономикалық кластерлерді нығайтуға ғана
емес, бұрын жетекші болған салалардың әлеуетін қалпына келтіруге де
көмектеседі.

Стратегиялық инвесторлар ағынын ынталандыру үшін
индустриялық-инновациялық қызмет саласындағы уәкілетті орган келісетін
мемлекеттік қолдау шараларының кешені көзделген. Инвестицияларды
тартудың ұлттық жоспары шеңберінде инвесторлар үшін қолайлы жағдайлар
жасауға, оның ішінде бизнесті жүргізудегі кедергілерді азайтуға және
қаржылық ынталандыруға бағытталған мемлекеттік қолдау тетіктері
қолданылатын болады.Сонымен қатар, шағын және моноқалалардағы
стратегиялық инвестициялық жобаларды іске асыруға қатысатын заңды
тұлғаларға бірқатар өндірістік жеңілдіктерге қол жеткізуге мүмкіндік
беріледі. Бұл жеңілдіктерге газ, электр энергиясы, жер телімін алу,
ғимараттар мен құрылыстарды салу немесе сатып алу шығындарын өтеу немесе
ішінара өтеу кіреді. Мұндай ынталандыру инвесторлардың шығындарын
азайтып, шағын қалалардағы жобалардың тартымдылығын арттырады, бұл
олардың экономикалық дамуы мен әртараптандыруын жеделдетеді {[}3{]}.
Шағын және орта бизнесті (ШОБ) қолдау стратегиялық тұрғыдан Қазақстан
экономикасының барлық секторларының бәсекеге қабілеттілігін арттыруға
бағытталуы тиіс. Нарықтық экономикасы дамыған елдердегідей, шағын және
орта бизнесті мемлекеттік қолдау кеңейтілген несиелік мүмкіндіктерді,
тәуекелдерді сақтандыруды және жергілікті бизнеске бағытталған
инвестициялық ынталандыруды қамтитын ынталандыруды қамтамасыз ететін көп
деңгейлі және жан-жақты болуы керек. Осы бағыттардың бірі экономикалық
өсу үшін қосымша ресурстарды құруға мүмкіндік беретін ШОБ секторындағы
инвестициялық жобаларды қаржыландыру үшін мемлекеттік және жеке
зейнетақы қорларының қаражатын пайдалану болуы мүмкін.

Осылайша, кәсіпкерлікті қолдау жүйесі келесі негізгі элементтерден тұруы
керек:

- құқықтық қорғауды және қызметтердің қолжетімділігін қамтамасыз ететін
заң көмегі;

- бизнестің тұрақты дамуына ықпал ететін қаржылық қауіпсіздік;

- кәсіпкерлерге жеңілдіктер мен жеңілдіктер беретін салықтық қолдау;

- ШОБ дамыту үшін қолайлы жағдайлар жасауға бағытталған кадрлық және
инфрақұрылымдық қолдау {[}4{]}.

Кәсіпкерлікті дамытудағы салық механизмінің рөлі кез келген ұлттық
экономикалық стратегияның негізгі құрамдас бөліктерінің бірі болып
табылады. Қазақстан Республикасының жағдайында салықтық реттеудің
қолданыстағы құралдары шағын және орта бизнесті, әсіресе өңдеуші
секторды қолдау және ынталандыру үшін жеткілікті түрде тиімді
пайдаланылмай отырғанын сеніммен айтуға болады. Бұл мәселе бизнесті
реттеудегі тереңірек жүйелік тұрақсыздықтың белгісі болып табылады.
Сондықтан салық заңнамасының тұрақтылығы мәселесі ерекше маңызға ие,
өйткені фискалдық саясаттағы тұрақты өзгерістер бизнестің ұзақ мерзімді
перспективада жоспарлау және даму мүмкіндігіне теріс әсер етеді {[}5{]}.
Салық заңнамасының тұрақсыздығы кәсіпкерлер үшін маңызды қауіптердің
бірі болып табылады. Салық ережелері үнемі өзгеріп отыратындықтан,
бизнес жаңа жағдайларға бейімделу қабілетіне кедергі келтіретін
белгісіздік жағдайына тап болады. Әрбір жаңа заң есептеулерді түбегейлі
өзгертіп, күтілетін пайданы шығынға айналдыра алатын болса, кәсіпкерлер
өз ресурстары мен стратегияларын тиімді басқара алмайды. Салық
реттеуіндегі белгісіздік тағы бір жағымсыз әсерді тудырады: көлеңкелі
экономиканың күшеюі. Көптеген кәсіпкерлер өз кірістерін сақтап қалуға
тырысып, пайданың нақты көлемін төмендете бастайды немесе кірісті
толығымен жасырады. Бұл мінез-құлық мемлекеттің фискалдық тұрақтылығына
нұқсан келтіріп қана қоймайды, сонымен қатар бизнес пен мемлекет
арасындағы сенімді азайтады. Бұған жол бермеу үшін бизнеске салық
тәуекелдерін барынша азайтуға емес, олардың дамуына назар аударуға
мүмкіндік беретін нақты және болжамды ойын ережелерін ұсына отырып,
салық жүйесінің тұрақтылығын қамтамасыз ету қажет. {[}6{]}.

Қазір Қазақстанның алдында, әсіресе, жастар арасында кәсіпкерліктің
дамуына кедергі келтіретін тереңірек құрылымдық міндеттер тұр. Сондай
өзекті мәселелердің бірі -- өскелең ұрпақтың бойында күшті кәсіпкерлік
рухтың болмауы. Қазақстанда кәсіпкерлік әлі де болса тәуекелдің жоғары
деңгейімен, белгісіздікпен және көптеген қиындықтармен байланысты қызмет
ретінде қабылданады. Жастар көбінесе оны тартымды мансап жолы ретінде
қарастырмайды, ірі бизнестегі немесе мемлекеттік қызметтегі тұрақты
лауазымдарды қалайды. Бұл елдегі кәсіпкерлік мәдениетінің дамудың
бастапқы кезеңінде екенін және мақсатты түрде ынталандыруды қажет
ететінін көрсетеді.

Екінші маңызды аспект -- кәсіпкерлікті әлеуметтік құбылыс ретінде
қабылдау. Бүгінгі таңда кәсіпкерлер жастардың кәсіби қалауы жүйесінде
жетекші орындарды иеленбейді. Бизнес, көптеген жастардың пікірінше,
табысқа жету мүмкіндігімен байланысты емес, керісінше өмір сүру үшін
тұрақты күреспен байланысты. Бұл жалпы түсінік жастардың өз ісін ашу
және дамыту жауапкершілігін алуға дайындығын шектейді, бұл елдің жалпы
кәсіпкерлік әлеуетін әлсіретеді.

Кәсіпкерлік көзқарастар мен дағдыларды дамытуда білім беру жүйесі де
маңызды рөл атқарады. Қазақстанда білім беру мекемелері әлі күнге дейін
тәуекелге бару, жаңашыл ойлау және басқару дағдылары сияқты мінез-құлық
құзыреттерін дамытпай, тек негізгі экономикалық білімді ұсынады.
Нәтижесінде оқу орындарының түлектері бизнесті табысты жүргізуге қажетті
білім мен дағдыларды ала алмайды. Білім беру жүйесі шығармашылық
ойлауды, кәсіпкерлік бастамаларды және экономикалық ортаның
өзгерістеріне бейімделу қабілетін дамытуға ықпал ететіндей етіп
жаңғыртылуы керек. {[}7{]}.

Әкімшілік-құқықтық кедергілер де өз ісін ашуға ұмтылған жастар үшін
айтарлықтай кедергі болып қала береді. Мемлекеттің кәсіпкерлікті
қолдауға бағытталған күш-жігеріне қарамастан, бюрократиялық кедергілер,
бизнес ашудың жоғары шығындары және қаржыландырудың жеткіліксіздігі
салдарынан жастар үшін нарыққа қол жеткізу қиын болып отыр. Жастарға
бағытталған шағын бизнесті қолдау шаралары әлі нақты нәтиже берген жоқ,
бұл қолданыстағы бағдарламаларды қайта қарау және кедергілерді жою
қажеттігін көрсетеді.

Қазақстанда кәсіпкерліктің жан-жақты дамуы үшін салық саясатын
тұрақтандыру, құқықтық және әкімшілік кедергілерді жою, білім беру
жүйесін жаңғырту және жастар арасында кәсіпкерлік мәдениетті белсенді
түрде насихаттауды қамтитын кешенді тәсіл қажет. Бар проблемаларды жою
және бизнес үшін қолайлы жағдайлар жасау арқылы ғана кәсіпкерлік
белсенділіктің өсуін ынталандыруға болады, бұл өз кезегінде экономикада
ұзақ мерзімді құрылымдық өзгерістерге әкеліп, елдің тұрақты дамуын
қамтамасыз етеді {[}8{]}.

Қазақстанның заманауи экономикасында шағын және орта бизнестің тиімді
жұмыс істеуі және дамуы үшін осы саланы мемлекеттік қолдауды түбегейлі
күшейту қажет. Ең алдымен, бірнеше жүйелі қадамдарды жүзеге асыру қажет,
олардың әрқайсысы бар кедергілерді жоюға және тұрақты өсу үшін қолайлы
жағдайлар жасауға бағытталуы тиіс:

- стратегиялық тұжырымдаманы қалыптастыру және басымдықтарды анықтау.
Мемлекет шағын бизнестің ел экономикасының жалпы құрылымындағы нақты
рөлін айқындап, оның басым бағыттарын айқындап, негізгі өсу нүктелерін
белгілеуі керек. Бұл ресурстар мен назарды экономиканың ең перспективалы
және стратегиялық маңызды секторларына шоғырландыруға мүмкіндік береді,
мұнда шағын бизнес экономиканы әртараптандыру мен тұрақтылығына елеулі
үлес қоса алады;

- шағын және орта бизнесті қолдаудың ұзақ мерзімді мемлекеттік саясатын
әзірлеу. Мұндағы ең маңызды элемент шағын кәсіпкерлікті кешенді қолдауға
бағытталған кешенді мемлекеттік бағдарламаны құру болып табылады --
басынан бастап өсу және масштабтау кезеңдері. Бұл бағдарлама тікелей
қаржыландыру шараларын да, кәсіпкерлікті қолдаудың инфрақұрылымын құруды
да, оның ішінде салықтық және инвестициялық ынталандыруды қамтуы тиіс;

- құқықтық және нормативтік базаны жетілдіру. Нақты және болжамды
ережелер болмаса, бизнестің дамуы шектеледі. Мемлекетке бизнесті
құқықтық қамтамасыз етуді жетілдіру, әкімшілік рәсімдерді жеңілдету
қажет, бұл нарыққа кірудегі кедергілерді азайтады және нормативтік
талаптарды сақтау үшін бизнес шығындарын азайтады. Бұл бизнесті тіркеуге
де, бухгалтерлік есеп пен есеп беруге де қатысты;

- қолайлы инвестициялық климат құру. Шағын және орта бизнеске инвестиция
тарту -- маңызды міндеттердің бірі. Ол үшін инвесторлардың құқықтарын
қорғауды қамтамасыз ететін заңнаманы жетілдіру, сондай-ақ ісін жаңа
бастаған кәсіпкерлер мен шағын бизнеске инвестиция салуды ынталандыруды
дамыту қажет. Инвестициялық саясат инвесторлар үшін тәуекелдерді
азайтуға және капиталдың ірі компанияларға ғана емес, сонымен қатар
шағын бизнеске де келіп, олардың дамуына ықпал ететін жағдай жасауға
бағытталуы тиіс;

- салық жүйесін реформалау. Шағын және орта кәсіпкерлікті басып-жаншудың
орнына, оларды қолдау үшін салық жүйесін түбегейлі қайта қарау керек.
Салық ставкаларын оңтайландыру және оларды, әсіресе, өндірістік немесе
инновациялық қызметпен айналысатын кәсіпорындар үшін икемді ету қажет.
Бұл экономикалық өсудің негізгі шарты болып табылатын кеңейтілген
өндіріс пен инновацияға инвестицияны ынталандырады;

- несие-қаржылық механизмдердің дамуы. Қазақстандағы шағын және орта
бизнес ұзақ мерзімді несиелерді, микроқаржыландыруды, венчурлық
капиталды инвестициялауды және жаңа қаржылық технологияларды қоса
алғанда, қолжетімді және әртүрлі қаржылық құралдарды қажет етеді. Бұл
кәсіпкерлерге олардың қызметінің ерекшеліктеріне бейімделген
қаржыландырудың икемді шарттарын ұсына алатын несиелік-қаржылық
инфрақұрылымды дамытуды және кеңейтуді талап етеді;

- тікелей және жанама қаржыландырудағы мемлекеттің рөлін белсендіру.
Мемлекет пайыздық мөлшерлемелерді субсидиялау, несиелер бойынша
мемлекеттік кепілдіктер беру, болашағы бар жобалардың капиталына қатысу
арқылы шағын бизнесті қолдауда белсендірек рөл атқаруы керек. Тікелей
қаржыландыру икемді несие шарттарымен үйлескенде бизнес тәуекелдерін
азайтып, ұзақ мерзімді негізде кәсіпорындардың тұрақты дамуын қамтамасыз
етеді {[}9{]}. Дамытудың негізгі бағыттарын, іс-шараларын, күтілетін
нәтижелерді және мүдделі тараптарды 1-кестеде көрсетейік.

{\bfseries 1-кесте. Дамытудың негізгі бағыттары, іс-шаралар, күтілетін
нәтижелер және}

{\bfseries мүдделі тараптар}

% \begin{longtable}[]{@{}
%   >{\raggedright\arraybackslash}p{(\columnwidth - 6\tabcolsep) * \real{0.2306}}
%   >{\raggedright\arraybackslash}p{(\columnwidth - 6\tabcolsep) * \real{0.2562}}
%   >{\raggedright\arraybackslash}p{(\columnwidth - 6\tabcolsep) * \real{0.2665}}
%   >{\raggedright\arraybackslash}p{(\columnwidth - 6\tabcolsep) * \real{0.2467}}@{}}
% \toprule\noalign{}
% \begin{minipage}[b]{\linewidth}\raggedright
% {\bfseries Дамыту бағыттары}
% \end{minipage} & \begin{minipage}[b]{\linewidth}\raggedright
% {\bfseries Іс-шаралар}
% \end{minipage} & \begin{minipage}[b]{\linewidth}\raggedright
% {\bfseries Ожидаемые результаты}
% \end{minipage} & \begin{minipage}[b]{\linewidth}\raggedright
% {\bfseries Заинтересованные стороны}
% \end{minipage} \\
% \midrule\noalign{}
% \endhead
% \bottomrule\noalign{}
% \endlastfoot
% \multirow{2}{=}{{\bfseries Қаржыландыруға қолжетімділікті жақсарту}} & -
% ШОБ үшін жеңілдікті несиелеу бағдарламаларын құру & - Шағын және орта
% кәсіпкерлік үшін қаржылық ресурстардың қолжетімділігін арттыру & -
% Қазақстанның даму банкі, коммерциялық банктер, үкімет \\
% & - Венчурлық қорларды дамыту және стартаптарға инвестициялау & -
% Инновациялық жобалар мен стартаптар санының артуы & - Инвесторлар, қаржы
% институттары, стартап-инкубаторлар \\
% \multirow{2}{=}{{\bfseries Салық саясатын жетілдіру}} & - ШОБ үшін салықтық
% жеңілдіктерді енгізу & - Салық жүктемесін азайту, шағын кәсіпкерліктің
% өсуін ынталандыру & - Салық органдары, шағын және орта кәсіпкерлер \\
% & - Салық есептілігін оңтайландыру & - Іскерлік процедураларды жеңілдету
% & - Қаржы министрлігі, кәсіпкерлер \\
% \multirow{2}{=}{{\bfseries ШОБ қолдау инфрақұрылымын дамыту}} & -
% Бизнес-инкубаторлар мен акселераторларды құру & - Ресурстарға қол
% жеткізу және тәлімгерлік арқылы шағын бизнестің өсуін жеделдету & -
% Акселераторлар, кәсіпкерлер қауымдастығы \\
% & - Арнайы экономикалық аймақтарды (АЭА) дамыту & - Шетелдік және
% жергілікті инвестицияларды тарту, өндіріс қуаттылығын арттыру & -
% Үкімет, инвесторлар, кәсіпкерлер \\
% \multirow{3}{=}{{\bfseries Иновация және цифрландыру}} & -
% Бизнес-процестерге цифрлық технологияларды енгізуді қолдау & - ШОБ
% субъектілерінің бәсекеге қабілеттілігін және өнімділігін арттыру & -
% IT-компаниялар, стартаптар, кәсіпкерлер, мемлекеттік бағдарламалар \\
% & - Әкімшілік процестерді цифрландыру бағдарламалары & - Құжат айналымын
% және мемлекеттік органдармен өзара іс-қимылды жеңілдету & - Үкімет,
% бизнес қауымдастығы \\
% & - Халықаралық нарықтарға шығу процедураларын жеңілдету & - Әлемдік
% нарықта сату географиясы мен шағын және орта бизнес мүмкіндіктерін
% кеңейту & - Министрліктер, экспорттық компаниялар \\
% \end{longtable}

Төмендегі 1-кестеде қаржыға қолжетімділікті, салық саясатын, білім
беруді, инфрақұрылымды қолдауды, цифрландыруды, заңнаманы және
экспорттық мүмкіндіктерді қоса алғанда, Қазақстандағы ШОБ жақсартудың
негізгі аспектілері қамтылған.Шағын және орта бизнесті қолдаудың кешенді
жүйесін құру, оның ішінде тек қаржылық ғана емес, сонымен қатар заңдық,
әкімшілік және салықтық құралдармен де инвестиция тарту және осы саланың
жедел дамуы үшін жағдай жасай алады. Салыстырмалы түрде тәуекелсіз
қаржыландыруды, тартымды салықтық шарттарды және болжамды реттеуші
ортаны ескере отырып, капитал міндетті түрде шағын және орта бизнеске
ағылады. Бұл саланың өзін нығайтып қана қоймай, жалпы ұлттық
экономиканың тиімділігін арттырып, тұрақты және әртараптандырылған өсім
үшін жағдай жасайды.

Кәсіпкерліктің дамуына әсер ететін әлеуметтік-экономикалық факторларға
монополияландыру мен жекешелендірудің төмен қарқыны жатады, бұл
бәсекелестік үшін кеңістік жасайды және шағын бизнеске нарықта тұрақты
тауашаларды иеленуге мүмкіндік береді. Дегенмен, нарықтық
инфрақұрылымның дамымағандығы, қаржыландыру көздерінің шектеулігі және
коммерциялық тәуекелдерді сақтандыру жүйесінің жоқтығы одан әрі өсуге
айтарлықтай кедергі болып қала береді. Бизнесті жүргізуге байланысты
тәуекелдерді тиімді сақтандыру мүмкін еместігі кәсіпкерлердің ұзақ
мерзімді инвестициялық шешімдер қабылдауын шектейді.Саяси факторлар
бизнестің дамуына бірдей маңызды әсер етеді. Мемлекеттік қолдаудың
жеткіліксіздігі мен бюрократиялық кедергілер кәсіпкерлерге үлкен
қиындықтар туғызуда. Бюрократиялық аппарат көбінесе бизнеске қосымша
кедергілер қойып, мемлекеттік қолдау бағдарламаларына, несиелер мен
субсидияларға қол жеткізуді қиындатады. Бұл проблемаларды шешу әкімшілік
кедергілерді азайтуға және неғұрлым икемді және қолайлы бизнес-ортаны
құруға бағытталған мемлекеттік реттеудің қолданыстағы жүйесін қайта
қарауды талап етеді {[}10{]}.

Шағын және орта бизнестің табысты дамуы мемлекеттік және экономикалық
саясат деңгейінде жүйелі өзгерістерді қажет етеді. Бизнеске қолайлы
жағдай жасауға бағытталған кешенді шаралар ғана кәсіпкерлік
белсенділіктің ұзақ мерзімді тұрақтылығын қамтамасыз етіп, ұлттық
экономиканың өсуіне ықпал ете алады.

Қоғамдық пікірді зерттеуге сәйкес, Қазақстандағы кәсіпкерлердің басым
бөлігі республикалық министрліктер мен жергілікті билік кәсіпкерлікті
дамытуға бағытталған реформаларды қолдайтынына сенімді. Алайда, бұл
реформаларды жүзеге асыруға кедергі келтіретін негізгі факторлар салық
жүйесіндегі күрделі проблемалар мен нормативтік-құқықтық базаның
жетілмегендігі болып қала береді. Заңнамадағы жиі өзгерістер бизнес үшін
белгісіздік пен тұрақсыздық атмосферасын тудырады, бұл кәсіпкерлерді
стратегиялық және ұзақ мерзімді жоспарлауға емес, қысқа мерзімді
табыстарға назар аударуға мәжбүр етеді. Бұл бизнестің тұрақты дамуының
негізін бұзады және жылдам пайда табуға бағытталған түнде ұшу схемаларын
әзірлеуді ынталандырады. Бұл міндеттер кәсіпкерлікті мемлекеттік
қолдауды күшейту қажеттігін көрсетеді. Қазақстанның қазіргі
экономикасында шағын және орта бизнес нарықтық институттармен және
тиісті инфрақұрылыммен тығыз өзара іс-қимылсыз тиімді жұмыс істей
алмайды. Бұл тұрғыда еліміздің алдында жүйелі шешімді қажет ететін
бірнеше негізгі міндеттер тұр.Біріншіден, әміршіл-әміршіл жүйенің
күйреуімен сипатталатын посткеңестік кезеңде биліктің барлық деңгейінде
кәсіпкерлерді аппараттық сыбайлас жемқорлықтың өктемдігінен қорғау өте
маңызды. Сыбайлас жемқорлық әрекеттер мемлекеттік ресурстар мен қолдауға
қолжетімділікті шектей отырып, бизнес үшін тең емес жағдай туғызады.
Кәсіпкерлер бәсекелестік пен әділетті ортада дамуы үшін озбырлық пен
сыбайлас жемқорлық қысымынан қорғауды қажет етеді.Екіншіден, кәсіпкерлік
белсенділікті қолдайтын инфрақұрылымды жаңғырту өте маңызды. Бұл
коммерциялық банктерді, сақтандыру және айырбас институттарын құру мен
дамытуға да, кәсіби қызмет көрсету секторын -- аудиторлық және
консалтингтік компанияларды дамытуға да қатысты. Мұндай институттар
бизнесті қажетті қаржылық және ақпараттық ресурстармен қамтамасыз ете
отырып, оның тиімді жұмыс істеуі үшін өмірлік маңызды негіз
жасайды.Үшіншіден, бәсекені ынталандыратын экономикалық жағдай жасау
қажет. Салауатты бәсеке болмаса, кәсіпкерлер инновациялық шешімдерді
іздеуге және өз өнімдері мен қызметтерін жақсартуға ынталы емес.
Бәсекелестік экономиканы жаңғырту мен жаңа нарықтарды құрудың драйвері
ретінде қызмет етеді, бұл әсіресе шағын және орта бизнестің өсуі үшін
маңызды. Дегенмен, мұндай бәсекелестік әділетті ойын ережелеріне
негізделуі керек, мұнда табыс әкімшілік ресурстарды пайдалануға емес,
басқару сапасына және инновациялық шешімдерге байланысты.

Төртіншіден, өркениетті бизнес-ортаны құруда мемлекет белсенді рөл
атқаруы тиіс. Бұл кәсіпкерлерді құқықтық және экономикалық ынталандыруды
ғана емес, сонымен қатар тұрақты бизнес-ортаны қамтамасыз ететін ашық
және болжамды ережелерді енгізуді де қамтиды. Мемлекет дамуға қосымша
кедергілер тудырмай, инфрақұрылымдық, құқықтық және қаржылық қолдау
көрсететін іскер серіктес болуы керек.

Осы аспектілерді талдай отырып, кәсіпкерлік Қазақстанның ұлттық
экономикасының дамуында шешуші рөл атқарады деп айтуға болады. Ол жұмыс
орындарын ашып, салық түсімдерін көбейтіп қана қоймайды, сонымен қатар
өсуді қамтамасыз ететін стратегиялық секторларда жұмыс істейді.
Кәсіпкерлік ұлттық байлық пен жалпы ішкі өнімді жасауға, елдің
экономикалық тұрақтылығын нығайтуға маңызды үлес қосуда {[}11{]}.

{\bfseries Қорытынды.}Алдағы жылдары Қазақстандағы шағын және орта бизнесті
дамытудың негізгі векторларына келесі стратегиялық бағыттар айналады.
Біріншіден, жаңа секторларды белсенді дамытуды және шикізаттық салаларға
тәуелділікті азайтуды көздейтін экономиканы жеделдетілген әртараптандыру
арқылы капиталдың тұрақты өсуін қамтамасыз етуге баса назар аударылатын
болады. Бизнес ортаны жақсарту және жеңілдетілген реттеу рәсімдері
арқылы кәсіпкерлікті қолдау, сондай-ақ сауданы ынталандыру бизнес үшін
неғұрлым қолайлы жағдай туғызады.Өңірлердегі экономикалық белсенділікті
айтарлықтай арттыра алатын маңызды сала ретінде туризмді дамытуға ерекше
көңіл бөлінеді. Бұдан басқа, нәтижелі жұмыспен қамтуға жәрдемдесу және
жұмыссыздық пен әлеуметтік қауіпсіздікті төмендетуге көмектесетін
әлеуметтік қорғаудың тиімділігін арттыру негізгі міндеттер болмақ.ШОБ-ты
дамытуда елордада кәсіпкерлік үшін инфрақұрылымды құру және нығайту
маңызды рөл атқаратын болады, бұл мемлекеттік қызмет көрсету сапасының
және тіршілікті қамтамасыз ету жүйелерінің сенімділігінің жоғарылауымен
бірге жүреді. Бұған сонымен қатар тұрғын үй құрылысындағы процестерді
жетілдіру және экономиканың тұрақты жұмыс істеуі үшін маңызды болып
табылатын энергетикалық инфрақұрылымның тиімді жұмыс істеуін қамтамасыз
ету кіреді.Сонымен қатар, елдің макроөңірлерінің неғұрлым тиімді
орналасуына ықпал ететін экономикалық әлеуеттің ұтымды аумақтық ұйымын
қалыптастыруға баса назар аударылатын болады. Урбанизация жағдайында
экологиялық тепе-теңдікті сақтау үшін жағдай жасауды талап ететін
қалалық экожүйені сақтау және қалпына келтіру маңызды міндет
болады.Бизнес пен үкіметтің өзара іс-қимылын жеңілдетуге және
реформаларды жүзеге асыруды жеделдетуге көмектесетін мемлекеттік
органдар қызметінің тиімділігін арттыру да бірдей маңызды бағыт болып
табылады. Бірлескен бұл шаралар Қазақстанда шағын және орта бизнестің
қарқынды дамуын қамтамасыз етуге қабілетті кешенді және тұрақты жүйені
құруға бағытталған.

{\bfseries Әдебиеттер}

1. Мочалова Л.А. Стратегический анализ и планирование: учебник / Л.А.
Мочалова, В.И. Власов; Уральский государственный горный университет. -
2-е изд. - Москва : Ай Пи Ар Медиа , 2024. - 167 с. ISBN
978-5-4497-1853-2.

2. Елшибаев Р.К. Современное состояние и направления развития малого и
среднего бизнеса Республики Казахстан//~Вестник университета «Туран».-
2021.-№1.- С.84-90.~

DOI 10.46914/1562-2959-2021-1-1-84-90

3. Как в Казахстане поддерживают малый и средний бизнес во время
пандемии?
//https://www.inform.kz/ru/kak-v-kazahstane-podderzhivayut-malyy-i-sredniy-biznes-vo-vremya-pandemii\_a3694302.Дата
обращения: 08.07.2024 г.

4. Дюсембаева Л.К. Роль малого предпринимательства в экономическом
развитии РК// International journal of Professional Science. - 2020.-
№10.- С.16-21.

5. Дингaзиевa Б.Д., Нурмуханбетова Д.Е., Бaйбaтыровa Г.Т. Развитие и
государственная поддержка малого бизнеса в Казахстане // Universum:
Технические науки: электрон. научн. журн. 2020.- № 3.1(72.1) - С.15-20.

6. Дюсембаева Л.К. Инфраструктура поддержки и развития экономического
роста Казахстана// Материалы Республиканской научно-теоретической
конференции «Сейфуллинские чтения-14: Молодежь, наука, инновации:
цифровизация -- новый этап развития». - 2018. - №.1. - С.148-151.
\url{https://kazatu.edu.kz/assets/i/science/} sf14\_mat\_104.pdf

7. Кошимова М.А. Экономика малого и среднего бизнеса: учебное пособие /
М.А. Кошимова, А.И. Естурлиева. - Алматы: Экономика, 2016. - 437с. -
ISBN 978-601-225-853-0.

8. Арыстанова Н.К., Протасова О.В. Становление и развитие малого бизнеса
в Республике Казахстан // Инновационная экономика: перспективы развития
и совершенствования, 2019.- №7 (41).-С.5-12.

9. Жаксылыкова А.А. Проблемы в развитии малого и среднего бизнеса
Казахстана и пути их решения //~Вестник университета «Туран». - 2019.-№
1-С.155-161.

10. Трансформация экономики Казахстана / Т. Хельм, Н. Шольц, Р.
Ошакбаев, Б. Уакпаев и др. - Астана: Фонд имени Конрада Аденауэра, 2017.
- 209 c. - ISBN 978-601-06-4166-2.

11. Мырзахметова А.М., Мухаметжан А.Е., Жакупова С.Т. Особенности и
проблемы развития малого и среднего предпринимательства Казахстана в
современных условиях//Вестник КазНУ,серия международные отношения и
международное право).-№~92(4).- С.65--75. DOI
10.26577/IRILJ.2020.v92.i4.07

{\bfseries References}

1. Mochalova L.A. Strategicheskij analiz i planirovanie: uchebnik / L.A.
Mochalova, V.I. Vlasov; Ural' skij gosudarstvennyj gornyj
universitet. - 2-e izd. - Moskva : Aj Pi Ar Media , 2024. - 167 s. ISBN
978-5-4497-1853-2.{[}in Russian{]}

2. Elshibaev R.K. Sovremennoe sostojanie i napravlenija razvitija malogo
i srednego biznesa Respubliki Kazahstan// Vestnik universiteta «Turan».-
2021.-№1.- S.84-90.

DOI 10.46914/1562-2959-2021-1-1-84-90.{[}in Russian{]}

3. Kak v Kazahstane podderzhivajut malyj i srednij biznes vo vremja
pandemii?
//https://www.inform.kz/ru/kak-v-kazahstane-podderzhivayut-malyy-i-sredniy-biznes-vo-vremya-pandemii\_a3694302.Data
obrashhenija: 08.07.2024 g.{[}in Russian{]}

4. Djusembaeva L.K. Rol'{} malogo
predprinimatel' stva v jekonomicheskom razvitii RK//
International journal of Professional Science. - 2020.- №10.-
S.16-21.{[}in Russian{]}

5. Dingazieva B.D., Nurmuhanbetova D.E., Bajbatyrova G.T. Razvitie i
gosudarstvennaja podderzhka malogo biznesa v Kazahstane // Universum:
Tehnicheskie nauki: jelektron. nauchn. zhurn. 2020.- № 3.1(72.1) -
S.15-20. {[}in Russian{]}

6. Djusembaeva L.K. Infrastruktura podderzhki i razvitija
jekonomicheskogo rosta Kazahstana// Materialy Respublikanskoj
nauchno-teoreticheskoj konferencii «Sejfullinskie chtenija-14:
Molodezh', nauka, innovacii: cifrovizacija -- novyj jetap
razvitija». - 2018. - №.1. - S.148-151.
https://kazatu.edu.kz/assets/i/science/ sf14\_mat\_104.pdf.{[}in
Russian{]}

7. Koshimova M.A. Jekonomika malogo i srednego biznesa: uchebnoe posobie
/ M.A. Koshimova, A.I. Esturlieva. - Almaty: Jekonomika, 2016. - 437s. -
ISBN 978-601-225-853-0. {[}in Russian{]}

8. Arystanova N.K., Protasova O.V. Stanovlenie i razvitie malogo biznesa
v Respublike Kazahstan // Innovacionnaja jekonomika: perspektivy
razvitija i sovershenstvovanija, 2019.- №7 (41).-S.5-12. {[}in
Russian{]}

9. Zhaksylykova A.A. Problemy v razvitii malogo i srednego biznesa
Kazahstana i puti ih reshenija // Vestnik universiteta «Turan». -
2019.-№ 1-S.155-161.{[}in Russian{]}

10. Transformacija jekonomiki Kazahstana / T. Hel' m, N.
Shol' c, R. Oshakbaev, B. Uakpaev i dr. - Astana: Fond
imeni Konrada Adenaujera, 2017. - 209 c. - ISBN 978-601-06-4166-2. {[}in
Russian{]}

11. Myrzahmetova A.M., Muhametzhan A.E., Zhakupova S.T. Osobennosti i
problemy razvitija malogo i srednego predprinimatel' stva
Kazahstana v sovremennyh uslovijah//Vestnik KazNU,serija mezhdunarodnye
otnoshenija i mezhdunarodnoe pravo).-№ 92(4).- S.65--75. DOI
10.26577/IRILJ.2020.v92.i4.07.{[}in Russian{]}

\emph{{\bfseries Авторлар туралы мәліметтер}}

Мукушев Т.Б{\bfseries . -} докторант, Л.Н.Гумилев ат Еуразия Ұлттық
Университеті, Астана, Қазақстан, e-mail:
\href{mailto:Tolegen1986@mail.ru}{\nolinkurl{Tolegen1986@mail.ru}};

Жуматаева Б.А.{\bfseries -}PhD, қауымдастырылған профессор, Қ.Құлажанова
атындағы Қазақ технология және бизнес университеті, Астана, Қазақстан,
e-mail:
\href{mailto:bahyt_jumataeva@mail.ru}{\nolinkurl{bahyt\_jumataeva@mail.ru}};

Амерханова А.Б.- PhD, қауымдастырылған профессор, Л.Н.Гумилев атындағы
Еуразия ұлттық университеті» КЕАҚ, Астана, Қазақстан, e-mail:
a\_ab85@mail.ru;

Баярлин А.Е. -э.ғ.к., аға оқытушы, С.Сейфуллин атындағы Қазақ
агротехникалық зерттеу университеті, Астана қ. Қазақстан, e-mail:
\href{mailto:mtbb1986@gmail.com}{\nolinkurl{mtbb1986@gmail.com}};

Кожабергенова К.Д.- техника ғылымдарының кандидаты, қауымдастырылған
профессор,Құлажанова атындағы Қазақ технология және бизнес
университеті,Астана қ. Қазақстан, e-mail:
\href{mailto:Everest-astana@mail.ru}{\nolinkurl{Everest-astana@mail.ru}};

\emph{{\bfseries Information about the authors}}

Mukushev T.B. - doctoral students, L.N. Gumilyov at Eurasia University,
Astana, Kazakhstan,
e-mail:\href{mailto:Tolegen1986@mail.ru}{\nolinkurl{Tolegen1986@mail.ru}};

Zhumatayeva B.A. -PhD, Acting Associate Professor, K.Kulazhanova
atyndagy Kazakh technology and business university, Astana, Kazakhstan,
e-mail:
\href{mailto:bahyt_jumataeva@mail.ru}{\nolinkurl{bahyt\_jumataeva@mail.ru}};

Amerkhanova B.A.- PhD, Acting Associate Professor, L.N. Gumilyov
Eurasian National University, Astana Kazakhstan, e-mail:
\href{mailto:a_ab85@mail.ru}{\nolinkurl{a\_ab85@mail.ru}};

Bayarlin A.E. - PhD in Economics , Senior Lecturer, Kazakh Research
Agrarian University named after S. Seifullin, Astana, Kazakhstan,
e-mail:\href{mailto:mtbb1986@gmail.com}{\nolinkurl{mtbb1986@gmail.com}};

Kodzhabergenova K.D. -PhD in Technical Sciences, Associate Professor,
Kazakh University of Technology and Business named after Kulazhanov,
Astana, Kazakhstan, e-mail:
\href{mailto:Everest-astana@mail.ru}{\nolinkurl{Everest-astana@mail.ru}}